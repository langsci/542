\ea\label{ex:text1-1}
O lagudi ma howo'o wo djaga-djaga'a\\
\gll o	lagudi	ma	howo'o	wo	jaga{\textasciitilde}jaga-'a\\
     \textsc{nm}	legundi	\textsc{rnm}	fruit	3\textsc{sg.m}.\textsc{a}	guard-?\\
\glt `The guardian of the \textit{legundi} fruits.'\footnote{MZ: Modole \textit{lagudi} is a loan of Indonesian \textit{legundi}. The name refers to several trees of the \textit{Vitex} genus, potentially \textit{Vitex trifolia} (simpleleaf chasetree).}
\z


\ea\label{ex:text1-2}
O roeae'a de jo loa ma eha de ma dea,\\
\gll o	roeae-o'a	de	yo	loa	ma	eha	de	ma	dea\\
     \textsc{nm}	?riot-\textsc{locv}	\textsc{conn}	3\textsc{hpl}.\textsc{a}	flee	\textsc{rnm}	mother	\textsc{conn}	\textsc{rnm}	father\\
\glt `During a riot they fled, the mother and the father,'\footnote{MZ: \textit{Roeae} was not recognized by my informant. Ellen apparently connected it to Tobelo \textit{ruae} `create an uproar, rage, scream, make noise' \citep[321]{hueting1908b}.}
\z

\ea\label{ex:text1-3}
de ma ngoa'a wi tai de wo ali;\\
\gll de	ma	ngoa'a	wi	tai	de	wo	ali\\
     \textsc{conn}	\textsc{rnm}	child	3\textsc{sg.m}.\textsc{u}	carry.on.back	\textsc{conn}	3\textsc{sg.m}.\textsc{a}	weep\\
\glt `and their child they carried on their backs, and he cried;'
\z

\ea\label{ex:text1-4}
de ma dea wo temo:\\
\gll de	ma	dea	wo	temo\\
     \textsc{conn}	\textsc{rnm}	father	3\textsc{sg.m}.\textsc{a}	say\\
\glt `and the father said:'
\z

\ea\label{ex:text1-5}
``Ano ge to temo uwau no wi tai.''\\
\gll 'a'ano	ge	to	temo	uwa-ou	no wi	tai\\
     just.now	\textsc{dist}	1\textsc{sg}.\textsc{a}	say	\textsc{proh}-\textsc{foc}	2\textsc{sg}.\textsc{a} 3\textsc{sg.m}.\textsc{u}	carry.on.back\\
\glt `{``}I just said, don't carry him.{''}'
\z

\ea\label{ex:text1-6}
I tugumia de ja iha\\
\gll i	togumu-i'a	de	ya	iha\\
     3\textsc{nh}.\textsc{a}	finish-\textsc{dir}.\textsc{itv}	\textsc{conn}	3\textsc{nh}>3\textsc{nh}	\textsc{dir}.\textsc{land}\\
\glt `After that they went landward'
\z

\newpage
\ea\label{ex:text1-7}
o daduhulu ma igutu'u de wi hodoau,\\
\gll o	daduhulu	ma	igutu-u'u	de	wi	ho-doa-ou\\
     \textsc{nm}	?ant	\textsc{rnm}	nest-\textsc{dir}.\textsc{down}	\textsc{conn}	3\textsc{sg.m}.\textsc{u}	?-put-\textsc{already}\\
\glt `[and] put him down on an anthill,'\footnote{MZ: \textit{Daduhulu} was not recognized by my informant.}
\z

\ea\label{ex:text1-8}
ma a wi diomaali gena'a,\\
\gll ma	'a	wi	dV-oma-oli	geena-o'a\\
     but	\textsc{foc}	3\textsc{sg.m}.\textsc{u}	\textsc{appl}-return-\textsc{again}	\textsc{dist}:\textsc{pro}.3\textsc{nh}-\textsc{locv}\\
\glt `but they returned to him again over there'
\z


\ea\label{ex:text1-9}
de o gota ma beleulu'u wi hodoau.\\
\gll de	o	gota	ma	beleulu-u'u	wi	ho-doa-ou\\
     \textsc{conn}	\textsc{nm}	wood	\textsc{rnm}	?root.spur-\textsc{dir}.\textsc{down}	3\textsc{sg.m}.\textsc{u}	?-put-\textsc{already}\\
\glt `and put him down on the root spur of a tree.'\footnote{MZ: \textit{Beleulu} is translated by Ellen as `root spur' (``worteluitlooper''). In the wordlist, it is given as `washed up sand in the sea or river' and in Pagu, it means `river delta' \citep{peranginangin2014}.}
\z

\ea\label{ex:text1-10}
Gena'a de jo tagi ho o de'u tumudingi de o ngaili tumudingi ja paha,\\
\gll geena-o'a	de	yo	tagi	ho	o	de'u	tumudingi	de	o	ngaili	tumudingi	ya	paha\\
     \textsc{dist}:\textsc{pro}.3\textsc{nh}-\textsc{locv}	\textsc{conn}	3\textsc{hpl}.\textsc{a}	go	thus	\textsc{nm}	mountain	seven	\textsc{conn}	\textsc{nm}	river	seven	3\textsc{nh}>3\textsc{nh}	leave\\
\glt `Then they went, over seven hills and seven rivers they left,'\footnote{MZ: The number seven plays a special role in Core North Halmaheran tales \citep[295]{hueting1908}. The cultural background is unknown to me.}
\z

\ea\label{ex:text1-11}
ma a jo ihenohi wo ali;\\
\gll ma	'a	yo	ihene-ohi	wo	ali\\
     but	\textsc{foc}	3\textsc{hpl}.\textsc{a}	hear-\textsc{still}	3\textsc{sg.m}.\textsc{a}	weep\\
\glt `but they still heard him cry,'
\z

\ea\label{ex:text1-12}
de ma eha mo mau mi diomali,\\
\gll de	ma	eha	mo	mau	mi	dV-oma-oli\\
     \textsc{conn}	\textsc{rnm}	mother	3\textsc{sg.f}.\textsc{a}	want	3\textsc{sg.f}>3\textsc{sg.m}	\textsc{appl}-return-\textsc{again}\\
\glt `and the mother wanted to return to him'
\z

\ea\label{ex:text1-13}
ma ma dea a wo olu'u, wo temo:\\
\gll ma	ma	dea	'a	wo	olu'u	wo	temo\\
     but	\textsc{rnm}	father	\textsc{foc}	3\textsc{sg.m}.\textsc{a}	refuse	3\textsc{sg.m}.\textsc{a}	say\\
\glt `but the father did not want that, he said:'
\z

\ea\label{ex:text1-14}
``Na'o no wi diomahi de ngoi to hilau.''\\
\gll na'o	no	wi	dV-oma-ohi	de	ngoi	to	hila-ou\\
     \textsc{cond}	2\textsc{sg}.\textsc{a}	3\textsc{sg.m}.\textsc{u}	\textsc{appl}-return-\textsc{still}	\textsc{conn}	\textsc{pro}.1\textsc{sg}	1\textsc{sg}.\textsc{a}	first-\textsc{already}\\
\glt `{``}If you (sg.) return to him, then I will [go] forward.{''}'
\z


\ea\label{ex:text1-15}
De ma ngoa ge wa odomo duga-duga a o heleene,\\
\gll de	ma	ngoa	ge	wa	odomo	duga{\textasciitilde}duga	'a	o	heleene\\
     \textsc{conn}	\textsc{rnm}	child	\textsc{dist}	3\textsc{sg.m}>3\textsc{nh}	eat	only	\textsc{foc}	\textsc{nm}	k.o.plant\\
\glt `And that child ate only eggplants,'\footnote{ MZ: According to my informant, \textit{heleene} is the name of a thorny, eggplant-like plant that grows in the forest.}
\z

\ea\label{ex:text1-16}
de na'o wo lamo'o'au, ma moi wo hupuo o a'ele Itio\\
\gll de	na'o	wo	lamo'o-o'au	ma	moi	wo	hupu-ou	o	a'ele	Itio\\
     \textsc{conn}	\textsc{cond}	3\textsc{sg.m}.\textsc{a}	big-\textsc{perf}	\textsc{rnm}	one	3\textsc{sg.m}.\textsc{a}	exit-\textsc{already}	\textsc{nm}	water	[place\_name]\\
\glt `and when he had grown up, he once left the Itio river,'\footnote{MZ: I have been unable to determine whether the Itio river exists. It was not recognized by people living on Halmahera.}
\z

\ea\label{ex:text1-17}
de wi ma'eo o bere'i mo moi ma 'oana o wange ma dumuno'a awi humu ma djodjaga,\\
\gll de	wi	ma'e-ou	o	bere'i	mo	moi	ma	'oana	o	wange	ma	dumunu-o'a	awi	humu	ma	jojaga\\
     \textsc{conn}	3\textsc{sg.m}>3\textsc{sg.f}	see-\textsc{already}	\textsc{nm}	old.person	3\textsc{sg.f}.\textsc{a}	one	\textsc{rnm}	king	\textsc{nm}	sun	\textsc{rnm}	dive-\textsc{locv}	3\textsc{sg.m}.\textsc{poss}	well	\textsc{rnm}	guardian\\
\glt `and he met an old woman, the well guardian of the Western King,'\footnote{MZ: The character of the well guardian also occurs in stories in other Core North Halmaheran languages and is always a woman (see \cite[170]{hueting1908}, \cite[418]{vanbaarda1904}).}
\z

\newpage
\ea\label{ex:text1-18}
de una ma ngoa'a ge wo temo:\\
\gll de	una	ma	ngoa'a	ge	wo	temo\\
     \textsc{conn}	\textsc{pro}.3\textsc{sg.m}	\textsc{rnm}	child	\textsc{dist}	3\textsc{sg.m}.\textsc{a}	say\\
\glt `and he, the child, said:'
\z

\ea\label{ex:text1-19}
``Apu, bote ni to'ata;''\\
\gll apu	bote	ni	to'ata\\
     grandmother	maybe	2\textsc{sg}.\textsc{u}	witch\\
\glt `{``}Grandmother, maybe you (sg.) [are] a witch,'''\footnote{MZ: \textit{To'ata} (< \textsc{pcnh} *cokat) is always translated as `witch' (`heks') by Ellen. According to other sources, it is a kind of evil spirit \citep[266]{hueting1921}.}
\z


\ea\label{ex:text1-20}
de muna mo temo:\\
\gll de	muna	mo	temo\\
     \textsc{conn}	\textsc{pro}.3\textsc{sg.f}	3\textsc{sg.f}.\textsc{a}	say\\
\glt `and she said:'
\z

\ea\label{ex:text1-21}
``Onu, ngoi 'o i to'atua.''\\
\gll onu	ngoi	'o	i	to'ata-ua\\
     child[address]	\textsc{pro}.1\textsc{sg}	\textsc{emph}	1\textsc{sg}.\textsc{u}	witch-\textsc{neg}\\
\glt `{``}Child, I am not a witch.'''
\z

\ea\label{ex:text1-22}
De muna mo hano mo temo:\\
\gll de	muna	mo	hano	mo	temo\\
     \textsc{conn}	\textsc{pro}.3\textsc{sg.f}	3\textsc{sg.f}.\textsc{a}	ask	3\textsc{sg.f}.\textsc{a}	say\\
\glt `And she asked, saying:'
\z

\ea\label{ex:text1-23}
``Ngona neena o'ia'a ngona de na ino?''\\
\gll ngona	neena	o'ia-o'a	ngona	de	na	ino\\
     \textsc{pro}.2\textsc{sg}	\textsc{prox}:\textsc{pro}.3\textsc{nh}	what-\textsc{locv}	\textsc{pro}.2\textsc{sg}	\textsc{conn}	2\textsc{sg}>3\textsc{nh}	\textsc{dir}.\textsc{ven}\\
\glt `{``}You (sg.) here, where are you (sg.) coming from?'''
\z

\ea\label{ex:text1-24}
De una wo temo:\\
\gll de	una	wo	temo\\
     \textsc{conn}	\textsc{pro}.3\textsc{sg.m}	3\textsc{sg.m}.\textsc{a}	say\\
\glt `And he said:'
\z

\ea\label{ex:text1-25}
``Ngoi o a'ele Itio de ta o'o'';\\
\gll ngoi	o	a'ele	Itio	de	ta	o'o\\
     \textsc{pro}.1\textsc{sg}	\textsc{nm}	water	[place\_name]	\textsc{conn}	1\textsc{sg}>3\textsc{nh}	\textsc{dir}.\textsc{sea}\\
\glt `{``}I have come from the Itio River seawards,'''
\z

\ea\label{ex:text1-26}
de ma ngoa'a ge wo hano o igono,\\
\gll de	ma	ngoa'a	ge	wo	hano	o	igono\\
     \textsc{conn}	\textsc{rnm}	child	\textsc{dist}	3\textsc{sg.m}.\textsc{a}	ask	\textsc{nm}	coconut\\
\glt `and the child asked for coconuts,'
\z

\ea\label{ex:text1-27}
wo temo:``Apu! bote de ani igono'a?''\\
\gll wo	temo	apu    bote	de	ani	igono-o'a\\
     3\textsc{sg.m}.\textsc{a}	say	grandmother   maybe	\textsc{conn}	2\textsc{sg}.\textsc{poss}	coconut-\textsc{locv}\\
\glt `he said: ``Grandmother! do you (sg.) maybe have coconuts?'''
\z

\ea\label{ex:text1-28}
De muna mo temo:\\
\gll de	muna	mo	temo\\
     \textsc{conn}	\textsc{pro}.3\textsc{sg.f}	3\textsc{sg.f}.\textsc{a}	say\\
\glt `And she said:'
\z

\ea\label{ex:text1-29}
``O igono ena naga, ma bote o no doa-doawa?''\\
\gll o	igono	ena	naga	ma	bote	'o	no	doa{\textasciitilde}doa-ua\\
     \textsc{nm}	coconut	\textsc{pro}.3\textsc{nh}	\textsc{exist}	but	maybe	\textsc{emph}	2\textsc{sg}.\textsc{a}	\textsc{rdpl}{\textasciitilde}climb-\textsc{neg}\\
\glt `{``}There are coconuts here, but maybe you (sg.) don't climb?'''
\z

\ea\label{ex:text1-30}
de una wo temo:\\
\gll de	una	wo	temo\\
     \textsc{conn}	\textsc{pro}.3\textsc{sg.m}	3\textsc{sg.m}.\textsc{a}	say\\
\glt `And he said:'
\z

\ea\label{ex:text1-31}
``Apu, a ta na'o to doa'';\\
\gll apu	'a	ta	na'o	to	doa\\
     grandmother	\textsc{foc}	1\textsc{sg}>3\textsc{nh}	know	1\textsc{sg}.\textsc{a}	climb\\
\glt `{``}Grandmother, I know [how to] climb;'''
\z

\newpage
\ea\label{ex:text1-32}
gena'ade wa doa ma igono,\\
\gll geena-o'a-de	wa	doa	ma	igono\\
     \textsc{dist}:\textsc{pro}.3\textsc{nh}-\textsc{locv}-\textsc{conn}	3\textsc{sg.m}>3\textsc{nh}	climb	\textsc{rnm}	coconut\\
\glt `then he climbed the coconut [tree],'\footnote{GJE: picked them}
\z

\ea\label{ex:text1-33}
o oti tumudingi, ho muna o oti hoata, una o oti haange.\\
\gll o	oti	tumudingi	ho	muna	o	oti	hoata	una	o	oti	haange\\
     \textsc{nm}	\textsc{class}.round\_thing	seven	thus	\textsc{pro}.3\textsc{sg.f}	\textsc{nm}	\textsc{class}.round\_thing	four	\textsc{pro}.3\textsc{sg.m}	\textsc{nm}	\textsc{class}.round\_thing	three\\
\glt `seven [coconuts], four [for] her, three [for him].'
\z

\ea\label{ex:text1-34}
Gena'ade j'odomo,\\
\gll geena-o'a-de	yo-odomo\\
     \textsc{dist}:\textsc{pro}.3\textsc{nh}-\textsc{locv}-\textsc{conn}	3\textsc{hpl}.\textsc{a}-eat\\
\glt `Then they ate'
\z

\ea\label{ex:text1-35}
de i duangou de mo temo:\\
\gll de	i	duanga-ou	de	mo	temo\\
     \textsc{conn}	3\textsc{nh}.\textsc{a}	finish-\textsc{already}	\textsc{conn}	3\textsc{sg.f}.\textsc{a}	say\\
\glt `and after that she said:'
\z

\ea\label{ex:text1-36}
``Onu! po ma hioru-oru'';\\
\gll onu    po	ma	hi-oru{\textasciitilde}oru\\
     child[address] 1\textsc{pl}.\textsc{in}.\textsc{a}	\textsc{mid}	\textsc{caus}-\textsc{rdpl}{\textasciitilde}row\\
\glt `{``}Child! [shall] we (in.) go rowing?'''
\z

\ea\label{ex:text1-37}
de una wo temo:\\
\gll de	una	wo	temo\\
     \textsc{conn}	\textsc{pro}.3\textsc{sg.m}	3\textsc{sg.m}.\textsc{a}	say\\
\glt `and he said:'
\z

\newpage
\ea\label{ex:text1-38}
``Ma i'au apu, po ma hioru-oru.''\\
\gll ma	i'a-ou	apu	po	ma	hi-oru{\textasciitilde}oru\\
     \textsc{rnm}	\textsc{dir}.\textsc{itv}-\textsc{foc}	grandmother	1\textsc{pl}.\textsc{in}.\textsc{a}	\textsc{mid}	\textsc{caus}-\textsc{rdpl}{\textasciitilde}row\\
\glt `{``}That's good grandmother, [let]'s (in.) [go] rowing.'''\footnote{MZ: \textit{(Ma) i'a} occurs several times as a kind of interjection and is translated as `good' by Ellen. I do not know whether this is an additional function of the \textsc{itive} directional or an unrelated root.}
\z

\ea\label{ex:text1-39}
Gena'ade jo ma hioru-oru\\
\gll geena-o'a-de	yo	ma	hi-oru{\textasciitilde}oru\\
     \textsc{dist}:\textsc{pro}.3\textsc{nh}-\textsc{locv}-\textsc{conn}	3\textsc{hpl}.\textsc{a}	\textsc{mid}	\textsc{caus}-\textsc{rdpl}{\textasciitilde}row\\
\glt `Then they went rowing'
\z

\ea\label{ex:text1-40}
ho jo i ma'eiha ma 'oana o wange ma dumuno'a awi moholehe jo tumudingi\\
\gll ho	yo	'i	ma'e-iha	ma	'oana	o	wange	ma	dumunu-o'a	awi	moholehe	yo	tumudingi\\
     thus	3\textsc{hpl}.\textsc{a}	3\textsc{hpl}.\textsc{u}	see-\textsc{dir}.\textsc{land}	\textsc{rnm}	king	\textsc{nm}	sun	\textsc{rnm}	dive-\textsc{locv}	3\textsc{sg.m}.\textsc{poss}	maiden	3\textsc{hpl}.\textsc{a}	seven\\
\glt `and landward they met the seven maidens of the Western King,'
\z

\ea\label{ex:text1-41}
o a'ele jo hiono'o,\\
\gll o	a'ele	yo	hi-ono'o\\
     \textsc{nm}	water	3\textsc{hpl}.\textsc{a}	\textsc{caus}-scoop.water\\
\glt `they scooped up water,'
\z

\ea\label{ex:text1-42}
de jo temo:\\
\gll de	yo	temo\\
     \textsc{conn}	3\textsc{hpl}.\textsc{a}	say\\
\glt `and they said:'
\z

\ea\label{ex:text1-43}
``Uai apu mia ilanga.''\\
\gll hai	apu	mia	ilanga\\
     hey	grandmother	1\textsc{pl}.\textsc{ex}.\textsc{poss}	brother\\
\glt `{``}Hey grandmother, our (ex.) brother.'''
\z

\newpage
\ea\label{ex:text1-44}
Gena'ade jo ma ohi'i,\\
\gll geena-o'a-de	yo	ma	ohi'i\\
     \textsc{dist}:\textsc{pro}.3\textsc{nh}-\textsc{locv}-\textsc{conn}	3\textsc{hpl}.\textsc{a}	\textsc{mid}	bathe\\
\glt `Then they went bathing,'
\z

\ea\label{ex:text1-45}
de ona ma moholehe ja tumudingi jo temo:\\
\gll de	ona	ma	moholehe	ya	tumudingi	yo	temo\\
     \textsc{conn}	\textsc{pro}.3\textsc{plh}	\textsc{rnm}	maiden	3\textsc{nh}>3\textsc{nh}	seven	3\textsc{hpl}.\textsc{a}	say\\
\glt `and they, the seven maidens, said:'
\z

\ea\label{ex:text1-46}
``Apu, po wi tila ne o humu'u'',\\
\gll apu	po	wi	tila	ne	o	humu-u'u\\
     grandmother	1\textsc{pl}.\textsc{in}.\textsc{a}	3\textsc{sg.m}.\textsc{u}	push	\textsc{prox}	\textsc{nm}	well-\textsc{dir}.\textsc{down}\\
\glt `{``}Grandmother! [let]'s (in.) push him here down the well,'''
\z

\ea\label{ex:text1-47}
de wi tila.\\
\gll de	wi	tila\\
     \textsc{conn}	3\textsc{sg.m}.\textsc{u}	push\\
\glt `and they pushed him [in].'
\z

\ea\label{ex:text1-48}
Ho wo puda,\\
\gll ho	wo puda\\
     thus	3\textsc{sg.m}.\textsc{a} float\\
\glt `But he floated'
\z

\ea\label{ex:text1-49}
de awi roehe ma hongona o hala'a, ma hongona o gurahi;\\
\gll de	awi	roehe	ma	hongona	o	hala'a	ma	hongona	o	gurahi\\
     \textsc{conn}	3\textsc{sg.m}.\textsc{poss}	body	\textsc{rnm}	half	\textsc{nm}	silver	\textsc{rnm}	half	\textsc{nm}	gold\\
\glt `and one half of his body was silver, and the [other] half was gold,'\footnote{MZ: Being half silver and half gold seems to be a marker of supernatural beings (compare \textref{chapter:text05}). It also occurs frequently in Tobelo folk tales (see \cite{hueting1908}).}
\z

\ea\label{ex:text1-50}
awi utu ma o geena.\\
\gll awi	utu	ma	'o geena\\
     3\textsc{sg.m}.\textsc{poss}	hair	\textsc{rnm}	\textsc{emph} \textsc{dist}:\textsc{pro}.3\textsc{nh}\\
\glt `his hair was like that as well.'
\z

\ea\label{ex:text1-51}
Gena'ade ma 'oana o wange ma hiwa'a awi ngoa'a ja tumudingi ja u'u jo ma pahiara,\\
\gll geena-o'a-de	ma	'oana	o	wange	ma	hiwa-o'a	awi	ngoa'a	ya	tumudingi	ya	u'u	yo	ma	pahiara\\
     \textsc{dist}:\textsc{pro}.3\textsc{nh}-\textsc{locv}-\textsc{conn}	\textsc{rnm}	king	\textsc{nm}	sun	\textsc{rnm}	shine-\textsc{locv}	3\textsc{sg.m}.\textsc{poss}	child	3\textsc{nh}>3\textsc{nh}	seven	3\textsc{nh}>3\textsc{nh}	\textsc{dir}.\textsc{down}	3\textsc{hpl}.\textsc{a}	\textsc{mid}	walk\\
\glt `Then the seven children of the Eastern King walked southwards,'
\z 

\ea\label{ex:text1-52}
i togumia de jo i timaju;\\
\gll i	togumu-i'a	de	yo	'i	timayu\\
     3\textsc{nh}.\textsc{a}	finish-\textsc{dir}.\textsc{itv}	\textsc{conn}	3\textsc{hpl}.\textsc{a}	3\textsc{hpl}.\textsc{u}	serve.betel\\
\glt `and after that they served them betel,'
\z

\ea\label{ex:text1-53}
de manga ilanga wo hupu,\\
\gll de	manga	ilanga	wo	hupu\\
     \textsc{conn}	3\textsc{hpl}.\textsc{poss}	brother	3\textsc{sg.m}.\textsc{a}	exit\\
\glt `and their brother came out,'\footnote{GJE: of the well}
\z

\ea\label{ex:text1-54}
ho jo i henewutuo'a,\\
\gll ho	yo	'i	henewutu-o'a\\
     thus	3\textsc{hpl}.\textsc{a}	3\textsc{hpl}.\textsc{u}	faint-\textsc{lim}\\
\glt `so they fainted,'\footnote{MZ: The index combination \textit{yo 'i} `3\textsc{hpl}>3\textsc{hpl}' is unexpected here since the brother (the causer of the fainting) is singular. In Tabaru, \textit{joki} occurs as third person plural human \textit{undergoer} index with experiencer predicates, e.g. \textit{jokisawini} `they are hungry' (\cite[352]{fortgens1928}). This may also be the case in Modole. Similar unclear occurrences of \textit{jo 'i} can be found in [B.\ref{ex:text2-71}] and [D.\ref{ex:text4-55}].}
\z

\ea\label{ex:text1-55}
de manga mumuu'u wa tuu'u aha jo mo todo'ana.\\
\gll de	manga	mumu-u'u	wa	tu-u'u	aha	yo	ma	todo'ana\\
     \textsc{conn}	3\textsc{hpl}.\textsc{poss}	head-\textsc{dir}.\textsc{down}	3\textsc{sg.m}>3\textsc{nh}	?-\textsc{dir}.\textsc{down}	as.a.consequence	3\textsc{hpl}.\textsc{a}	\textsc{mid}	startle\\
\glt `and he ?came down on their heads so that they were startled.'
\z

\ea\label{ex:text1-56}
De ma 'oana o wange ma dumuno'a wo hano wo temo:\\
\gll de	ma	'oana	o	wange	ma	dumunu-o'a	wo	hano	wo	temo\\
     \textsc{conn}	\textsc{rnm}	king	\textsc{nm}	sun	\textsc{rnm}	dive-\textsc{locv}	3\textsc{sg.m}.\textsc{a}	ask	3\textsc{sg.m}.\textsc{a}	say\\
\glt `And the Western King asked, saying:'
\z

\ea\label{ex:text1-57}
``Neena nia u'u nio hiatubele o hepa-hepa,\\
\gll neena	nia	u'u	nio	hi-'a-tubele	o	hepa{\textasciitilde}hepa\\
     \textsc{prox}:\textsc{pro}.3\textsc{nh}	2\textsc{pl}>3\textsc{nh}	\textsc{dir}.\textsc{down}	2\textsc{pl}.\textsc{a}	\textsc{caus}-\textsc{vpl}-provoke	\textsc{nm}	\textsc{rdpl}{\textasciitilde}kick\\
\glt `{``}Now you (pl.) go down\footnote{GJE: the King's abode is usually on an elevation} to play ball\footnote{MZ: \textit{Hepa-hepa} is some kind of ball game. \citet[296]{hueting1908} notes that it was explained to him as `tossing up and catching fruit' but that he had never seen it played among the Tobelo (also see \cite[326]{vanbaarda1904}). In Indonesian, \textit{sepa} means `kick' but \citet[327]{vanbaarda1904} notes that the semantic connection is rather vague. He connects it to the homonymous lexeme in Sangir (compare \citet[102]{sneddon1984}: Sangir \textit{sepa} `to play with a rattan ball').},'
\z

\ea\label{ex:text1-58}
e'ola nio hiatubele o namo?''\\
\gll e'ola	nio	hi-'a-tubele	o	namo\\
     or	2\textsc{pl}.\textsc{a}	\textsc{caus}-\textsc{vpl}-provoke	\textsc{nm}	chicken\\
\glt `or will you (pl.) let the cocks fight?'''
\z

\ea\label{ex:text1-59}
de ona jo temo: ``Neena ngomi mia u'u,\\
\gll de	ona	yo	temo	neena	ngomi	mia	u'u\\
     \textsc{conn}	\textsc{pro}.3\textsc{plh}	3\textsc{hpl}.\textsc{a}	say	\textsc{prox}:\textsc{pro}.3\textsc{nh}	\textsc{pro}.1\textsc{pl}.\textsc{ex}	1\textsc{pl}.\textsc{ex}>3\textsc{nh}	\textsc{dir}.\textsc{down}\\
\glt `And they said: ``Now we'll (ex.) go down,'
\z

\ea\label{ex:text1-60}
mio hiatubele, o hepa-hepa ma 'oua,\\
\gll mio	hi-'a-tubele	o	hepa{\textasciitilde}hepa	ma	'o-ua\\
     1\textsc{pl}.\textsc{ex}.\textsc{a}	\textsc{caus}-\textsc{vpl}-provoke	\textsc{nm}	\textsc{rdpl}{\textasciitilde}kick	but	\textsc{emph}-\textsc{neg}\\
\glt `we'll (ex.) play ball, no,'
\z

\ea\label{ex:text1-61}
o namo ma 'oua,\\
\gll o	namo	ma	'o-ua\\
     \textsc{nm}	chicken	\textsc{rnm}	\textsc{emph}-\textsc{neg}\\
\glt `[we'll let] the cocks [fight], no,'
\z

\ea\label{ex:text1-62}
ho a ngomi mio papahiara.''\\
\gll ho	'a	ngomi	mio	CV{\textasciitilde}pahiara\\
     thus	\textsc{foc}	\textsc{pro}.1\textsc{pl}.\textsc{ex}	1\textsc{pl}.\textsc{ex}.\textsc{a}	\textsc{rdpl}{\textasciitilde}walk\\
\glt `so we'll (ex.) just go for a walk.'''
\z

\ea\label{ex:text1-63}
De gena'ade ma 'oana o hupera 'a wo tuoau,\\
\gll de	geena-o'a-de	ma	'oana	o	hupera	'a	wo	tu'u-ou\\
     \textsc{conn}	\textsc{dist}:\textsc{pro}.3\textsc{nh}-\textsc{locv}-\textsc{conn}	\textsc{rnm}	king	\textsc{nm}	k.o.small.cannon	\textsc{foc}	3\textsc{sg.m}.\textsc{a}	shoot-\textsc{already}\\
\glt `And then the king fired a cannon,'
\z

\ea\label{ex:text1-64}
ho a jo lioau,\\
\gll ho	'a	yo	lio-ou\\
     thus	\textsc{foc}	3\textsc{hpl}.\textsc{a}	return-\textsc{already}\\
\glt `so they returned,'\footnote{GJE: Instead of calling his children, the king fired a cannon as a sign that they have to come back.}
\z

\ea\label{ex:text1-65}
de onali ja ie jo ma peleti 'a jo ma boha.\\
\gll de	ona-oli	ya	ie	yo	ma	peleti	'a	yo	ma	boha\\
     \textsc{conn}	\textsc{pro}.3\textsc{plh}-\textsc{again}	3\textsc{nh}>3\textsc{nh}	\textsc{dir}.\textsc{up}	3\textsc{hpl}.\textsc{a}	\textsc{mid}	do.hair	\textsc{foc}	3\textsc{hpl}.\textsc{a}	\textsc{mid}	put.hair.in.wreath\\
\glt `and they climbed up again [and] they did their hair in a wreath.'
\z

\ea\label{ex:text1-66}
De 'a jo ma ena'a,\\
\gll de	'a	yo	ma	ena'a\\
     \textsc{conn}	\textsc{foc}	3\textsc{hpl}.\textsc{a}	\textsc{mid}	areca\\
\glt `And they chewed betel,'\footnote{MZ: \textit{Ena'a} is the name of the areca nut (Indonesian \textit{pinang}). It is chewed in combination with betel leaves (Modole \textit{bidoho}, Indonesian \textit{sirih}). Therefore, the nut is also known as `betel nut' in English. I have translated \textit{ena'a} used as a verb as `chew betel' even though technically it is the nut and not the leaves that are referred to.}
\z

\newpage
\ea\label{ex:text1-67}
ma duangino, ma 'oana o wange ma dumuno'a o hupera 'a wo tuoau\\
\gll ma	duanga-ino	ma	'oana	o	wange	ma	dumunu-o'a	o	hupera	'a	wo	tu'u-ou\\
     \textsc{rnm}	finish-\textsc{dir}.\textsc{ven}	\textsc{rnm}	king	\textsc{nm}	sun	\textsc{rnm}	dive-\textsc{locv}	\textsc{nm}	k.o.small.cannon	\textsc{foc}	3\textsc{sg.m}.\textsc{a}	shoot-\textsc{already}\\
\glt `afterwards the Western King fired a cannon,'
\z

\ea\label{ex:text1-68}
ho 'a jo lioau,\\
\gll ho	'a	yo	lio-ou\\
     thus	\textsc{foc}	3\textsc{hpl}.\textsc{a}	return-\textsc{already}\\
\glt `so they returned,'
\z

\ea\label{ex:text1-69}
de aha onali ja u'u jo pahiara,\\
\gll de	aha	ona-oli	ya	u'u	yo	pahiara\\
     \textsc{conn}	as.a.consequence	\textsc{pro}.3\textsc{plh}-\textsc{again}	3\textsc{nh}>3\textsc{nh}	\textsc{dir}.\textsc{down}	3\textsc{hpl}.\textsc{a}	walk\\
\glt `and after that they went down again to go for a walk,'
\z

\ea\label{ex:text1-70}
jo ma peleti ja tjara Hilamu.\\
\gll yo	ma	peleti	ya	cara	Hilamu\\
     3\textsc{hpl}.\textsc{a}	\textsc{mid}	do.hair	3\textsc{nh}>3\textsc{nh}	manner	Muslim\\
\glt `they did their hair in the manner of the Muslims.'
\z

\ea\label{ex:text1-71}
De jo i timaju,\\
\gll de	yo	'i	timayu\\
     \textsc{conn}	3\textsc{hpl}.\textsc{a}	3\textsc{hpl}.\textsc{u}	serve.betel\\
\glt `And they served them betel,'
\z

\ea\label{ex:text1-72}
manga tupa 'a o nahinara,\\
\gll manga	tupa	'a	o	nahinara\\
     3\textsc{hpl}.\textsc{poss}	betel.box	\textsc{foc}	\textsc{nm}	k.o.betel.box\\
\glt `their betel box was a \textit{nahinara}'\footnote{GJE: A small kind of box.}
\z

\newpage
\ea\label{ex:text1-73}
de 'a jo i timaju\\
\gll de	'a	yo	'i	timayu\\
     \textsc{conn}	\textsc{foc}	3\textsc{hpl}.\textsc{a}	3\textsc{hpl}.\textsc{u}	serve.betel\\
\glt `and they served them betel'
\z

\ea\label{ex:text1-74}
ma duangino, ma 'oana o wange ma hiwa'a o hupera wo tu'u\\
\gll ma	duanga-ino	ma	'oana	o	wange	ma	hiwa-o'a	o	hupera	wo	tu'u\\
     \textsc{rnm}	finish-\textsc{dir}.\textsc{ven}	\textsc{rnm}	king	\textsc{nm}	sun	\textsc{rnm}	shine-\textsc{locv}	\textsc{nm}	k.o.small.cannon	3\textsc{sg.m}.\textsc{a}	shoot\\
\glt `[and] after that, the Eastern King fired a cannon,'
\z

\ea\label{ex:text1-75}
ho 'a jo lioau.\\
\gll ho	'a	yo	lio-ou\\
     thus	\textsc{foc}	3\textsc{hpl}.\textsc{a}	return-\textsc{already}\\
\glt `so they returned.'
\z

\ea\label{ex:text1-76}
De ma lagudi ma howo'o wo djaga-djaga wa ino mamane'a,\\
\gll de	ma	lagudi	ma	howo'o	wo	jaga{\textasciitilde}jaga	wa	ino	mamane-o'a\\
     \textsc{conn}	\textsc{rnm}	legundi	\textsc{rnm}	fruit	3\textsc{sg.m}.\textsc{a}	guard	3\textsc{sg.m}>3\textsc{nh}	\textsc{dir}.\textsc{ven}	lover-\textsc{locv}\\
\glt `And the \textit{legundi} guardian approached his lover'
\z

\ea\label{ex:text1-77}
ma lagudi ma howo'o o beta-beta moi wo hidoadoa'a,\\
\gll ma	lagudi	ma	howo'o	o	beta{\textasciitilde}beta	moi	wo	hi-doa{\textasciitilde}doa-o'a\\
     \textsc{rnm}	legundi	\textsc{rnm}	fruit	\textsc{nm}	bunch[fruit]	one	3\textsc{sg.m}.\textsc{a}	\textsc{caus}-\textsc{rdpl}{\textasciitilde}put-\textsc{lim}\\
\glt `and he handed [her] a bunch of \textit{legundi} fruits,'
\z

\ea\label{ex:text1-78}
ma oma dawonguau,\\
\gll ma	'o-ma	dawongo-ua-ou\\
     but	\textsc{emph}-3\textsc{sg.f}>3\textsc{nh}	take-\textsc{neg}-\textsc{already}\\
\glt `but she did not accept it anymore,'
\z

\newpage
\ea\label{ex:text1-79}
de wo temo:\\
\gll de	wo	temo\\
     \textsc{conn}	3\textsc{sg.m}.\textsc{a}	say\\
\glt `and he said:'
\z

\ea\label{ex:text1-80}
``i dodoa ho ona dawongua?''\\
\gll i	dodoa	ho	'o-na	dawongo-ua\\
     3\textsc{nh}.\textsc{a}	why	thus	\textsc{emph}-2\textsc{sg}>3\textsc{nh}	take-\textsc{neg}\\
\glt `{``}Why don't you (sg.) take it?'''
\z

\ea\label{ex:text1-81}
De ma 'oana o wange ma dumuno'a awi ngoa'a ja u'u\\
\gll de	ma	'oana	o	wange	ma	dumunu-o'a	awi	ngoa'a	ya	u'u\\
     \textsc{conn}	\textsc{rnm}	king	\textsc{nm}	sun	\textsc{rnm}	dive-\textsc{locv}	3\textsc{sg.m}.\textsc{poss}	child	3\textsc{nh}>3\textsc{nh}	\textsc{dir}.\textsc{down}\\
\glt `And the children of the Western King went down'
\z

\ea\label{ex:text1-82}
de jo temo:\\
\gll de	yo	temo\\
     \textsc{conn}	3\textsc{hpl}.\textsc{a}	say\\
\glt `and they said:'
\z

\ea\label{ex:text1-83-1}
``Baba! ano mia ie de ma lagudi ma howo'o wo djaga-djaga wa ino\\
\gll baba   'a'ano	mia	ie	de	ma	lagudi	ma	howo'o	wo	jaga{\textasciitilde}jaga	wa	ino\\
     father just.now	1\textsc{pl}.\textsc{ex}>3\textsc{nh}	\textsc{dir}.\textsc{up}	\textsc{conn}	\textsc{rnm}	legundi	\textsc{rnm}	fruit	3\textsc{sg.m}.\textsc{a}	guard	3\textsc{sg.m}>3\textsc{nh}	\textsc{dir}.\textsc{ven}\\
\glt `{``}Father! a moment ago we (ex.) went up and the \textit{legundi} guardian came'
\z

\ea\label{ex:text1-84}
wo hidoadoa'a o beta-beta moi mamane'a,\\
\gll wo	hi-doa{\til}doa-o'a	o	beta{\textasciitilde}beta	moi	mamane-i'a\\
     3\textsc{sg.m}.\textsc{a}	\textsc{caus}-\textsc{rdpl}{\textasciitilde}put-\textsc{lim}	\textsc{nm}	bunch[fruit]	one	lover-\textsc{dir.itv}\\
\glt `and gave his lover a bunch [of \textit{legundi} fruits],'
\z

\ea\label{ex:text1-85}
ma o ma dawongua.''\\
\gll ma	'o	ma	dawongo-ua\\
     but	\textsc{emph}	3\textsc{sg.f}>3\textsc{nh}	take-\textsc{neg}\\
\glt `but she didn't take it.'''
\z

\ea\label{ex:text1-86}
De ma moioli ja ie,\\
\gll de	ma	moi-oli	ya	ie\\
     \textsc{conn}	\textsc{rnm}	one-\textsc{again}	3\textsc{nh}>3\textsc{nh}	\textsc{dir}.\textsc{up}\\
\glt `And once again they went up,'
\z

\ea\label{ex:text1-87}
o dodoto ami babarihi a mu ma gaha,\\
\gll o	dodoto	ami	babarihi	'a	mu	ma	gaha\\
     \textsc{nm}	younger.sibling	3\textsc{sg.f}.\textsc{poss}	broom	\textsc{foc}	3\textsc{sg.f}.\textsc{a}	\textsc{mid}	take\\
\glt `the youngest took her broom with her,'
\z

\ea\label{ex:text1-88}
de ma ie 'a mo gila-gila ma lagudi ma howo'o ma goa'a mo parihi.\\
\gll de	ma	ie	'a	mo	gila{\textasciitilde}gila	ma	lagudi	ma	howo'o	ma	goa-o'a	mo	parihi\\
     \textsc{conn}	3\textsc{sg.f}>3\textsc{nh}	\textsc{dir}.\textsc{up}	\textsc{foc}	3\textsc{sg.f}.\textsc{a}	straight	\textsc{rnm}	legundi	\textsc{rnm}	fruit	\textsc{rnm}	?lower.part-\textsc{locv}	3\textsc{sg.f}.\textsc{a}	sweep\\
\glt `and she, going up, went straight to the ?lower part of the \textit{legundi} fruit [tree], [and] she swept [there].'\footnote{MZ: In many Core North Halmahera languages, cognates of the total duplication of \textit{gila} `long' mean `straight'.}
\z

\ea\label{ex:text1-89}
De ma moioli wa ino,\\
\gll de	ma	moi-oli	wa	ino\\
     \textsc{conn}	\textsc{rnm}	one-\textsc{again}	3\textsc{sg.m}>3\textsc{nh}	\textsc{dir}.\textsc{ven}\\
\glt `And he came once again'
\z

\ea\label{ex:text1-90}
de ma lagudi ma dumulia mi datatara,\\
\gll de	ma	lagudi	ma	dumulia	mi	dV-CV{\textasciitilde}tara\\
     \textsc{conn}	\textsc{rnm}	legundi	\textsc{rnm}	residual.rainwater	3\textsc{sg.m}>3\textsc{sg.f}	\textsc{appl}-\textsc{rdpl}{\textasciitilde}shake\\
\glt `and he shook the remaining rain water from the \textit{legundi} tree onto her,'\footnote{MZ: My informant told me that \textit{legundi} trees have bowl-like leaves which after a rain fall are filled with water. The water will splash on people standing below the tree.}
\z

\ea\label{ex:text1-91}
de muna mo temo:\\
\gll de	muna	mo	temo\\
     \textsc{conn}	\textsc{pro}.3\textsc{sg.f}	3\textsc{sg.f}.\textsc{a}	say\\
\glt `and she said:'
\z

\ea\label{ex:text1-92}
``Ma ngona de no pa'ihana.''\\
\gll ma	ngona	de	no	pa'ihana\\
     but	\textsc{pro}.2\textsc{sg}	\textsc{conn}	2\textsc{sg}.\textsc{a}	force\\
\glt `{``}But you (sg.), you're (sg.) forcing [me].'''
\z

\ea\label{ex:text1-93}
De muna mo hano mo temo:\\
\gll de	muna	mo	hano	mo	temo\\
     \textsc{conn}	\textsc{pro}.3\textsc{sg.f}	3\textsc{sg.f}.\textsc{a}	ask	3\textsc{sg.f}.\textsc{a}	say\\
\glt `And she asked, saying:'
\z

\ea\label{ex:text1-94}
``Neena to ngona ani lagudi ma howo'o?''\\
\gll neena	to	ngona	ani	lagudi	ma	howo'o\\
     \textsc{prox}:\textsc{pro}.3\textsc{nh}	\textsc{poss}.\textsc{hum}	\textsc{pro}.2\textsc{sg}	2\textsc{sg}.\textsc{poss}	legundi	\textsc{rnm}	fruit\\
\glt `{``}Are these your (sg.) \textit{legundi} fruits?'''
\z

\ea\label{ex:text1-95}
De una wo temo:\\
\gll de	una	wo	temo\\
     \textsc{conn}	\textsc{pro}.3\textsc{sg.m}	3\textsc{sg.m}.\textsc{a}	say\\
\glt `And he said:'
\z

\ea\label{ex:text1-96}
``To ngoi ai lagudi ma howo'o,''\\
\gll to	ngoi	ai	lagudi	ma	howo'o\\
     \textsc{poss}.\textsc{hum}	\textsc{pro}.1\textsc{sg}	1\textsc{sg}.\textsc{poss}	legundi	\textsc{rnm}	fruit\\
\glt `{``}These are my \textit{legundi} fruits,'''
\z

\ea\label{ex:text1-97}
``na'o to ngona ani lagudi ma howo'o,\\
\gll na'o	to	ngona	ani	lagudi	ma	howo'o\\
     \textsc{cond}	\textsc{poss}.\textsc{hum}	\textsc{pro}.2\textsc{sg}	2\textsc{sg}.\textsc{poss}	legundi	\textsc{rnm}	fruit\\
\glt `{``}If these are your (sg.) \textit{legundi} fruits,'
\z

\ea\label{ex:text1-98}
de o'ia'a de na ino no habea?''\\
\gll de	o'ia-i'a	de	na	ino	no	habea\\
     \textsc{conn}	what-\textsc{dir.itv}	\textsc{conn}	2\textsc{sg}>3\textsc{nh}	\textsc{dir}.\textsc{ven}	2\textsc{sg}.\textsc{a}	greet\\
\glt `where do you (sg.) come from here to greet [me]?'''\footnote{MZ: \textit{Sabea} means `to pray' in several Core North Halmahera languages (from Indonesian \textit{sembahyang} `pray'). \textit{Tabea}, on the other hand, means `to greet'. Since the meaning `to pray' makes no sense in this context, I assume that either the transcriber confused \textit{sabea} and \textit{tabea} or that the two forms merged in Modole.}
\z

\ea\label{ex:text1-99}
De una wo temo:\\
\gll de	una	wo	temo\\
     \textsc{conn}	\textsc{pro}.3\textsc{sg.m}	3\textsc{sg.m}.\textsc{a}	say\\
\glt `And he said: '
\z

\ea\label{ex:text1-100}
``Ngoi to habea o Limau ma ituo'a de ta u'u,''\\
\gll ngoi	to	habea	o	{Limau ma ituo'a}	de	ta	u'u\\
     \textsc{pro}.1\textsc{sg}	1\textsc{sg}.\textsc{a}	greet	\textsc{nm}	[place\_name]	\textsc{conn}	1\textsc{sg}>3\textsc{nh}	\textsc{dir}.\textsc{down}\\
\glt `{``}I'm coming from Limau ma ituo'a down to greet [you],'''\footnote{MZ: \textit{Limau} is a Ternate word for `town'. There is a place called \textit{Limau} in the Galela region of north Halmahera but Limau ma ituo'a may simply mean `town at the Itio river' (see above note for [A.\ref{ex:text1-16}]).}
\z

\ea\label{ex:text1-101}
de muna mo temo:\\
\gll de	muna	mo	temo\\
     \textsc{conn}	\textsc{pro}.3\textsc{sg.f}	3\textsc{sg.f}.\textsc{a}	say\\
\glt `and she said:'
\z

\ea\label{ex:text1-102}
``Na'o 'ogeena de o to ngonawau ani lagudi ma howo'o.''\\
\gll na'o	'o-geena	de	o	to	ngona-ua-ou	ani	lagudi	ma	howo'o\\
     \textsc{cond}	\textsc{emph}-\textsc{dist}:\textsc{pro}.3\textsc{nh}	\textsc{conn}	\textsc{nm}	\textsc{poss}.\textsc{hum}	\textsc{pro}.2\textsc{sg}-\textsc{neg}-\textsc{foc}	2\textsc{sg}.\textsc{poss}	legundi	\textsc{rnm}	fruit\\
\glt `{``}If that's the case, then your (sg.) \textit{legundi} fruits are no longer yours.'''
\z

\ea\label{ex:text1-103}
Wa i'a de wo hingahu ma dea'a, wo temo:\\
\gll wa	i'a	de	wo	hi-ngahu	ma	dea-i'a	wo	temo\\
     3\textsc{sg.m}>3\textsc{nh}	\textsc{dir}.\textsc{itv}	\textsc{conn}	3\textsc{sg.m}.\textsc{a}	\textsc{caus}-report	\textsc{rnm}	father-\textsc{dir}.\textsc{itv}	3\textsc{sg.m}.\textsc{a}	say\\
\glt `He went away and he told [it] to his father, saying:'
\z

\ea\label{ex:text1-104}
``Baba! 'a'ano ta i'a de jo temo\\
\gll baba   'a'ano	ta	i'a	de	yo	temo\\
     father just.now	1\textsc{sg}>3\textsc{nh}	\textsc{dir}.\textsc{itv}	\textsc{conn}	3\textsc{hpl}.\textsc{a}	say\\
\glt `{``}Father! I just went and they said'
\z

\ea\label{ex:text1-105}
o to ngoi- wau ai lagudi ma howo'o;\\
\gll 'o	to	ngoi	-ua-ou	ai	lagudi	ma	howo'o\\
     \textsc{emph}	\textsc{poss}.\textsc{hum}	\textsc{pro}.1\textsc{sg}	-\textsc{neg}-\textsc{foc}	1\textsc{sg}.\textsc{poss}	legundi	\textsc{rnm}	fruit\\
\glt `that my \textit{legundi} fruits are no longer mine;'
\z

\ea\label{ex:text1-106}
i hano jo temo:\\
\gll i	hano	yo	temo\\
     3\textsc{nh}.\textsc{a}	ask	3\textsc{hpl}.\textsc{a}	say\\
\glt `they asked, saying:'
\z

\ea\label{ex:text1-107}
`O 'ia'a de na ino no habea?'\\
\gll o'ia-i'a	de	na	ino	no	habea\\
     what-\textsc{dir}.\textsc{itv}	\textsc{conn}	2\textsc{sg}>3\textsc{nh}	\textsc{dir}.\textsc{ven}	2\textsc{sg}.\textsc{a}	greet\\
\glt `{`}Where have you (sg.) come from here to greet [us]?{'}'
\z

\ea\label{ex:text1-108}
De to temo:\\
\gll de	to	temo\\
     \textsc{conn}	1\textsc{sg}.\textsc{a}	say\\
\glt `And I said:'
\z

\ea\label{ex:text1-109}
`Limau ma ituo'a de ta u'u to habea.'{''}\\
\gll {Limau ma ituo'a}	de	ta	u'u	to	habea\\
     [place\_name]	\textsc{conn}	1\textsc{sg}>3\textsc{nh}	\textsc{dir}.\textsc{down}	1\textsc{sg}.\textsc{a}	greet\\
\glt ` {`}I came down from Limau ma ituo'a to greet [you].{'}''{'}
\z

\ea\label{ex:text1-110}
De una, ma dea wo temo:\\
\gll de	una	ma	dea	wo	temo\\
     \textsc{conn}	\textsc{pro}.3\textsc{sg.m}	\textsc{rnm}	father	3\textsc{sg.m}.\textsc{a}	say\\
\glt `And he, the father, he said:'
\z

\ea\label{ex:text1-111}
``Gemutu o to ngona-wau ani lagudi ma howo'o.''\\
\gll gemutu	'o	to	ngona-ua-ou	ani	lagudi	ma	howo'o\\
     ?thus	\textsc{emph}	\textsc{poss}.\textsc{hum}	\textsc{pro}.2\textsc{sg}-\textsc{neg}-\textsc{foc}	2\textsc{sg}.\textsc{poss}	legundi	\textsc{rnm}	fruit\\
\glt `{``}Thus your (sg.) \textit{legundi} fruits are no longer yours (sg.).{''}'
\z

\newpage
\ea\label{ex:text1-112}
De una wo temo:\\
\gll de	una	wo	temo\\
     \textsc{conn}	\textsc{pro}.3\textsc{sg.m}	3\textsc{sg.m}.\textsc{a}	say\\
\glt `And he said:'
\z

\ea\label{ex:text1-113}
``na'o 'ogeena de 'a i togumo'u.''\\
\gll na'o	'o-geena	de	'a	i	togumu-u'u\\
     \textsc{cond}	\textsc{emph}-\textsc{dist}:\textsc{pro}.3\textsc{nh}	\textsc{conn}	\textsc{foc}	3\textsc{nh}.\textsc{a}	finish-\textsc{dir}.\textsc{down}\\
\glt `{``}If that's the case, then let it be so.'''
\z

\ea\label{ex:text1-114}
De ma 'oana o wange ma dumunuali awi ngo'a wa modo'a,\\
\gll de	ma	'oana	o	wange	ma	dumunu-oli	awi	ngo'a	wa	modo'a\\
     \textsc{conn}	\textsc{rnm}	king	\textsc{nm}	sun	\textsc{rnm}	dive-\textsc{again}	3\textsc{sg.m}.\textsc{poss}	child	3\textsc{sg.m}>3\textsc{nh}	marry\\
\glt `?And he wed the child of the Western King too,'\footnote{MZ: The index combination \textit{wa} `3\textsc{sg.m}>3\textsc{nh}' is unexpected here. Usually the person who is wedded is referred to by the undergoer index.}
\z

\ea\label{ex:text1-115}
de ma 'o'ano o wange ma hiwa'a awi ngoa'a mo modo'a,\\
\gll de	ma	'oana	o	wange	ma	hiwa-o'a	awi	ngoa'a	mo	modo'a\\
     \textsc{conn}	\textsc{rnm}	king	\textsc{nm}	sun	\textsc{rnm}	shine-\textsc{locv}	3\textsc{sg.m}.\textsc{poss}	child	3\textsc{sg.f}.\textsc{a}	marry\\
\glt `and the child of the Eastern King wed'
\z

\ea\label{ex:text1-116}
ma 'o'ana o wange ma dumuno'a awi doguulu moi,\\
\gll ma	'oana	o	wange	ma	dumunu-o'a	awi	doguulu	moi\\
     \textsc{rnm}	king	\textsc{nm}	sun	\textsc{rnm}	dive-\textsc{locv}	3\textsc{sg.m}.\textsc{poss}	young.man	one\\
\glt `the son of the Western King'
\z


\ea\label{ex:text1-117}
de mi hirame-rame o wutu tumudingi de o wange tumudingi.\\
\gll de	mi	hi-rame{\textasciitilde}rame	o	wutu	tumudingi	de	o	wange	tumudingi\\
     \textsc{conn}	3\textsc{sg.f}.\textsc{u}	\textsc{caus}-\textsc{rdpl}{\textasciitilde}feast	\textsc{nm}	night	seven	\textsc{conn}	\textsc{nm}	sun	seven\\
\glt `and they celebrated her seven nights and seven days.'
\z
