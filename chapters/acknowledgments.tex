\addchap{\lsAcknowledgementTitle}

When G.J. Ellen published the texts edited in this book in 1916 he did not consider it necessary to acknowledge their narrators or even record their names. This injustice cannot be undone and I wish to thank the narrators with all my heart. Without them this book as well as my research would not be possible.

The edition of the texts in this book commenced as a side project to my PhD project about the linguistic history of the North Halmahera languages (supported by the OUTOFPAPUA project funded by the European Research Council (grant agreement no. 848532)). Modole is the least described of the North Halmahera languages. My analysis of the texts initially served the purpose of including Modole data in my study. This work is addressed to linguists and members of the Modole community who wish to learn more about the language. I pursue no religious or ideological goals.

I thank Christian Döhler for making me aware of \textit{Open Text Collections} and his continuous encouragment and support in publishing this book; and Uri Tadmor of Brill for the permission to republish the texts.
I wish to thank Annika Schiefner, Antoinette Schapper, Jacob Menschel, Norea Beaujean, Silvie Strauß, and especially Robert Tegethoff for reading early drafts of chapters; two reviewers for their helpful comments; and Lena Pointner and my parents for their persistent enthusiasm. 
I further wish to thank Edward Kotynski for our discussions on all things Tabaru and North Halmahera in general; and Dalan Perangin Angin for answering numerous questions on Pagu.

My final thanks go to a Modole speaker who patiently answered my weird questions about the texts and Modole language and culture.\footnote{Due to paternalistic privacy laws of the European Union, I am not allowed to name my language informant. They are not allowed to deanonymize themself either.} Their views do not represent the whole Modole community, which was not further involved in the creation of this work.