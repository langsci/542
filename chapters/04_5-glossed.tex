\ea\label{ex:text4-1}
O Gogunane\\
\gll o	gogunane\\
     \textsc{nm}	ant\\
\glt `The ants'\footnote{MZ: It is unclear which species is designated by \textit{gogunane}.}
\z


\ea\label{ex:text4-2}
Naga o bere'i moi ami pihanga o gogunane ja odomo.\\
\gll naga	o	bere'i	moi	ami	pihanga	o	gogunane	ya	odomo\\
     \textsc{exist}	\textsc{nm}	old.person	one	3\textsc{sg.f}.\textsc{poss}	banana	\textsc{nm}	ant	3\textsc{nh}>3\textsc{nh}	eat\\
\glt `There was an old woman [and] her bananas [were] eaten by ants.'
\z

\ea\label{ex:text4-3}
Ma bere'i mo temo: ``ai pihanga neena o'ia ja odomo?''\\
\gll ma	bere'i	mo	temo	ai	pihanga	neena	o'ia	ya	odomo\\
     \textsc{rnm}	old.person	3\textsc{sg.f}.\textsc{a}	say	1\textsc{sg}.\textsc{poss}	banana	\textsc{prox}:\textsc{pro}.3\textsc{nh}	what	3\textsc{nh}>3\textsc{nh}	eat\\
\glt `The old woman said: ``These bananas of mine, who has eaten them?'''
\z

\ea\label{ex:text4-4}
De mo temo: ``ai pihanga ja odo-odomo ho to goanohi'';\\
\gll de	mo	temo	ai	pihanga	ya	(C)V(C)V{\textasciitilde}odomo	ho	to	goana-ohi\\
     \textsc{conn}	3\textsc{sg.f}.\textsc{a}	say	1\textsc{sg}.\textsc{poss}	banana	3\textsc{nh}>3\textsc{nh}	\textsc{rdpl}{\textasciitilde}eat	thus	1\textsc{sg}.\textsc{a}	guard-\textsc{still}\\
\glt `And she said: ``They have eaten my bananas, so I'll guard [them] still;'''
\z

\ea\label{ex:text4-5}
de mo goana.\\
\gll de	mo	goana\\
     \textsc{conn}	3\textsc{sg.f}.\textsc{a}	guard\\
\glt `and she guarded [them].'
\z

\ea\label{ex:text4-6}
Hoato! ma dadaao ena, a ma gogunane modidio'a i odomo;\\
\gll ho-ato ma	dadaao	ena	'a	ma	gogunane	modidi-o'a	i	odomo\\
     thus-\textsc{excl} \textsc{rnm}	after	\textsc{pro}.3\textsc{nh}	\textsc{foc}	\textsc{rnm}	ant	two-?\textsc{locv}	3\textsc{nh}.\textsc{a}	eat\\
\glt `And look! after that the two ants ate;'
\z

\ea\label{ex:text4-7}
de mo temo:\\
\gll de	mo	temo\\
     \textsc{conn}	3\textsc{sg.f}.\textsc{a}	say\\
\glt `and she said:'
\z


\ea\label{ex:text4-8}
 ``o gogunane dau ena ai pihanga ja odo-odomo ho ta pohaahi.''\\
\gll o	gogunane	dau	ena	ai	pihanga	ya	(C)V(C)V{\textasciitilde}odomo	ho	ta	poha-ohi\\
     \textsc{nm}	ant	\textsc{loc}.\textsc{down}	\textsc{pro}.3\textsc{nh}	1\textsc{sg}.\textsc{poss}	banana	3\textsc{nh}>3\textsc{nh}	\textsc{rdpl}{\textasciitilde}eat	thus	1\textsc{sg}>3\textsc{nh}	hit-\textsc{still}\\
\glt `{``}The ants down here are eating my bananas, I will strike them [dead].'''
\z

\ea\label{ex:text4-9}
De ma gogunane i temo:\\
\gll de	ma	gogunane	i	temo\\
     \textsc{conn}	\textsc{rnm}	ant	3\textsc{nh}.\textsc{a}	say\\
\glt `And the ants said:'
\z

\ea\label{ex:text4-10}
 ``Apu uwa ni mi poha la ni mi aha ani woa'a,\\
\gll apu	uwa	no	mi	poha	la	no	mi	aha	ani	wo'a-i'a\\
     grandmother	\textsc{proh}	2\textsc{sg}.\textsc{a}	1\textsc{pl}.\textsc{ex}.\textsc{u}	hit	so.that	2\textsc{sg}.\textsc{a}	1\textsc{pl}.\textsc{ex}.\textsc{u}	carry	2\textsc{sg}.\textsc{poss}	house-\textsc{dir}.\textsc{itv}\\
\glt `{``}Grandmother, don't strike us (ex.) [dead], but take us (ex.) to your (sg.) home,'''
\z

\ea\label{ex:text4-11}
de ma aha ami woa'a,\\
\gll de	ma	aha	ami	wo'a-i'a\\
     \textsc{conn}	3\textsc{sg.f}>3\textsc{nh}	carry	3\textsc{sg.f}.\textsc{poss}	house-\textsc{dir}.\textsc{itv}\\
\glt `and she took them to her house,'
\z

\ea\label{ex:text4-12}
ma i'a de mo temo:\\
\gll ma	i'a	de	mo	temo\\
     \textsc{rnm}	\textsc{dir}.\textsc{itv}	\textsc{conn}	3\textsc{sg.f}.\textsc{a}	say\\
\glt `and after that she said:'\footnote{GJE: to her husband}
\z

\ea\label{ex:text4-13}
 ``bere'i 'aano neena ai pihanga o gogunane ja odomo\\
\gll bere'i	'a'ano	neena	ai	pihanga	o	gogunane	ya	odomo\\
     old.person	just.now	\textsc{prox}:\textsc{pro}.3\textsc{nh}	1\textsc{sg}.\textsc{poss}	banana	\textsc{nm}	ant	3\textsc{nh}>3\textsc{nh}	eat\\
\glt `{``}Old man, a moment ago my bananas [were] eaten by ants,'
\z

\ea\label{ex:text4-14}
de ato ta poha,\\
\gll de	ato	ta	poha\\
     \textsc{conn}	\textsc{excl}	1\textsc{sg}>3\textsc{nh}	hit\\
\glt `and look, I [wanted] to strike them'
\z

\ea\label{ex:text4-15}
de i temo: ``Apu! uwa ni mi poha la ani woa'a aha.''\\
\gll de	i	temo	apu  uwa	no	mi	poha	la	ani	wo'a-i'a	aha\\
     \textsc{conn}	3\textsc{nh}.\textsc{a}	say	grandmother  \textsc{proh}	2\textsc{sg}.\textsc{a}	1\textsc{pl}.\textsc{ex}.\textsc{u}	hit	so.that	2\textsc{sg}.\textsc{poss}	house-\textsc{dir}.\textsc{itv}	carry\\
\glt `and they said: `Grandmother! don't strike us (ex.) [dead], but take [us] to your (sg.) house.{'}''{'}
\z

\ea\label{ex:text4-16}
De una mo bere'i wo temo:\\
\gll de	una	ma	bere'i	wo	temo\\
     \textsc{conn}	\textsc{pro}.3\textsc{sg.m}	\textsc{rnm}	old.person	3\textsc{sg.m}.\textsc{a}	say\\
\glt `And he, the old man, said:'
\z

\ea\label{ex:text4-17}
 ``geenade na paliarano.''\\
\gll geena-de	na	paliara-ino\\
     \textsc{dist}:\textsc{pro}.3\textsc{nh}-\textsc{conn}	2\textsc{sg}>3\textsc{nh}	raise-\textsc{dir}.\textsc{ven}\\
\glt `{``}Then you'll (sg.) raise them.'''\footnote{MZ: I glossed the index \textit{na} \textsc{2sg>3nh} but it could also be a first person plural inclusive undergoer index. Ellen interprets the verbs as ditransitive `you raise them for us' which would explain the occurrence of the undergoer index and the \textsc{ventive} directional \textit{-ino}. However, if the verb was ditransitive I would also expect the second person singular actor index \textit{no}. I hence chose the \textsc{2sg>3nh} gloss.}
\z

\ea\label{ex:text4-18}
De ma moi ma bere'i mo tagi ami baili'a,\\
\gll de	ma	moi	ma	bere'i	mo	tagi	ami	baili-i'a\\
     \textsc{conn}	\textsc{rnm}	one	\textsc{rnm}	old.person	3\textsc{sg.f}.\textsc{a}	go	3\textsc{sg.f}.\textsc{poss}	garden-\textsc{dir}.\textsc{itv}\\
\glt `And once the old woman went to her garden,'
\z

\ea\label{ex:text4-19}
ami baili mo daapu'u.\\
\gll ami	baili	mo	dV-apu'u\\
     3\textsc{sg.f}.\textsc{poss}	garden	3\textsc{sg.f}.\textsc{a}	\textsc{appl}-pull.out\\
\glt `to weed her garden.'
\z


\ea\label{ex:text4-20}
De mo liono de ma gogunane i hano, i temo:\\
\gll de	mo	lio-ino	de	ma	gogunane	i	hano	i	temo\\
     \textsc{conn}	3\textsc{sg.f}.\textsc{a}	return-\textsc{dir}.\textsc{ven}	\textsc{conn}	\textsc{rnm}	ant	3\textsc{nh}.\textsc{a}	ask	3\textsc{nh}.\textsc{a}	say\\
\glt `And she returned [home] and the ants asked, saying:'
\z

\ea\label{ex:text4-21}
 ``Bere'i! ani baili no daapu'u na botoau?''\\
\gll bere'i  ani	baili	no	dV-apu'u	na	boto-ou\\
     old.person 2\textsc{sg}.\textsc{poss}	garden	2\textsc{sg}.\textsc{a}	\textsc{appl}-pull.out	2\textsc{sg}>3\textsc{nh}	finish-\textsc{already}\\
\glt `{``}Old woman! are you (sg.) done weeding your (sg.) garden?'''
\z

\ea\label{ex:text4-22}
De mo temo: ``ouahi o ta botuahi'';\\
\gll de	mo	temo	'o-ua-ohi	'o	ta	boto-ua-ohi\\
     \textsc{conn}	3\textsc{sg.f}.\textsc{a}	say	\textsc{emph}-\textsc{neg}-\textsc{still}	\textsc{emph}	1\textsc{sg}>3\textsc{nh}	finish-\textsc{neg}-\textsc{still}\\
\glt `And she said: ``Not yet, I am not yet done with it,'''\footnote{MZ: A question mark occurs at the end of the Dutch translation which is certainly a typo.}
\z

\ea\label{ex:text4-23}
de ma gogunane i temo:\\
\gll de	ma	gogunane	i	temo\\
     \textsc{conn}	\textsc{rnm}	ant	3\textsc{nh}.\textsc{a}	say\\
\glt `and the ants said:'
\z

\ea\label{ex:text4-24}
 ``o dewela de ni mi ni'i.''\\
\gll o	dewela	de	ni	mi	ni'i\\
     \textsc{nm}	tomorrow	\textsc{conn}	2\textsc{sg}.\textsc{u}	1\textsc{pl}.\textsc{ex}.\textsc{a}	follow\\
\glt `{``}Tomorrow, we (ex.) [will] follow you (sg.).'''\footnote{MZ: The index combination \textit{ni mi} could also be \textsc{1pl.excl.a>2sg.u}: `you [will] follow us.'}
\z

\ea\label{ex:text4-25}
De ma bere'i mo temo:\\
\gll de	ma	bere'i	mo	temo\\
     \textsc{conn}	\textsc{rnm}	old.person	3\textsc{sg.f}.\textsc{a}	say\\
\glt `And the old woman said:'
\z

\ea\label{ex:text4-26}
 ``Mai'au, o dewela aha po tagi.''\\
\gll ma-i'a-ou	o	dewela	aha	po	tagi\\
     \textsc{rnm}-\textsc{dir}.\textsc{itv}-\textsc{foc}	\textsc{nm}	tomorrow	as.a.consequence	1\textsc{pl}.\textsc{in}.\textsc{a}	go\\
\glt `{``}Go ahead, tomorrow we (in.) will go.'''
\z

\ea\label{ex:text4-27}
De ma dewelano de jo tagi,\\
\gll de	ma	dewela-ino	de	yo	tagi\\
     \textsc{conn}	\textsc{rnm}	tomorrow-\textsc{dir}.\textsc{ven}	\textsc{conn}	3\textsc{hpl}.\textsc{a}	go\\
\glt `And the next day they went,'
\z

\ea\label{ex:text4-28}
ja i'ade ja rari\\
\gll ya	i'a-de	ya	rari\\
     3\textsc{nh}>3\textsc{nh}	\textsc{dir}.\textsc{itv}-\textsc{conn}	3\textsc{nh}>3\textsc{nh}	cut\\
\glt `they went and weeded'
\z

\ea\label{ex:text4-29}
a ma moi de a i ma hidapatau.\\
\gll 'a	ma	moi	de	'a	i	ma	hi-dV-pata-ou\\
     \textsc{foc}	\textsc{rnm}	one	\textsc{conn}	\textsc{foc}	3\textsc{nh}.\textsc{a}	\textsc{mid}	\textsc{caus}-\textsc{appl}-open-\textsc{already}\\
\glt `and at once everything was done.'\footnote{MZ: In the wordlist, \textit{pata} is glossed as `open' and this meaning is found in [G.\ref{ex:text7-70}]. Here and in [D.\ref{ex:text4-33}], \textit{pata} probably refers to the extraction of weeds to `open' a space for gardening. Ellen translates it as \textit{klaar}, usually meaning `done' but note that it can also mean `clear.' Ellen takes the index \textit{i} to refer to the garden and not the ants. If \textit{i} refers to the ants, \textit{i ma hidapatau} is better translated as `they opened it' (also in [D.\ref{ex:text4-33}]).}
\z

\ea\label{ex:text4-30}
De a jo liou ja i'a\\
\gll de	'a	yo	lio-ou	ya	i'a\\
     \textsc{conn}	\textsc{foc}	3\textsc{hpl}.\textsc{a}	return-\textsc{already}	3\textsc{nh}>3\textsc{nh}	\textsc{dir}.\textsc{itv}\\
\glt `And they returned, they went'
\z

\ea\label{ex:text4-31}
de mo hingahu ma berei'a mo temo:\\
\gll de	mo	hi-ngahu	ma	bere'i-i'a	mo	temo\\
     \textsc{conn}	3\textsc{sg.f}.\textsc{a}	\textsc{caus}-report	\textsc{rnm}	old.person-\textsc{dir}.\textsc{itv}	3\textsc{sg.f}.\textsc{a}	say\\
\glt `and she told the old man, saying:'
\z

\ea\label{ex:text4-32}
 ``Bere'i 'aano neena mia iha\\
\gll bere'i	'a'ano	neena	mia	iha\\
     old.person	just.now	\textsc{prox}:\textsc{pro}.3\textsc{nh}	1\textsc{pl}.\textsc{ex}>3\textsc{nh}	\textsc{dir}.\textsc{land}\\
\glt `{``}Old man, a moment ago we (ex.) went landwards'
\z

\ea\label{ex:text4-33}
de jo rari a ma moi de a i ma hidapatou.''\\
\gll de	yo	rari	'a	ma	moi	de	'a	i	ma	hi-dV-pata-ou\\
     \textsc{conn}	3\textsc{hpl}.\textsc{a}	cut	\textsc{foc}	\textsc{rnm}	one	\textsc{conn}	\textsc{foc}	3\textsc{nh}.\textsc{a}	\textsc{mid}	\textsc{caus}-\textsc{appl}-open-\textsc{already}\\
\glt `and they weeded and at once it was done.'''\footnote{MZ: The index <jo> may be a typo for \textit{ya}. In [E.\ref{ex:text4-23}], \textit{rari} `weed' occurs with \textit{ya}.}
\z

\ea\label{ex:text4-34}
De ona jo temo: ``Bere'i, ni mi taäni;''\\
\gll de	ona	yo	temo	bere'i	ni	mi	taani\\
     \textsc{conn}	\textsc{pro}.3\textsc{plh}	3\textsc{hpl}.\textsc{a}	say	old.person	2\textsc{pl}.\textsc{u}	1\textsc{pl}.\textsc{ex}.\textsc{a}	look.for.lice\\
\glt `And they [the ants] said: ``Old people, [let] us (ex.) search you (pl.) for lice\footnote{GJE: a very popular and highly esteemed occupation among the natives},'''\footnote{MZ: Starting in this line, the ants are referred to by the third person plural \textit{human} index \textit{yo} instead of the third person plural \textit{non-human} index \textit{i}. 
Ellen translates the verb \textit{ni mi taani} as `you search us for lice'. Based on the contexts, I think it is evident that the ants are the ones doing the searching.}
\z

\ea\label{ex:text4-35}
de jo temo: ``danongo!\\
\gll de	yo	temo	danongo\\
     \textsc{conn}	3\textsc{hpl}.\textsc{a}	say	grandchild\\
\glt `and they said: ``Grandchildren!'
\z

\ea\label{ex:text4-36}
ho ma i'a ni mi taäni,\\
\gll ho	ma	i'a	ni	mi	taani\\
     thus	\textsc{rnm}	\textsc{dir}.\textsc{itv}	2\textsc{pl}.\textsc{a}	1\textsc{pl}.\textsc{ex}.\textsc{u}	look.for.lice\\
\glt `go ahead, search us (ex.) for lice,'''\footnote{MZ: As is the case for [D.\ref{ex:text4-34}], Ellen and I disagree on the analysis of the index combination \textit{ni mi}.}
\z

\ea\label{ex:text4-37}
de jo 'i taani jo temo:\\
\gll de	yo	'i	taani	yo	temo\\
     \textsc{conn}	3\textsc{hpl}.\textsc{a}	3\textsc{hpl}.\textsc{u}	look.for.lice	3\textsc{hpl}.\textsc{a}	say\\
\glt `and they searched them for lice [and] they said:'
\z

\ea\label{ex:text4-38}
 ``ni ma leo-leotie!''\\
\gll ni	ma	(C)V(C)V{\textasciitilde}leoto-ie\\
     2\textsc{pl}.\textsc{a}	\textsc{mid}	\textsc{rdpl}{\textasciitilde}supine-\textsc{dir}.\textsc{up}\\
\glt `{``}Look up!'''
\z

\ea\label{ex:text4-39}
De jo ma leo-leotie,\\
\gll de	yo	ma	(C)V(C)V{\textasciitilde}leoto-ie\\
     \textsc{conn}	3\textsc{hpl}.\textsc{a}	\textsc{mid}	\textsc{rdpl}{\textasciitilde}supine-\textsc{dir}.\textsc{up}\\
\glt `And they looked up,'
\z


\ea\label{ex:text4-40}
de jo 'i hibaumu o gogunane de o gao,\\
\gll de	yo	'i	hi-baumu	o	gogunane	de	o	gao\\
     \textsc{conn}	3\textsc{hpl}.\textsc{a}	3\textsc{hpl}.\textsc{u}	\textsc{caus}-scatter	\textsc{nm}	ant	\textsc{conn}	\textsc{nm}	lime\\
\glt `and they, the ants\footnote{GJE: who later will turn out to be disguised humans (MZ: In fact, they have already been identified as humans by use of the third person plural human actor index).}, threw lime at them,'\footnote{ MZ: The root \textit{baumu} is not attested without the prefix \textit{hi-} (also in [D.\ref{ex:text4-47}]) and there are no cognates in other Core North Halmahera languages either. I prefer to analyze \textit{hi-} as the causative prefix since \textit{hibaumu} does not fit the usual structure of basic roots in Modole and because there is an explicit instrument phrase \textit{de o gao} `with lime'.}
\z

\ea\label{ex:text4-41}
de ma bere'i awi la'o ja hiihi ho wo ore-orehe,\\
\gll de	ma	bere'i	awi	la'o	ya	hi-hihi	ho	wo	(C)V(C)V{\textasciitilde}orehe\\
     \textsc{conn}	\textsc{rnm}	old.person	3\textsc{sg.m}.\textsc{poss}	eye	3\textsc{nh}>3\textsc{nh}	\textsc{caus}-hurt	thus	3\textsc{sg.m}.\textsc{a}	\textsc{rdpl}{\textasciitilde}scream\\
\glt `and the eyes of the old man hurt so that he screamed;'
\z

\ea\label{ex:text4-42}
de jo loa, de ja i'a de o gota de jo doade,\\
\gll de	yo	loa	de	ya	i'a	de	o	gota	de	yo	doa-de\\
     \textsc{conn}	3\textsc{hpl}.\textsc{a}	flee	\textsc{conn}	3\textsc{nh}>3\textsc{nh}	\textsc{dir}.\textsc{itv}	\textsc{conn}	\textsc{nm}	wood	\textsc{conn}	3\textsc{hpl}.\textsc{a}	climb-\textsc{conn}\\
\glt `and they [the ants] ran away, and they went and climbed a tree'
\z

\ea\label{ex:text4-43}
de ma bere'i wo 'i tuulu.\\
\gll de	ma	bere'i	wo	'i	tuulu\\
     \textsc{conn}	\textsc{rnm}	old.person	3\textsc{sg.m}.\textsc{a}	3\textsc{hpl}.\textsc{u}	follow\\
\glt `and the old man followed them.'
\z

\ea\label{ex:text4-44}
Wo 'i ma'e ha o gota'a jo doa-doa de,\\
\gll wo 'i	ma'e	ho	o	gota-o'a	yo	doa{\textasciitilde}doa	de\\
     3\textsc{sg.m}.\textsc{a} 3\textsc{hpl}.\textsc{u}	see	thus	\textsc{nm}	tree-\textsc{locv}	3\textsc{hpl}.\textsc{a}	\textsc{rdpl}{\textasciitilde}climb	\textsc{conn}\\
\glt `He saw them climbing a tree'
\z

\ea\label{ex:text4-45}
de ma bere'i wo temo:\\
\gll de	ma	bere'i	wo	temo\\
     \textsc{conn}	\textsc{rnm}	old.person	3\textsc{sg.m}.\textsc{a}	say\\
\glt `and the old man said:'
\z

\ea\label{ex:text4-46}
 ``Ti ni pohau, ngomi o ni mi dodoawa,\\
\gll to	ni	poha-ou	ngomi	'o	ni	mi	dodoa-ua\\
     1\textsc{sg}.\textsc{a}	2\textsc{pl}.\textsc{u}	hit-\textsc{already}	\textsc{pro}.1\textsc{pl}.\textsc{ex}	\textsc{emph}	2\textsc{pl}.\textsc{u}	1\textsc{pl}.\textsc{ex}.\textsc{a}	do-\textsc{neg}\\
\glt `{``}I will strike you (pl.) [dead], we (ex.) haven't done anything to you (pl.),'\footnote{MZ: The form \textit{dodoa} may be a reduplicated form of \textit{doa} `climb, put.'}
\z

\ea\label{ex:text4-47}
ma o gogunane ni mi hibaumu mia la'o'o ho.''\\
\gll ma	o	gogunane	ni	mi	hibaumu	mia	la'o	ho\\
     but	\textsc{nm}	ant	2\textsc{pl}.\textsc{a}	1\textsc{pl}.\textsc{ex}.\textsc{u}	scatter	1\textsc{pl}.\textsc{ex}.\textsc{poss}	eye	thus\\
\glt `but you (pl.) ants threw [lime] in our (ex.) eyes.'''
\z

\ea\label{ex:text4-48}
De wo 'i hidotodanga,\\
\gll de	wo 'i	hi-dV-todanga\\
     \textsc{conn}	3\textsc{sg.m}.\textsc{a} 3\textsc{hpl}.\textsc{u}	\textsc{caus}-\textsc{appl}-cut.down\\
\glt `And he cut them down,'\footnote{GJE: with the tree}
\z

\ea\label{ex:text4-49}
ho na'o i luba, jo ma palitana ma gota ma homoa'a,\\
\gll ho	na'o	i	luba	yo	ma	palitana	ma	gota	ma	homoa-i'a\\
     thus	\textsc{cond}	3\textsc{nh}.\textsc{a}	fall.down	3\textsc{hpl}.\textsc{a}	\textsc{mid}	jump.\textsc{up}	\textsc{rnm}	wood	\textsc{rnm}	other-\textsc{dir}.\textsc{itv}\\
\glt `but when it fell down, they jumped to another tree;'
\z

\ea\label{ex:text4-50}
i'a wa todanga ho i luba jo palitana ma gota ma homoa'a;\\
\gll i'a	wa	todanga	ho	i	luba	yo	palitana	ma	gota	ma	homoa-i'a\\
     \textsc{dir}.\textsc{itv}	3\textsc{sg.m}>3\textsc{nh}	cut.down	thus	3\textsc{nh}.\textsc{a}	fall.down	3\textsc{hpl}.\textsc{a}	jump.\textsc{up}	\textsc{rnm}	wood	\textsc{rnm}	other-\textsc{dir}.\textsc{itv}\\
\glt `then he hacked it down, so [when] it fell, they jumped to another tree;'
\z

\ea\label{ex:text4-51}
i'a wa todanga ho i luba jo mo palitana ma homoa'a,\\
\gll i'a	wa	todanga	ho	i	luba	yo	ma	palitana	ma	homoa-i'a\\
     \textsc{dir}.\textsc{itv}	3\textsc{sg.m}>3\textsc{nh}	cut.down	thus	3\textsc{nh}.\textsc{a}	fall.down	3\textsc{hpl}.\textsc{a}	\textsc{mid}	jump.\textsc{up}	\textsc{rnm}	other-\textsc{dir}.\textsc{itv}\\
\glt `then he hacked it down, so [when] it fell, they jumped to another tree;'
\z

\ea\label{ex:text4-52}
de a wo 'angelau\\
\gll de	'a	wo	'angela-ou\\
     \textsc{conn}	\textsc{foc}	3\textsc{sg.m}.\textsc{a}	tired-\textsc{already}\\
\glt `and he got tired'
\z

\ea\label{ex:text4-53}
de ma ligihoro-horo ja u'u de jo baleno'a.\\
\gll de	ma ligihoro-horo	ya	u'u	de	yo	balene-o'a\\
     \textsc{conn} \textsc{rnm}	flying.palace	3\textsc{nh}>3\textsc{nh}	\textsc{dir}.\textsc{down}	\textsc{conn}	3\textsc{hpl}.\textsc{a}	embark-\textsc{lim}\\
\glt `and a flying palace came down and they [the ants] embarked [in it].'\footnote{MZ: The \textit{ma ligihoro-horo} `zweefpaleis' (`hovering palace') also occurs in folk tales of other Core North Halmaheran ethnic groups: Tobelo \textit{maligai horo-horo} `zweefpaleis' \citep{hueting1908a}, Loloda \textit{malige-soro-soro} `vliegend paleis (hemelwagen)' (`flying palace (sky carriage)'; \cite{vanbaarda1904}) and Tabaru \textit{malige isoro-soro} `vluighuis' (`flying house'; \cite{fortgens1928}). 
In his Tobelo dictionary, \citet[238]{hueting1908b} gives \textit{malige} as `paleis, vorstelijke woning, fraaie woning' (`palace,  royal dwelling, beautiful dwelling') and \citet{vanfraassennd} cite both \textit{maligai} and \textit{malige} for `palace' in their Ternate dictionary. These are borrowings from Malay where \textit{maligai} means `palace, princess’s bower' (see \citet[82]{hoogervorst2015} for the Tamil and Sanskrit origin of the word). The writing of \textit{ma} in isolation in the Modole texts is a reanalysis of the element as the relational noun marker, either by the speakers or the transcriber. 
The second element \textit{horo-horo}/\textit{soro-soro} is  
Ternate \textit{soro} `fly' (< \textsc{pcnh} *soSoC `fly'). Since the form was most likely borrowed already reduplicated, I do not analyze reduplication.
A literary model for a flying palace could be the \textit{Pu\d{s}paka vim\=ana} of the \textit{R\=am\=aya\d{n}a} (see \cite{feller2023}).}
\z

\ea\label{ex:text4-54}
Ho ma ligihoro-horo i'a de ma tonau'u i ma hidutuou,\\
\gll ho	ma ligihoro-horo	i'a	de	ma	tona'a-u'u	i	ma	hi-dutu'u-ou\\
     thus \textsc{rnm}	flying.palace	\textsc{dir}.\textsc{itv}	\textsc{conn}	\textsc{rnm}	earth-\textsc{dir}.\textsc{down}	3\textsc{nh}.\textsc{a}	\textsc{mid}	\textsc{caus}-?arrive-\textsc{already}\\
\glt `[They embarked] the flying palace ?and they came to the ground with it,'\footnote{MZ: In the Modole text, the previous sentences is ended with a full stop after \textit{jo baleno'a}. It seems likely that \textit{Ho ma ligihoro-horo i'a} in this line is part of the previous sentence and is the object of \textit{jo baleno'a} (`they embarked the flying palace'). 
The verb form \textit{hidutuou} is unclear to me. I tentatively analyze it as \textit{hi-dutu'u-ou} with \textit{dutu'u} being the onset-mutated counterpart of \textit{tutu'u} `arrive'.}
\z

\ea\label{ex:text4-55}
dauie jo i hiduba jo lutu-lutu de.\\
\gll da'u-ie	yo	'i	hi-duba	yo	lutu{\textasciitilde}lutu	de\\
     \textsc{loc}.\textsc{up}-\textsc{dir}.\textsc{up}	3\textsc{hpl}.\textsc{a}	3\textsc{hpl}.\textsc{u}	\textsc{caus}-?fall	3\textsc{hpl}.\textsc{a}	\textsc{rdpl}{\textasciitilde}sink	\textsc{conn}\\
\glt `?from above they fell with it [and] they sank.'\footnote{MZ: The form \textit{hiduba} is translated as `er mee vallen' (`fall with something') in the wordlist. I take this as an indication that \textit{hi-} is the causative prefix. The root \textit{duba} is not attested elsewhere in Modole or any other {\cnh} language. 
Based on Ellen's translation I assume that the function of the index combination <jo i> matches the one noted in [A.\ref{ex:text1-54}] (see above). Alternatively, the verb may mean `they made them fall.'}
\z

\ea\label{ex:text4-56}
Ho ma bere'i jo temo:\\
\gll ho	ma	bere'i	yo	temo\\
     thus	\textsc{rnm}	old.person	3\textsc{hpl}.\textsc{a}	say\\
\glt `So the old people said:'
\z

\ea\label{ex:text4-57}
 ``ngaro a nia u'u o ni mi dodoawa ho.''\\
\gll ngaro	'a	nia	u'u	'o	ni	mi	dodoa-ua	ho\\
     just	\textsc{foc}	2\textsc{pl}>3\textsc{nh}	\textsc{dir}.\textsc{down}	\textsc{emph}	2\textsc{pl}.\textsc{u}	1\textsc{pl}.\textsc{ex}.\textsc{a}	do-\textsc{neg}	thus\\
\glt `{``}Just come down, we (ex.) won't do anything to you (pl.).'''
\z

\ea\label{ex:text4-58}
Ena ma ligihoro-horo'a jo tatabua,\\
\gll ena	ma ligihoro-horo-o'a	yo	tatabua\\
     \textsc{pro}.3\textsc{nh} \textsc{rnm} flying.palace-\textsc{locv}	3\textsc{hpl}.\textsc{a}	bamboo.guitar\\
\glt `In the flying palace they played the bamboo guitar,'\footnote{MZ: \textit{Tatabua} or \textit{tatabuang} is given as the name of a stringed instrument for several Core North Halmahera languages. Notably, it refers to a set of gongs or cymbals in Ternate \citep{vanfraassennd}.}
\z

\ea\label{ex:text4-59}
de jo arababu, de jo huiha,\\
\gll de	yo	arababu	de	yo	huiha\\
     \textsc{conn}	3\textsc{hpl}.\textsc{a}	native.violin	\textsc{conn}	3\textsc{hpl}.\textsc{a}	play.the.flute\\
\glt `and they played the violin, and they played the flute'
\z

\ea\label{ex:text4-60}
i ma hididiota'a de jo djidjini.\\
\gll i	ma	hididiota-o'a	de	yo	ji{\textasciitilde}jini\\
     3\textsc{nh}.\textsc{a}	\textsc{mid}	drum.beating-\textsc{lim}	\textsc{conn}	3\textsc{hpl}.\textsc{a}	CV{\textasciitilde}jinn\\
\glt `and they beat the drum and had a jinn feast.'\footnote{MZ: The form \textit{hididiota'a} may contain the causative prefix \textit{hi-}.
Jinns entered Halmaheran traditional religion from Islam and are now well established there (see \cite[270-285]{hueting1921}).}
\z

\ea\label{ex:text4-61}
Ho ma doguulu jo mau jo 'i modo'a,\\
\gll ho	ma	doguulu	yo	mau	yo	'i	modo'a\\
     thus	\textsc{rnm}	young.man	3\textsc{hpl}.\textsc{a}	want	3\textsc{hpl}.\textsc{a}	3\textsc{hpl}.\textsc{u}	marry\\
\glt `And the young men wanted to marry them,'
\z

\newpage
\ea\label{ex:text4-62}
de jo temo: ``Na'o ni mi modo'a,\\
\gll de	yo	temo	na'o	ni	mi	modo'a\\
     \textsc{conn}	3\textsc{hpl}.\textsc{a}	say	\textsc{cond}	2\textsc{pl}.\textsc{a}	1\textsc{pl}.\textsc{ex}.\textsc{u}	marry\\
\glt `and they said: ``If you (pl.) [want] to marry us (ex.),'\footnote{GJE: the two humans disguised as ants that now appear to be girls}
\z

\ea\label{ex:text4-63}
halingou daue o todo'u ma ho'a mio hibelohu'u\\
\gll halingou	daue	o	todo'u	ma	ho'a	nio	hi-beloho-u'u\\
     necessary	\textsc{loc}.\textsc{down}	\textsc{nm}	k.o.bamboo	\textsc{rnm}	leaf	2\textsc{pl}.\textsc{a}	\textsc{caus}-impale-\textsc{dir}.\textsc{down}\\
\glt `you (pl.) have to impale bamboo leaves\footnote{GJE: on a lance or sharp object; a kind of game or contest},'\footnote{MZ: The index \textit{mio} is likely a typo for \textit{nio} since second person plural referents occur in the following.}
\z

\ea\label{ex:text4-64}
ho o nio dedebulaahi'';\\
\gll ho	'o	nio	dV-debulu-ohi\\
     thus	\textsc{emph}	2\textsc{pl}.\textsc{a}	\textsc{appl}-throw-\textsc{still}\\
\glt `you (pl.) must first throw [them],'''
\z

\ea\label{ex:text4-65}
de jo dedebulu ho iho jo i pupuhie;\\
\gll de	yo	dV-debulu	ho	{iho jo}	i	pupuhu-ie\\
     \textsc{conn}	3\textsc{hpl}.\textsc{a}	\textsc{appl}-throw	thus	?	3\textsc{nh}.\textsc{a}	spear-\textsc{dir}.\textsc{up}\\
\glt `?and they threw [them] and they speared them up\footnote{GJE: they hit};'\footnote{MZ: Based on the parallelism between this and the following lines [D.\ref{ex:text4-68}], [D.\ref{ex:text4-71}], [D.\ref{ex:text4-74}], [D.\ref{ex:text4-78}], [D.\ref{ex:text4-81}] I assume that <iho jo>, <iho po> and <i ho pa> should be written in the same manner and have the same meaning. 
Unfortunately, both the correct spelling and the meaning is unclear.
\textit{Hopa} could be cognate to Galela \textit{sopa} `wrap into each other' (indicating the layering of leaves). Alternatively, a connection to Malay \textit{supaya} `so that' is possible (`they threw so that they were speared up'). This subordination was borrowed into several Mainland North Halmahera languages.
My translations of the respective lines are based on Ellen's translation.}
\z

\ea\label{ex:text4-66}
i'a ja ino jo mau jo 'i modo'a,\\
\gll i'a	ya	ino	yo	mau	yo	'i	modo'a\\
     \textsc{dir}.\textsc{itv}	3\textsc{nh}>3\textsc{nh}	\textsc{dir}.\textsc{ven}	3\textsc{hpl}.\textsc{a}	want	3\textsc{hpl}.\textsc{a}	3\textsc{hpl}.\textsc{u}	marry\\
\glt `after that they came [and] they wanted to marry them,'
\z


\ea\label{ex:text4-67}
jo temo: ``halingou o todo'u ma hoa daue nio dedebuloahi'',\\
\gll yo	temo	halingou	o	todo'u	ma	ho'a	daue	nio	dV-debulu-ohi\\
     3\textsc{hpl}.\textsc{a}	say	necessary	\textsc{nm}	k.o.bamboo	\textsc{rnm}	leaf	\textsc{loc}.\textsc{down}	2\textsc{pl}.\textsc{a}	\textsc{appl}-throw-\textsc{still}\\
\glt `and they said: ``You (pl.) first have to throw bamboo leaves!'''
\z

\ea\label{ex:text4-68}
de jo dedebulu ho iho po i pupuhie;\\
\gll de	yo	dV-debulu	ho	{iho po}	i	pupuhu-ie\\
     \textsc{conn}	3\textsc{hpl}.\textsc{a}	\textsc{appl}-throw	thus	?	3\textsc{nh}.\textsc{a}	spear-\textsc{dir}.\textsc{up}\\
\glt `?and they threw [them] and the [leaves] were speared up,'
\z

\ea\label{ex:text4-69}
i'a ja ino jo mau,\\
\gll i'a	ya	ino	yo	mau\\
     \textsc{dir}.\textsc{itv}	3\textsc{nh}>3\textsc{nh}	\textsc{dir}.\textsc{ven}	3\textsc{hpl}.\textsc{a}	want\\
\glt `then they came [and] they wanted [to marry them],'
\z

\ea\label{ex:text4-70}
jo temo: ``halingou o todo'u ma ho'a daue nio dedebuloahi'',\\
\gll yo	temo	halingou	o	todo'u	ma	ho'a	daue	nio	dV-debulu-ohi\\
     3\textsc{hpl}.\textsc{a}	say	necessary	\textsc{nm}	k.o.bamboo	\textsc{rnm}	leaf	\textsc{loc}.\textsc{down}	2\textsc{pl}.\textsc{a}	\textsc{appl}-throw-\textsc{still}\\
\glt `and they said: ``You (pl.) first have to throw bamboo leaves,'''
\z

\ea\label{ex:text4-71}
de jo dedebulu ho iho po i pupuhie,\\
\gll de	yo	dV-debulu	ho	{iho po}	i	pupuhu-ie\\
     \textsc{conn}	3\textsc{hpl}.\textsc{a}	\textsc{appl}-throw	thus	?	3\textsc{nh}.\textsc{a}	spear-\textsc{dir}.\textsc{up}\\
\glt `?and they threw [them] and they speared [them] up,'
\z

\ea\label{ex:text4-72}
i'a ja ino jo mau jo 'i modo'a,\\
\gll i'a	ya	ino	yo	mau	yo	'i	modo'a\\
     \textsc{dir}.\textsc{itv}	3\textsc{nh}>3\textsc{nh}	\textsc{dir}.\textsc{ven}	3\textsc{hpl}.\textsc{a}	want	3\textsc{hpl}.\textsc{a}	3\textsc{hpl}.\textsc{u}	marry\\
\glt `then they came [and] they wanted to marry them,'
\z

\newpage
\ea\label{ex:text4-73}
jo temo: ``halingou o todo'u ma ho'a daue nio dedebuloahi'',\\
\gll yo	temo	halingou	o	todo'u	ma	ho'a	daue	nio	dV-debulu-ohi\\
     3\textsc{hpl}.\textsc{a}	say	necessary	\textsc{nm}	k.o.bamboo	\textsc{rnm}	leaf	\textsc{loc}.\textsc{down}	2\textsc{pl}.\textsc{a}	\textsc{appl}-throw-\textsc{still}\\
\glt `and they said: ``You (pl.) first have to throw bamboo leaves,'''
\z

\ea\label{ex:text4-74}
de jo dedebulu ho iho po i pupuhie.\\
\gll de	yo	dV-debulu	ho	{iho po}	i	pupuhu-ie\\
     \textsc{conn}	3\textsc{hpl}.\textsc{a}	\textsc{appl}-throw	thus	?	3\textsc{nh}.\textsc{a}	spear-\textsc{dir}.\textsc{up}\\
\glt `?and they threw [them] and they speared [them] up,'
\z

\ea\label{ex:text4-75}
I'a ja ino jo mau jo 'i modo'a,\\
\gll i'a	ya	ino	yo	mau	yo	'i	modo'a\\
     \textsc{dir}.\textsc{itv}	3\textsc{nh}>3\textsc{nh}	\textsc{dir}.\textsc{ven}	3\textsc{hpl}.\textsc{a}	want	3\textsc{hpl}.\textsc{a}	3\textsc{hpl}.\textsc{u}	marry\\
\glt `Then they came [and] they wanted to marry them,'
\z

\ea\label{ex:text4-76}
jo temo: ``halingou o todo'u ma ho'a daue nio belohu'u,\\
\gll yo	temo	halingou	o	todo'u	ma	ho'a	daue	nio	beloho-u'u\\
     3\textsc{hpl}.\textsc{a}	say	necessary	\textsc{nm}	k.o.bamboo	\textsc{rnm}	leaf	\textsc{loc}.\textsc{down}	2\textsc{pl}.\textsc{a}	impale-\textsc{dir}.\textsc{down}\\
\glt `and they said: ``You (pl.) first have to spear bamboo leaves,'
\z

\ea\label{ex:text4-77}
ho nio dedebuluahi'';\\
\gll ho	nio	debulu-ohi\\
     thus	2\textsc{pl}.\textsc{a}	throw-\textsc{still}\\
\glt `you (pl.) [have to] throw [them],'''
\z

\ea\label{ex:text4-78}
de jo dedebulu ho i ho pa i tamide;\\
\gll de	yo	dV-debulu	ho	{i ho pa}	i	tami-de\\
     \textsc{conn}	3\textsc{hpl}.\textsc{a}	\textsc{appl}-throw	thus	?	3\textsc{nh}.\textsc{a}	sit-\textsc{conn}\\
\glt `?and they threw [them], raising [them],'\footnote{GJE: throwing them on top of those already impaled earlier}
\z

\newpage
\ea\label{ex:text4-79}
i'a ja inojo mau jo 'i modo'a,\\
\gll i'a	ya	ino-yo	mau	yo	'i	modo'a\\
     \textsc{dir}.\textsc{itv}	3\textsc{nh}>3\textsc{nh}	\textsc{dir}.\textsc{ven}-3\textsc{hpl}.\textsc{a}	want	3\textsc{hpl}.\textsc{a}	3\textsc{hpl}.\textsc{u}	marry\\
\glt `then they came [and] they wanted to marry them,'
\z

\ea\label{ex:text4-80}
jo temo: ``halingohu o todo'u ma ho'a daue nio dedebuloahi'';\\
\gll yo	temo	halingohu	o	todo'u	ma	ho'a	daue	nio	dV-debulu-ohi\\
     3\textsc{hpl}.\textsc{a}	say	necessary	\textsc{nm}	k.o.bamboo	\textsc{rnm}	leaf	\textsc{loc}.\textsc{down}	2\textsc{pl}.\textsc{a}	\textsc{appl}-throw-\textsc{still}\\
\glt `[and] they said: ``You (pl.) first have to throw bamboo leaves,'''
\z

\ea\label{ex:text4-81}
de jo dedebuluho i ho pa i pupuhie.\\
\gll de	yo	de-debuluho	{i ho pa}	i	pupuhu-ie\\
     \textsc{conn}	3\textsc{hpl}.\textsc{a}	\textsc{appl-}throw	?	3\textsc{nh}.\textsc{a}	spear-\textsc{dir}.\textsc{up}\\
\glt `?and they threw [them] and speared [them] up.'
\z

\ea\label{ex:text4-82}
De o wange ma hohoruoali ma 'oana awi ngoa wa ino, de wo temo:\\
\gll de	o	wange	ma	hohoru-o'a-oli	ma	'oana	awi	ngoa	wa	ino	de	wo	temo\\
     \textsc{conn}	\textsc{nm}	sun	\textsc{rnm}	west-\textsc{locv}-\textsc{again}	\textsc{rnm}	king	3\textsc{sg.m}.\textsc{poss}	child	3\textsc{sg.m}>3\textsc{nh}	\textsc{dir}.\textsc{ven}	\textsc{conn}	3\textsc{sg.m}.\textsc{a}	say\\
\glt `And the son of the Western King came too and said:'
\z

\ea\label{ex:text4-83}
 ``Na'o nio mode'e de ti ni modo'a,''\\
\gll na'o	nio	mode'e	de	to	ni	modo'a\\
     \textsc{cond}	2\textsc{pl}.\textsc{a}	want	\textsc{conn}	1\textsc{sg}.\textsc{a}	2\textsc{pl}.\textsc{u}	marry\\
\glt `{``}If you (pl.) want, I will marry you (pl.),'''
\z

\ea\label{ex:text4-84}
de jo temo: ``halingou o todo'u ma ho'a daue no dedebuloahi,\\
\gll de	yo	temo	halingou	o	todo'u	ma	ho'a	daue	no	dV-debulu-ohi\\
     \textsc{conn}	3\textsc{hpl}.\textsc{a}	say	necessary	\textsc{nm}	k.o.bamboo	\textsc{rnm}	leaf	\textsc{loc}.\textsc{down}	2\textsc{sg}.\textsc{a}	\textsc{appl}-throw-\textsc{still}\\
\glt `and they said: ``You (sg.) have to throw bamboo leaves first,'
\z

\newpage
\ea\label{ex:text4-85}
aha ni mi modo'a.''\\
\gll aha	ni	mi	modo'a\\
     as.a.consequence	2\textsc{sg}.\textsc{a}	1\textsc{pl}.\textsc{ex}.\textsc{u}	marry\\
\glt `then you (sg.) [can] marry us (ex.).'''
\z

\ea\label{ex:text4-86}
De wo dedebulu, ma o wi ha'awa,\\
\gll de	wo	dV-debulu	ma	'o	wi	ha'a-ua\\
     \textsc{conn}	3\textsc{sg.m}.\textsc{a}	\textsc{appl}-throw	but	\textsc{emph}	3\textsc{sg.m}.\textsc{u}	stab.with.a.long.object-\textsc{neg}\\
\glt `And he threw them, ?but he missed,'\footnote{MZ: The index \textit{wi} is unexpected here. It may be a typo for \textit{wo} `\textsc{3sgm.A}' or \textit{wa} `\textsc{3sgm>3nh}.}
\z

\ea\label{ex:text4-87}
de a wo 'i modo'au.\\
\gll de	'a	wo	'i	modo'a-ou\\
     \textsc{conn}	\textsc{foc}	3\textsc{sg.m}.\textsc{a}	3\textsc{hpl}.\textsc{u}	marry-\textsc{already}\\
\glt `and he married them.'
\z

\ea\label{ex:text4-88}
Ho manga damunu, de manga liwanga a i ma hididiotau.\\
\gll ho	manga	damunu	de	manga	liwanga	'a	i	ma	hididiota-ou\\
     thus	3\textsc{hpl}.\textsc{poss}	drum	\textsc{conn}	3\textsc{hpl}.\textsc{poss}	gong	\textsc{foc}	3\textsc{nh}.\textsc{a}	\textsc{mid}	drum.beating-\textsc{already}\\
\glt `So their drums and their gongs\footnote{GJE: copper cymbals} [were] beaten.'
\z
