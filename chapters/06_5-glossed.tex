\ea\label{ex:text6-1}
O Uho\\
\gll o	'uho\\
     \textsc{nm}	cuscus\\
\glt `The cuscus'\footnote{MZ: \textit{'uho} refers to an animal of the Phalanger genus. It is very likely that the English word \textit{cuscus} goes back to a Core North Halmahera language (\textsc{pcnh} *kusoC), potentially Ternate \textit{kuso}.}
\z

\ea\label{ex:text6-2}
Naga o njawa moi mo hingoa'a o 'uho.\\
\gll naga	o	nyawa	moi	mo	hi-ngoa'a	o	'uho\\
     \textsc{exist}	\textsc{nm}	person	one	3\textsc{sg.f}.\textsc{a}	\textsc{caus}-child	\textsc{nm}	cuscus\\
\glt `There was a woman [and] she gave birth to a cuscus.'
\z

\ea\label{ex:text6-3}
De ma eha mo hihano mo temo:\\
\gll de	ma	eha	mo	hi-hano	mo	temo\\
     \textsc{conn}	\textsc{rnm}	mother	3\textsc{sg.f}.\textsc{a}	\textsc{caus}-ask	3\textsc{sg.f}.\textsc{a}	say\\
\glt `And the mother asked, saying:'
\z

\ea\label{ex:text6-4}
 ``Ai 'uho, nagoona ja tana?''\\
\gll ai	'uho	naga-ona	ya	tana\\
     1\textsc{sg}.\textsc{poss}	cuscus	\textsc{exist}-\textsc{pro}.3\textsc{plh}	3\textsc{nh}>3\textsc{nh}	raise\\
\glt `{``}My cuscus, who shall raise it?'''
\z


\ea\label{ex:text6-5}
De ma 'oana awi ngoa'a ja tumudingo'a ho ma dodoto mo mode'e.\\
\gll de	ma	'oana	awi	ngoa'a	ya	tumudingi-o'a	ho	ma	dodoto	mo	mode'e\\
     \textsc{conn}	\textsc{rnm}	king	3\textsc{sg.m}.\textsc{poss}	child	3\textsc{nh}>3\textsc{nh}	seven-\textsc{locv}	thus	\textsc{rnm}	younger.sibling	3\textsc{sg.f}.\textsc{a}	want\\
\glt `And of the seven children\footnote{GJE: daughters} of the king the youngest wanted [it].'
\z

\newpage
\ea\label{ex:text6-6}
De jo temo: ``Tole, o naulu no ali-ali nage o 'uho de na modo'a.''\\
\gll de	yo	temo	tole	o	naulu	no	ali{\textasciitilde}ali	nage	o	'uho	de	na	modo'a\\
     \textsc{conn}	3\textsc{hpl}.\textsc{a}	say	sister	\textsc{nm}	man	2\textsc{sg}.\textsc{a}	\textsc{rdpl}{\textasciitilde}weep	\textsc{exist}.\textsc{prox}	\textsc{nm}	cuscus	\textsc{conn}	2\textsc{sg}>3\textsc{nh}	marry\\
\glt `And they\footnote{GJE: the other six daughters} said: ``Sister, [are] you (sg.) so weeping over men\footnote{GJE: Do you have an aversion to men?} that you (sg.) marry a cuscus?'''
\z

\ea\label{ex:text6-7}
De muna a mu ma liliili,\\
\gll de	muna	'a	mu	ma	liliili\\
     \textsc{conn}	\textsc{pro}.3\textsc{sg.f}	\textsc{foc}	3\textsc{sg.f}.\textsc{a}	\textsc{mid}	silent\\
\glt `And she was silent,'
\z

\ea\label{ex:text6-8}
ena manga rame i ho o dewelau.\\
\gll ena	manga	rame	i	ho	o	dewela-ou\\
     \textsc{pro}.3\textsc{nh}	3\textsc{hpl}.\textsc{poss}	feast	3\textsc{nh}.\textsc{a}	thus	\textsc{nm}	tomorrow-\textsc{foc}\\
\glt `and then the next day was their feast.'
\z

\ea\label{ex:text6-9}
De jo temo: ``Tole, o dewelau po rame po ma hiwange hau'u,\\
\gll de	yo	temo	tole	o	dewela-ou	po	rame	po	ma	hi-wange	hau'u\\
     \textsc{conn}	3\textsc{hpl}.\textsc{a}	say	sister	\textsc{nm}	tomorrow-\textsc{foc}	1\textsc{pl}.\textsc{in}.\textsc{a}	feast	1\textsc{pl}.\textsc{in}.\textsc{a}	\textsc{mid}	\textsc{caus}-sun	hot\\
\glt `And they\footnote{GJE: The sisters.} said: ``Sister, tomorrow we (in.) will celebrate ?during daytime,'
\z

\ea\label{ex:text6-10}
'one ani 'o'ata daengo'o wo uti dja'o ma wehara.''\\
\gll 'o-ne	ani	'o'ata	dai-ngo'o	wo	uti	ja'o	ma	wehara\\
     \textsc{emph}-\textsc{prox}	2\textsc{sg}.\textsc{poss}	husband	\textsc{loc}.\textsc{sea}-\textsc{n:dir.sea}	3\textsc{sg.m}.\textsc{a}	descend	appearance	\textsc{rnm}	ugly\\
\glt `like this your (sg.) husband ?[shall] descend\footnote{GJE: To celebrate with us.} seawards [with] his ugly appearance.'''\footnote{MZ: Ellen interprets \textit{daengo'o} `\textsc{loc.sea-n:dir.sea}' as the current location of the cuscus but it may also refer to the goal of the verb \textit{wo uti} `he descends'. }
\z

\ea\label{ex:text6-11}
De a mu ma liliili,\\
\gll de	'a	mu	ma	liliili\\
     \textsc{conn}	\textsc{foc}	3\textsc{sg.f}.\textsc{a}	\textsc{mid}	silent\\
\glt `And she was silent,'
\z

\ea\label{ex:text6-12}
genaade i ho o nenau jo rame\\
\gll geena-o'a-de	i	ho	'o	neena-ou	yo	rame\\
     \textsc{dist}:\textsc{pro}.3\textsc{nh}-\textsc{locv}-\textsc{conn}	3\textsc{nh}.\textsc{a}	thus	\textsc{emph}	\textsc{prox}:\textsc{pro}.3\textsc{nh}-\textsc{foc}	3\textsc{hpl}.\textsc{a}	feast\\
\glt `then they celebrated in this way'
\z

\ea\label{ex:text6-13}
de ma uho i temo:\\
\gll de	ma	'uho	i	temo\\
     \textsc{conn}	\textsc{rnm}	cuscus	3\textsc{nh}.\textsc{a}	say\\
\glt `and the cuscus said:'
\z

\ea\label{ex:text6-14}
 ``O igono nage no na dupuniti la po ma gowo'',\\
\gll o	igono	nage	no	na	dV-puniti	la	po	ma	gowo\\
     \textsc{nm}	coconut	\textsc{exist}.\textsc{prox}	2\textsc{sg}.\textsc{a}	1\textsc{pl}.\textsc{in}.\textsc{u}	\textsc{appl}-coconut.husk	so.that	1\textsc{pl}.\textsc{in}.\textsc{a}	\textsc{mid}	wash.hair.with.coconut.milk\\
\glt `{``}Strip this coconut of its husk for us (in.), so that we (in.) can wash and oil our hair,'''
\z

\ea\label{ex:text6-15}
de ma lia'a jo temo:\\
\gll de	ma	lia'a	yo	temo\\
     \textsc{conn}	\textsc{rnm}	older.sibling	3\textsc{hpl}.\textsc{a}	say\\
\glt `and the older sisters said:'
\z

\ea\label{ex:text6-16}
 ``tole, ho 'one awi 'iau wa gowo.''\\
\gll tole	ho	'o-ne	awi	'iau	wa	gowo\\
     sister	thus	\textsc{emph}-\textsc{prox}	3\textsc{sg.m}.\textsc{poss}	?part	3\textsc{sg.m}>3\textsc{nh}	wash.hair.with.coconut.milk\\
\glt `{``}Sister, so like this what part of him [shall] he wash and oil'''\footnote{GJE: Because cuscuses have only very short hair.}
\z

\ea\label{ex:text6-17}
De ma puniti ho jo ma gowo,\\
\gll de	ma	puniti	ho	yo	ma	gowo\\
     \textsc{conn}	3\textsc{sg.f}>3\textsc{nh}	coconut.husk	thus	3\textsc{hpl}.\textsc{a}	\textsc{mid}	wash.hair.with.coconut.milk\\
\glt `And she dehusked coconuts to wash and oil their hair,'\footnote{MZ: Following Ellen's translation, I analyze \textit{ma} as an \textsc{3sgf>3nh} index combination. It is also possible that this is the relational noun marker and that \textit{puniti} is used as a noun. The line would then mean `They washed and oiled their hair with coconut husk'.}
\z


\ea\label{ex:text6-18}
ja o'o dai o dowongi ma baha o de jo ma gowo.\\
\gll ya	o'o	dai	o	dowongi	ma	baha	'o	de	yo	ma	gowo\\
     3\textsc{nh}>3\textsc{nh}	\textsc{dir}.\textsc{sea}	\textsc{loc}.\textsc{sea}	\textsc{nm}	sand	\textsc{rnm}	border	?\textsc{emph}	\textsc{conn}	3\textsc{hpl}.\textsc{a}	\textsc{mid}	wash.hair.with.coconut.milk\\
\glt `they went seawards to the edge of the beach and they washed and oiled their hair.'\footnote{The letter <o> may also represent the \textsc{sea} directional \textit{o'o}.}
\z

\ea\label{ex:text6-19}
De wo temo: ``ngonahi no hila;''\\
\gll de	wo	temo	ngona-ohi	no	hila\\
     \textsc{conn}	3\textsc{sg.m}.\textsc{a}	say	\textsc{pro}.2\textsc{sg}-\textsc{still}	2\textsc{sg}.\textsc{a}	first\\
\glt `And he said: ``You (sg.) go first,'''
\z

\ea\label{ex:text6-20}
de munahi mu ma gowo.\\
\gll de	muna-ohi	mu	ma	gowo\\
     \textsc{conn}	\textsc{pro}.3\textsc{sg.f}-\textsc{still}	3\textsc{sg.f}.\textsc{a}	\textsc{mid}	wash.hair.with.coconut.milk\\
\glt `and she first washed and oiled her hair.'
\z


\ea\label{ex:text6-21}
De mu ma gowo i botode, mo temo: ``ngonali.''\\
\gll de	mu	ma	gowo	i	boto-de	mo	temo	ngona-oli\\
     \textsc{conn}	3\textsc{sg.f}.\textsc{a}	\textsc{mid}	wash.hair.with.coconut.milk	3\textsc{nh}.\textsc{a}	finish-\textsc{conn}	3\textsc{sg.f}.\textsc{a}	say	\textsc{pro}.2\textsc{sg}-\textsc{again}\\
\glt `And [when] she was done with washing and oiling her hair, she said: ``You (sg.) too,'''
\z

\ea\label{ex:text6-22}
De una ali, wo temo:\\
\gll de	una	oli	wo	temo\\
     \textsc{conn}	\textsc{pro}.3\textsc{sg.m}	\textsc{again}	3\textsc{sg.m}.\textsc{a}	say\\
\glt `and he [did it] too, [and] he said:'
\z

\ea\label{ex:text6-23}
 ``no ma ruwuto'a,''\\
\gll no	ma	ruwutu-o'a\\
     2\textsc{sg}.\textsc{a}	\textsc{mid}	close.eyes-\textsc{lim}\\
\glt `{``}Close your (sg.) eyes,'''
\z


\ea\label{ex:text6-24}
de mu ma ruwuto'a,\\
\gll de	mu	ma	ruwutu-o'a\\
     \textsc{conn}	3\textsc{sg.f}.\textsc{a}	\textsc{mid}	close.eyes-\textsc{lim}\\
\glt `and she closed her eyes,'
\z

\ea\label{ex:text6-25}
ho mo temo: ``'a i leletongo,''\\
\gll ho	mo	temo	'a	i	CV{\textasciitilde}letongo\\
     thus	3\textsc{sg.f}.\textsc{a}	say	\textsc{foc}	3\textsc{nh}.\textsc{a}	\textsc{rdpl}{\textasciitilde}shine\\
\glt `and he said: ``It is a mask,'''\footnote{MZ: Ellen translates \textit{leletongo} as `it shines' (``het schittert''). \textit{Leletongo} means `mask' elsewhere ([E.\ref{ex:text5-97}], [E.\ref{ex:text5-123}]) and therefore I translate the clause as `it is a mask'. Alternatively, \textit{'a} and \textit{i} belong together as the first person singular possessive pronoun so that the clause means `my mask'.}
\z

\ea\label{ex:text6-26}
ena awi pahita'a w'ai'iau.\\
\gll ena	awi	pahita'a	wa-'ai'i-ou\\
     \textsc{pro}.3\textsc{nh}	3\textsc{sg.m}.\textsc{poss}	mask	3\textsc{sg.m}>3\textsc{nh}-take.out-\textsc{already}\\
\glt `then he took off his mask.'
\z

\ea\label{ex:text6-27}
Awi utu ma hongona o hala'a, ma hongona o gurahi,\\
\gll awi	utu	ma	hongona	o	hala'a	ma	hongona	o	gurahi\\
     3\textsc{sg.m}.\textsc{poss}	hair	\textsc{rnm}	half	\textsc{nm}	silver	\textsc{rnm}	half	\textsc{nm}	gold\\
\glt `Half of his hair was silver, the [other] half was gold,'
\z

\ea\label{ex:text6-28}
awi ilingi ma hongona o hala'a, ma hongona o gurahi.\\
\gll awi	ilingi	ma	hongona	o	hala'a	ma	hongona	o	gurahi\\
     3\textsc{sg.m}.\textsc{poss}	tooth	\textsc{rnm}	half	\textsc{nm}	silver	\textsc{rnm}	half	\textsc{nm}	gold\\
\glt `[and] half of his teeth were silver [and] the [other] half was gold.'
\z

\ea\label{ex:text6-29}
De jo liou, wi ma'e, de jo temo ma lia'a:\\
\gll de	yo	lio-ou	wi	ma'e	de	yo	temo	ma	lia'a\\
     \textsc{conn}	3\textsc{hpl}.\textsc{a}	return-\textsc{already}	3\textsc{sg.m}.\textsc{u}	see	\textsc{conn}	3\textsc{hpl}.\textsc{a}	say	\textsc{rnm}	older.sibling\\
\glt `And they\footnote{GJE: The other sisters.} returned [and] they saw him and they, the older sisters, said:'
\z


\ea\label{ex:text6-30}
 ``tole, nanga 'o'ata,''\\
\gll tole	nanga	'o'ata\\
     sister	1\textsc{pl}.\textsc{in}.\textsc{poss}	husband\\
\glt `{``}Sister, our (in.) husband'''\footnote{GJE: That shall be our husband.}
\z

\ea\label{ex:text6-31}
de una wo temo:\\
\gll de	una	wo	temo\\
     \textsc{conn}	\textsc{pro}.3\textsc{sg.m}	3\textsc{sg.m}.\textsc{a}	say\\
\glt `and he said:'
\z

\ea\label{ex:text6-32}
 ``uwa ai modo'a ho,''\\
\gll uwa	ai	modo'a	ho\\
     \textsc{proh}	1\textsc{sg}.\textsc{poss}	marry	thus\\
\glt `{``}No, my wife,'''\footnote{GJE: She alone is my wife; MZ: or, alternatively: `You can't be my wives.'}
\z

\ea\label{ex:text6-33}
jo temo: ``o'ia ma mudu'o,''\\
\gll yo	temo	o'ia	ma	mudu'o\\
     3\textsc{hpl}.\textsc{a}	say	what	\textsc{rnm} brother-in-law\\
\glt `[and] they said: ``What [is he our] brother-in-law?'''
\z

\ea\label{ex:text6-34}
de wo temo: ``uwa''.\\
\gll de	wo	temo	uwa\\
     \textsc{conn}	3\textsc{sg.m}.\textsc{a}	say	\textsc{proh}\\
\glt `and he said: ``No.'''\footnote{GJE: He preferred to not reveal himself because earlier, before he had taken off his disguise, they had mocked him.}
\z

\ea\label{ex:text6-35}
Gena'adau de awi ruae de wo tagi,\\
\gll geena-o'a-dau-ou	de	awi	ruae	de	wo	tagi\\
     \textsc{dist}:\textsc{pro}.3\textsc{nh}-\textsc{locv}-\textsc{loc}.\textsc{down}-\textsc{foc}	\textsc{conn}	3\textsc{sg.m}.\textsc{poss}	?troubled	\textsc{conn}	3\textsc{sg.m}.\textsc{a}	go\\
\glt `Then he went [away] ?with a troubled heart'\footnote{MZ: Ellen translates this line as `So he also went [away] with a troubled heart' (``Alzoo ging hij ook (weg) met een bezwaard hart''). \textit{Ruae} may be cognate to Tobelo \textit{ruae} `create an uproar, rage, scream, make noise' (\citet[321]{hueting1908b}; also see [A.\ref{ex:text1-2}]) and hence I translate it as `troubled'.}
\z

\ea\label{ex:text6-36}
de ma we'ati'a wo djadji wo temo:\\
\gll de	ma	we'ata-i'a	wo	jaji	wo	temo\\
     \textsc{conn}	\textsc{rnm}	wife-\textsc{dir}.\textsc{itv}	3\textsc{sg.m}.\textsc{a}	promise	3\textsc{sg.m}.\textsc{a}	say\\
\glt `and to his wife he promised, saying:'
\z

\ea\label{ex:text6-37}
 ``Na'o no ngo'a, de ani ngo'a modidi,\\
\gll na'o	no	ngo'a	de	ani	ngo'a	modidi\\
     \textsc{cond}	2\textsc{sg}.\textsc{a}	child	\textsc{conn}	2\textsc{sg}.\textsc{poss}	child	two\\
\glt `{``}When you (sg.) give birth, your (sg.) children will be two,'
\z

\ea\label{ex:text6-38}
moi de ma mede ma giauo'a,\\
\gll moi	de	ma	mede	ma	giau-o'a\\
     one	\textsc{conn}	\textsc{rnm}	moon	\textsc{rnm}	new-\textsc{locv}\\
\glt `one [will be born] at the new moon,'
\z

\ea\label{ex:text6-39}
de moi ma wange ma njonjie'a.''\\
\gll de	moi	ma	wange	ma	nyonyie-o'a\\
     \textsc{conn}	one	\textsc{rnm}	sun	\textsc{rnm}	sunrise-\textsc{locv}\\
\glt `and one [will be born] at sunrise.'''
\z

\ea\label{ex:text6-40}
Gena'ade wo tagi,\\
\gll geena-o'a-de	wo	tagi\\
     \textsc{dist}:\textsc{pro}.3\textsc{nh}-\textsc{locv}-\textsc{conn}	3\textsc{sg.m}.\textsc{a}	go\\
\glt `Then he went,'\footnote{MZ: This line is not translated in the Dutch original but has been combined with the following line.}
\z

\ea\label{ex:text6-41}
ho wo paha'a de mo ngo'a'a.\\
\gll ho	wo	paha-o'a	de	mo	ngo'a-o'a\\
     thus	3\textsc{sg.m}.\textsc{a}	leave-\textsc{lim}	\textsc{conn}	3\textsc{sg.f}.\textsc{a}	child-\textsc{lim}\\
\glt `when he was gone, she gave birth.'
\z

\ea\label{ex:text6-42}
De ma lia'a jo temo:\\
\gll de	ma	lia'a	yo	temo\\
     \textsc{conn}	\textsc{rnm}	older.sibling	3\textsc{hpl}.\textsc{a}	say\\
\glt `And the older sisters said:'
\z

\ea\label{ex:text6-43}
 ``Ani la'o mia ditio'a,''\\
\gll ani	la'o	mia	dV-tio'a\\
     2\textsc{sg}.\textsc{poss}	eye	1\textsc{pl}.\textsc{ex}>3\textsc{nh}	\textsc{appl}-?seal\\
\glt `{``}We (ex.) [will] seal your (sg.) eyes,'''\footnote{MZ: The form \textit{tio'a} in this and the next line is unclear to me.}
\z

\ea\label{ex:text6-44}
de ja ditio'a, de mo ngo'a'a.\\
\gll de	ya	dV-tio'a	de	mo	ngo'a-o'a\\
     \textsc{conn}	3\textsc{nh}>3\textsc{nh}	\textsc{appl}-?seal	\textsc{conn}	3\textsc{sg.f}.\textsc{a}	child-\textsc{lim}\\
\glt `and they sealed them, and she gave birth.'
\z

\ea\label{ex:text6-45}
De i huputu'u\\
\gll de	i	hupu-u'u\\
     \textsc{conn}	3\textsc{nh}.\textsc{a}	exit-\textsc{dir}.\textsc{down}\\
\glt `And [when] they were born,'\footnote{MZ: The <t> in the form <huputu'u> is unclear to me. It may be a typo for <i> and the line may actually read \textit{De i hupu i u'u} `And they went out and down'.}
\z

\ea\label{ex:text6-46}
de o borua i jo hihupuo'a\\
\gll de	o	borua	i	yo	hi-hupu-o'a\\
     \textsc{conn}	\textsc{nm}	box	?	3\textsc{hpl}.\textsc{a}	\textsc{caus}-exit-\textsc{lim}\\
\glt `they brought them outside in a box'\footnote{MZ: I do not know what the function of <i> is in this clause. Maybe <i jo> is an unconventional spelling of the index \textit{yo}.}
\z

\ea\label{ex:text6-47}
de jo hidaahini,\\
\gll de	yo	hi-dahini\\
     \textsc{conn}	3\textsc{hpl}.\textsc{a}	\textsc{caus}-float\\
\glt `and let [the children] float away,'
\z

\ea\label{ex:text6-48}
i togumiade o u'u ma bitino jo doho-doaha ami oleha, de o puniti.\\
\gll i	togumu-i'a-de	o	u'u	ma	bitino	yo	dV-ho-	doa-iha	ami	ole-iha	de	o	puniti\\
     3\textsc{nh}.\textsc{a}	finish-\textsc{dir}.\textsc{itv}-\textsc{conn}	\textsc{nm}	fire	\textsc{rnm}	burned.object	3\textsc{hpl}.\textsc{a}	\textsc{appl}-?-	put-\textsc{dir}.\textsc{land}	3\textsc{sg.f}.\textsc{poss}	?vagina-\textsc{dir}.\textsc{land}	\textsc{conn}	\textsc{nm}	coconut.husk\\
\glt `after that they put charcoal in her vagina, and coconut bark.'
\z

\ea\label{ex:text6-49}
De ma mi la'o jo dauhu ho i pelangali,\\
\gll de	ma-ami	la'o	yo	dauhu	ho	i	pelanga-oli\\
     \textsc{conn}	\textsc{rnm}-3\textsc{sg.f}.\textsc{poss}	eye	3\textsc{hpl}.\textsc{a}	oil	thus	3\textsc{nh}.\textsc{a}	open-\textsc{again}\\
\glt `And her eyes they rubbed with oil, so they opened again,'
\z

\ea\label{ex:text6-50}
de mo hano ma eha, mo temo:\\
\gll de	mo	hano	ma	eha	mo	temo\\
     \textsc{conn}	3\textsc{sg.f}.\textsc{a}	ask	\textsc{rnm}	mother	3\textsc{sg.f}.\textsc{a}	say\\
\glt `and she, the mother, asked, saying:'
\z

\ea\label{ex:text6-51}
 ``tole, ho ma ngo'a?''\\
\gll tole	ho	ma	ngo'a\\
     sister	thus	\textsc{rnm}	child\\
\glt `{``}Sisters, how are the kids?'''
\z

\ea\label{ex:text6-52}
De jo temo: ``o ngoa'a o'iano?\\
\gll de	yo	temo	o	ngoa'a	o'ia-ino\\
     \textsc{conn}	3\textsc{hpl}.\textsc{a}	say	\textsc{nm}	child	what-\textsc{dir}.\textsc{ven}\\
\glt `And they said: ``Children, where?'
\z

\ea\label{ex:text6-53}
Begeena 'a o u'u bobitino, de o puniti.''\\
\gll be-geena	'a	o	u'u	CV{\textasciitilde}bitino	de	o	puniti\\
     ?here-\textsc{dist}:\textsc{pro}.3\textsc{nh}	\textsc{foc}	\textsc{nm}	fire	\textsc{rdpl}{\textasciitilde}burned.object	\textsc{conn}	\textsc{nm}	coconut.husk\\
\glt `Here are only charcoal and coconut bark.'''
\z

\newpage

\ea\label{ex:text6-54}
De ma ngoa'a ge i bawa ma Djindaali awi hohoniha.\\
\gll de	ma	ngoa'a	ge	i	bawa	ma	jindaali	awi	hohana-iha\\
     \textsc{conn}	\textsc{rnm}	child	\textsc{dist}	3\textsc{nh}.\textsc{a}	?float	\textsc{rnm}	general	3\textsc{sg.m}.\textsc{poss}	landing.place-\textsc{dir}.\textsc{land}\\
\glt `And those children\footnote{GJE: In the box.} floated landwards to the landing place of the general\footnote{MZ: \textit{Jindaali} is likely a borrowing of Indonesian \textit{jenderal} `general'. People on Halmahera at the beginning of the 20th century were familiar with the \textit{Gouverneur-Generaal} (`Governor-general') of Ternate, the local head of the Dutch colonial administration. The general and his wife in this story are comic figures (and cannibals) rather than people who command authority.}.'\footnote{MZ: The form \textit{bawa} is unattested elsewhere. It may be a borrowing of Indonesian \textit{bawa} `carry.'}
\z

\ea\label{ex:text6-55}
De ma Djindaali ma we'ata ma ma o'o\\
\gll de	ma	jindaali	ma	we'ata	mu	ma	o'o\\
     \textsc{conn}	\textsc{rnm}	general	\textsc{rnm}	wife	3\textsc{sg.f}.\textsc{a}	\textsc{mid}	defecate\\
\glt `And the wife of the general defecated,'
\z

\ea\label{ex:text6-56}
de mu ma tami ma boruade,\\
\gll de	mu	ma	tami	ma	borua-de\\
     \textsc{conn}	3\textsc{sg.f}.\textsc{a}	\textsc{mid}	sit	\textsc{rnm}	box-\textsc{conn}\\
\glt `and she sat down on the box,'
\z

\ea\label{ex:text6-57}
de ma doda'a ma ngo'a'a i ali.\\
\gll de	ma	doda-o'a	ma	ngoa'a	i	ali\\
     \textsc{conn}	\textsc{rnm}	inside-\textsc{locv}	\textsc{rnm}	child	3\textsc{nh}.\textsc{a}	weep\\
\glt `and in there the children cried.'
\z

\ea\label{ex:text6-58}
De mu ma popolitana ma iha mo hingahu mo temo:\\
\gll de	mu	ma	CV{\textasciitilde}politana	ma	iha	mo	hi-ngahu	mo	temo\\
     \textsc{conn}	3\textsc{sg.f}.\textsc{a}	\textsc{mid}	\textsc{rdpl}{\textasciitilde}run	3\textsc{sg.f}>3\textsc{nh}	\textsc{dir}.\textsc{land}	3\textsc{sg.f}.\textsc{a}	\textsc{caus}-report	3\textsc{sg.f}.\textsc{a}	say\\
\glt `And she ran landwards [and] she reported, saying:'
\z

\newpage
\ea\label{ex:text6-59}
 ``Bere'i, dai, o hohano'a o gota ma doda'a o ngoa'a jo ali''\\
\gll bere'i	dai	o	hohana-o'a	o	gota	ma	doda-o'a	o	ngoa'a	yo	ali\\
     old.person	\textsc{loc}.\textsc{sea}	\textsc{nm}	landing.place-\textsc{locv}	\textsc{nm}	wood	\textsc{rnm}	inside-\textsc{locv}	\textsc{nm}	child	3\textsc{hpl}.\textsc{a}	weep\\
\glt `{``}Old man, seawards at the landing place inside a [piece of] wood children are crying,'''
\z

\ea\label{ex:text6-60}
de wo temo: ``Berei, geenade pa poga.''\\
\gll de	wo	temo	bere'i	geena-de	pa	poga\\
     \textsc{conn}	3\textsc{sg.m}.\textsc{a}	say	old.person	\textsc{dist}:\textsc{pro}.3\textsc{nh}-\textsc{conn}	1\textsc{pl}.\textsc{in}>3\textsc{nh}	split\\
\glt `and he said: ``Old woman, then [let]'s (in.) split it open.'''
\z

\ea\label{ex:text6-61}
De ja poga,\\
\gll de	ya	poga\\
     \textsc{conn}	3\textsc{nh}>3\textsc{nh}	split\\
\glt `And they split it, '
\z

\ea\label{ex:text6-62}
ja poga de ena ma ngo'a'a ja mididio'a.\\
\gll ya	poga	de	ena	ma	ngoa'a	ya	mididi-o'a\\
     3\textsc{nh}>3\textsc{nh}	split	\textsc{conn}	\textsc{pro}.3\textsc{nh}	\textsc{rnm}	child	3\textsc{nh}>3\textsc{nh}	two-\textsc{locv}\\
\glt `and they split it and there were two children.'
\z

\ea\label{ex:text6-63}
De ne ma we'ata mo temo:\\
\gll de	ne	ma	we'ata	mo	temo\\
     \textsc{conn}	\textsc{prox}	\textsc{rnm}	wife	3\textsc{sg.f}.\textsc{a}	say\\
\glt `And now the wife said:'
\z

\ea\label{ex:text6-64}
 ``Bere'i po ma dapoha'a'',\\
\gll bere'i	po	ma	dV-poha-'a\\
     old.person	1\textsc{pl}.\textsc{in}.\textsc{a}	\textsc{mid}	\textsc{appl}-hit-?\textsc{lim}\\
\glt `{``}Old man, [let]'s (in.) slaughter them for us,'''
\z

\ea\label{ex:text6-65}
de wo temo: ``uwa la po ma duduhulo'o.''\\
\gll de	wo	temo	uwa	la	po	ma	CV{\textasciitilde}dV-hulo'o\\
     \textsc{conn}	3\textsc{sg.m}.\textsc{a}	say	\textsc{proh}	so.that	1\textsc{pl}.\textsc{in}.\textsc{a}	\textsc{mid}	\textsc{rdpl}{\textasciitilde}\textsc{appl}-send\\
\glt `and he said: ``[Let's] not [do that], so that we (in.) can send them.'''\footnote{GJE: Use them as servants.}
\z

\ea\label{ex:text6-66}
Genaade jo 'i tana ho jo amooau.\\
\gll geena-o'a-de	yo	'i	tana	ho	yo	lamo'o-o'au\\
     \textsc{dist}:\textsc{pro}.3\textsc{nh}-\textsc{locv}-\textsc{conn}	3\textsc{hpl}.\textsc{a}	3\textsc{hpl}.\textsc{u}	raise	thus	3\textsc{hpl}.\textsc{a}	big-\textsc{perf}\\
\glt `Then they brought them up until they had grown up.'
\z

\ea\label{ex:text6-67}
Ja no'ui'a o njawa o dadami moi,\\
\gll ya	nou'u-i'a	o	nyawa	o	dadami	moi\\
     3\textsc{nh}>3\textsc{nh}	smoke-\textsc{dir}.\textsc{itv}	\textsc{nm}	person	\textsc{nm}	smoking.spit	one\\
\glt `[Once] they smoked one spit of people'\footnote{GJE: Human meat.}
\z

\ea\label{ex:text6-68}
de o ode o dadami moi.\\
\gll de	o	ode	o	dadami	moi\\
     \textsc{conn}	\textsc{nm}	pig	\textsc{nm}	smoking.spit	one\\
\glt `and one spit of pork.'
\z

\ea\label{ex:text6-69}
De jo odomo de jo temo, ma ngoa'age:\\
\gll de	yo	odomo	de	yo	temo	ma	ngoa'a-ge\\
     \textsc{conn}	3\textsc{hpl}.\textsc{a}	eat	\textsc{conn}	3\textsc{hpl}.\textsc{a}	say	\textsc{rnm}	child-\textsc{dist}\\
\glt `And they ate and said, those children:'
\z

\ea\label{ex:text6-70}
 ``ete, ngomi nage o inomo ia aulie mi bohono,''\\
\gll ete	ngomi	nage	o	inomo	ya	aulu-ie	mi	bohono\\
     grandfather	\textsc{pro}.1\textsc{pl}.\textsc{ex}	\textsc{exist}.\textsc{prox}	\textsc{nm}	food	3\textsc{nh}>3\textsc{nh}	extensive-\textsc{dir}.\textsc{up}	1\textsc{pl}.\textsc{ex}.\textsc{u}	forbidden\\
\glt `{``}Grandfather, there is lots of food that is forbidden to us (ex.),'''
\z

\ea\label{ex:text6-71}
de wo temo: ``Bere'i, nage o inomo i titipo'o'a,\\
\gll de	wo	temo	bere'i	nage	o	inomo	i	CV{\textasciitilde}tipo'o-'a\\
     \textsc{conn}	3\textsc{sg.m}.\textsc{a}	say	old.person	\textsc{exist}.\textsc{prox}	\textsc{nm}	food	3\textsc{nh}.\textsc{a}	\textsc{rdpl}{\textasciitilde}short-?\textsc{locv}\\
\glt `and he said: ``Old woman, there is a little bit of food left,'
\z

\newpage
\ea\label{ex:text6-72}
no 'i gai'ino, nanga danongo o ja odo-odomo'a ma inomo i aaulie ho.''\\
\gll no	'i	gai'i-ino	nanga	danongo	'o	ya	(C)V(C)V{\textasciitilde}odomo-'a	ma	inomo	i	aulu-ie	ho\\
     2\textsc{sg}.\textsc{a}	3\textsc{hpl}.\textsc{u}	take-\textsc{dir}.\textsc{ven}	1\textsc{pl}.\textsc{in}.\textsc{poss}	grandchild	\textsc{emph}	3\textsc{nh}>3\textsc{nh}	\textsc{rdpl}{\textasciitilde}eat-?\textsc{lim}	\textsc{rnm}	food	3\textsc{nh}.\textsc{a}	extensive-\textsc{dir}.\textsc{up}	thus\\
\glt `take [it] for them, [so that] our (in.) grandchildren [may] eat the abundant food.'''
\z

\ea\label{ex:text6-73}
De jo odomo;\\
\gll de	yo	odomo\\
     \textsc{conn}	3\textsc{hpl}.\textsc{a}	eat\\
\glt `And they ate;'
\z

\ea\label{ex:text6-74}
ho i duangino de wo tagi o 'aho wo aho'o,\\
\gll ho	i	duanga-ino	de	wo	tagi	o	'aho	wo	aho'o\\
     thus	3\textsc{nh}.\textsc{a}	finish-\textsc{dir}.\textsc{ven}	\textsc{conn}	3\textsc{sg.m}.\textsc{a}	go	\textsc{nm}	dog	3\textsc{sg.m}.\textsc{a}	call\\
\glt `and after that he went [and] called the dogs'\footnote{GJE: For hunting.}
\z

\ea\label{ex:text6-75}
de wo temo: ``bere'i nanga danongo uwa no 'i tinga-tinga'a\\
\gll de	wo	temo	bere'i	nanga	danongo	uwa	no	'i	tinga{\textasciitilde}tinga-'a\\
     \textsc{conn}	3\textsc{sg.m}.\textsc{a}	say	old.person	1\textsc{pl}.\textsc{in}.\textsc{poss}	grandchild	\textsc{proh}	2\textsc{sg}.\textsc{a}	3\textsc{hpl}.\textsc{u}	\textsc{rdpl}{\textasciitilde}separate-?\textsc{lim}\\
\glt `and he said: ``Old woman, don't [let] our (in.) grandchildren alone,'
\z

\ea\label{ex:text6-76}
jo lalaagomo'au ho.''\\
\gll yo	CV{\textasciitilde}lagomo-o'au	ho\\
     3\textsc{hpl}.\textsc{a}	\textsc{rdpl}{\textasciitilde}big-\textsc{perf}	thus\\
\glt `[since] they are grown up.'''
\z

\ea\label{ex:text6-77}
De muna mo temo: ``Bere'i, ngaro'a no tagi,\\
\gll de	muna	mo	temo	bere'i	ngaro-'a	no	tagi\\
     \textsc{conn}	\textsc{pro}.3\textsc{sg.f}	3\textsc{sg.f}.\textsc{a}	say	old.person	just-\textsc{foc}	2\textsc{sg}.\textsc{a}	go\\
\glt `And she said: ``Old man, just go,'
\z

\ea\label{ex:text6-78}
o to 'i madawa ho.''\\
\gll 'o	to	'i	mada-ua	ho\\
     \textsc{emph}	1\textsc{sg}.\textsc{a}	3\textsc{hpl}.\textsc{u}	abandon-\textsc{neg}	thus\\
\glt `I won't abandon them.'''
\z

\ea\label{ex:text6-79}
De wo tagi, wo aho'o ho wo paha'a\\
\gll de	wo	tagi	wo	aho'o	ho	wo	paha-o'a\\
     \textsc{conn}	3\textsc{sg.m}.\textsc{a}	go	3\textsc{sg.m}.\textsc{a}	call	thus	3\textsc{sg.m}.\textsc{a}	leave-\textsc{lim}\\
\glt `And he went, he called\footnote{GJE: The dogs.} and was gone,'
\z

\ea\label{ex:text6-80}
de jo temo: ``Bere'i, ni mi 'ulano o bebeoto,\\
\gll de	yo	temo	bere'i	no	mi	'ula-ino	o	bebeoto\\
     \textsc{conn}	3\textsc{hpl}.\textsc{a}	say	old.person	2\textsc{sg}.\textsc{a}	1\textsc{pl}.\textsc{ex}.\textsc{u}	give-\textsc{dir}.\textsc{ven}	\textsc{nm}	knife\\
\glt `and they\footnote{GJE: The two children.} said: ``Old woman, give us (ex.) a knife,'
\z

\ea\label{ex:text6-81}
dai o a'elo'a mi ma guguule ho.''\\
\gll dai	o	a'ele-o'a	mi	ma	gugule	ho\\
     \textsc{loc}.\textsc{sea}	\textsc{nm}	water-\textsc{locv}	1\textsc{pl}.\textsc{ex}.\textsc{a}	\textsc{mid}	toy	thus\\
\glt `[because] seawards by the water we (ex.) will play.'''
\z

\ea\label{ex:text6-82}
De muna mo temo: ``Danongo, ma i'a ni ma guguule;''\\
\gll de	muna	mo	temo	danongo	ma	i'a	ni	ma	gugule\\
     \textsc{conn}	\textsc{pro}.3\textsc{sg.f}	3\textsc{sg.f}.\textsc{a}	say	grandchild	\textsc{rnm}	\textsc{dir}.\textsc{itv}	2\textsc{pl}.\textsc{a}	\textsc{mid}	toy\\
\glt `And she said: ``Grandchildren, go ahead, play,'''
\z

\ea\label{ex:text6-83}
de jo ma guguule,\\
\gll de	yo	ma	gugule\\
     \textsc{conn}	3\textsc{hpl}.\textsc{a}	\textsc{mid}	toy\\
\glt `and they played,'
\z


\ea\label{ex:text6-84}
ja i'a de jo diai o ngotili moi o aharu ja tjara djuangana.\\
\gll ya	i'a	de	yo	diai	o	ngotili	moi	o	aharu	ya	cara	juangana\\
     3\textsc{nh}>3\textsc{nh}	\textsc{dir}.\textsc{itv}	\textsc{conn}	3\textsc{hpl}.\textsc{a}	make	\textsc{nm}	proa	one	\textsc{nm}	stone	3\textsc{nh}>3\textsc{nh}	manner	boat.used.by.the.Sultan\\
\newpage
\glt `they went away and they made one proa of stone in the manner of a juangana\footnote{GJE: a boat used by the Sultan; MZ: Described by \citet{vanfraassennd} as ``large outrigger canoe with a length of up to 25 meters, a width of up to 3 meters and a strongly upwards curving stem and stern richly decorated with wood carvings. The Ternatan war ship par excellence.''}.'
\z

\ea\label{ex:text6-85}
Jo diai ma hidete, ma liaro, ma halimi, ma aoto;\\
\gll yo	diai	ma	hidete	ma	liaro	ma	halimi	ma	aoto\\
     3\textsc{hpl}.\textsc{a}	make	\textsc{rnm}	sail	\textsc{rnm}	mast	\textsc{rnm}	oar	\textsc{rnm}	plank\\
\glt `They made its sails, its masts, its oars, its planks;'
\z

\ea\label{ex:text6-86}
ena ma bere'i mo ahoou mo temo:\\
\gll ena	ma	bere'i	mo	aho'o-ou	mo	temo\\
     \textsc{pro}.3\textsc{nh}	\textsc{rnm}	old.person	3\textsc{sg.f}.\textsc{a}	call-\textsc{already}	3\textsc{sg.f}.\textsc{a}	say\\
\glt `and then called the old woman, saying:'
\z

\ea\label{ex:text6-87}
 ``Danongo nia ihau''\\
\gll danongo	nia	iha-ou\\
     grandchild	2\textsc{pl}>3\textsc{nh}	\textsc{dir}.\textsc{land}-\textsc{already}\\
\glt `{``}Grandchildren, [come] landwards;'''
\z

\ea\label{ex:text6-88}
de jo temo: ``Bere'i uwahi mia ngagootili mio diaiohi ho.''\\
\gll de	yo	temo	bere'i	uwa-ohi	mia	ngagootili	mio	diai-ohi	ho\\
     \textsc{conn}	3\textsc{hpl}.\textsc{a}	say	old.person	\textsc{proh}-\textsc{still}	1\textsc{pl}.\textsc{ex}.\textsc{poss}	small.toy.proa	1\textsc{pl}.\textsc{ex}.\textsc{a}	make-\textsc{still}	thus\\
\glt `and they said: ``Old woman, not yet, we (ex.) are still making our (ex.) proa.'''\footnote{MZ: The form <ngagootili> is probably a typo of the reduplicated form \textit{ngongotili} of \textit{ngotili} `proa'. Reduplication can express mimicry, i.e. a toy proa.}
\z

\ea\label{ex:text6-89}
De ma hidete jo higo'o de jo tagiou.\\
\gll de	ma	hidete	yo	hi-go'o	de	yo	tagi-ou\\
     \textsc{conn}	\textsc{rnm}	sail	3\textsc{hpl}.\textsc{a}	\textsc{caus}-rise.up	\textsc{conn}	3\textsc{hpl}.\textsc{a}	go-\textsc{already}\\
\glt `And they set up the sail and they went [away].'
\z

\newpage
\ea\label{ex:text6-90}
Ma dadaao ena wo boau de wo hano:\\
\gll ma	dadaao	ena	wo	boa-ou	de	wo	hano\\
     \textsc{rnm}	after	\textsc{pro}.3\textsc{nh}	3\textsc{sg.m}.\textsc{a}	come-\textsc{already}	\textsc{conn}	3\textsc{sg.m}.\textsc{a}	ask\\
\glt `Then he\footnote{GJE: The general.} came [back] and asked: '
\z

\ea\label{ex:text6-91}
 ``Ho nanga danongo?''\\
\gll ho	nanga	danongo\\
     thus	1\textsc{pl}.\textsc{in}.\textsc{poss}	grandchild\\
\glt `{``}How are our (in.) grandchildren?'''
\z

\ea\label{ex:text6-92}
De mo temo: ``'Aano dai o hohano'a jo guguule.''\\
\gll de	mo	temo	'a'ano	dai	o	hohana-o'a	yo	gugule\\
     \textsc{conn}	3\textsc{sg.f}.\textsc{a}	say	just.now	\textsc{loc}.\textsc{sea}	\textsc{nm}	landing.place-\textsc{locv}	3\textsc{hpl}.\textsc{a}	toy\\
\glt `And she said: ``Just now they are playing at sea at the landing place.'''
\z

\ea\label{ex:text6-93}
De wo doa o huburu de,\\
\gll de	wo	doa	o	huburu	de\\
     \textsc{conn}	3\textsc{sg.m}.\textsc{a}	climb	\textsc{nm}	kenari	\textsc{conn}\\
\glt `And he climbed a kenari tree,'\footnote{MZ: \textit{Kenari} is the Indonesian word for several species of the canarium genus. The \textit{huburu} is described by \citet[139]{hueting1908b} as `a kind of kenari tree with small, round fruits' (`een soort kanarieboom met kleine, ronde vrucht').}
\z

\ea\label{ex:text6-94}
ho u ma tutumu ena manga hidete,\\
\gll ho	u	ma	tutumu	ena	manga	hidete\\
     thus	3\textsc{sg.m}.\textsc{a}	\textsc{mid}	go.see	\textsc{pro}.3\textsc{nh}	3\textsc{hpl}.\textsc{poss}	sail\\
\glt `so he saw their sail'
\z

\ea\label{ex:text6-95}
i ho o duduahi ohi.\\
\gll i	ho	o	dudu-ohi ohi\\
     3\textsc{nh}.\textsc{a}	thus	\textsc{nm}	cigarette-\textsc{still} \textsc{still}\\
\glt `[as small]  as a cigarette.'\footnote{GJE: So far they had already gone.}
\z


\ea\label{ex:text6-96}
De awi hogo wo hidodowanga de wo 'i li'o,\\
\gll de	awi	hogo	wo	hi-dV-dowanga	de	wo	'i	li'o\\
     \textsc{conn}	3\textsc{sg.m}.\textsc{poss}	pubic.hair	3\textsc{sg.m}.\textsc{a}	\textsc{caus}-\textsc{appl}-extend	\textsc{conn}	3\textsc{sg.m}.\textsc{a}	3\textsc{hpl}.\textsc{u}	tie\\
\glt `And he extended his pubic hair\footnote{GJE: turned it into a long thread} to bind them\footnote{GJE: the children}
\z

\ea\label{ex:text6-97}
ma o ja adonuwa,\\
\gll ma	'o	ya	adono-ua\\
     but	\textsc{emph}	3\textsc{nh}>3\textsc{nh}	reach-\textsc{neg}\\
\glt `but it did not reach it,'\footnote{GJE: the sail of the children}
\z

\ea\label{ex:text6-98}
de wo temo: ``Bere'i, ani hogo na rari la po hiadowanga la po 'i li'o.''\\
\gll de	wo	temo	bere'i	ani	hogo	na	rari	la	po	hi-'a-dowanga	la	po	'i	li'o\\
     \textsc{conn}	3\textsc{sg.m}.\textsc{a}	say	old.person	2\textsc{sg}.\textsc{poss}	pubic.hair	2\textsc{sg}>3\textsc{nh}	cut	so.that	1\textsc{pl}.\textsc{in}.\textsc{a}	\textsc{caus}-\textsc{vpl}-extend	so.that	1\textsc{pl}.\textsc{in}.\textsc{a}	3\textsc{hpl}.\textsc{u}	tie\\
\glt `and he said: ``Old woman, pull out your (sg.) pubic hair so that we (in.) can extend [mine] with it so that we (in.) [can] bind them.'''
\z

\ea\label{ex:text6-99}
De wo hiadowanga de wo 'i dugumo.\\
\gll de	wo	hi-'a-dowanga	de	wo'i	dV-gumo\\
     \textsc{conn}	3\textsc{sg.m}.\textsc{a}	\textsc{caus}-\textsc{vpl}-extend	\textsc{conn}	3\textsc{sg.m}.\textsc{a}3\textsc{hpl}.\textsc{u}	\textsc{appl}-throw\\
\glt `And he extended [it] and threw [it] at them.'
\z

\ea\label{ex:text6-100}
De jo 'i li'o,\\
\gll de	yo	'i	li'o\\
     \textsc{conn}	3\textsc{hpl}.\textsc{a}	3\textsc{hpl}.\textsc{u}	tie\\
\glt `And they bound them,'\footnote{GJE: with the thread made of pubic hair}
\z

\newpage
\ea\label{ex:text6-101}
ma dodoao ena daingiha manga hidete 'a i hori-hori.\\
\gll ma	dadaao	ena	dai-ngiha	manga	hidete	'a	i	hori{\textasciitilde}hori\\
     \textsc{rnm}	after	\textsc{pro}.3\textsc{nh}	\textsc{loc}.\textsc{sea}-\textsc{n:dir.land}	3\textsc{hpl}.\textsc{poss}	sail	\textsc{foc}	3\textsc{nh}.\textsc{a}	\textsc{rdpl}{\textasciitilde}near\\
\glt `and following that their sail came closer and closer landwards from the sea.'
\z

\ea\label{ex:text6-102}
De ja patu-patu ma 'a i olu'u,\\
\gll de	ya	patu{\textasciitilde}patu	ma	'a	i	olu'u\\
     \textsc{conn}	3\textsc{nh}>3\textsc{nh}	\textsc{rdpl}{\textasciitilde}hoe	but	\textsc{foc}	3\textsc{nh}.\textsc{a}	refuse\\
\glt `And they\footnote{GJE: the children in the boat} hacked it\footnote{GJE: the thread} with an axe but it refused [to break],'
\z

\ea\label{ex:text6-103}
ja dia-dia ma 'a i olu'u,\\
\gll ya	dia{\textasciitilde}dia	ma	'a	i	olu'u\\
     3\textsc{nh}>3\textsc{nh}	\textsc{rdpl}{\textasciitilde}chopper	but	\textsc{foc}	3\textsc{nh}.\textsc{a}	refuse\\
\glt `they chopped it but it refused [to break]'
\z

\ea\label{ex:text6-104}
o igono ma ngo'ono jo hiiili la ato o porogi ja reno,\\
\gll o	igono	ma	ngo'ono	yo	hi-iili	la	ato	o	porogi	ya	reno\\
     \textsc{nm}	coconut	\textsc{rnm}	dregs	3\textsc{hpl}.\textsc{a}	\textsc{caus}-rub	so.that	\textsc{excl}	\textsc{nm}	mouse	3\textsc{nh}>3\textsc{nh}	gnaw\\
\glt `[and] they rubbed [it] with waste of coconuts, so that, look, mice would gnaw on it,'
\z

\ea\label{ex:text6-105}
ma 'a i olu'u,\\
\gll ma	'a	i	olu'u\\
     but	\textsc{foc}	3\textsc{nh}.\textsc{a}	refuse\\
\glt `but it refused [to break]'
\z

\ea\label{ex:text6-106}
de 'a jo 'i hori-jo 'i hori.\\
\gll de	'a	yo	'i	hori yo	'i	hori\\
     \textsc{conn}	\textsc{foc}	3\textsc{hpl}.\textsc{a}	3\textsc{hpl}.\textsc{u}	near 3\textsc{hpl}.\textsc{a} 3\textsc{hpl}.\textsc{u}	near\\
\glt `and they [came] closer and closer to them.'
\z

\ea\label{ex:text6-107}
De jo tagioli ho jo tutu'u ma 'o'ana awi hohaniha,\\
\gll de	yo	tagi-oli	ho	yo	tutu'u	ma	'oana	awi	hohana-iha\\
     \textsc{conn}	3\textsc{hpl}.\textsc{a}	go-\textsc{again}	thus	3\textsc{hpl}.\textsc{a}	arrive	\textsc{rnm}	king	3\textsc{sg.m}.\textsc{poss}	landing.place-\textsc{dir}.\textsc{land}\\
\glt `And they went again and they arrived at the landing place of the king'
\z


\ea\label{ex:text6-108}
de ma ilanga wo temo: ``no ma haana de,''\\
\gll de	ma	ilanga	wo	temo	no	ma	haana	de\\
     \textsc{conn}	\textsc{rnm}	brother	3\textsc{sg.m}.\textsc{a}	say	2\textsc{sg}.\textsc{a}	\textsc{mid}	pants	\textsc{conn}\\
\glt `and the brother said: ``Put on pants,'''
\z

\ea\label{ex:text6-109}
de mu ma haana de.\\
\gll de	mu	ma	haana	de\\
     \textsc{conn}	3\textsc{sg.f}.\textsc{a}	\textsc{mid}	pants	\textsc{conn}\\
\glt `and she put on pants.'\footnote{GJE: of these two children one seems to be a boy and the other a girl}
\z

\ea\label{ex:text6-110}
De jo utiou, de ma 'o'ana wo temo:\\
\gll de	yo	uti-ou	de	ma	'oana	wo	temo\\
     \textsc{conn}	3\textsc{hpl}.\textsc{a}	descend-\textsc{already}	\textsc{conn}	\textsc{rnm}	king	3\textsc{sg.m}.\textsc{a}	say\\
\glt `And they disembarked and the king said:'
\z

\ea\label{ex:text6-111}
 ``Nage ani bilanga to mi modoa'a,''\\
\gll nage	ani	bilanga	to	mi	modo'a-'a\\
     \textsc{exist}.\textsc{prox}	2\textsc{sg}.\textsc{poss}	sister	1\textsc{sg}.\textsc{a}	3\textsc{sg.f}.\textsc{u}	marry-?\textsc{lim}\\
\glt `{``}Your (sg.) sister here, I [want] to marry her,'''
\z

\ea\label{ex:text6-112}
de una wo temo: ``Djou lamo-lamo ma'a mi ma nau-naulo'a.''\\
\gll de	una	wo	temo	jou	lamo{\textasciitilde}lamo	ma-'a	mi	ma	(C)V(C)V{\textasciitilde}naulu-o'a\\
     \textsc{conn}	\textsc{pro}.3\textsc{sg.m}	3\textsc{sg.m}.\textsc{a}	say	lord	great	but-\textsc{foc}	1\textsc{pl}.\textsc{ex}.\textsc{a}	\textsc{mid}	\textsc{rdpl}{\textasciitilde}man-\textsc{locv}\\
\glt `and he [the brother] said: ``Great lord, but we (ex.) are only men.'''
\z

\newpage
\ea\label{ex:text6-113}
Mo ma 'o'ana a wo nga'uwa,\\
\gll ma	ma	'oana	'a	wo	nga'u-ua\\
     but	\textsc{rnm}	king	\textsc{foc}	3\textsc{sg.m}.\textsc{a}	believe-\textsc{neg}\\
\glt `But the king did not believe [it],'
\z

\ea\label{ex:text6-114}
de una ma ilanga wo temo:\\
\gll de	una	ma	ilanga	wo	temo\\
     \textsc{conn}	\textsc{pro}.3\textsc{sg.m}	\textsc{rnm}	brother	3\textsc{sg.m}.\textsc{a}	say\\
\glt `and he, the brother, said:'
\z

\ea\label{ex:text6-115}
 ``Abei'a, 'o no nga'uwa, ho u ma ohihi.''\\
\gll abei'a	'o	no	nga'u-ua	ho	u	ma	ohihi\\
     well	\textsc{emph}	2\textsc{sg}.\textsc{a}	believe-\textsc{neg}	thus	3\textsc{sg.m}.\textsc{a}	\textsc{mid}	urinate\\
\glt `{``}Well then, [if] you (sg.) don't believe [it], so [let] him\footnote{GJE: actually her} urinate.'''
\z

\ea\label{ex:text6-116}
De u ma ohihi,\\
\gll de	u	ma	ohihi\\
     \textsc{conn}	3\textsc{sg.m}.\textsc{a}	\textsc{mid}	urinate\\
\glt `And he urinated,'
\z

\ea\label{ex:text6-117}
ma ma heli de o upa-upaha\\
\gll ma	ma	heli	de	o	(C)V(C)V{\textasciitilde}upaha\\
     but	\textsc{rnm}	spout	\textsc{conn}	\textsc{nm}	\textsc{rdpl}{\textasciitilde}bamboo.joint\\
\glt `but the spout [he let] through a bamboo joint'
\z

\ea\label{ex:text6-118}
ho o doangu'u aha itubu'u.\\
\gll ho	'o	do'a-ngu'u	aha	i-tubu-u'u\\
     thus	\textsc{emph}	\textsc{locv}-\textsc{n}:\textsc{dir}.\textsc{down}	as.a.consequence	3\textsc{nh}.\textsc{a}-top-\textsc{dir.down}\\\
\glt `so it dripped down far away.'
\z

\ea\label{ex:text6-119}
De wo temo: ``i goungu!''\\
\gll de	wo	temo	i	goungu\\
     \textsc{conn}	3\textsc{sg.m}.\textsc{a}	say	3\textsc{nh}.\textsc{a}	true\\
\glt `And he\footnote{GJE: the king} said: ``It's true!'''
\z

\ea\label{ex:text6-120}
ma i togumiade ami teleme ia aunu\\
\gll ma	i	togumu-i'a-de	ami	teleme	ya	aunu\\
     but	3\textsc{nh}.\textsc{a}	finish-\textsc{dir}.\textsc{itv}-\textsc{conn}	3\textsc{sg.f}.\textsc{poss}	vagina	3\textsc{nh}>3\textsc{nh}	blood\\
\glt `but afterwards her vagina bled'
\z

\ea\label{ex:text6-121}
de ami haana ja hihi.\\
\gll de	ami	haana	ya	hihi\\
     \textsc{conn}	3\textsc{sg.f}.\textsc{poss}	pants	3\textsc{nh}>3\textsc{nh}	hurt\\
\glt `and her pants hurt it.'
\z

\ea\label{ex:text6-122}
De ma 'oana wa mae', de wo temo:\\
\gll de	ma	'oana	wa	ma'e	de	wo	temo\\
     \textsc{conn}	\textsc{rnm}	king	3\textsc{sg.m}>3\textsc{nh}	see	\textsc{conn}	3\textsc{sg.m}.\textsc{a}	say\\
\glt `And the king saw it and he said:'
\z

\ea\label{ex:text6-123}
 ``'Ouwa a o ngewe'a,''\\
\gll 'o-uwa	'a	o	ngewe'a\\
     \textsc{emph}-\textsc{proh}	\textsc{foc}	\textsc{nm}	woman\\
\glt `{``}It's not [true], [it's] a woman,'''
\z

\ea\label{ex:text6-124}
de ma holoibi ami hae'ino i boa de ami bebeoto mo hilegono,\\
\gll de	ma	holoibi	ami	hae'e-ino	i	boa	de	ami	bebeoto	mo	hi-legono\\
     \textsc{conn}	\textsc{rnm}	hummingbird	3\textsc{sg.f}.\textsc{poss}	head-\textsc{dir}.\textsc{ven}	3\textsc{nh}.\textsc{a}	come	\textsc{conn}	3\textsc{sg.f}.\textsc{poss}	knife	3\textsc{sg.f}.\textsc{a}	\textsc{caus}-catch\\
\glt `but a hummingbird came to her head here and she caught [it] with her knife,'\footnote{MZ: The species referred to as \textit{holoibi} is unclear.}
\z

\ea\label{ex:text6-125}
ho ma aunu ami giamu'u, de mo temo:\\
\gll ho	ma	aunu	ami	giama-u'u	de	mo	temo\\
     thus	\textsc{rnm}	blood	3\textsc{sg.f}.\textsc{poss}	hand-\textsc{dir}.\textsc{down}	\textsc{conn}	3\textsc{sg.f}.\textsc{a}	say\\
\glt `so blood [dripped] down on her hand, and she said:'
\z

\ea\label{ex:text6-126}
 ``beneena 'a o holaibi ma aunu,''\\
\gll be-neena	'a	o	holaibi	ma	aunu\\
     ?here-\textsc{prox}:\textsc{pro}.3\textsc{nh}	\textsc{foc}	\textsc{nm}	hummingbird	\textsc{rnm}	blood\\
\glt `{``}Here, [it] is just the blood of the hummingbird,'''
\z

\ea\label{ex:text6-127}
de wo temo: ``i goungu.''\\
\gll de	wo	temo	i	goungu\\
     \textsc{conn}	3\textsc{sg.m}.\textsc{a}	say	3\textsc{nh}.\textsc{a}	true\\
\glt `and he\footnote{GJE: the king} said: ``It's true.'''
\z

\ea\label{ex:text6-128}
De wo temo: ``i'au a nio tagiou,''\\
\gll de	wo	temo	i'a-ou	'a	nio	tagi-ou\\
     \textsc{conn}	3\textsc{sg.m}.\textsc{a}	say	\textsc{dir}.\textsc{itv}-\textsc{foc}	\textsc{foc}	2\textsc{pl}.\textsc{a}	go-\textsc{already}\\
\glt `And he said: ``It is good, go,'''
\z


\ea\label{ex:text6-129}
de jo tagioli, tara dai jo ma hiowou, mu ma he'ono.\\
\gll de	yo	tagi-oli	tara	dai	yo	ma	hi-oru-ou	mu	ma	he'ono\\
     \textsc{conn}	3\textsc{hpl}.\textsc{a}	go-\textsc{again}	?down	\textsc{loc}.\textsc{sea}	3\textsc{hpl}.\textsc{a}	\textsc{mid}	\textsc{caus}-row-\textsc{already}	3\textsc{sg.f}.\textsc{a}	\textsc{mid}	naked\\
\glt `and they went again, they rowed ?down seawards, [and] she was naked.'\footnote{MZ: \textit{Tara} may be a borrowing of the Ternate \textsc{down} directional.}
\z

\ea\label{ex:text6-130}
De wo temo: ``'a o ngewe'a ho ni mi dongoru.''\\
\gll de	wo	temo	'a	o	ngewe'a	ho	ni	mi	dV-ngoru\\
     \textsc{conn}	3\textsc{sg.m}.\textsc{a}	say	\textsc{foc}	\textsc{nm}	woman	thus	2\textsc{sg}.\textsc{a}	\textsc{3sg.f.u}	\textsc{appl}-row\\
\glt `And he said: ''?It is a woman so you (pl.) [must] row her.'''\footnote{MZ: 
Ellen translates the direct speech as `like women (row), you must row us' (``gelijk de vrouwen (roeien) moet gij ons roeien''). However, the actor index \textit{ni} can also be second person plural and in the following lines, third person plural indices occur. I therefore assume that in addition to the siblings, there are several rowers in the boat. The undergoer index \textit{mi} is either first person plural exclusive (i.e. the siblings) or third person singular feminine (i.e. the sister). An interpretation as third person singular feminine is more likely in light of the next line.}
\z

\ea\label{ex:text6-131}
De mi dongoru,\\
\gll de	mi	dV-ngoru\\
     \textsc{conn}	3\textsc{sg.f.u}	\textsc{appl}-row\\
\glt `And they rowed her,'\footnote{MZ: Ellen translated `And she rowed it' (``En zij roeide het''). This translation is not licensed by the index \textit{mi}.}
\z

\ea\label{ex:text6-132}
ja tila o 'a i lutu;\\
\gll ya	tila	'o	'a	i	lutu\\
     3\textsc{nh}>3\textsc{nh}	push	\textsc{emph}	\textsc{foc}	3\textsc{nh}.\textsc{a}	sink\\
\glt `they pushed it\footnote{GJE: the boat while rowing} [and] it sank,'\footnote{MZ: In this and the following lines, Ellen translates \textit{ya tila} as `she pushed' (```zij duwde''). However, the index combination is \textsc{3nh>3nh}. I therefore translate `they pushed it' (i.e. the rowers).}
\z

\ea\label{ex:text6-133}
i'a moioli ja tila 'o 'a i lutu;\\
\gll i'a	moi-oli	ya	tila	'o	'a	i	lutu\\
     \textsc{dir}.\textsc{itv}	one-\textsc{again}	3\textsc{nh}>3\textsc{nh}	push	\textsc{emph}	\textsc{foc}	3\textsc{nh}.\textsc{a}	sink\\
\glt `ahead, once again, they pushed it [and] it sank,'
\z

\ea\label{ex:text6-134}
i'a ja tila 'o 'a i lutu,\\
\gll i'a	ya	tila	'o	'a	i	lutu\\
     \textsc{dir}.\textsc{itv}	3\textsc{nh}>3\textsc{nh}	push	\textsc{emph}	\textsc{foc}	3\textsc{nh}.\textsc{a}	sink\\
\glt `ahead, once again, they pushed it [and] it sank,'
\z

\ea\label{ex:text6-135}
i'a ja tila 'o 'a i lutu;\\
\gll i'a	ya	tila	'o	'a	i	lutu\\
     \textsc{dir}.\textsc{itv}	3\textsc{nh}>3\textsc{nh}	push	\textsc{emph}	\textsc{foc}	3\textsc{nh}.\textsc{a}	sink\\
\glt `ahead, once again, they pushed it [and] it sank,'\footnote{MZ: This line is untranslated in the original (i.e. there are only two repetitions of the line in the translation instead of three as in the Modole text).}
\z

\ea\label{ex:text6-136}
de ma ngootili ja boto'au de 'a mo temou:\\
\gll de	ma	ngootili	ya	boto-o'au	de	'a	mo	temo-ou\\
     \textsc{conn}	\textsc{rnm}	proa	3\textsc{nh}>3\textsc{nh}	finish-\textsc{perf}	\textsc{conn}	\textsc{foc}	3\textsc{sg.f}.\textsc{a}	say-\textsc{already}\\
\glt `and [then] the boat [was] gone\footnote{GJE: completely sunken; MZ: Literally, `the boat was finished' or `they finished the boat'.} and she said:'
\z

\newpage
\ea\label{ex:text6-137}
 ``nia madaau ho pa a'unua ho.''\\
\gll nia	mada-ou	ho	pa	a'unu-ua	ho\\
     2\textsc{pl}>3\textsc{nh}	abandon-\textsc{already}	thus	1\textsc{pl}.\textsc{in}>3\textsc{nh}	can-\textsc{neg}	thus\\
\glt `?{``}You (pl.) abandon it\footnote{GJE: The boat} so we (in.) can't'''\footnote{MZ: A more natural translation may be `you abandoned it so we can't use it'. However, the boat is used in the following lines since the siblings do not directly disembark from it.}
\z

\ea\label{ex:text6-138}
De 'a jo tagioli,\\
\gll de	'a	yo	tagi-oli\\
     \textsc{conn}	\textsc{foc}	3\textsc{hpl}.\textsc{a}	go-\textsc{again}\\
\glt `And they went again,'
\z

\ea\label{ex:text6-139}
de jo tutu'u ma dea awi hohaniha,\\
\gll de	yo	tutu'u	ma	dea	awi	hohana-iha\\
     \textsc{conn}	3\textsc{hpl}.\textsc{a}	arrive	\textsc{rnm}	father	3\textsc{sg.m}.\textsc{poss}	landing.place-\textsc{dir}.\textsc{land}\\
\glt `and they came landwards to the landing place of the father,'
\z

\ea\label{ex:text6-140}
de dai awi hohano'a jo tutu'u.\\
\gll de	dai	awi	hohana-o'a	yo	tutu'u\\
     \textsc{conn}	\textsc{loc}.\textsc{sea}	3\textsc{sg.m}.\textsc{poss}	landing.place-\textsc{locv}	3\textsc{hpl}.\textsc{a}	arrive\\
\glt `and at sea at his landing place they arrived.'
\z

\ea\label{ex:text6-141}
De ma 'oana wo temo:\\
\gll de	ma	'oana	wo	temo\\
     \textsc{conn}	\textsc{rnm}	king	3\textsc{sg.m}.\textsc{a}	say\\
\glt `And the king\footnote{GJE: their father, the previously disguised cuscus} said:'
\z

\ea\label{ex:text6-142}
 ``Nagoona jo tutu'u ai hohano'a?\\
\gll naga-ona	yo	tutu'u	ai	hohana-o'a\\
     \textsc{naga}-\textsc{pro}.3\textsc{plh}	3\textsc{hpl}.\textsc{a}	arrive	1\textsc{sg}.\textsc{poss}	landing.place-\textsc{locv}\\
\glt `{``}Who is arriving at my landing place?'''
\z

\ea\label{ex:text6-143}
de wo temo ma 'oana: ``daio nio 'i ehe,''\\
\gll de	wo	temo	ma	'oana	dai-o'o	nio	'i	ehe\\
     \textsc{conn}	3\textsc{sg.m}.\textsc{a}	say	\textsc{rnm}	king	\textsc{loc}.\textsc{sea}-\textsc{dir}.\textsc{sea}	2\textsc{pl}.\textsc{a}	3\textsc{hpl}.\textsc{u}	fetch\\
\glt `and he, the king, said: ``There seawards, fetch them,'''\footnote{MZ: I am not sure about the analysis of <daio>. It may also be \textit{dai-'o} `\textsc{loc.sea-emph}.'}
\z

\ea\label{ex:text6-144}
de ja o'o jo 'i ehe jo temo:\\
\gll de	ya	o'o	yo	'i	ehe	yo	temo\\
     \textsc{conn}	3\textsc{nh}>3\textsc{nh}	\textsc{dir}.\textsc{sea}	3\textsc{hpl}.\textsc{a}	3\textsc{hpl}.\textsc{u}	fetch	3\textsc{hpl}.\textsc{a}	say\\
\glt `and they\footnote{GJE: certainly the servants of the king} went seawards [to] fetch them [and] they said:'
\z

\ea\label{ex:text6-145}
 ``Dinao'o''\\
\gll dina-o'o\\
     \textsc{loc}.\textsc{land}-\textsc{dir}.\textsc{sea}\\
\glt `{``}[We've come] from the land seawards,'''
\z

\ea\label{ex:text6-146}
ma 'oana wo temo:\\
\gll ma	'oana	wo	temo\\
     \textsc{rnm}	king	3\textsc{sg.m}.\textsc{a}	say\\
\glt `the king said:'
\z


\ea\label{ex:text6-147}
 ``i dodoa ho to ngoi ai hohaniha de jo tutu'u?''\\
\gll i	dodoa	ho	to	ngoi	ai	hohana-iha	de	yo	tutu'u\\
     3\textsc{nh}.\textsc{a}	why	thus	\textsc{poss}.\textsc{hum}	\textsc{pro}.1\textsc{sg}	1\textsc{sg}.\textsc{poss}	landing.place-\textsc{dir}.\textsc{land}	\textsc{conn}	3\textsc{hpl}.\textsc{a}	arrive\\
\glt `{``}Why do they arrive at my landing place landwards?'''
\z

\ea\label{ex:text6-148}
De ona jo temo: ``na'o o geena de 'a una wa o'o,''\\
\gll de	ona	yo	temo	na'o	'o geena	de	'a	una	wa	o'o\\
     \textsc{conn}	\textsc{pro}.3\textsc{plh}	3\textsc{hpl}.\textsc{a}	say	\textsc{cond}	\textsc{emph} \textsc{dist}:\textsc{pro}.3\textsc{nh}	\textsc{conn}	\textsc{foc}	\textsc{pro}.3\textsc{sg.m}	3\textsc{sg.m}>3\textsc{nh}	\textsc{dir}.\textsc{sea}\\
\glt `And they said: ``If that's the case, ?[let] him come seawards,'''
\z

\ea\label{ex:text6-149}
de wao'o wo hano wo temo:\\
\gll de	wa-o'o	wo	hano	wo	temo\\
     \textsc{conn}	3\textsc{sg.m}>3\textsc{nh}-\textsc{dir}.\textsc{sea}	3\textsc{sg.m}.\textsc{a}	ask	3\textsc{sg.m}.\textsc{a}	say\\
\glt `and he came seawards [and] asked, saying:'
\z

\newpage
\ea\label{ex:text6-150}
 ``O'iano ngini ma njawa?''\\
\gll o'ia-ino	ngini	ma	nyawa\\
     what-\textsc{dir}.\textsc{ven}	\textsc{pro}.2\textsc{pl}	\textsc{rnm}	person\\
\glt `{``}Where have you (pl.) [come from] here?'''
\z

\ea\label{ex:text6-151}
De ona jo temo:\\
\gll de	ona	yo	temo\\
     \textsc{conn}	\textsc{pro}.3\textsc{plh}	3\textsc{hpl}.\textsc{a}	say\\
\glt `And they said:'
\z

\ea\label{ex:text6-152}
 ``Ngomi neena ma 'oana o 'uho awi ngoa'a.\\
\gll ngomi	neena	ma	'oana	o	'uho	awi	ngoa'a\\
     \textsc{pro}.1\textsc{pl}.\textsc{ex}	\textsc{prox}:\textsc{pro}.3\textsc{nh}	\textsc{rnm}	king	\textsc{nm}	cuscus	3\textsc{sg.m}.\textsc{poss}	child\\
\glt `{``}We (ex.) here are the children of King Cuscus.'
\z

\ea\label{ex:text6-153}
Ma mudua awi ruae de wo tagi,\\
\gll ma	mudua	awi	ruae	de	wo	tagi\\
     \textsc{rnm}	?	3\textsc{sg.m}.\textsc{poss}	?troubled	\textsc{conn}	3\textsc{sg.m}.\textsc{a}	go\\
\glt `He went with ?a burdened heart'\footnote{MZ: The form <mudua> is completely obscure to me. Based on the translation, it could mean `heart' but no potential cognates are attested. Also note that the parallel line [F.\ref{ex:text6-35}] does not contain it.}
\z

\ea\label{ex:text6-154}
de wo djadji mia eha'a wo temo:\\
\gll de	wo	jaji	mia	eha-i'a	wo	temo\\
     \textsc{conn}	3\textsc{sg.m}.\textsc{a}	promise	1\textsc{pl}.\textsc{ex}.\textsc{poss}	mother-\textsc{dir}.\textsc{itv}	3\textsc{sg.m}.\textsc{a}	say\\
\glt `and he promised our (ex.) mother, saying:'
\z

\ea\label{ex:text6-155}
 `Ngona neena no ngoa'a, de ani ngoa'a modidi,\\
\gll ngona	neena	no	ngoa'a	de	ani	ngoa'a	modidi\\
     \textsc{pro}.2\textsc{sg}	\textsc{prox}:\textsc{pro}.3\textsc{nh}	2\textsc{sg}.\textsc{a}	child	\textsc{conn}	2\textsc{sg}.\textsc{poss}	child	two\\
\glt `{`}You (sg.) here [shall] give birth, and your (sg.) children [will be] two,'
\z

\ea\label{ex:text6-156}
moi de ma mede ma giauo'a, de moi de ma wange ma njonjie'a.'\\
\gll moi	de	ma	mede	ma	giau-o'a	de	moi	de	ma	wange	ma	nyonyie-o'a\\
     one	\textsc{conn}	\textsc{rnm}	moon	\textsc{rnm}	new-\textsc{locv}	\textsc{conn}	one	\textsc{conn}	\textsc{rnm}	sun	\textsc{rnm}	sunrise-\textsc{locv}\\
\glt `and one [will be born] at new moon and one at sunrise.'{'}
\z

\ea\label{ex:text6-157}
De o ngo jai mo ngoa'a, de ami la'o ja diti'o,\\
\gll de	o	ngo	yai	mo	ngoa'a	de	ami	la'o	ya	dV-tio'a\\
     \textsc{conn}	\textsc{nm}	\textsc{hon}.\textsc{fem}	mother	3\textsc{sg.f}.\textsc{a}	child	\textsc{conn}	3\textsc{sg.f}.\textsc{poss}	eye	3\textsc{nh}>3\textsc{nh}	\textsc{appl}-?seal\\
\glt `And mother gave birth, and they sealed her eyes'
\z

\ea\label{ex:text6-158}
de ami oleha o u'u ma bitino jo hongodaika de o puniti,\\
\gll de	ami	ole-iha	'o	u'u	ma	bitino	yo	ho-ngoda-ika	de	o	puniti\\
     \textsc{conn}	3\textsc{sg.f}.\textsc{poss}	?vagina-\textsc{dir}.\textsc{land}	\textsc{emph}	\textsc{dir}.\textsc{down}	\textsc{rnm}	burned.object	3\textsc{hpl}.\textsc{a}	?-?put-\textsc{dir}.\textsc{itv}	\textsc{conn}	\textsc{nm}	coconut.husk\\
\glt `and in her vagina they put charcoal and coconut bark'\footnote{MZ: The form <hongodaika> is tentatively analyzed here. The <k> suggests that it is borrowed since Modole has no native /k/. Tabaru has a form \textit{songoda} `fill, put inside.'}
\z
\largerpage
\ea\label{ex:text6-159}
i togumi'a de ma ngo jai o lia'a o boruau mi hihipu'u de mi hidaahini.\\
\gll i	togumu-i'a	de	ma	ngo	yai	o	lia'a	o	borua-ou	mi	hi-hipu'u	de	mi	hi-dahini\\
     3\textsc{nh}.\textsc{a}	finish-\textsc{dir}.\textsc{itv}	\textsc{conn}	\textsc{rnm}	\textsc{hon}.\textsc{fem}	mother	\textsc{nm}	older.sibling	\textsc{nm}	box-\textsc{foc}	1\textsc{pl}.\textsc{ex}.\textsc{u}	\textsc{caus}-put.into	\textsc{conn}	1\textsc{pl}.\textsc{ex}.\textsc{u}	\textsc{caus}-float\\
\glt `and after that the older mothers\footnote{GJE: the older sisters of the mother who are here usually called `older mothers'} put us (ex.) into a box and [let] us (ex.) float away.'
\z


\ea\label{ex:text6-160}
De mio tutu'u ma Djindaali awi hohaniha\\
\gll de	mio	tutu'u	ma	jindaali	awi	hohana-iha\\
     \textsc{conn}	1\textsc{pl}.\textsc{ex}.\textsc{a}	arrive	\textsc{rnm}	general	3\textsc{sg.m}.\textsc{poss}	landing.place-\textsc{dir}.\textsc{land}\\
\glt `And we (ex.) arrived landwards at the landing place of the general'
\z

\ea\label{ex:text6-161}
de wi paliara hiadono mio lamo'o'au.\\
\gll de	wi	paliara	hiadono	mio	lamo'o-o'au\\
     \textsc{conn}	3\textsc{sg.m}>1\textsc{pl}.\textsc{excl}	raise	until	1\textsc{pl}.\textsc{ex}.\textsc{a}	big-\textsc{perf}\\
\glt `and he raised us (ex.) until we had grown up.'\footnote{MZ: The form \textit{hiadono} is derived from \textit{hi-adono} `\textsc{caus}-reach.'}
\z

\newpage
\ea\label{ex:text6-162}
De o 'aho wo aho'o'a,\\
\gll de	o	'aho	wo	aho'o-o'a\\
     \textsc{conn}	\textsc{nm}	dog	3\textsc{sg.m}.\textsc{a}	call-\textsc{lim}\\
\glt `And he\footnote{GJE: the general} called the dogs'\footnote{GJE: went hunting}
\z

\ea\label{ex:text6-163}
de mio tagi mia ngotiliau aharu mia diai.''\\
\gll de	mio	tagi	mia	ngotili-ou	aharu	mia	diai\\
     \textsc{conn}	1\textsc{pl}.\textsc{ex}.\textsc{a}	go	1\textsc{pl}.\textsc{ex}.\textsc{poss}	proa-\textsc{foc}	stone	1\textsc{pl}.\textsc{ex}>3\textsc{nh}	make\\
\glt `and we (ex.) went, we (ex.) made our (ex.) proa of stone.'''
\z

\ea\label{ex:text6-164}
De ma 'oana wo temo:\\
\gll de	ma	'oana	wo	temo\\
     \textsc{conn}	\textsc{rnm}	king	3\textsc{sg.m}.\textsc{a}	say\\
\glt `And the king said:'
\z

\ea\label{ex:text6-165}
 ``Na'o o geena de ngoiou nia dea,\\
\gll na'o	'o	geena	de	ngoi-ou	nia	dea\\
     \textsc{cond}	\textsc{emph}	\textsc{dist}:\textsc{pro}.3\textsc{nh}	\textsc{conn}	\textsc{pro}.1\textsc{sg}-\textsc{foc}	2\textsc{pl}.\textsc{poss}	father\\
\glt `{``}If that's the case, I am your (pl.) father,'
\z

\ea\label{ex:text6-166}
ho nio utihau,''\\
\gll ho	nio	uti-iha-ou\\
     thus	2\textsc{pl}.\textsc{a}	descend-\textsc{dir}.\textsc{land}-\textsc{already}\\
\glt `so disembark landwards,'''
\z


\ea\label{ex:text6-167}
de jo hano jo temo: ``Ho ma ngo jai?''\\
\gll de	yo	hano	yo	temo	ho	ma	ngo	yai\\
     \textsc{conn}	3\textsc{hpl}.\textsc{a}	ask	3\textsc{hpl}.\textsc{a}	say	thus	\textsc{rnm}	\textsc{hon}.\textsc{fem}	mother\\
\glt `and they asked, saying: ``How about mother?'''
\z

\ea\label{ex:text6-168}
De una wo temo:\\
\gll de	una	wo	temo\\
     \textsc{conn}	\textsc{pro}.3\textsc{sg.m}	3\textsc{sg.m}.\textsc{a}	say\\
\glt `And he said:'
\z

\ea\label{ex:text6-169}
 ``Nia eha muna dai mu o tetengo'a mi hidupuhuie.''\\
\gll nia	eha	muna	dai	mu	'o	CV{\textasciitilde}tengo-o'a	mi	hi-dupuhu-ie\\
     2\textsc{pl}.\textsc{poss}	mother	\textsc{pro}.3\textsc{sg.f}	\textsc{loc}.\textsc{sea}	3\textsc{sg.f}.\textsc{a}	\textsc{emph}	\textsc{rdpl}{\textasciitilde}alone-\textsc{lim}	3\textsc{sg.f}>1\textsc{pl}.\textsc{ex}	\textsc{caus}-follow-\textsc{dir}.\textsc{up}\\
\glt `{``}Your (pl.) mother is by the sea, she is alone, [and] she will follow us (ex.) up.'''\footnote{GJE: the residence of a king is usually on a hill by the sea}
\z

\ea\label{ex:text6-170}
De ona jo temo: ``Na'o mio uti,\\
\gll de	ona	yo	temo	na'o	mio	uti\\
     \textsc{conn}	\textsc{pro}.3\textsc{plh}	3\textsc{hpl}.\textsc{a}	say	\textsc{cond}	1\textsc{pl}.\textsc{ex}.\textsc{a}	descend\\
\glt `And they said: ``When we (ex.) disembark,'
\z

\ea\label{ex:text6-171}
halingou ani we'ata nage ja butanga mia ngootili'a mio hilangi.''\\
\gll halingou	ani	we'ata	nage	ya	butanga	mia	ngootili-'a	mio	hi-langi\\
     necessary	2\textsc{sg}.\textsc{poss}	wife	\textsc{exist}.\textsc{prox}	3\textsc{nh}>3\textsc{nh}	six	1\textsc{pl}.\textsc{ex}.\textsc{poss}	proa-?\textsc{locv}	1\textsc{pl}.\textsc{ex}.\textsc{a}	\textsc{caus}-roller\\
\glt `your (sg.) six wives must serve as rollers [for] our (ex.) proa [when] we (ex.) drag it up.'''\footnote{MZ: A more literal translation of this line may be `We must roller up our proa with your six wives' with the causative prefix \textit{hi-} in \textit{hilangi} marking \textit{ani we'ata nage ya butanga} as an instrument.}
\z

\ea\label{ex:text6-172}
De ma 'o'ana wo temo: ``I'au,''\\
\gll de	ma	'oana	wo	temo	i'a-ou\\
     \textsc{conn}	\textsc{rnm}	king	3\textsc{sg.m}.\textsc{a}	say	\textsc{dir}.\textsc{itv}-\textsc{foc}\\
\glt `And the king said: ``That's good,'''
\z

\ea\label{ex:text6-173}
de wo hilangi ho manga 'obongo i tobi-tobi'e'a.\\
\gll de	wo	hi-langi	ho	manga	'obongo	i	(C)V(C)V{\textasciitilde}tobi'i-'a\\
     \textsc{conn}	3\textsc{sg.m}.\textsc{a}	\textsc{caus}-roller	thus	3\textsc{hpl}.\textsc{poss}	bone	3\textsc{nh}.\textsc{a}	\textsc{rdpl}{\textasciitilde}break-?\textsc{lim}\\
\glt `and he used [them as rollers], so their bones broke.'\footnote{MZ: Ellen translates the second part of this line as `dus hunne beenderen braken en de uiteinden gingen naar boven staan' (`thus their bones broke and their ends went upwards'). The latter part of this translation does not correspond to the Modole original, as far as I can tell.}
\z

\ea\label{ex:text6-174}
I togumi'a de ma eha mï ehe,\\
\gll i	togumu-i'a	de	ma	eha	mï	ehe\\
     3\textsc{nh}.\textsc{a}	finish-\textsc{dir}.\textsc{itv}	\textsc{conn}	\textsc{rnm}	mother	3\textsc{sg.f}.\textsc{u}	fetch\\
\glt `After that, they fetched the mother'
\z

\ea\label{ex:text6-175}
de o puniti mï iili,\\
\gll de	o	puniti	mi	iili\\
     \textsc{conn}	\textsc{nm}	coconut.husk	3\textsc{sg.f}.\textsc{u}	rub\\
\glt `and they rubbed her with coconut bark,'
\z

\ea\label{ex:text6-176}
ma duangino o tero ma a'elu'u mï hiohi'i,\\
\gll ma	duangaino	'o	tero	ma	a'ele-u'u	mi	hi-ohi'i\\
     \textsc{rnm}	finish.\textsc{dir}.\textsc{ven}	\textsc{emph}	beautiful	\textsc{rnm}	water-\textsc{dir}.\textsc{down}	3\textsc{sg.f}.\textsc{u}	\textsc{caus}-wash\\
\glt `after that, they washed her beautiful in water,'
\z

\ea\label{ex:text6-177}
aha de hala'a ma a'elu'u mï hiohi'i, aha de o gurahi ma a'elu'uli.\\
\gll aha	de	hala'a	ma	a'ele-u'u	mi	hi-ohi'i	aha	de	o	gurahi	ma	a'ele-u'u-oli\\
     as.a.consequence	\textsc{conn}	silver	\textsc{rnm}	water-\textsc{dir}.\textsc{down}	3\textsc{sg.f}.\textsc{u}	\textsc{caus}-wash	as.a.consequence	\textsc{conn}	\textsc{nm}	gold	\textsc{rnm}	water-\textsc{dir}.\textsc{down}-\textsc{again}\\
\glt `[and] then they washed her with silver water and and after that with gold water.'
\z

\ea\label{ex:text6-178}
Ho ami utu ma hongona o hala'a, ma hongona o gurahi,\\
\gll ho	ami	utu	ma	hongona	o	hala'a	ma	hongona	o	gurahi\\
     thus	3\textsc{sg.f}.\textsc{poss}	hair	\textsc{rnm}	half	\textsc{nm}	silver	\textsc{rnm}	half	\textsc{nm}	gold\\
\glt `So half of her hair was silver [and] the [other] half gold,'
\z

\ea\label{ex:text6-179}
ami ilingi ma hongona o hala'a, ma hongona o gurahi.\\
\gll ami	ilingi	ma	hongona	o	hala'a	ma	hongona	o	gurahi\\
     3\textsc{sg.f}.\textsc{poss}	tooth	\textsc{rnm}	half	\textsc{nm}	silver	\textsc{rnm}	half	\textsc{nm}	gold\\
\glt `her teeth were half silver, [and] the [other] half was gold.'
\z
