\ea\label{ex:text3-1}
O Goda\\
\gll o	goda\\
     \textsc{nm}	k.o.spirit\\
\glt `The \textit{goda} spirit'\footnote{GJE: who lives in large stones and is feared; MZ: The \textit{goda} spirit is also known among other Halmaheran ethnic groups. It is said to live in stones close to the water (see \cite[279]{hueting1921}).}
\z

\ea\label{ex:text3-2}
Naga o njawa moi o modo'a.\\
\gll naga	o	nyawa	moi	o	modo'a\\
     \textsc{exist}	\textsc{nm}	person	one	\textsc{nm}	marry\\
\glt `There was someone, a married woman.'\footnote{MZ: \textit{Mod'a} and its cognates in other {\cnhl} seems to be a general address for a married woman (compare \cite[259]{vanbaarda1895}, \cite[248]{hueting1908}). Ellen translates it as `daughter-in-law' (``schoondochter'').}
\z

\ea\label{ex:text3-3}
Wo hungi-hungi,\\
\gll wo	hungi{\textasciitilde}hungi\\
     3\textsc{sg.m}.\textsc{a}	search.for.game.with.dogs\\
\glt `He\footnote{her husband} was a hunter\footnote{GJE: a game seeker},'
\z

\ea\label{ex:text3-4}
wo tagi, o 'aho wo aho'o\\
\gll wo	tagi	o	'aho	wo	aho'o\\
     3\textsc{sg.m}.\textsc{a}	go	\textsc{nm}	dog	3\textsc{sg.m}.\textsc{a}	call\\
\glt `he went and called the dogs,'
\z

\ea\label{ex:text3-5}
de wo djadji ma eha'a wo temo:\\
\gll de	wo	jaji	ma	eha-i'a	wo	temo\\
     \textsc{conn}	3\textsc{sg.m}.\textsc{a}	promise	\textsc{rnm}	mother-\textsc{dir}.\textsc{itv}	3\textsc{sg.m}.\textsc{a}	say\\
\glt `and he urged his mother, saying:'
\z

\ea\label{ex:text3-6}
``Jai! ani dunungu uwa mu ma tinga-tinga'a,\\
\gll yai ani	dunungu	uwa	mu	ma	tinga{\textasciitilde}tinga-'a\\
     mother 2\textsc{sg}.\textsc{poss}	in-law	\textsc{proh}	3\textsc{sg.f}.\textsc{a}	\textsc{mid}	\textsc{rdpl}{\textasciitilde}separate-?\textsc{lim}\\
\glt `{``}Mother! Don't [let] your (sg.) daughter-in-law stay on her own,'
\z

\newpage
\ea\label{ex:text3-7}
'one o goda mi haiti ho.''\\
\gll 'o-ne	o	goda	mi	haiti	ho\\
     \textsc{emph}-\textsc{prox}	\textsc{nm}	k.o.spirit	3\textsc{sg.f}.\textsc{u}	seize.by.spirit	thus\\
\glt `because then the \textit{goda} spirit will seize her.'''\footnote{GJE: will stick to her}
\z

\ea\label{ex:text3-8}
De mo temo: ``o uwa, o to mi higumadawa, ngaro'a no tagi.''\\
\gll de	mo	temo	'o	uwa	'o	to	mi	hi-gu-mada-ua	ngaro-'a	no	tagi\\
     \textsc{conn}	3\textsc{sg.f}.\textsc{a}	say	\textsc{emph}	\textsc{proh}	\textsc{emph}	1\textsc{sg}.\textsc{a}	3\textsc{sg.f}.\textsc{u}	\textsc{caus}-?-abandon-\textsc{neg}	just-\textsc{foc}	2\textsc{sg}.\textsc{a}	go\\
\glt `And she said: ``No, I won't leave her alone, just go.'''
\z

\ea\label{ex:text3-9}
De wo tagi, ho wo paha'a.\\
\gll de	wo	tagi	ho	wo	paha-o'a\\
     \textsc{conn}	3\textsc{sg.m}.\textsc{a}	go	thus	3\textsc{sg.m}.\textsc{a}	leave-\textsc{lim}\\
\glt `And he went, and was gone.'
\z

\ea\label{ex:text3-10}
De muna ma we'ata mo temo:\\
\gll de	muna	ma	we'ata	mo	temo\\
     \textsc{conn}	\textsc{pro}.3\textsc{sg.f}	\textsc{rnm}	wife	3\textsc{sg.f}.\textsc{a}	say\\
\glt `And she, the woman, said:'
\z

\ea\label{ex:text3-11}
``Dodo, po ma ohi-ohi'i,\\
\gll dodo	po	ma	(C)V(C)V{\textasciitilde}ohi'i\\
     younger.sibling	1\textsc{pl}.\textsc{in}.\textsc{a}	\textsc{mid}	\textsc{rdpl}{\textasciitilde}bathe\\
\glt `{``}Younger sister, [let]'s (in.) bathe'\footnote{MZ: Ellen translates \textit{dodo} as `younger sister' (``jongere zuster''). It may be a loan of the Ternate or Galela cognate of Modole \textit{dodoto} `younger sibling' (both languages show apocope of the final syllable). The clause suggests that it can refer to a younger relative in general.}
\z

\ea\label{ex:text3-12}
la nanga utu pa gowo.''\\
\gll la	nanga	utu	pa	gowo\\
     so.that	1\textsc{pl}.\textsc{in}.\textsc{poss}	hair	1\textsc{pl}.\textsc{in}>3\textsc{nh}	wash.hair.with.coconut.milk\\
\glt `and wash our (in.) hair with coconut milk.'''
\z

\newpage
\ea\label{ex:text3-13}
De ma dunungu mo temo:\\
\gll de	ma	dunungu	mo	temo\\
     \textsc{conn}	\textsc{rnm}	in-law	3\textsc{sg.f}.\textsc{a}	say\\
\glt `And the daughter-in-law said:'
\z

\ea\label{ex:text3-14}
``ma i'a, po ma ohi-ohi'i,\\
\gll ma	i'a	po	ma	(C)V(C)V{\textasciitilde}ohi'i\\
     \textsc{rnm}	\textsc{dir}.\textsc{itv}	1\textsc{pl}.\textsc{in}.\textsc{a}	\textsc{mid}	\textsc{rdpl}{\textasciitilde}bathe\\
\glt `{``}That's good, [let]'s (in.) bathe,'
\z

\ea\label{ex:text3-15}
la nanga utu pa gowo.''\\
\gll la	nanga	utu	pa	gowo\\
     so.that	1\textsc{pl}.\textsc{in}.\textsc{poss}	hair	1\textsc{pl}.\textsc{in}>3\textsc{nh}	wash.hair.with.coconut.milk\\
\glt `and wash our (in.) hair with coconut milk.'''
\z

\ea\label{ex:text3-16}
De jo ma ohi-ohi'i ho jo ma gowo\\
\gll de	yo	ma	(C)V(C)V{\textasciitilde}ohi'i	ho	yo	ma	gowo\\
     \textsc{conn}	3\textsc{hpl}.\textsc{a}	\textsc{mid}	\textsc{rdpl}{\textasciitilde}bathe	thus	3\textsc{hpl}.\textsc{a}	\textsc{mid}	wash.hair.with.coconut.milk\\
\glt `And they bathed and washed their hair'
\z

\ea\label{ex:text3-17}
ma duangino, de jo palenou.\\
\gll ma	duanga-ino	de	yo	palene-ou\\
     \textsc{rnm}	finish-\textsc{dir}.\textsc{ven}	\textsc{conn}	3\textsc{hpl}.\textsc{a}	climb-\textsc{already}\\
\glt `[and] after that, they embarked.'\footnote{GJE: perhaps in a proa in order to return}
\z

\ea\label{ex:text3-18}
De ma dunungu mo hiduga, mo temo:\\
\gll de	ma	dunungu	mo	hi-duga	mo	temo\\
     \textsc{conn}	\textsc{rnm}	in-law	3\textsc{sg.f}.\textsc{a}	\textsc{caus}-deliberate	3\textsc{sg.f}.\textsc{a}	say\\
\glt `And the daughter-in-law thought [and] said:'
\z

\ea\label{ex:text3-19}
``modo'a po ma idu-idu,''\\
\gll modo'a	po	ma	idu{\textasciitilde}idu\\
     marry	1\textsc{pl}.\textsc{in}.\textsc{a}	\textsc{mid}	\textsc{rdpl}{\textasciitilde}sleep\\
\glt `{``}Mother-in-law, [let]'s (in.) sleep,'''
\z

\ea\label{ex:text3-20}
de mo temo: ``ma i'a.''\\
\gll de	mo	temo	ma	i'a\\
     \textsc{conn}	3\textsc{sg.f}.\textsc{a}	say	\textsc{rnm}	\textsc{dir}.\textsc{itv}\\
\glt `and she said: ``That's good.'''
\z

\ea\label{ex:text3-21}
De jo ma idu-idu, ho ma dunungu mo bolutuo'a.\\
\gll de	yo	ma	idu{\textasciitilde}idu	ho	ma	dunungu	mo	bolutu-o'a\\
     \textsc{conn}	3\textsc{hpl}.\textsc{a}	\textsc{mid}	\textsc{rdpl}{\textasciitilde}sleep	thus	\textsc{rnm}	in-law	3\textsc{sg.f}.\textsc{a}	fast.asleep-\textsc{lim}\\
\glt `And they lay down, so the mother-in-law fell asleep.'
\z

\ea\label{ex:text3-22}
Una ma 'aho wo aho-aho'o ma we'ata o tiba mo tobi'i,\\
\gll una	ma	'aho	wo	(C)V(C)V{\textasciitilde}aho'o	ma	we'ata	o	tiba	mo	tobi'i\\
     \textsc{pro}.3\textsc{sg.m}	\textsc{rnm}	dog	3\textsc{sg.m}.\textsc{a}	\textsc{rdpl}{\textasciitilde}call	\textsc{rnm}	wife	\textsc{nm}	bamboo	3\textsc{sg.f}.\textsc{a}	break\\
\glt `The wife of the dog-caller\footnote{GJE: hunter} cut bamboo,'
\z

\ea\label{ex:text3-23}
de mo dohodoa ma tiba ma godau ma tataruu'u,\\
\gll de	mo	dV-ho-doa	ma	tiba	ma	goda-ou	ma	tataru-u'u\\
     \textsc{conn}	3\textsc{sg.f}.\textsc{a}	\textsc{appl}-?-put	\textsc{rnm}	bamboo	\textsc{rnm}	k.o.spirit-\textsc{foc}	\textsc{rnm}	?shade-\textsc{dir}.\textsc{down}\\
\glt `?she put the bamboo close to the \textit{goda} spirit'\footnote{MZ: Ellen translates this line as `and she chopped the bamboo close to the spirit' (``en zij hakte de bamboe af naar den geest toe''). The verb form \textit{dohodoa} surely contains \textit{doa} `put'. The word form \textit{tataruu'u} could be related to Galela \textit{taru} `shade, shelter' and Tabaru \textit{tarusu} `darkness' and may refer to the dark place where the \textit{goda} spirit is lurking. I therefore tentatively gloss it `shade'.}
\z

\ea\label{ex:text3-24}
de mi haiti.\\
\gll de	mi	haiti\\
     \textsc{conn}	3\textsc{sg.f}.\textsc{u}	seize.by.spirit\\
\glt `and [the spirit] seized her.'
\z

\ea\label{ex:text3-25}
Ena ma 'o'ata wo boaau\\
\gll ena	ma	'o'ata	wo	boa-ou\\
     \textsc{pro}.3\textsc{nh}	\textsc{rnm}	husband	3\textsc{sg.m}.\textsc{a}	come-\textsc{already}\\
\glt `In this moment, the husband came'
\z

\ea\label{ex:text3-26}
de wo hano ma eha'a wo temo:\\
\gll de	wo	hano	ma	eha-i'a	wo	temo\\
     \textsc{conn}	3\textsc{sg.m}.\textsc{a}	ask	\textsc{rnm}	mother-\textsc{dir}.\textsc{itv}	3\textsc{sg.m}.\textsc{a}	say\\
\glt `and he asked his mother, saying:'
\z

\ea\label{ex:text3-27}
``Jai ho ani dunungu?''\\
\gll yai	ho	ani	dunungu\\
     mother	thus	2\textsc{sg}.\textsc{poss}	in-law\\
\glt `{``}Mother, how is your (sg.) daughter-in-law?'''
\z

\ea\label{ex:text3-28}
de mo temo: ``'ano to nihu'uo'a\\
\gll de	mo	temo	'a'ano	to	nihu'u-o'a\\
     \textsc{conn}	3\textsc{sg.f}.\textsc{a}	say	just.now	1\textsc{sg}.\textsc{a}	sleep-\textsc{lim}\\
\glt `And she said: ``Just now I was sleeping'
\z

\ea\label{ex:text3-29}
de o tiba mo tobi'i de o goda mi haitiau.''\\
\gll de	o	tiba	mo	tobi'i	de	o	goda	mi	haiti-ou\\
     \textsc{conn}	\textsc{nm}	bamboo	3\textsc{sg.f}.\textsc{a}	break	\textsc{conn}	\textsc{nm}	k.o.spirit	3\textsc{sg.f}.\textsc{u}	seize.by.spirit-\textsc{already}\\
\glt `and she chopped bamboo and the \textit{goda} spirit seized her.'''
\z

\ea\label{ex:text3-30}
De una wo temo:\\
\gll de	una	wo	temo\\
     \textsc{conn}	\textsc{pro}.3\textsc{sg.m}	3\textsc{sg.m}.\textsc{a}	say\\
\glt `And he said:'
\z

\ea\label{ex:text3-31}
``'ano to temo, ua no mi higumada,\\
\gll 'a'ano	to	temo	uwa	no	mi	hi-gu-mada\\
     just.now	1\textsc{sg}.\textsc{a}	say	\textsc{proh}	2\textsc{sg}.\textsc{a}	3\textsc{sg.f}.\textsc{u}	\textsc{caus}-?-abandon\\
\glt `{``}I just said, don't let her alone,'''
\z

\ea\label{ex:text3-32}
de wa i'a mi tutumu ena ami pupunguau ja ie mi haiti.\\
\gll de	wa	i'a	mi	tutumu	ena	ami	pupungu-ou	ya	ie	mi	haiti\\
     \textsc{conn}	3\textsc{sg.m}>3\textsc{nh}	\textsc{dir}.\textsc{itv}	3\textsc{sg.m}>3\textsc{sg.f}	go.see	\textsc{pro}.3\textsc{nh}	3\textsc{sg.f}.\textsc{poss}	ankle-\textsc{foc}	3\textsc{nh}>3\textsc{nh}	\textsc{dir}.\textsc{up}	3\textsc{sg.f}.\textsc{u}	seize.by.spirit\\
\glt `and he went away to look for her [and] ?above her ankle [the spirit] had seized her.'
\z

\ea\label{ex:text3-33}
De wa ehe ami hala'a, de ami tjutju'ode de mi hinoa;\\
\gll de	wa	ehe	ami	hala'a	de	ami	cucu'ode	de	mi	hi-noa\\
     \textsc{conn}	3\textsc{sg.m}>3\textsc{nh}	fetch	3\textsc{sg.f}.\textsc{poss}	silver	\textsc{conn}	3\textsc{sg.f}.\textsc{poss}	hairpin	\textsc{conn}	3\textsc{sg.m}>3\textsc{sg.f}	\textsc{caus}-put\\
\glt `And he took her silver\footnote{GJE: silver bracelets} and her hairpins and put [them] on her'
\z

\ea\label{ex:text3-34}
ho ato! mi tutumu ena mi lutuoau.\\
\gll ho	ato mi	tutumu	ena	mi	lutu-o'au\\
     thus	\textsc{excl} 3\textsc{sg.m}>3\textsc{sg.f}	go.see	\textsc{pro}.3\textsc{nh}	3\textsc{sg.f}.\textsc{u}	sink-\textsc{perf}\\
\glt `but look! He looked at her and [the spirit] had already drowned her.'\footnote{MZ: Ellen translates \textit{lutuoau} as `seized completely' (``geheel gepakt'') but \textit{lutu} means `sink' and `drown' in several {\cnhl}. Since the \textit{goda} spirit lives in stones in a body of water this probably refers to the act of the spirit drowning the woman.}
\z

\ea\label{ex:text3-35}
De a wo liou awi woa'a\\
\gll de	'a	wo	lio-ou	awi	wo'a-i'a\\
     \textsc{conn}	\textsc{foc}	3\textsc{sg.m}.\textsc{a}	return-\textsc{already}	3\textsc{sg.m}.\textsc{poss}	house-\textsc{dir}.\textsc{itv}\\
\glt `And he returned to his house'
\z

\ea\label{ex:text3-36}
de i'a de wo temo: ``a to mi tuulu'',\\
\gll de	i'a	de	wo	temo	'a	to	mi	tuulu\\
     \textsc{conn}	\textsc{dir}.\textsc{itv}	\textsc{conn}	3\textsc{sg.m}.\textsc{a}	say	\textsc{foc}	1\textsc{sg}.\textsc{a}	3\textsc{sg.f}.\textsc{u}	follow\\
\glt `and afterwards he said: ``I will follow her,'''
\z

\ea\label{ex:text3-37}
de wo tagi u ma lio.\\
\gll de	wo	tagi	u	ma	lio\\
     \textsc{conn}	3\textsc{sg.m}.\textsc{a}	go	3\textsc{sg.m}.\textsc{a}	\textsc{mid}	return\\
\glt `and he went, returning [to her].'
\z

\ea\label{ex:text3-38}
Ho ato ma wutu tumudingi ena do'a manga ngio'a jo arababu,\\
\gll ho	ato	ma	wutu	tumudingi	ena	do'a	manga	ngi-o'a	yo	arababu\\
     thus	\textsc{excl}	\textsc{rnm}	night	seven	\textsc{pro}.3\textsc{nh}	\textsc{locv}	3\textsc{hpl}.\textsc{poss}	place-\textsc{locv}	3\textsc{hpl}.\textsc{a}	native.violin\\
\glt `And look, [after] seven days there at their place they were playing the violin,'\footnote{MZ: \textit{Arababu} is used in Indonesia as the name for several string instruments usually called \textit{rebab} in English. According to \citet[45]{vanbaarda1895}, the Halmaheran type has one string. The name was borrowed from Indonesian \textit{arababu}, itself a borrowing of Arabic \textit{ar-rab{\=a}ba}.}
\z

\ea\label{ex:text3-39}
de jo bangiheli,\\
\gll de	yo	bangiheli\\
     \textsc{conn}	3\textsc{hpl}.\textsc{a}	flute\\
\glt `and playing the flute,'
\z

\ea\label{ex:text3-40}
de mo i tutumu, ma eha,\\
\gll de	mo	'i	tutumu	ma	eha\\
     \textsc{conn}	3\textsc{sg.f}.\textsc{a}	3\textsc{hpl}.\textsc{u}	go.see	\textsc{rnm}	mother\\
\glt `and she, the mother, looked for them,'
\z

\ea\label{ex:text3-41}
ho ato ma waliie ena ona do'a enau,\\
\gll ho	ato	ma	wali-ie	ena	ona	do'a	ena-ou\\
     thus	\textsc{excl}	\textsc{rnm}	?door-\textsc{dir}.\textsc{up}	\textsc{pro}.3\textsc{nh}	\textsc{pro}.3\textsc{plh}	\textsc{locv}	\textsc{pro}.3\textsc{nh}-\textsc{foc}\\
\glt `and look, ?they [came] up to the door,'\footnote{MZ: Ellen translates \textit{wali} as `door'. I could not find any potential cognates in other {\cnhl}.  Ellen further translates the clause as `and look here above the door they were' (``en kijk hier boven de deur waren zij'').  I do not know whether it was possible to be `above the door' in a traditional Modole house but I decided to assume a movement upwards to the door in my translation.}
\z

\ea\label{ex:text3-42}
de a jo liooli.\\
\gll de	'a	yo	lio-oli\\
     \textsc{conn}	\textsc{foc}	3\textsc{hpl}.\textsc{a}	return-\textsc{again}\\
\glt `and they returned again.'
\z
