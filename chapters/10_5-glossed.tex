\ea\label{ex:text10-1}
Tjumu tjagulu.\\
\gll cumu	cagulu\\
     riddle	riddle\\
\glt `Riddles'
\z

\ea\label{ex:text10-2}
1. O a'ele ma muhuti'a o'ia?\\
\gll o	a'ele	ma	muhuti'a	o'ia\\
     \textsc{nm}	water	\textsc{rnm}	pearl	what\\
\glt `What are the pearls of water?'
\z

\ea\label{ex:text10-3}
O abo-abo.\\
\gll o	abo{\textasciitilde}abo\\
     \textsc{nm}	foam\\
\glt `The foam.'\footnote{GJE: In which eyes glittering like pearls are found.}
\z

\ea\label{ex:text10-4}
2. Pahita'a i ma tero, ma rahai regu-regu, geena o'ia?\\
\gll pahita'a	i	ma	tero	ma	rahai	regu{\textasciitilde}regu	geena	o'ia\\
     mask	3\textsc{nh}.\textsc{a}	\textsc{mid}	beautiful	\textsc{rnm}	beautiful	other	\textsc{dist}:\textsc{pro}.3\textsc{nh}	what\\
\glt `The form is identical, the beauty is different, what's that?'
\z

\ea\label{ex:text10-5}
O a'ele, de o gahi ma a'ele.\\
\gll o	a'ele	de	o	gahi	ma	a'ele\\
     \textsc{nm}	water	\textsc{conn}	\textsc{nm}	salt	\textsc{rnm}	water\\
\glt `Water and seawater.'
\z

\largerpage
\ea\label{ex:text10-6}
3. Wange tja'o mo dodengo, wange tjumu mo halai; geena o'ia?\\
\gll wange	ca'o	mo	dodengo	wange	cumu	mo	halai	geena	o'ia\\
     sun	?hour	3\textsc{sg.f}.\textsc{a}	k.o.dance	sun	?evening	3\textsc{sg.f}.\textsc{a}	k.o.dance	\textsc{dist}:\textsc{pro}.3\textsc{nh}	what\\
\glt `?In the morning she dances the dodengo dance\footnote{MZ: \textit{Dodengo} is a kind of war dance (see \citet[434-435]{vanbaarda1895}).}, in the evening\footnote{MZ: \textit{Cumu} means `riddle' in multiple \cnhl. The meaning `evening' as suggested by Ellen is not attested. Also note that \textit{cumu} is translated as `riddles' in the title (see [J.\ref{ex:text10-1}]).} she dances the cakalele dance\footnote{MZ: \textit{Halai} is a spiritual dance that accompanies jinn rituals (see \cite[270-271]{hueting1921}). Ellen associates it with the \textit{cakalele} but this term pertains to war dances. A video of Modole men dancing the \textit{cakalele} can be found at \url{https://youtu.be/HttEemJj64o?si=1UqCcunVtKfPhYjk} (accessed 4 October 2025).}, what's that?'
\z

\ea\label{ex:text10-7}
Po parihi.\\
\gll po	parihi\\
     1\textsc{pl}.\textsc{in}.\textsc{a}	sweep\\
\glt `Sweeping.'\footnote{GJE: Done by the women in the morning and evening; the body moves up and down as during a dance.}
\z

\ea\label{ex:text10-8}
4. Naga o njawa awi wo'a ma tadi'a moio'a; geena o'ia?\\
\gll naga	o	nyawa	awi	wo'a	ma	tadi-o'a	moi-o'a	geena	o'ia\\
     \textsc{exist}	\textsc{nm}	person	3\textsc{sg.m}.\textsc{poss}	house	\textsc{rnm}	pole-\textsc{locv}	one-\textsc{locv}	\textsc{dist}:\textsc{pro}.3\textsc{nh}	what\\
\glt `There is someone whose house has just one pole, what's that?'
\z

\ea\label{ex:text10-9}
O biara.\\
\gll o	biara\\
     \textsc{nm}	k.o.plant\\
\glt `The \textit{biara} plant.'
\z

\ea\label{ex:text10-10}
5. O ngo'u i tangi, o wogono i hoho, geena o'ia?\\
\gll o	ngo'u	i	tangi	o	wogono	i	hoho	geena	o'ia\\
     \textsc{nm}	forest.pigeon	3\textsc{nh}.\textsc{a}	sit	\textsc{nm}	crow	3\textsc{nh}.\textsc{a}	fly	\textsc{dist}:\textsc{pro}.3\textsc{nh}	what\\
\glt `The [white] forest pigeon sits down, the [black] crow flies [away], what's that?'
\z

\ea\label{ex:text10-11}
O wange de o wutu.\\
\gll o	wange	de	o	wutu\\
     \textsc{nm}	sun	\textsc{conn}	\textsc{nm}	night\\
\glt `The day and the night.'
\z

\ea\label{ex:text10-12}
6. O tona'a ma muhuti'a o'ia?\\
\gll o	tona'a	ma	muhuti'a	o'ia\\
     \textsc{nm}	earth	\textsc{rnm}	pearl	what\\
\glt `What is the pearl of the soil?'
\z

\ea\label{ex:text10-13}
O ulu bitanga.\\
\gll o	ulubitanga\\
     \textsc{nm}	worm\\
\glt `The worm.'
\z

\ea\label{ex:text10-14}
7. O Tjina ma njawa de o Papua'a ma njawa jo ma'a paranga,\\
\gll o	Cina	ma	nyawa	de	o	Papua-o'a	ma	nyawa	yo	ma-'a	paranga\\
     \textsc{nm}	China	\textsc{rnm}	person	\textsc{conn}	\textsc{nm}	Papua-\textsc{locv}	\textsc{rnm}	person	3\textsc{hpl}.\textsc{a}	\textsc{mid}-\textsc{vpl}	war\\
\glt `The Chinese and the Papuans fight each other,'
\z

\ea\label{ex:text10-15}
halingou o Lada'a ja ino, aha jo aunu, geena o'ia?\\
\gll halingou	o	Lada-o'a	ya	ino	aha	yo	aunu	geena	o'ia\\
     necessary	\textsc{nm}	Netherlands-\textsc{locv}	3\textsc{nh}>3\textsc{nh}	\textsc{dir}.\textsc{ven}	as.a.consequence	3\textsc{hpl}.\textsc{a}	blood	\textsc{dist}:\textsc{pro}.3\textsc{nh}	what\\
\glt `[and then] the Dutch must join, then they bleed, what's that?'
\z

\ea\label{ex:text10-16}
O ena'a, de o bidoho, de o gau.\\
\gll o	ena'a	de	o	bidoho	de	o	gau\\
     \textsc{nm}	areca	\textsc{conn}	\textsc{nm}	betel	\textsc{conn}	\textsc{nm}	lime\\
\glt `Areca and betel and lime.'\footnote{GJE: [These are] the ingredients for betel chewing which is highly loved by the natives. Only after adding lime (the white Dutchmen) it becomes red as blood.}
\z

\ea\label{ex:text10-17}
8. O dodai ma doda'a a o palengu i hara-hara geena o'ia?\\
\gll o	dodai	ma	doda-o'a	'a	o	palengu	i	hara{\textasciitilde}hara	geena	o'ia\\
     \textsc{nm}	barrel	\textsc{rnm}	inside-\textsc{locv}	\textsc{foc}	\textsc{nm}	bullet	3\textsc{nh}.\textsc{a}	\textsc{rdpl}{\textasciitilde}time	\textsc{dist}:\textsc{pro}.3\textsc{nh}	what\\
\glt `A barrel whose inside is all bullets, what's that?'
\z

\ea\label{ex:text10-18}
O tapaja ma mui.\\
\gll o	tapaya	ma	mui\\
     \textsc{nm}	papaya	\textsc{rnm}	kernel\\
\glt `The kernels of a papaya.'\footnote{GJE: In which hundreds of seeds are found.}
\z
