\subsection*{De biaraplant (een soort plant met lange bladeren op een langen stamstengel; de bladeren worden als groente gegeten)}

Op een keer ging zij (uit het verhaal blijkt zij eene moeder te zijn van twee kinderen) naar haar tuin, en zij deed haar (dochtertje) beloven, zij zeide: ``Nene'' (een meisjesnaam) ``als je broertje huilt, geef hem dan eten, doch geef hem mijn biara niet.''

Zij ging dus en toen zij weg was huilde hij, en zij gaf hem (te) eten, maar hij wilde (dat) niet, hij wilde slechts die biara (hebben), toen gaf zij die, en hij huilde niet meer.

Kijk, daarna was de moeder gekomen en zij vroeg naar haar biara, zij zeide: ``Hoe staat het met mijn biara?'' En zij zeiden: ``Moeder, zooeven huilde de kleine er om, dus heb ik ze hem gegeven.''

En de moeder zeide: ``Ik deed (je) beloven, ik zeide: `gij moogt hem mijn biara niet geven, al huilt hij er ook om'; ik zeide: `al geeft gij hem (ook) al het eten, dat hier is, tot er niets meer over is, als ge hem maar niet mijn biara geeft.'{''} En zij (de dochter) zeide: ``maar moeder, het eten daarginds heb ik hem gegeven, maar hij wilde (het) niet (eten), hij wilde slechts de biara (hebben)'' (in het gewone leven der inlanders gaat het steeds zooals hier verhaald wordt, dwingende kinderen krijgen namelijk steeds hun zin.)

En zij zeide: ``Mijn biara heeft hij opgegeten, dus ga ik (weg)''; toen ging zij, en zij (de kinderen namelijk) zeiden: ``Moeder, de kleine heeft wel de biara opgegeten, doch ga toch niet (weg).'' Maar zij zeide: ``Ik wil niet (blijven), ik ga (weg).''

En zij volgden (haar) maar, doch zij zeide: ``volgt mij (toch) niet, opdat gij bij uw vader blijft,'' maar zij wilden niet, zij volgden slechts, en de zuster riep, zij zeide: ``Moeder! de koning der Molukken (haar broertje namelijk) zijn keel is droog (hij heeft dorst) dus geef hem eerst de borst.''

En zij de moeder zeide: ``Keert toch terug tot uw vader en blijft aldaar;'' toen drukte zij hare melk uit in een palmblad en zij gingen landwaarts en zij (de zuster) gaf hem ervan te drinken, vervolgens was zijn keel (weer) droog (en) zij riepen de moeder: ``Moeder, de koning der Molukken zijne keel is droog, dus geef hem eerst (weer) te drinken.'' Toen drukte zij haar borst wederom uit in een palmblad, ook (gingen) zij landwaarts en zij, de zuster, gaf hem eerst weer melk, vervolgens was zijn keel (alweer) droog, (en) de zuster riep, zij zeide: ``Moeder, de koning der Molukken zijn keel is droog, dus geef hem eerst de borst;'' en zij drukte haar borst weer uit in een palmblad, en zij gingen en de zuster gaf hem die eerst te drinken; daarna (werd) zijn keel weer droog, (en) zij riepen: ``Moeder: de koning der Molukken zijn keel is droog, dus geef hem eerst de borst;'' daar landwaarts op de steenen stond een huis en aldaar wachtte zij hen. En zij gingen landwaarts en zij gaf hem de borst tot hij geheel verzadigd was, en zij zeide: ``Vooruit, keert toch naar huis terug,'' en zij zeiden: ``Moeder, hoe kunnen wij nu terugkeeren?'' En (toen) brak zij haar eene borst af voor hem. En zij deed (hen) beloven, zij zeide: ``Neem deze mijne eene borst voor hem mee, opdat gij hem er mee voedt, doch als hij geen moedermelk meer gebruikt, plant haar (die moederborst) dan (stop die dan in den grond, zooals men vruchten plant.)'' Toen hij dan de moedermelk niet meer gebruikte, plantten zij die (borst).

Ziet, daarna ontsproot zij en groeide op (als een boom) en later (toen die boom groot geworden was) klommen de kinderen er in, maar zij vielen er niet mee om.

Ziet, toen bloeide die (boom), en de bloesems die hij voortbracht waren kains (lijnwaden), witte baadjes, broeken, baadjes, hakmessen, patjols, bijlen, kralen, zilveren ringen, gouden (ringen), haarkammen, oorringen, enkelringen, oorhangers.

En aldaar, wat er ook voor verschillende dingen in de wereld zijn, hij (die boom namelijk) bloeide ze (bracht ze voort) en daar aan zijn top hing een kist.

Kijk, zij braken die open (en) hier de moeder zelf (was daarin).
