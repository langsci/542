\subsection*{Het Bibitivischje}

Op een keer gingen zij vischscheppen (met een mandje); ('t waren) een moeder met haar kind en een bibitivischje vingen zij.
De moeder zeide: ``(Laten) wij het (dood) slaan.'' En zij zeide: ``Sla het (toch) niet (dood), (laat) het toch mijn speelgoed (zijn)!'' Toen legde zij het in den takoksel van den gohapanataboom. Doch kijk! zij zag het, zij (de vischjes) vulden den gohapanataboom (de ruimte tusschen tak en stam) geheel op, en het kind (meisje) zeide: ``Vader! maak toch eene prauw (bootje), opdat ik mijn bibitivischjes daarin doe.''
Toen deed zij ze erin, doch kijk! zij keek er naar (en zag), dat ze (de vischjes) het bootje vulden tot boven toe.
En zij zeide: ``Vader! gij moest toch nog eens mijn tuin openkappen (de boomen daarin omhakken), want die van mijn vrienden hebben zij reeds opengekapt.''
En hij zeide: ``Ik zal openkappen, doch ga gij vooruit!'' En zij ging landwaarts, en kijk, toen zij landwaarts kwam had men deze reeds opengekapt (en) de wortels zelfs had men er uitgetrokken.
En zij ging zeewaarts en zij zeide: ``Vader, 't is niet noodig (dat) gij landwaarts gaat, ik ben reeds met het openkappen klaar.''
En dien nacht kwam iemand om met haar te vrijen, en zij wilde niet, zij zeide: ``Ik kwam hierheen, niet om te vrijen.'' En hij zeide: ``Als het zoo is, wil dan (maar even) mijn haar opmaken!'' (een door de jongens zeer gewaardeerd teeken van genegenheid), maar zij wilde niet, ``wil dan mijn haar beoliën!'' maar zij wilde (dat) ook niet, ``maar als gij dan voor ons pinangnoten van de schil ontdoet, opdat wij sirihpinang pruimen!'' maar zij wilde ook (dat) niet, daarna daalde hij af (van het huis op palen) (en) zijn pruimdoos was een pinang-sirihtasch, (welke op den rug gedragen wordt).\\\largerpage

En toen bescheen de zon hem en zij zag hem goed ('t was dus goed licht geworden) en zij zeide: ``Kom hier opdat ik je haar opmaak,'' maar (toen) wilde hij niet, hij zeide: ``Doe dat niet! want dan zullen je vader en je moeder boos op je zijn'' en zij vond dat niet goed (van hem) dus zij volgde hem.
En als hij afdaalde van de omgehakte boomstammen klom zij op aan 't ondereind van de stammen, maar als hij opklom, (dan) daalde zij neer.
Alzoo vervolgden zij hun weg tot over zeven rivieren en over zeven bergen, doch als hij (een berg) besteeg, dan daalde zij af, en als hij afdaalde dan beklom zij (een berg).
En zij vonden eene rivier waarboven een boschliaan naar beneden hing, aldaar wachtte zij hem, (en) zij pruimden sirihpinang, en (toen) zeide hij: ``Keer terug! anders (zullen) je moeder en je vader je niet vinden'' en zij zeide: ``Ik wil niet, ik (wil) je volgen,'' (en) toen zeide hij: ``doe dan je oogen dicht,'' en zij deed ze dicht, doch ziet! (toen) zij ze weer opende (was) zij hier midden op den berg. Aldaar hing een boschliaan (naar beneden) en aldaar legde zij zich neer en haar hoofdhaar spreidde zij uit zelfs over twee rivierbochten. En de visschen zetten zich in dat haar tot boven aan toe, gelijk een werpnet dat geheel vol is.
Eens vischte hij (iemand) met een net en kreeg haar haar in dat net, en hij zeide: ``Hier nu, van wie is dit hun haar toch?'' Hij zeide: ``Kom, nog wat landwaarts,'' en zij (er schijnen nu twee personen geweest te zijn) gingen landwaarts (en) hier op den boschliaan lag zij te slapen en zij laadden (tilden) haar in de prauw.
En zij gingen en aldaar (waren) de vrouwen des konings boos, zij zeiden: ``gij verzamelt maar vrouwen; nu zijn uw vrouwen zeker wel zeven.'' (De visscher blijkt dus een koning geweest te zijn). En zij (die vrouwen) zeiden: ``Zij (is) toch mooi; welaan, laat ons poepen'' zij dan poepten, maar de lucht van haar ontlasting was aangenaam (geurig), en zij zeiden weer: ``zij (is) mooi, kom, laten wij een wind laten,'' en zij lieten een wind, doch kijk! de lucht van haar wind was aangenaam (geurig).
En zij zeiden weer: ``Kom, laten wij van hier met een vlot den stroom volgen naar zee,'' en zij volgden met een vlot den stroom. Zij die zeven (gingen) eerst, en zij stootten er mee (met dat vlot tegen den oever) en (zij) in eens naar beneden (verongelukten allen). En zij ook (ging op een vlot). En zij dreef er ook mee weg, doch zij (dreef) er mee landwaarts (dus tegen den stroom in) zij keerde (dus) terug.
En de koning zeide: ``Ik zal je trouwen.'' En zij zeide: ``dan moet ik eerst een van uwe bedienden (dood) steken.'' En zij stak hem, zoodat hij er bij omviel (dood neer viel) en men omwond hem met kaladibladeren, en zij lieten hem achter in het huis van iemand anders. Doch kijk! Na zeven nachten alhier geweest te zijn speelde hij op de fluit, en hij leefde weer. En de helft van zijn uiterlijk was zilver (en) zijn (andere) helft (was) goud, de helft zijner haren (was) zilver, zijn (andere) helft goud, zijne bovenste tanden (waren) zilver en zijn onderste (waren) goud.
En de koning zeide: ``Steek gij mij ook!'' en zij zeide: ``(Laat ik dat) niet doen; want dan wordt gij ook half.'' Maar hij haalde haar over (om het toch maar te doen) alzoo stak zij hem; en hij is half geworden en hij is niet terug gekeerd, (d. w. z. voor de helft zilver en de helft goud).
