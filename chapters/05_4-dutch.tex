\subsection*{De oudere en de jongere (broeder)}

Er waren twee menschen, een oudere en een jongere (broeder). De oudste riep de honden (ging op de jacht) en de jongste ging sagokloppen. De oudste ging de honden roepen (met de honden op de jacht) en tot zijne vrouw zeide hij: ``Als straks uw zwager hierheen komt en hij vraagt sagobroodjes, dan (moet) gij hem die geven.''

En de vrouw zeide: ``ga maar, ik zal hem wel geven;'' en toen hij gegaan was kwam zijn jongere broer. En hij kwam en vroeg sagobroodjes, hij zeide: ``schoonzuster, (mag) ik wat sagobroodjes eten, ik heb honger?'' En zij zeide: ``sagobroodjes, van waar zou ik die hebben?'' en zijne ingewanden rommelden (van honger).

En hij, de zwager, was beschaamd en hij ging (weg), over zeven bergen steeg hij.

De oudste (broeder) kwam (terug) en hij vroeg aan zijne vrouw, hij zeide: ``Hoe (is) het met je zwager?'' En zij zeide: ``Weet ik van hem af?'' En hij zeide bij zich zelven: ``dit is het, hij (heeft) sagobroodjes gevraagd en zij (heeft) ze hem niet gegeven, alzoo is hij (ook) weggegaan.''

En hij volgde hem (namelijk zijn jongeren broer) over zeven bergen en hij klopte op een boomwortel en (toen) hield hij op (met loopen). En aldaar hielden zij (beiden) op, zij pruimden pinangsirih en de oudste vroeg, hij zeide: ``Waarom (zijt) gij toch weggegaan?'' En hij zeide: ``nergens om, doch ik vroeg sagobroodjes aan je vrouw en zij heeft ze mij niet gegeven.''

En de oudste zeide: ``kom hierheen (laten) wij terugkeeren!'' En hij (de jongste) zeide: ``ik ga niet terug.''

En zij keerden elkaar den rug toe (scheidden van elkaar), hij, de oudste keerde terug naar hun huis, en hij, de jongste ging den vreemde in, en hij ging en kwam bij eene oude vrouw.

En hij zond de honden, hij zeide: ``honden, gaat landwaarts en als grootmoeder eene heks is, geeft (dit dan) te kennen!''

En zij gingen landwaarts en keerden terug en zij zeiden: ``zij (is) geen heks;'' en hij zond ook zijn ring en die (kwam weer) zeewaarts en zeide: ``zij (is) geen heks.''

En hij riep en vertoonde zich (en) hij zeide: ``Grootmoeder! grootmoeder!'' en zij zeide: ``hei, ik heb geen kleinzoon.'' En hij zeide: ``Grootmoeder, misschien zijt gij eene heks, neem mij toch niet mee!'' En zij zeide: ``ik (ben) geen heks, (doch) een werkelijk mensch.''

En zij gingen naar haar huis en zij kookte rijst, dus aten zij, en na den eten was het nacht geworden, alzoo gingen zij slapen.

\largerpage
En te middernacht viel er een ringworm in hun huis neer, en hij schrok er van en hij zeide: ``stellig (is) zij dat'' en hij riep, hij zeide: ``Grootmoeder, wat is er?'' en zij zeide: ``daar is een ringworm gevallen.'' En zij stak een harsfakkel aan, en dat (was) een ringworm, die gevallen was.

En den volgenden dag kwamen zeven dochters van den koning van het Oosten zuidwaarts aanwandelen; zij gingen zuidwaarts en zij zeiden: ``Grootmoedertje, uw geur is aangenaam, zooals de geur van bovenaardsche wezens zal zijn.''

En zij zeide: ``kleinkinderen, er is niets, (in) mijn huis (heb ik) jasmijnbloemen, die ruiken zoo;'' toen daalden zij af in het water.

En het oudje vroeg (aan dien jongeren broer, die bij haar gekomen was), zij zeide: ``Zooeven, (die) zeven daar beneden, van welke houdt gij? En hij zeide: ``Van die, zij (die) volgt;'' (de achterste dus) en de koning schoot een kanon af (en) alzoo keerden zij terug.

Toen zij daarboven (bij 's konings paleis) aangekomen waren, werd op hunne trommen en koperen bekkens geslagen (werd er feest gevierd) en hij vroeg, hij zeide: ``Grootmoeder, wat is dit voor een feest?'' En zij zeide: ``Men zegt, men werpt elkaar met de toppen van de banahanaplant'' (een soort spel). En hij zeide: ``Grootmoeder, als zij bloeden, (zullen) wij dan gaan kijken?'' En zij gingen en op het eind van den weg zetten zij zich neer. En zij schopte (speelde ook, doch waar zij mee speelde is niet duidelijk) en het viel op zijn hoofd en daarna ook nog op zijn schoot neer. En hij ook schopte (speelde), alzoo viel het naar beneden op haar hoofd (en) daarna ook op haar schoot neer.

Toen keerden zij terug, doch kijk daarna kwamen zij weer, de zeven namelijk en zij (hadden) zich vermomd.

En hij, de jongeling verborg zich onder de jasmijnstruik, en de mom van de jongste raakte hem bij toeval aan en toen zij in het water waren afgedaald, verborg hij haar (namelijk haar mom).

Kijk, de oudere (zusters) gingen (en) toen was haar (de jongere) mom er niet; toen schoot de koning een kanon af en zij de oudere (zusters) keerden terug, toen waren zij reeds teruggekeerd.

En het oudje zeide (tegen de jongere, die haar mom kwijt was): ``huil (mom) niet, ga naar boven''; en zij ging naar boven, en toen zeide zij: ``dit is mijn mom, die hij vertoonde''; en zij trouwde met hem.

En (toen) gaf de koning bevel, dat ze haar en haar man zouden (gaan) halen, en zij daalden af en zij haalden hen en zij klommen met hen naar boven en zij vierden feest zeven nachten en zeven dagen (lang).

En de oudste (broeder) volgde, ook kwam hij tot aan het oudje, (en) toen daalden zij zes (zusters) (ook af in de rivier), zij (hadden) zich vermomd.

En het oudje vroeg, zij zeide: ``Kleinzoon, welke van de zes dochters des koning van zooeven begeert gij?'' En hij zeide: ``die in het midden (is);'' toen schoot de koning een kanon af, dus keerden zij weer terug. Daarna daalden zij weer af (en) zij hadden zich vermomd; en hij verborg zich onder de jasmijnstruik en de mom van de middelste raakte hen bij toeval aan en zij daalden af in het water en hij verborg haar (namelijk haar mom).

Kijk, de koning schoot toen een kanon af en haar mom was er niet (en) toen keerden de oudere (zusters) terug en zij huilde en het oudje zeide: ``huil gij (maar) niet, (ga) maar naar boven''; en zij ging naar boven en zeide ``mijn mom (is) deze bij hem.'' En de helft van zijn haar was zilver en de (andere) helft goud, en zijn bovenste tanden waren goud, zijn benedenste zilver.

En de koning beval, (dat) zij hen zouden halen, en zij daalden af (en) zij haalden hen en het oudje weende. En hij (de oudere broeder namelijk) zeide: ``huil (maar) niet, doch kom met ons mee'' en zij gingen met hen naar boven; (toen) zij boven (waren) vroeg hij (de koning namelijk), hij zeide: ``Gij, wat voor een mensch (zijt) gij, want uw uiterlijk en uw voorkomen is gelijk (dat) van de djin (geest) en nimf?''

En hij zeide: ``Groote heer, ik hier, met ons tweeën waren wij, ik riep de honden (was jager) en hij, mijn jongere broer hij klopte sago en ik zeide (tegen zijne vrouw zeide hij dat, volgens het bovenstaande): `Als je zwager hier komt, geef hem (dan) sagokoeken.'

En hij kwam (en) hij vroeg en zij gaf (ze) niet en hij (werd) beschaamd en hij ging (weg), over zeven bergen en over zeven rivieren was hij, en ik volgde hem.

\largerpage
En ik vond hem en nadat wij sirih-pinang gepruimd hadden zei ik: `kom hier, (laat) ons terugkeeren' en hij zeide: `Wel, ik ben beschaamd om weer terug te keeren; keer gij maar terug, dan ga ik maar verder.'{''}

En de koning zeide: ``Kom haal je krisschede (eens) uit (je gordel), opdat ik er mijn mes indoe;'' en hij deed het er in en het paste. En hij (de koning namelijk) zeide: ``Kom (laat) ik je ring (eens) aandoen (passen)'' en hij deed hem aan, en hij paste. En hij zeide (weer): ``Kom (laat) ons je klapperdoppen op elkaar passen''; en zij pasten (en) zij waren gelijk (deze drie middelen schijnt men wel eens te probeeren om daardoor te weten te komen of men nog familie van elkaar is) en de koning sloeg zich op de borst en hij zeide: ``als het zoo is, dan ben ik het,'' en zij vierden feest zeven nachten en zeven dagen (lang).
\newpage
