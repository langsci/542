\section{Introduction}

In this chapter, I will outline the grammatical structure of Modole based on the texts in this collection. References to examples from the texts are in the format ``[A.1]'', where the letter refers to the text and the second digit refers to the respective line. Working with a corpus of such limited size has consequences for the grammatical description. Naturally, some phenomena will be unattested in the texts. Especially in the area of syntax the data is often inconclusive. Occasionally I asked a native speaker of Modole for judgments regarding certain structures but I refrained from incorporating any other Modole sources so that this sketch grammar would faithfully reflect the language of only the texts. My Modole informant found the texts comprehensible in general but struggled with the phasal polarity system, which seems to have changed a lot in the past 100 years.

Modole is structurally very similar to the closely related languages Tabaru and Pagu. I am of course influenced by descriptions of the latter two languages (especially by Perangin Angin's (\citeyear{peranginangin2018}) Pagu grammar). The identification of morpheme boundaries is partly based on descriptions of Pagu and Tabaru.
Nonetheless, I have tried as much as possible to describe the grammar of Modole in its own terms. Therefore, I only draw comparisons to other languages to improve understanding of rarely attested features. I also kept notes on diachronic issues to a minimum. The chapter is longer than what may be expected of a grammar sketch. Since this is the first extensive linguistic description of Modole, I discuss as many features as possible. 

Modole is a head-marking language. Predicates are morphologically more complex than arguments. The alignment is nominative--accusative in the sense that the single argument of a intransitive predicate is marked like the actor of an transitive verb. Predicates usually occur with actor and undergoer indices. Verbs are preceded by a number of morphemes that mostly mark changes in valency. Tense is not marked but there is a set of suffixes pertaining to the temporal domain, such as phasal polarity markers. Nominal modifiers (``adjectives'') have the form of intransitive verbs. Pronominals and indices distinguish clusivity and human vs. non-human referents. There is no nominal number or overt gender marking. Noun phrases as undergoer arguments are morphologically marked in certain contexts. Modole has an elaborate system of spatial notions along the three axes: \textsc{sea} vs. \textsc{land}, \textsc{up} vs. \textsc{down} and \textsc{ventive} vs. \textsc{itive}. Demonstratives are morphologically complex. The default word order is SOV but variations frequently occur.

\section{Orthography}

This section deals with orthography of Modole used in the texts as well as some basic phonology issues that can be gathered from the written text alone. Ellen used a Dutch-based orthography for Modole inspired by the works of his missionary colleague M.J. van Baarda for Galela \citep{vanbaarda1895, vanbaarda1904, vanbaarda1908}. Van Baarda was considered the best ``linguist'' by the other missionaries and they followed his standards (see \citet[11]{hueting1908}). This does not only concern the spelling of sounds but also which grammatical items were written independently or conjoined with another verb.\footnote{Van Baarda's theoretical considerations on this issue can be found in the works cited above.} 

\tabref{tab:consonant_phonemes} displays the graphemes and digraphs used to write consonants in Modole.\footnote{ The approximate pronunciations are based on evidence from other North Halmahera languages and modern recordings of Modole.} The modern orthography is put in parentheses where it deviates from Ellen's orthography. The graphemes <k> and <ny> are only found in loanwords (\textit{waktu} `time' (Malay, ultimately Arabic), \textit{nyawa} `person' (Malay), \textit{nyonyie} `sunrise' (Ternate)) and they are likely not phonemic. Proto-Core North Halmahera (\textsc{pcnh}) *r is reflected as /l/ or zero in most cases, but there are 52 roots attested in the stories that contain /r/. Most of these are loans from non-North Halmaheran languages. A root such as \textit{gorona}, which is likely native, could be a borrowing from another North Halmahera language (expected outcome in Modole is **gona\footnote{Two asterisks introduce an unattested form.}).

In (\ref{ex:minimalpairs_consonants}), I give at least one minimal pair for each grapheme/digraph to demonstrate that they represent phonemes. There is no minimal pair for /l/ vs. /r/ and the phonemic status of the latter is uncertain.

\begin{table}
    \caption{Consonant inventory of Modole}
    \label{tab:consonant_phonemes}
    \begin{tabularx}{\textwidth}{llllll}
    \lsptoprule
         &  \textsc{labial}&  \textsc{dental}/\textsc{alveolar}&  \textsc{palatal}&  \textsc{velar}& \textsc{glottal}\\
         \midrule
         \textsc{plosive}&  p b&  t d&  tj (c) dj (j)&  k g& ' \\
         \textsc{nasal}&  m&  n&   nj (ny)& ng\\
         \textsc{trill}&  &  r &  &  & \\
         \textsc{approximant}&  w&  l&  &  j (y)& h\\
         \lspbottomrule
    \end{tabularx}
\end{table}

\begin{exe}
    \ex 
    \label{ex:minimalpairs_consonants}
    \begin{xlist}
    \ex /b/ : /h/ : /l/ : /n/ \\
        \textit{abo} `foam' : \textit{aho} `dog' : \textit{alo} `beat sago' : \textit{ano} `a moment ago'
    \ex /p/ : /m/ \\
        \textit{hupu} `exit' : \textit{humu} `well'
    \ex /d/ : /g/ \\
        \textit{tadi} `stab' : \textit{tagi} `walk'
    \ex /t/ : /d/ : /y/ \\
        \textit{tai} `carry on the back' : \textit{dai} `\textsc{sea} locational' : \textit{yai}  `mother'
    \ex /c/ : /d/ \\
        \textit{cina} `China' : \textit{dina} `\textsc{land} locational'
        \ex /j/ : /n/ \\
        \textit{jaga{\textasciitilde}jaga} `guard' : \textit{naga} `\textsc{exist}'
    \ex /ng/ : /n/ \\
        \textit{ngamo} `angry' : \textit{namo} `chicken' 
    \ex /g/ : /r/ \\
        \textit{jaga{\textasciitilde}jaga} `guard' : \textit{jara} `horse'
    \ex /w/ : /'/ \\
        \textit{dawu} `sister in law' : \textit{da'u} `\textsc{up} locational' 
    \end{xlist}
\end{exe}

There seems to be a general rule that only one glottal stop occurs per word form (with a few exceptions). I do not know whether this is a synchronic phonological rule, or a mere orthographic convention. Some roots occur with and without a final syllable of the form /'V/, where V is identical to the preceding vowel. Examples are given in (\ref{ex:nmlz}). 

\begin{exe}
    \ex 
    \label{ex:nmlz}
    \begin{xlist}
    \ex \textit{wutu} `night' vs. once \textit{wutu'u} [B.\ref{ex:text2-23}]\footnote{or < \textit{wutu-u'u} `night-\textsc{dir.down}'}
    \ex \textit{ena'a} `areca' vs. once \textit{enaou} `chew betel' [B.\ref{ex:text2-46}]
    \ex \textit{la'o} `eye' vs. once \textit{la'o'o} [D.\ref{ex:text4-47}]
    \end{xlist}
\end{exe}

Ellen uses five graphemes to represent vowels: <a>, <e>, <i>, <o> and <u>. The minimal pairs given in (\ref{ex:minimalpairs_vowels}) show they represent phonemes. Some vowels are marked with a trema (e.g. <ä>, <ï>). Its function is unclear.

\begin{exe}
    \ex 
    \label{ex:minimalpairs_vowels}
    \begin{xlist}
    \ex /a/ : /u/ \\
        \textit{tana} `raise' : \textit{tanu} `modality marker'
    \ex /a/ : /e/ \\
        \textit{da'u} `\textsc{up} locational' : \textit{de'u} `mountain' 
    \ex /i/ : /o/ \\
        \textit{ai} `first person singular possessive' : \textit{ao} `take' 
    \end{xlist}
\end{exe}

There are several cases of roots where one vowel is spelled either <a> or <o> (e.g. \textit{gohapanata} vs. \textit{gahapanata} `kind of tree').
It is possible that /a/ and /o/ had phonetic realizations that were difficult to tell apart for the transcriber(s). Note that the Modole are called \textit{Madole} in most Dutch sources.\footnote{Alternatively, <a> and <o> may have been difficult to distinguish in handwriting by the printers.}

There is some evidence for a change of lexeme-final /u/ > /o/ if the following lexeme starts with /o/, e.g. \textit{no dapa'o o ngotili} [B.\ref{ex:text2-11}] instead of **dapa'u o.
The suffix \textit{-ou} is spelled <o> word-finally when the following word starts with an /o/; see \textit{hupu-o} in (\ref{ex:u>o}).

\begin{exe}
    \ex 
        \gll de na'o wo lamo'o'au, {ma moi} wo hupu-o o a'ele Itio \\
        \textsc{conn} \textsc{cond} 3\textsc{sg.m}.\textsc{a} big:\textsc{perf} once 3\textsc{sg.m}.\textsc{a} exit-\textsc{already} \textsc{nm} water Itio \\
        \trans ‘and when he had grown up, he once left the Itio river’ [A.\ref{ex:text1-16}] 
        \label{ex:u>o}
\end{exe}

The final vowel of a root is often lost when a suffix is attached. This seems to more frequently the case with directionals (see \sectref{sec:directionals_locationals}) than with other suffixes (see \sectref{sec:other_suffixes}).

\subsection{Onset mutation}\label{subsec:onset_mutation}

The Core North Halmahera languages share the feature of onset mutation. The initial consonant of a root is replaced with another consonant and vowel initial roots are prefixed with \textit{g-} or \textit{ng-}. Onset mutation has several functions and the most tangible one is deriving nouns from verbs. In the Modole texts, onset mutation seems to play a minor role and it is questionable whether it is productive. Attested mutation pairs are displayed in Table (\ref{tab:onset mutation}).

\begin{table}
    \caption{Onset mutation in Modole}
    \label{tab:onset mutation}
    \begin{tabularx}{.55\textwidth}{lcl}
        \lsptoprule
        \textsc{unmutated} & & \textsc{mutated} \\
        \midrule
        \textit{aha} `carry'&  :&\textit{gaha} `take'\\
        \textit{aho'o} `call'&  :&\textit{gaaho'o} `request'\\
        \textit{ai'i} `take out of'&  :&\textit{gai'i} `take'\\
        \textit{ali} `cry'&  :& \textit{gali} `cry'\\
        \textit{aliti} `exchange'&  :&\textit{ngaliti} `replace'\\
        \textit{oa} `end (n)'&  :&\textit{goa} `end (n)'\\
        \textit{oru} `row'&  :&\textit{ngoru} `row'\\
        \textit{parihi} `sweep'&  :&\textit{babarihi} `broom'\\
        \textit{tahe} `court (v)'&  :&\textit{dahe} `court (v)'\\
        \textit{tuulu} `follow'&  :&\textit{duulu} `after'\\
        \textit{umo} `throw'& :&\textit{gumo} `throw'\\
        \lspbottomrule
    \end{tabularx}
\end{table}


\subsection{Reduplication}\label{subsec:redupl}

Reduplication is a common morphological process in Modole. There are three attested reduplication patterns: total reduplication, partial reduplication of the first two syllables of a root ((C\textsubscript{1})V\textsubscript{1}(C\textsubscript{2})V\textsubscript{2}{\til}) and partial reduplication of only the first syllable (C\textsubscript{1}V\textsubscript{1}{\til}). 
Only bisyllabic roots can undergo total reduplication. The distribution of the other two patterns is not entirely clear to me. 

The same is true for the mapping of functions onto patterns. 
Reduplication of only the first syllable commonly derives noun from verbs, e.g. \textit{le{\til}letongo} `mask' (from \textit{letongo} `shine') or \textit{ba{\til}barihi} `broom' (from \textit{parihi} `sweep').

Reduplication can have an itensifying function as in (\ref{ex:redpl_intens}).\footnote{Notice that the \textsc{down} directional \textit{u'u} has the same function.}
In other examples, reduplication yields an imperfective reading (eg. durative, frequentative or habitual). For example, the reduplication of \textit{gogele} `sit' means `stay' (compare [E.\ref{ex:text5-78}] vs. [G.\ref{ex:text7-30}]). In (\ref{ex:total_rdpl}), the reduplicated form \textit{hungi-hungi} `search for game with dogs' refers to someones occupation.
Reduplicated forms with imperfective reading frequently occur in relative clauses (see \sectref{subsec:relative_clause}).

\begin{exe}
    \ex
    \begin{xlist}
    \ex 
    \gll mi huhu ho mi punu{\til}punuhu'u\\
    3\textsc{sg.f}>3\textsc{sg.m}	breast	thus	3\textsc{sg.f}>3\textsc{sg.m}	\textsc{rdpl}{\textasciitilde}satisfied:\textsc{dir}.\textsc{down}\\
\glt `she nursed him so that he was completely satisfied' [G.\ref{ex:text7-50}]\label{ex:redpl_intens}
    \ex 
        \gll wo hungi{\til}hungi \\
        3\textsc{sg.m}.\textsc{a} search.for.game.with.dogs \\
        \trans `he was a hunter' [C.\ref{ex:text3-3}]
        \label{ex:total_rdpl}
    \end{xlist}
\end{exe}

Prefixes are reduplicated as well, for example the applicative \textit{dV-} (\ref{ex:rdpl_dV}).\footnote{Notably, causative \textit{hi-} is never reduplicated}.

\begin{exe}
    \ex 
    \gll abei'a dauwe-ngo'o mi ma hi-du{\textasciitilde}du-tuumo \\
     well	\textsc{loc}.\textsc{down}-\textsc{n:dir.sea}	1\textsc{pl}.\textsc{ex}.\textsc{a}	\textsc{mid}	\textsc{caus-rdpl}{\textasciitilde}\textsc{appl}-?follow.with.raft\\
    \trans `come, [let]'s (ex.) ?follow the river with a raft down to the sea' [B.\ref{ex:text2-68}]
    \label{ex:rdpl_dV}
\end{exe}

Since I am unable to match patterns to functions I gloss reduplication as \textsc{rdpl}. If a root only occurs in reduplicated form, as is the case with \textit{hungi-hungi}, the whole form is glossed with the lexical meaning once. In the glosses texts, the two partial reduplication patterns are indicated as \textit{CV} and \textit{(C)V(C)V}.

Occasionally repetition of the whole verb form including indices occurs (\ref{ex:redpl_VP}). The meaning seems to be durative/frequentative.

\begin{exe}
    \ex 
    \gll gena-'a-de yo tagi-yo tagi\\
     \textsc{dist}:\textsc{pro}.3\textsc{nh}-\textsc{locv}-\textsc{conn}	3\textsc{hpl}.\textsc{a}	go-3\textsc{hpl}.\textsc{a}	go\\
    \trans `then they walked on' [H.\ref{ex:text8-24}]
    \label{ex:redpl_VP}
\end{exe}


\section{Parts of speech}

Verbs and nouns can be distinguished in Modole based on the grammatical elements they co-occur with. Verbs occur with indices while nouns are preceded by noun markers or possessives. Many roots can function as both verbs and nouns. There are two means of nominalization (onset mutation and reduplication). Nominalized stems sometimes occur with indices as well. 

%arguments and predicates
Most predicates are verbs. Types of predicates belonging to the realm of non-verbal predication in many languages (e.g. equative clauses) are treated like intransitive verbs and take verbal morphology.
Arguments are personal pronouns, demonstratives or nouns.  

%predication & attribution
A distinction between predication and attribution is seldom made and there is no adjective class. Nominal modifiers usually come in the form of intransitive verbs. The same is true for some kinds of adverbials. 

%affixes
As mentioned in the orthography section above, Ellen used an orthography devised by van Baarda for Galela. In this orthography, several grammatical elements are written separately that would now be considered affixes or clitics. For example, the Modole middle marker \textit{ma} is written separately while the causative marker \textit{si-} is attached to the following verb. Since I cannot say anything about the phonological status of these elements, I will only call elements affixes if they are written conjoined with another lexeme.

\section{Personal pronouns and indices}

Person (and in most cases number as well) is obligatorily expressed on verbs in the form of indices.\footnote{That is bound, referential person forms that express an referent without the obligatory overt expression of these referents by other linguistic means; see \citet{haspelmath2013}.} Optionally, a free personal pronoun can be added as well. Both personal pronouns and indices are written independently in the texts. There are just a few examples where an index is merged with a verb, e.g. \textit{j'odomo} `they eat' [A.\ref{ex:text1-34}].

Third person personal pronouns and indices show a gender distinction between human and non-human referents and, within the human category, also between singular feminine and masculine referents.
For human referents, there are separate personal pronouns and indices expressing feminine singular, masculine singular, and plural, while for non-human referents only one set of personal pronouns and indices exists, without number or gender distinctions (see \tabref{tab:gender_pp_ind}).\footnote{The distinction between plural human and non-human actors is not found in actor-undergoer combinations, see \sectref{subsubsec:pronouns_indices-indices-actor_undergoer}.}

\begin{table}
    \caption{Gender distinction in third person personal pronouns, actor indices and undergoer indices}
    \label{tab:gender_pp_ind}
    \begin{tabularx}{.55\textwidth}{Xll}
        \lsptoprule
         &  +\textsc{human}& --\textsc{human}\\
         \midrule 
         \textsc{feminine}&  \textit{muna}, \textit{mo}, \textit{mi}& \multirow{3}{*}{\textit{ena, i, a}}\\
         \textsc{masculine}&  \textit{una, wo, wi}& \\
         \textsc{plural}&  \textit{ona, yo, 'i}& \\
         \lspbottomrule
    \end{tabularx}
\end{table}

\subsection{Personal pronouns}

The personal pronouns of Modole are displayed in \tabref{tab:pronouns}. They function as expressions of actor (\ref{ex:PP_actor}), undergoer (\ref{ex:PP_undergoer}), and possessor roles (\ref{ex:PP_possessor}), and as predicates (\ref{ex: PP_predicate}). They can co-occur with indices. 

\begin{table}
    \caption{Modole independent pronouns}
    \label{tab:pronouns}
    \begin{tabularx}{.25\textwidth}{ll}
         \lsptoprule
         1\textsc{sg}& \textit{ngoi}\\
         2\textsc{sg}& \textit{ngona}\\
         3\textsc{sg.f}& \textit{muna}\\
         3\textsc{sg.m}& \textit{una}\\
         1\textsc{pl.in}& \textit{ngone}\\
         1\textsc{pl.ex}& \textit{ngomi}\\
         2\textsc{pl}& \textit{ngini}\\
         3\textsc{hpl}& \textit{ona}\\
         3\textsc{nh}& \textit{ena}\\
         \lspbottomrule 
    \end{tabularx}
\end{table}

\begin{exe}
    \ex 
    \begin{xlist}
    \ex 
        \gll de ngoi to hila-u \\
        \textsc{conn} \textsc{pro}.1\textsc{sg} 1\textsc{sg}.\textsc{a} first-\textsc{already} \\
        \trans `then I will [go] forward' [A.\ref{ex:text1-14}]
        \label{ex:PP_actor}
    \ex 
        \gll o jara i na tulu ngone \\
        \textsc{nm} horse 3\textsc{nh}.\textsc{a} 1\textsc{pl.in}.\textsc{u} transport \textsc{pro}.1\textsc{pl.in} \\
        \trans `a horse will pick us (in.) up' [H.\ref{ex:text8-32}]
        \label{ex:PP_undergoer}
    \ex 
        \gll neena to ngona ani lagudi ma howo'o? \\
        \textsc{prox}:\textsc{pro}.3\textsc{nh} \textsc{poss.hum} \textsc{pro}.2\textsc{sg} 2\textsc{sg}.\textsc{poss} legundi \textsc{rnm} fruit \\
        \trans `are these your (sg.) \textit{legundi} fruits?' [A.\ref{ex:text1-94}]
        \label{ex:PP_possessor}
    \ex 
\gll o'ia-no ngini ma nyawa? \\
what-\textsc{dir.ven} \textsc{pro}.2\textsc{pl} \textsc{rnm} person \\
\trans `where have you (pl.) [come from] here?' [F.\ref{ex:text6-150}]
\label{ex: PP_predicate}
    \end{xlist}
\end{exe}

The third person non-human personal pronoun \textit{ena} has additional functions but the full scope is not clear to me. In the example in (\ref{ex:ena_cons}) it seems to separate constituents in a clause, signaling that \textit{do'a} modifies \textit{manga ngio'a} `their place' and not \textit{ma wutu tumudingi} `seven nights'.

\begin{exe}
    \ex
        \gll ma wutu tumudingi ena do'a manga ngi-o'a yo arababu \\
        \textsc{rnm} night seven \textsc{pro}.3\textsc{nh} \textsc{locv} 3\textsc{hpl}.\textsc{poss} place-\textsc{locv} 3\textsc{hpl}.\textsc{a} native.violin \\
        \trans `[after] seven days there at their place they were playing the violin' [C.\ref{ex:text3-38}]
        \label{ex:ena_cons}
\end{exe}

\subsection{Predicate indices}

Actor\footnote{an agent-type role} and undergoer\footnote{a patient-type role} indices are the only obligatory verbal inflection.\footnote{See \citet[80]{naess2007} for an overview on the terms `actor' and `undergoer'.} They precede the verb in the ordering actor--undergoer.

\subsubsection{Actor indices}

The actor indices are displayed in \tabref{tab:actor_indices}.
The label \textit{actor} may be misleading since the referent of the actor index does not always have the semantic role of an agent. It can also be a theme (in locational clauses) or experiencer.\footnote{I chose the labels \textit{actor} and \textit{undergoer} over \textit{subject} and \textit{object} since the data does not allow for subjecthood tests.}

\begin{table}
    \caption{Actor indices}
    \label{tab:actor_indices}
    \begin{tabularx}{.25\textwidth}{ll} 
    \lsptoprule
        1\textsc{sg} & \textit{to} \\ 
        2\textsc{sg} & \textit{no} \\ 
        3\textsc{sg.f} & \textit{mo (mu)}\\ 
        3\textsc{\textsc{sg.m}} & \textit{wo (u)}\\ 
        1\textsc{pl.in} & \textit{po} \\ 
        1\textsc{pl.ex} & \textit{mio (mi)}\\ 
        2\textsc{pl} & \textit{nio (ni)}\\ 
        3\textsc{hpl} & \textit{yo} \\ 
        3\textsc{nh} & \textit{i} \\
    \lspbottomrule 
    \end{tabularx}
\end{table}

The first person plural exclusive and the second person plural indices occur without the final /o/ as \textit{mi} and \textit{ni}, respectively, before the middle marker \textit{ma}: \textit{mi mao'o} `we (ex.) defecate' [B.\ref{ex:text2-64}], \textit{ni ma guguule} `you (pl.) play' [F.\ref{ex:text6-82}] (see \sectref{subsec:middle} on the middle marker).
The third person singular feminine and masculine indices have the allomorphs \textit{mu} and \textit{u}, respectively, before the middle marker \textit{ma}: \textit{de ami utu mu ma bilinganaau} `and she spread out her hair' [B.\ref{ex:text2-53}], \textit{de una u ma iuno'a} `and he hid himself' [E.\ref{ex:text5-113}].

The third person non-human actor index \textit{i} is sometimes used for third person plural human referents if they are continued actors (\ref{ex:continued_actor}).

\begin{exe}
    \ex 
        \gll de yo iha de i lio-o de i temo \\
        \textsc{conn} 3\textsc{hpl}.\textsc{a} \textsc{dir.land} \textsc{conn} 3\textsc{nh}.\textsc{a} return-\textsc{already} \textsc{conn} 3\textsc{nh}.\textsc{a} say \\
        \trans `and they went landwards and they returned and they said' [E.\ref{ex:text5-41}]
    \label{ex:continued_actor}
\end{exe}

Some actor indices change their form when they are combined with certain undergoer indices. This is discussed in \sectref{subsubsec:pronouns_indices-indices-actor_undergoer}.

\subsubsection{Undergoer indices}

The forms of the isolated undergoer indices are given in \tabref{tab:undergoer_indices}. The undergoer indices comprise more semantic roles than Patient, including Recipient and Beneficiary. In terms of grammatical roles, they mark both direct and indirect objects (see \sectref{subsec:ditranstive_clauses}).

\begin{table}
    \caption{Undergoer indices}
    \label{tab:undergoer_indices}
    \begin{tabularx}{.2\textwidth}{ll} 
        \lsptoprule
        1\textsc{sg} & \textit{i} \\ 
        2\textsc{sg}& \textit{ni}\\ 
        3\textsc{sg.f}& \textit{mi}\\ 
        3\textsc{sg.m}& \textit{wi}\\ 
        1\textsc{pl.in} & \textit{na}\\ 
        1\textsc{pl.ex}& \textit{mi}\\ 
        2\textsc{pl}& \textit{ni}\\ 
        3\textsc{hpl}& \textit{'i}\\ 
        3\textsc{nh}& \textit{a}\\ 
        \lspbottomrule
    \end{tabularx}
\end{table}

Undergoer indices (with the exception of the third person plural human and the third person non-human) often occur without actor indices. This is discussed in the next section.

\subsubsection{Actor-undergoer combinations}\label{subsubsec:pronouns_indices-indices-actor_undergoer}

A verb can occur with both actor and undergoer indices. The undergoer either expresses the patient of a transitive verb or the (human) recipient or beneficiary of a ditransitive verb.
In some combinations, one of the two indices has a form different from the index in isolation, and sometimes the two indices have fused. 
The third person plural human actor index \textit{yo} only occurs in combinations if the undergoer is third person plural human as well \textit{yo 'i}. Otherwise, third person plural human actors are marked like third person non-human actors. However, the actor index only surfaces in combinations with first person plural inclusive (\textit{i na}) and third person non-human undergoers (\textit{ya}).

\tabref{tab:actor-undergoer-indices} shows the attested combinations of actor and undergoer indices.
The label \textsc{mid} indicates that the middle marker \textit{ma} (see Section \ref{subsec:middle}) is used instead of an undergoer index. En-dashes symbolize unattested combinations.


\begin{table}
    \caption{Attested actor-undergoer combinations}
    \label{tab:actor-undergoer-indices}
    \fittable{
    \begin{tabular}{llllllllll}
        \lsptoprule
        &  1\textsc{sg}.\textsc{u}&  \textsc{2sg.u}&  3\textsc{sg.f}.\textsc{u}&  3\textsc{sg.m}.\textsc{u}&  1\textsc{pl.in}.\textsc{u}&  1\textsc{pl.ex}.\textsc{u}&  2\textsc{pl}.\textsc{u}&  3\textsc{hpl}.\textsc{u}& 3\textsc{nh}.\textsc{u}\\ \midrule
        1\textsc{sg}.\textsc{a}&  \textsc{mid}&  \textit{to ni} &  \textit{to mi}&  \textit{to wi}&  \textsc{mid}&  \textsc{mid}&  \textit{ti ni}&  \textit{to 'i}& \textit{ta}\\ 
        2\textsc{sg}.\textsc{a}&  \textit{no i}/\textit{ni}&  \textsc{mid}&  \textit{no mi}&  \textit{no wi}&  \textit{no na}& \textit{ni mi} &  \textsc{mid}&  \textit{no 'i}& \textit{na}\\ 
        3\textsc{sg.f}.\textsc{a}&  \textit{mo i}&  --&  --& \textit{mi} &  --&  \textit{mi}&  --&  \textit{mo'i}& \textit{ma}\\ 
        3\textsc{sg.m}.\textsc{a}&  --&  --&  \textit{wi}&  \textit{wi}&  --&  \textit{wi}&  --&  \textit{wo 'i}& \textit{wa}\\ 
        1\textsc{pl.in}.\textsc{a}&  \textsc{mid}&  --&  --&  \textit{po wi}&  \textsc{mid}&  \textsc{mid}&  \textsc{mid}&  \textit{po 'i}& \textit{pa}\\ 
        1\textsc{pl.ex}.\textsc{a}&  \textsc{mid}&  --&  --&  --&  \textsc{mid}&  \textsc{mid}&  \textit{ni mi}&  --& \textit{mia}\\ 
        2\textsc{pl}.\textsc{a}&  \textit{ni}&  \textsc{mid}&  \textit{ni mi}&  --&  --&  \textit{ni mi}&  \textsc{mid}&  \textit{nio 'i}& \textit{nia}\\ 
        3\textsc{hpl}.\textsc{a}&  --&  --&  --&  --&  --&  --&  --&  \textit{yo 'i}& --\\ 
        3\textsc{nh}.\textsc{a}&  --&  \textit{ni}&  \textit{mi}&  \textit{wi}&  \textit{i na}&  \textit{mi}&  \textit{ni}&  --& \textit{ya}\\ 
        \lspbottomrule
    \end{tabular}
    }
\end{table}

As mentioned above, the third person non-human actor index \textit{i} surfaces in combination with first person plural inclusive undergoers (\textit{i na}, (\ref{ex:3nhA_1pl.inU}))\footnote{The same exception is found in Tabaru \citep[262]{holton2008}.} and third person non-human undergoers (\textit{ya} < *i-a, (\ref{ex:3nhA_3nhU})). The same form \textit{ya} is used for third person plural human actors acting on third person non-human undergoers. There is no example of a third person non-human actor acting on a third person plural human undergoer.

\begin{exe}
    \ex 
    \begin{xlist}
    \ex
        \gll o jara i na tulu ngone \\
        \textsc{nm} horse 3\textsc{nh}.\textsc{a} 1\textsc{pl.in}.\textsc{u} transport \textsc{pro}.1\textsc{pl.in} \\
        \trans `a horse will pick us (in.) up' [H.\ref{ex:text8-32}]
    \label{ex:3nhA_1pl.inU}
    \ex 
        \gll ai pihanga o gogunane ya odomo \\
        1\textsc{sg}.\textsc{poss} banana \textsc{nm} ant 3\textsc{nh}>3\textsc{nh} eat \\
        \trans `my bananas [were] eaten by ants' [D.\ref{ex:text4-13}]
    \label{ex:3nhA_3nhU}
    \end{xlist}
\end{exe}

For the second person singular and plural the same undergoer index \textit{ni} is used. However, number is disambiguated by a preceding singular actor index which ends in /o/ if the undergoer is a second person \textit{singular} (e.g. \textit{to ni} `1\textsc{sg}>2\textsc{sg}', see (\ref{ex:toni})) and in /i/ if the undergoer is a second person \textit{plural} (e.g. \textit{ti ni} `1\textsc{sg}>2\textsc{pl}', see (\ref{ex:tini})).

\begin{exe}
    \ex 
    \begin{xlist}
    \ex
        \gll to ni modoa-'a \\
        1\textsc{sg}.\textsc{a} 2\textsc{sg.u} marry-\textsc{lim} \\
        \trans `I will marry you (sg.)' [B.\ref{ex:text2-74}]
        \label{ex:toni}
    \ex 
        \gll ti ni modo'a \\
        1\textsc{sg.a} 2\textsc{pl.u} marry \\
        \trans `I will marry you (pl.)' [D.\ref{ex:text4-83}]
        \label{ex:tini}
    \end{xlist}
\end{exe}

Likewise, third person singular feminine and the first person plural exclusive undergoer referents are both expressed by \textit{mi}. They are only disambiguated with a second person singular actor: \textit{no mi} `2\textsc{sg}>3\textsc{sg.f}' (\ref{ex:nomi}) vs. \textit{ni mi} `2\textsc{sg}>1\textsc{pl.ex}' (\ref{ex:nimi}).

\begin{exe}
    \ex 
    \begin{xlist}
    \ex 
        \gll ua no mi hi-gu-mada \\
        \textsc{proh} 2\textsc{sg.a} 3\textsc{sgf.u} \textsc{caus}-?-abandon \\
        \trans `don't abandon her' [C.\ref{ex:text3-31}]
        \label{ex:nomi}
    \ex 
        \gll uwa ni mi poha \\
        \textsc{proh} 2\textsc{sg.a} 1\textsc{pl.ex.u} hit \\
        \trans `don't strike us (ex.)' [D.\ref{ex:text4-10}]
        \label{ex:nimi}
    \end{xlist}
\end{exe}

The isolated third person singular feminine undergoer index is \textit{mi} (\ref{ex:3sgf.U}) but \textit{mi} also expresses a third person singular feminine actor acting on a third person singular masculine undergoer (\ref{ex:3sgf>3sgm}), or first person plural exclusive undegoer (\ref{ex:3sgf>1pl.ex}). 

\begin{exe}
    \ex 
    \begin{xlist}
    \ex 
        \gll o goda mi haiti-au \\
        \textsc{nm} k.o.spirit 3\textsc{sg.f}.\textsc{u} seize-\textsc{already} \\
        \trans `the \textit{goda} spirit seized her' [C.\ref{ex:text3-29}]
        \label{ex:3sgf.U}
    \ex
        \gll de mi ma'e i tiai \\
        \textsc{conn} 3\textsc{sg.f}>3\textsc{sg.m} see 3\textsc{nh}.\textsc{a} straight \\
        \trans `and she saw him well' [B.\ref{ex:text2-35}]
        \label{ex:3sgf>3sgm}
    \ex 
        \gll nia eha muna dai mu o te{\til}teng-o'a mi hi-dupuhu-ie \\
        \textsc{2pl.poss} mother \textsc{pro.3sgf} \textsc{loc.sea} \textsc{3sgf.a} \textsc{emph} \textsc{rdpl}{\til}alone-\textsc{lim} \textsc{3sgf>1pl.ex} \textsc{caus-}follow\textsc{-dir.up} \\
        \trans `your (pl.) mother is by the sea, she is alone, [and] she will follow us (ex.) up' [F.\ref{ex:text6-169}]
        \label{ex:3sgf>1pl.ex}
    \end{xlist} 
\end{exe}

Likewise, \textit{wi} expresses an isolated third person singular masculine undergoer (\ref{ex:3sgm.U}) but also a third person singular masculine actor acting on a third person singular feminine undergoer (\ref{ex:3sgm>3sgf}), a third person singular masculine undergoer (\ref{ex:3sgm>3sgm}) or a first person plural exclusive undergoer (\ref{ex:3sgm>1pl.excl}).

\begin{exe}
    \ex 
    \begin{xlist}
    \ex %3sgm.U
        \gll ami pahita'a wi da-haba'a una-'a \\
        3\textsc{sg.f}.\textsc{poss} mask 3\textsc{sg.m}.\textsc{u} \textsc{appl-}touch.by.accident \textsc{pro}.3\textsc{sg.m}-\textsc{dir.itv} \\
        \trans `her mask hit him by accident' [E.\ref{ex:text5-87}]
        \label{ex:3sgm.U}
    \ex %3sgm>3sgf
        \gll wi ma'e-o o bere'i mo moi \\
        3\textsc{sg.m}>3\textsc{sg.f} meet-\textsc{already} \textsc{nm} old.person 3\textsc{sg.f}.\textsc{a} one \\
        \trans `he met an old woman' [A.\ref{ex:text1-17}]
        \label{ex:3sgm>3sgf}
    \ex %3sgm>3sgm
        \gll 'o-ne ma lia'a wi tomang-ou ma dodoto \\
        \textsc{emph}-\textsc{prox} \textsc{rnm} older.sibling 3\textsc{sg.m}>3\textsc{sg.m} wake.up-\textsc{already} \textsc{rnm} younger.sibling \\
        \trans `so the older brother woke up the younger one' [H.\ref{ex:text8-36}]
        \label{ex:3sgm>3sgm}
    \ex %3sgm>1pl.excl
        \gll wi paliara hiadono mio lamo'o'au \\
        3\textsc{sg.m}>1\textsc{pl.ex} raise until 1\textsc{pl.ex}.\textsc{a} big:\textsc{perf} \\
        \trans `he raised us (ex.) until we (ex.) had grown up' [F.\ref{ex:text6-161}]
        \label{ex:3sgm>1pl.excl}
    \end{xlist}
\end{exe}
 
The expected combination of a first person plural exclusive actor acting on a second person plural undergoer (1\textsc{pl.ex}>2\textsc{pl}) is **mi ni but the combination exhibits metathesis and only \textit{ni mi} occurs (\ref{ex:1pl.excl>2pl}).\footnote{This was confirmed by a Modole speaker.} The same combination marks the reversed scenario (\ref{ex:2pl>1pl.excl}), a second person singular actor acting on first person plural exclusive undergoer (\ref{ex:2sg>1pl.excl}) and second person plural acting on third person singular feminine (\ref{ex:2pl>3sgf}).

\begin{exe}
    \ex 
    \begin{xlist}
    \ex %1pl.excl>2pl
        \gll ni mi dodoa-wa\footnotemark \\
        2\textsc{pl.u} 1\textsc{pl.ex.a} do-\textsc{neg} \\
        \trans `we (ex.) don't do anything to you (pl.)' [D.\ref{ex:text4-46}]
        \label{ex:1pl.excl>2pl}
    \ex %2pl>1pl.excl
        \gll o gogunane ni mi hi-baumu mia la'o'o ho \\
        \textsc{nm} ant 2\textsc{pl.a} 1\textsc{pl.ex.u} \textsc{caus}-pour 1\textsc{pl.ex}.\textsc{poss} eye \\
        \trans `you (pl.), ants, threw [lime] in our (ex.) eyes' [D.\ref{ex:text4-47}]
        \label{ex:2pl>1pl.excl}
    \ex %2sg>1pl.excl
\gll
        apu uwa ni mi poha \\
        grandmother \textsc{proh} 2\textsc{sg.a} 1\textsc{pl.ex.u} hit \\
        \trans `grandmother, don't strike us' [D.\ref{ex:text4-10}]
        \label{ex:2sg>1pl.excl}
    \ex %2pl>3sgf
        \gll ni mi hi-baiti de ngoi \\
        2\textsc{pl.a} 3\textsc{sgf.u} \textsc{caus}-bury \textsc{conn} \textsc{pro}.1\textsc{sg} \\
        \trans `bury her with me' [I.\ref{ex:text9-22}]
        \label{ex:2pl>3sgf}
    \end{xlist}
\end{exe}
\footnotetext{If \textit{ni mi} was `2\textsc{pl}>1\textsc{pl.ex}' the prohibitive marker \textit{uwa} would be expected.}

\section{Further grammatical elements preceding the verb}

A number of grammatical elements occur between Modole indices and verbs. Some of them are written independently by Ellen.
They can be combined in the following order: \\
\textit{ma} `middle'  -- \textit{hi-} `causative' -- \textit{tu-} (function unknown) -- \textit{dV-} `applicative' -- \textit{'o} `emphatic' -- \textit{a-} `pluractionality' -- \textit{ho-} (function unknown).\footnote{Not all elements are attested together but this is the most likely ordering based on the data.}

It is very likely that the occurrence of these morphemes is governed by the lexical valency and lexical aspect of a verb, e.g. a verb that is lexically intransitive and stative requires certain markers in order to be used as a transitive, dynamic verb. Determining these properties for each verb goes beyond the scope of this grammar sketch and would require a much larger corpus, as well as further elicitation.\footnote{See \citet{holton2008} for the impact of lexical aspect on the indexing system of North Halmahera languages.}

\subsection{Applicative: \textit{dV-}}\label{subsec:applicative}

The vowel in the prefix \textit{dV-} assimilates to the following vowel. Hence, the allomorphs \textit{da}, \textit{de}, \textit{di}, \textit{do} and \textit{du} occur. It increases the valency of verbs by opening an additional slot for an undergoer argument and is hence glossed `applicative' (\textsc{appl}). 
\textit{dV-} occurs with intransitive verbs\footnote{Intransitive verbs are here defined as verbs that occur with only an actor index. I make no claims about the intrinsic valency of these verbs.} when they govern an object (see examples in (\ref{ex:dV-intr})).

\begin{exe}
    \ex 
    \label{ex:dV-intr}
    \begin{xlist}
    \ex 
        \gll ami baili mo da-apu'u \\
        3\textsc{sg.f}.\textsc{poss} garden 3\textsc{sg.f}.\textsc{a} \textsc{appl}-pull.out \\
        \trans `she weeded her garden' [D.\ref{ex:text4-19}]
    \ex 
        \gll de mo da-hano {ma mi} biara \\
        \textsc{conn} 3\textsc{sg.f}.\textsc{a} \textsc{appl}-ask \textsc{rnm}:3\textsc{sg.f}.\textsc{poss} biara \\
        \trans `and she asked for her \textit{biara}' [G.\ref{ex:text7-13}]
    \end{xlist}
\end{exe}

If \textit{dV-} is marked on a transitive verb, its undergoer index refers to an indirect object -- a goal in (\ref{ex:dV_trans_goal})  and a beneficiary in (\ref{ex:dV_trans_ben}) -- not the direct object (see \sectref{subsec:ditranstive_clauses}).

\begin{exe}
    \ex 
    \label{ex:dV-tr}
    \begin{xlist}
    \ex 
        \gll de ma eha mo mau mi di-oma-li \\
        \textsc{conn} \textsc{rnm} mother 3\textsc{sg.f}.\textsc{a} want 3\textsc{sg.f}>3\textsc{sg.m} \textsc{appl}-return-\textsc{again} \\
        \trans `and the mother wanted to return to him' [A.\ref{ex:text1-12}]
        \label{ex:dV_trans_goal}
    \ex 
        \gll de ami huhu ma hongona mi do-towi \\
        \textsc{conn} 3\textsc{sg.f}.\textsc{poss} breast \textsc{rnm} half 3\textsc{sg.f}>3\textsc{sg.m} \textsc{appl}-break \\
        \trans `and she broke off half of her breast for him' [G.\ref{ex:text7-53}]
        \label{ex:dV_trans_ben}
    \end{xlist} 
\end{exe}

In contrast to causative \textit{hi-} and middle \textit{ma}, \textit{dV-} participates in reduplication, for example in \textit{yo ma hi-du{\til}du-tuumo} `3\textsc{hpl.a} \textsc{mid} \textsc{caus-}\textsc{rdpl}{\til}\textsc{appl}-?follow.with.raft' [B.\ref{ex:text2-75}] where the reduplicated syllable /du/ is the applicative prefix.

\subsection{Causative: \textit{hi-}}

The prefix \textit{hi-} seems to be a kind of causative marker (\textsc{caus}) whose main function is to open an additional slot for an actor argument.
This argument does not need to be a causer though since \textit{hi-} also functions as a verbalizer. In the example in (\ref{ex:hi_VBLZ}) where the loanword \textit{bunga} `flower' is used in the meaning `to blossom'.

\begin{exe}
    \ex 
        \gll i hi-bunga o moha, o papago, o haana, o baju \\
        3\textsc{nh}.\textsc{a} \textsc{caus}-flower \textsc{nm} sarong \textsc{nm} k.o.clothing \textsc{nm} pants \textsc{nm} k.o.shirt \\
        \trans `it blossomed sarongs, white shirts, pants, bajus' [G.\ref{ex:text7-64}]
        \label{ex:hi_VBLZ}
\end{exe}

Mostly, \textit{hi-} occurs in prototypical causative situations, marking that the referent of the actor index is the causer (\ref{ex:hi_CAUS}). The causee can optionally be expressed by the undergoer index (\ref{ex:hi_CAUS_U}).

\begin{exe}
    \ex
    \label{ex:hi_CAUS}
    \begin{xlist}
    \ex
        \gll yo hi-ie \\
        3\textsc{hpl}.\textsc{a} \textsc{caus}-\textsc{dir.up} \\
        \trans `they brought [them] up' (lit. `they made [them] go up') [E.\ref{ex:text5-100}, E.\ref{ex:text5-130}] \\
    \ex 
        \gll mi hi-balene o ngootil-u'u \\
        3\textsc{sg.f}.\textsc{u} \textsc{caus}-embark \textsc{nm} proa-\textsc{dir.down} \\
        \trans `they loaded her into the proa' (lit. `they made her embark in the proa') [B.\ref{ex:text2-59}] \\
        \label{ex:hi_CAUS_U}
    \end{xlist}
\end{exe}

%instrumental
The causee can be an instrument (\ref{ex:hi_INST}).

\begin{exe}
    \ex 
    \label{ex:hi_INST}
        \gll de ami bebeoto mo hi-legono \\
        \textsc{conn} 3\textsc{sg.f}.\textsc{poss} knife 3\textsc{sg.f}.\textsc{a} \textsc{caus}-catch \\
        \trans `and she caught [it] with her knife' (lit. `she made her knife catch [it]') [F.\ref{ex:text6-124}] \\
\end{exe}

\subsection{Middle: \textit{ma}}\label{subsec:middle}

The morpheme \textit{ma} takes the slot of the undergoer index and is incompatible with it. Its basic function is to signal that the actor is also the undergoer of the verb. This includes reflexive situations but they are rare in the data. \textit{Ma} also is part of reciprocal marking (see \sectref{subsec:recip}). I gloss \textit{ma} as a middle marker (\textsc{mid}), following \citet{bahrt2021} in calling the syncretism of reflexive and reciprocal `middle'. 
Examples include grooming (\ref{ex:ma_rflx}) and translational motion verbs (\ref{ex:ma_aff}), which commonly occur with middle markers cross-linguistically (see \cite{kemmer1993}).

\begin{exe}
    \ex 
    \begin{xlist}
    \ex 
        \gll gena-'a-de yo ma ohi'i \\
        \textsc{dist}:\textsc{pro.3nh}-\textsc{locv}-\textsc{conn} 3\textsc{hpl}.\textsc{a} \textsc{mid} bathe \\
        \trans `then they went bathing' [A.\ref{ex:text1-44}]
        \label{ex:ma_rflx}
    \ex 
        \gll mu {ma} po{\til}politana ma iha \\
        3\textsc{sg.f}.\textsc{a} \textsc{mid} \textsc{rdpl}{\til}run 3\textsc{sg.f}>3\textsc{nh} \textsc{dir.land} \\
        \trans `she ran landwards' [F.\ref{ex:text6-58}]
        \label{ex:ma_aff}
    \end{xlist}
\end{exe}

While \textit{ma} precludes the occurrence of an undergoer index, direct objects sometimes occur in the form of noun phrases (\ref{ex:ma_DO}).

\begin{exe}
    \ex 
        \gll o dodoto ami babarihi a mu {ma} gaha	\\
        \textsc{nm} younger.sibling 3\textsc{sg.f}.\textsc{poss} broom \textsc{foc} 3\textsc{sg.f}.\textsc{a} \textsc{mid} take \\ 
        \trans `the youngest took her broom with her' [A.\ref{ex:text1-87}]
        \label{ex:ma_DO}
\end{exe}

Sometimes the verb acquires a passive-like reading.\footnote{Reflexive-reciprocal-passive syncretism also falls under Bahrt's (\citeyear{bahrt2021}) middle syncretism.}

\begin{exe}
    \ex 
        \gll ho manga damunu, de manga liwanga a {i} ma hididiotau \\
        thus 3\textsc{hpl}.\textsc{poss} drum \textsc{conn} 3\textsc{hpl}.\textsc{poss} gong \textsc{foc} 3\textsc{nh}.\textsc{a} \textsc{mid} beat \\
        \trans `so their drums and their gongs [were] beaten' [D.\ref{ex:text4-88}]
\end{exe}

\subsection{Reciprocal: \textit{ma- 'a}} \label{subsec:recip}

There are two attestations of the middle marker \textit{ma-} follow by the pluractionality morpheme \textit{'a} in the texts. In combination, these two morphemes have a reciprocal function (\ref{ex:RCIP}). The morpheme \textit{'o-} can be inserted between \textit{ma} and \textit{'a} (see \sectref{subsec:other_prefixes}).

\begin{exe}
    \ex 
    \label{ex:RCIP}
    \begin{xlist}
    \ex 
        \gll o Cina ma nyawa de o Papua-'a ma nyawa yo ma-'a paranga \\
        \textsc{nm} China \textsc{rnm} person \textsc{conn} \textsc{nm} Papua-\textsc{locv} \textsc{rnm} person 3\textsc{hpl}.\textsc{a} \textsc{mid-vpl} fight \\
        \trans `the Chinese and the Papuans fight each other' [J.\ref{ex:text10-14}]
    \ex 
        \gll de a yo ma-'a tili'ut-ou \\
        \textsc{conn} \textsc{foc} 3\textsc{hpl}.\textsc{a} \textsc{mid-vpl} turn.back-\textsc{already} \\
        \trans `and they turned their backs to each other' [E.\ref{ex:text5-35}]
    \end{xlist}
\end{exe}

\subsection{Other Morphemes}\label{subsec:other_prefixes}

A prefix \textit{a-} is attested several times between the root and the causative prefix \textit{hi-} (\ref{ex:a-}). These are probably instances of the morpheme \textit{'a} also found in reciprocal situations (see \sectref{subsec:recip}).
Based on the evidence from other Core North Halmahera languages, I analyze \textsc{pcnh} *ka (from which Modole \textit{'a} is derived) as a marker of pluractionality (glossed \textsc{vpl} for `verbal plural').
In Tabaru, the cognate prefix \textit{ka} (without \textit{ma}) expresses that several objects are attached to each other \citep[379]{fortgens1928}. This function is found in examples (\ref{ex:a-attach1}) and (\ref{ex:a-attach2}). 
Pluractionality may be the function of \textit{'a} in (\ref{ex:a-?}).

\begin{exe}
    \ex 
    \label{ex:a-}
    \begin{xlist}
    \ex 
        \gll neena nia u'u nio hi-{a-}tubele o hepa{\til}hepa, e'ola nio hi-{a-}tubele o namo? \\
        \textsc{prox:pro.3nh} 2\textsc{pl}>3\textsc{nh} \textsc{dir.down} 2\textsc{pl}.\textsc{a} \textsc{caus}-\textsc{vpl}-provoke \textsc{nm} \textsc{rdpl}{\til}kick or 2\textsc{pl}.\textsc{a} \textsc{caus}-\textsc{vpl}-provoke \textsc{nm} chicken \\
        \trans `now you (pl.) go down to play football, or will you (pl.) let the cocks fight?' [A.\ref{ex:text1-57}--\ref{ex:text1-58}]
        \label{ex:a-?}
    \ex 
        \gll ani hahawi po hi-{a-}togo'o \\
       2\textsc{sg}.\textsc{poss} coconut.shell 1\textsc{pl.in}.\textsc{a} \textsc{caus}-\textsc{vpl}-add \\
        \trans `[let]'s (in.) connect your (sg.) coconut shells' [E.\ref{ex:text5-153}]
        \label{ex:a-attach1}
    \ex 
        \gll de wo hi-{a-}dowanga \\
        \textsc{conn} 3\textsc{sg.m}.\textsc{a} \textsc{caus}-\textsc{vpl}-extend \\
        \trans `and he extended [it]' [F.\ref{ex:text6-99}]
        \label{ex:a-attach2}
    \end{xlist}
\end{exe}

The morpheme \textit{'o} in (\ref{ex:ma'o a}) is likely cognate to the element \textit{ko-} found in other Core North Halmahera languages and related to the emphatic marker \textit{'o} in Modole (see \sectref{subsec:'o}). It serves to intensify the function of \textit{ka}/\textit{'a} and is hence glossed \textsc{emph} for `emphatic'. In [F.\ref{ex:text6-169}] it occurs without \textit{ma}.

\begin{exe}
    \ex 
        \gll o banahana ma doa'a ho yo ma-'o a-topo'o \\
        \textsc{nm} banahana \textsc{rnm} ?top thus 3\textsc{hpl}.\textsc{a} \textsc{mid-emph} \textsc{vpl}-stab \\
        \trans `they stab with the tops of \textit{banahana} plants at each other' [E.\ref{ex:text5-76}]
        \label{ex:ma'o a}
\end{exe}

The prefix \textit{ho-} is attested five times ([A.\ref{ex:text1-7}], [A.\ref{ex:text1-9}], [C.\ref{ex:text3-23}], [F.\ref{ex:text6-48}], [F.\ref{ex:text6-158}], see (\ref{ex:ho-})). Its function is unclear. It cannot be an allomorph of \textit{hi-} since it follows the applicative \textit{dV-} while \textit{hi-} always precedes \textit{dV-}.

\begin{exe}
    \ex
    \label{ex:ho-}
        \gll mo do-ho-doa ma tiba ma godau ma tataru-u'u \\
        3\textsc{sg.f}.\textsc{a} \textsc{appl}-?-put \textsc{rnm} bamboo \textsc{rnm} k.o.spirit \textsc{rnm} shade-\textsc{dir.down} \\
        \trans `?she put the bamboo close to the \textit{goda} spirit' [C.\ref{ex:text3-23}]
\end{exe}

The verb \textit{mada} `abandon' is twice ([C.\ref{ex:text3-8}], [C.\ref{ex:text3-31}]) marked with a prefix \textit{gu-} of unclear function.

Two verb forms \textit{tuu'u} `?come down' [A.\ref{ex:text1-55}] (< \textit{tu-u'u}) and \textit{tudu(u)'u} `plant' [G.\ref{ex:text7-58}, G.\ref{ex:text7-59}] (< \textit{tu-du-u'u}) contain an element \textit{tu-} of unknown function.

\vspace{-.3cm}
\section{Noun markers}\label{sec:noun_phrase-noun_markers}

Nouns in Modole are usually preceded by a noun marker. The only context where noun markers are mostly absent is when the noun is preceded by a possessive. The two noun markers are \textit{o} and \textit{ma}. \textit{O} is glossed \textsc{nm} for `noun marker'. \textit{Ma} also functions as the third person non-human possessive and links nouns in relational constructions (see Sections \ref{subsec:possession-possessives} and \ref{subsec:possession-relational_cons}). I therefore always gloss it as \textsc{rnm}, `relational noun marker'.

Roughly speaking, \textit{o} occurs with nouns whose referents are not accessible because they are newly introduced into the narrative. Once the referent is accessible, the noun is marked with \textit{ma}.
In the example in (\ref{ex:noun_markers}), the old woman and the ants are introduced in the first sentence of the story (\ref{ex:noun_markers_a}). Both \textit{bere'i} `old woman' and \textit{gogunane} `ant' are preceded by \textit{o} since they are not accessible. When \textit{bere'i} occurs again in the next sentence (\ref{ex:noun_markers_b}) it is preceded by \textit{ma}, marking its higher accessibility. The ants are only mentioned again several sentences further on in the story (\ref{ex:noun_markers_c}) but \textit{gogunane} is still preceded by \textit{ma} since they have already been introduced and hence are still accessible.
 
\begin{exe}
	\ex 
	\label{ex:noun_markers}
	\begin{xlist}
	\ex 
		\gll naga {o} bere'i moi ami pihanga o gogunane ya odomo \\
		\textsc{exist} \textsc{nm} old.person one 3\textsc{sg.f}.\textsc{poss} banana \textsc{nm} ant 3\textsc{nh}>3\textsc{nh} eat \\
		\trans `there was an old woman [and] her bananas [were] eaten by ants' [D.\ref{ex:text4-2}]
		\label{ex:noun_markers_a}
	\ex 
		\gll {ma} bere'i mo temo \\
		\textsc{rnm} old.person 3\textsc{sg.f}.\textsc{a} say \\
		\trans `the old woman said' [D.\ref{ex:text4-3}]
		\label{ex:noun_markers_b}
	\ex 
		\gll ma dadaao ena, a ma gogunane modidi-o'a i odomo \\
		\textsc{rnm} after \textsc{pro}.3\textsc{nh} \textsc{foc} \textsc{rnm} ant two-\textsc{locv} 3\textsc{nh}.\textsc{a} eat \\
		\trans `and after that two ants ate' [D.\ref{ex:text4-6}]
		\label{ex:noun_markers_c}
	\end{xlist}
\end{exe}

Some referents are intrinsically more accessible than others. Nouns denoting kinship terms usually occur with \textit{ma}, no matter whether their actual referent has been previously introduced or not. 
The example in (\ref{ex:rnm_kinship}) is again the first sentence of a story but \textit{eha} `mother' and \textit{dea} `father' occur with \textit{ma} since they are kinship terms. 

\begin{exe}
	\ex 
		\gll o roeae'a de yo loa ma eha de ma dea \\
		\textsc{nm} riot:\textsc{locv} \textsc{conn} 3\textsc{hpl}.\textsc{a} flee \textsc{rnm} mother \textsc{conn} \textsc{rnm} father \\
		\trans `during a riot they fled, the mother and the father' [A.\ref{ex:text1-2}]
		\label{ex:rnm_kinship}
\end{exe}

\textit{O} does not co-occur with possessives. There are four instances of the sequence <ma mi> and one instance of <mami>, which I analyze as \textit{ma=ami} `\textsc{rnm}=3\textsc{sg.f}.\textsc{poss}', i.e. the relational noun marker \textit{ma} plus third person singular feminine possessive (\ref{ex:ma_mi}). Further, there is one example with the form <m'ai>, probably \textit{ma=ai} `\textsc{rnm}=1\textsc{sg}.\textsc{poss}'.

\begin{exe}
	\ex 
		\gll ena to muna {ma mi} pahita'a 'o'iwa-u \\
		\textsc{pro}.3\textsc{nh} \textsc{poss.hum} \textsc{pro}.3\textsc{sg.f} \textsc{rnm}=3\textsc{sg.f}.\textsc{poss} mask \textsc{neg.exist-foc} \\
		\trans `[and] her mask was gone' [E.\ref{ex:text5-90}]
		\label{ex:ma_mi}
\end{exe}

Another function of \textit{o} is to link nouns denoting materials to the noun they modify (\ref{ex:o_materials}).\footnote{See \citet[410]{fortgens1928} for the same construction in Tabaru.}

\begin{exe}
\ex
\label{ex:o_materials}
    \begin{xlist}
    \ex 
        \gll awi ilingi ma dau-ie o hala'a, dau-u'u o gurahi \\
        3\textsc{sg.m}.\textsc{poss} teeth \textsc{rnm} \textsc{loc.up}-\textsc{dir.up} \textsc{nm} silver \textsc{loc.down}-\textsc{dir.down} \textsc{nm} gold \\
        \trans `his upper teeth (were) silver and his lower (were) gold' [B.\ref{ex:text2-83}]
    \ex
        \gll yo diai o ngotili moi o aharu\\
     	3\textsc{hpl}.\textsc{a}	make	\textsc{nm}	proa	one	\textsc{nm}	stone\\
        \trans `they made one proa of stone' [F.\ref{ex:text6-84}]
    \end{xlist}
\end{exe}

\section{Directionals and Locationals}\label{sec:directionals_locationals}

Modole has two sets of spatial markers, called \textit{directionals} and \textit{locationals}. Their forms are displayed in \tabref{tab:directionals_locationals}. 

\begin{table}
    \caption{Directional and Locational in Modole}
    \label{tab:directionals_locationals}
    \begin{tabularx}{.63\textwidth}{lll}
    \lsptoprule
         &  \textsc{directional}& \textsc{locational}\\ 
         \midrule
         \textsc{up}&  \textit{ie}& \textit{da'u}\\
         \textsc{down}&  \textit{u'u}& \textit{dau(e)}\\
         \textsc{sea}&  \textit{o'o}& \textit{dai}\\
         \textsc{land}&  \textit{iha}& \textit{dina}\\
         \textsc{itive}&  \textit{i'a}& --\\
         \textsc{ventive}&  \textit{ino}& --\\
         general locative&  \textit{o'a}& \textit{do'a}\\
    \lspbottomrule 
    \end{tabularx}
\end{table}

Directionals and locationals both mark four directions or locations along two axes: \textsc{up} vs. \textsc{down} and \textsc{sea} vs. \textsc{land}.\footnote{I use the capitalized terms to indicate the four abstract points of the two axes.} 
\textsc{up} and \textsc{down} can express vertical elevation, e.g. up and down a house on poles, but more commonly they refer to up and down a river or up and down the coast (see \cite{holton2017}). \textsc{up} and \textsc{down} do not match specific cardinal directions (\textit{u'u} `\textsc{dir.down}' is translated both as `northwards' [A.\ref{ex:text1-100}] and `southwards' [A.\ref{ex:text1-51}] by Ellen) but mostly \textit{u'u} is associated with a southerly direction.\footnote{Notably, Pagu \textit{uku} `\textsc{down}' down usually refers to the North and \textit{iye} `\textsc{up}' to the South \citep[91]{peranginangin2018}. This is also the more common association on Halmahera \citep{holton2017}.}
\textsc{sea} and \textsc{land} can refer to the direction towards or away from the coast but also to and away from a river bank.

The directionals express an additional axis: \textsc{itive} (towards the deictic center) vs. \textsc{ventive} (away from the deictic center). The \textsc{itive} and \textsc{ventive} markers \textit{i'a} and \textit{ino} also have temporal functions.
They also function as applicatives in ditransitive clauses (see \sectref{subsec:ditranstive_clauses}).

Finally, there is also a Locative (\textsc{locv}) marker \textit{-o'a} that expresses location without reference to one of the six axis points. The Locative resembles the directionals in that it can be bound to noun phrases. It does not, however, occur as an independent motion verb or bound to verbs. The form of the corresponding locational is \textit{do'a}.



\subsection{Directionals}
\largerpage
Directionals usually express motion in a certain direction (upwards, downwards, seawards, landwards\footnote{``Landwards'' does not always mean that the movement starts at sea and hence includes the notion of `inland'. I use \textit{landwards} in all contexts to keep the translations consistent.}).
They function as
1) directed motion verbs,
2) suffixes on verbs, conveying motion in a direction,
3) suffixes on nouns, conveying that its referent is located in a certain direction,
4) parts of demonstratives (see \sectref{subsec:ge+direc}). The initial vowel is often lost.

As directed motion verbs, the directionals express movement in the respective direction (see examples in (\ref{ex:directionals})). These verbs in their basic form without prefixes are transitive. They occur with the appropriate actor index plus the third person non-human undergoer index.

Source and goal can be optionally expressed by noun phrases serving as adverbials. Source adverbials usually precede the directional (\ref{ex:directionals_source}), while goal adverbials follow it ((\ref{ex:directionals_goal1}), (\ref{ex:directionals_goal2})). The goal adverbial is marked with the locative \textit{-o'a}\footnote{This could also be the \textsc{itive} directional \textit{-i'a}. In other Core North Halmahera languages, forms cognate to \textit{-o'a} are used for this purpose (see \cite[40]{gane2019b} for examples in Loloda) and I hence assume the same for Modole. Also see example (\ref{ex:directionals_verbs_oka1}) below, which clearly shows \textit{-o'a}.} or the appropriate directional.

\begin{exe}
	\ex 
	\label{ex:directionals}
	\begin{xlist}
	\ex 
		\gll ya ie \\
		3pl>3\textsc{nh} \textsc{dir.up} \\
		\trans `they went up' [E.\ref{ex:text5-131}]\footnotemark
	\ex 
		\gll ma {ligihoro-horo} ya u'u \\
		\textsc{rnm} flying.palace 3\textsc{nh}>3\textsc{nh} \textsc{dir.down} \\
		\trans `a flying palace came down' [D.\ref{ex:text4-53}]
	\ex 
		\gll ngoi o a'ele Itio de ta o'o \\
		\textsc{pro}.1\textsc{sg} \textsc{nm} water Itio \textsc{conn} 1\textsc{sg}>3\textsc{nh} \textsc{dir.sea} \\
		\trans `I have come from the Itio River seaward' [A.\ref{ex:text1-25}]
		\label{ex:directionals_source}
	\ex 
		\gll i tugum-ia de ya iha \\
		3\textsc{nh}.\textsc{a} finish-\textsc{dir.itv} \textsc{conn} 3\textsc{hpl}>3\textsc{nh} \textsc{dir.land} \\
		\trans `after that they went landward' [A.\ref{ex:text1-6}]
	\ex 
			\gll de ya i'a ami wo'a-'a \\
		\textsc{conn} 3\textsc{hpl}>3\textsc{nh} \textsc{dir.itv} 3\textsc{sg.f}.\textsc{poss} house-\textsc{locv} \\
		\trans `and they went to her house' [E.\ref{ex:text5-50}] 
		\label{ex:directionals_goal1}
	\ex 
		\gll de ma lagudi ma howo'o wo jaga{\textasciitilde}jaga wa ino mamane-'a \\
		\textsc{conn} \textsc{rnm} legundi \textsc{rnm} fruit 3\textsc{sg.m}.\textsc{a} guard 3\textsc{sg.m}>3\textsc{nh} 		\textsc{dir.ven} lover-\textsc{locv} \\
		\trans `and the \textit{legundi} guardian approached his lover' [A.\ref{ex:text1-76}]
		\label{ex:directionals_goal2}
	\end{xlist}
\end{exe}
\footnotetext{As a directed motion verb, \textit{ie} is only attested in the context of climbing up and down an elevation to a house, i.e. the palace of a king on a hill. \textit{U'u} often denotes the opposite movement down the elevation but occurs in other contexts as well.}

In one case (\ref{ex:directional_undergoer}), an undergoer index different from a third person non-human occurs with a directed motion verb, marking the goal of the motion.

\begin{exe}
	\ex 
		\label{ex:directional_undergoer}
		\gll de mi o'o  \\
		\textsc{conn} 3\textsc{sg.f}>3\textsc{sg.m} \textsc{dir} sea \\
		\trans `and she came seawards to him' [H.\ref{ex:text8-67}]
\end{exe}

As directed motion verbs, the directionals can be suffixed with temporal markers (\ref{ex:directional_ou}). They can also be prefixed with the causative marker \textit{hi-} (\ref{ex:directional_hi}). In this case, no undergoer index occurs, even if there is an implicit undergoer argument.

\begin{exe}
	\ex 
	\begin{xlist}
	\ex 
	\label{ex:directional_ou}
		\gll de yo {hi-o'o-u} \\
		\textsc{conn} 3\textsc{hpl.a} \textsc{caus}-\textsc{dir.sea}-\textsc{already} \\
		\trans `?and they brought [her] seawards' [I.\ref{ex:text9-14}] 
	\ex 
	\label{ex:directional_hi}
		\gll yo 'i ehe, de yo {hi-ie} \\
		3\textsc{hpl}.\textsc{a} 3\textsc{hpl}.\textsc{u} fetch \textsc{conn} 3\textsc{pl}.\textsc{a} \textsc{caus}-\textsc{dir.up} \\
		\trans `they fetched them and they brought [them] up' [E.\ref{ex:text5-100}]
	\end{xlist}
\end{exe}

Directionals also function as suffixes on undirected motion verbs.\footnote{I define \textit{motion verb} in the broadest sense here, meaning a verb that entails some kind of movement along a path, whether by the actor or some other entity.} Directionals specify the direction of motion (see examples in (\ref{ex:directionals_verbs})).
The goal of the motion can be marked with the locative \textit{-o'a} (see examples (\ref{ex:directionals_verbs_oka1}) and (\ref{ex:directionals_verbs_oka2})).

\begin{exe}
    \ex 
    \label{ex:directionals_verbs}
    \begin{xlist}
    \ex 
        \gll gena-'a-de mo hi-leot{-ie} o gohapanat-o'a \\ 
        \textsc{dist:pro.3nh-locv-conn} 3\textsc{sg.f}.\textsc{a} \textsc{caus}-supine-\textsc{dir.up} \textsc{nm} gohapanata-\textsc{locv} \\
        \trans `then she placed [it] up in a \textit{gohapanata} tree' [B.\ref{ex:text2-8}]
        \label{ex:directionals_verbs_oka1}
    \ex 
        \gll de yo tagi ho nge'omo ma oa de yo ma gogel{-u'u} \\
        \textsc{conn} 3\textsc{hpl}.\textsc{a} go thus way \textsc{rnm} ?end \textsc{conn} 3\textsc{hpl}.\textsc{a} \textsc{mid} sit-\textsc{dir.down} \\
        \trans `and they went and at the end of the path they sat down' [E.\ref{ex:text5-78}]
    \ex 
        \gll ho yo i ma'e{-iha} ma 'oana o wange ma dumun-o'a awi moholehe yo tumudingi \\
        thus 3\textsc{hpl}.\textsc{a} 3\textsc{hpl}.\textsc{u} see-\textsc{dir.land} \textsc{rnm} king \textsc{nm} sun \textsc{rnm} dive-\textsc{locv} 3\textsc{sg.m}.\textsc{poss} maiden 3\textsc{hpl}.\textsc{a} seven \\
        \trans `and landward they met seven maidens of the Western King' [A.\ref{ex:text1-40}]
    \ex 
        \gll ani hahawi po hi-a-tog{-o'o}\footnotemark \\
        2\textsc{sg}.\textsc{poss} coconut.shell 1\textsc{pl.in}.\textsc{a} \textsc{caus}-\textit{a-}connect-\textsc{dir.sea} \\
        \trans `[let]'s (in.) connect your (sg.) coconut shells' [E.\ref{ex:text5-153}] 
    \ex 
        \gll de o wo'a ma homoa{-'a} wi tinga{-i'a} \\
        \textsc{conn} \textsc{nm} house \textsc{rnm} other-\textsc{locv} 3\textsc{sg.m}.\textsc{u} separate-\textsc{dir.itv} \\
        \trans `and they separated him in another house' [B.\ref{ex:text2-78}]
        \label{ex:directionals_verbs_oka2}
    \ex 
        \gll de mo lio{-no} \\
        \textsc{conn} 3\textsc{sg.f}.\textsc{a} return-\textsc{dir.ven} \\
        \trans `and she returned home' [D.\ref{ex:text4-20}]
    \end{xlist}
\end{exe}
\footnotetext{I cannot rule out that the root is actually \textit{togo'o}.}

Lastly, directionals occur as suffixes on nouns. The noun then refers to the goal or location of the motion (see examples in (\ref{ex:directionals_nouns})).

\begin{exe}
    \ex 
    \label{ex:directionals_nouns}
    \begin{xlist}
    \ex 
        \gll ma wali{-ie} ena ona do'a ena-u \\
        \textsc{rnm} door-\textsc{dir.up} \textsc{pro.}3\textsc{nh} \textsc{pro}.3\textsc{hpl} \textsc{locv} \textsc{pro.3nh-foc} \\
        \trans `?they [came] up to the door' [C.\ref{ex:text3-41}] 
    \ex 
        \gll ho ma aunu ami giam{-u'u} \\
        thus \textsc{rnm} blood 3\textsc{sg.f}.\textsc{poss} hand-\textsc{dir.down} \\
        \trans `so blood [dripped] down on her hand' [F.\ref{ex:text6-125}] 
    \ex 
        \gll ena ma heleo{-iha} o wo'a o utu moi \\
        \textsc{pro.}3\textsc{nh} \textsc{rnm} stone-\textsc{dir.land} \textsc{nm} house \textsc{nm} \textsc{class} one \\
        \trans `there [on] the landward stones was one house' [G.\ref{ex:text7-49}]
    \ex 
        \gll de {ma moi} ma bere'i mo tagi ami bail{-i'a} \\
        \textsc{conn} once \textsc{rnm} old.person 3\textsc{sg.f}.\textsc{a} go 3\textsc{sg.f}.\textsc{poss} garden-\textsc{dir.itv} \\
        \trans `and once the old woman went to her garden' [D.\ref{ex:text4-18}]
    \ex 
\gll de ma holoibi ami hae'-ino i boa \\
\textsc{conn} \textsc{rnm} hummingbird 3\textsc{sg.f}.\textsc{poss} head-\textsc{dir.ven} 3\textsc{nh}.\textsc{a} come \\
\trans `and a hummingbird came here to her head' [F.\ref{ex:text6-124}]
    \end{xlist}
\end{exe}

Direction can be expressed both on the noun and as a directed motion verb within the same clause (\ref{ex:directionals_noun+verb}). Notice that the goal adverbial \textit{o a'el{-u'u}} `down the water' is marked with \textit{-u'u} instead of \textit{-o'a} and precedes the verb instead of following it.

\begin{exe}
    \ex 
        \label{ex:directionals_noun+verb}
        \gll de o a'el{-u'u} yo uti{-o'u} \\
        \textsc{conn} \textsc{nm} water-\textsc{dir.down} 3\textsc{hpl}.\textsc{a} descend-\textsc{dir.down} \\
        \trans `and they descended into the water' [E.\ref{ex:text5-115}] 
\end{exe}

\subsection{Locationals}

Locationals denote places in four directions (at sea, on the land side, etc.) and mostly function as adverbs.
The orthography of the texts does not always distinguish between the \textsc{down} locational \textit{dau}  (< \textsc{pcnh} *dahu) and the \textsc{up} locational \textit{da'u}  (< \textsc{pcnh} *daku). In the example in (\ref{ex:dau}) both are spelled \textit{dau}.

\begin{exe}
    \ex 
        \gll awi ilingi ma dau-ie o hala'a, dau-u'u o gurahi \\
        3\textsc{sg.m}.\textsc{poss} teeth \textsc{rnm} \textsc{loc.up}-\textsc{dir.up} \textsc{nm} silver \textsc{loc.down}-\textsc{dir.down} \textsc{nm} gold \\
        \trans `his upper teeth (were) silver and his lower (were) gold' [B.\ref{ex:text2-83}]
        \label{ex:dau}
\end{exe}

The form \textit{daue} occurs several times in Story 4 ([D.\ref{ex:text4-63},\ref{ex:text4-67},\ref{ex:text4-70},\ref{ex:text4-73},\ref{ex:text4-76},\ref{ex:text4-80},\ref{ex:text4-84}]). I do not know what the function of this form is in comparison to the \textsc{down} locational \textit{dau} but notice that Galela features proximate locationals with final /e/ \citep[148,150]{shelden1991a} and that \textit{daue} is translated as `hier beneden' (`down here') in Ellen's (\citeyear{ellen1916b}) Modole wordlist. The form \textit{dauwe} that occurs in [B.\ref{ex:text2-68}] may be a spelling variant of \textit{daue}.

Bare locationals function as adverbials (see examples in (\ref{ex:locationals_adverbs})).\footnote{The landwards locational \textit{dina} is not attested in this function in the texts.}

\begin{exe}
    \ex 
    \label{ex:locationals_adverbs}
    \begin{xlist}
    \ex 
        \gll de {da'u} ma tubu{\til}tubu-o'a-u de o borua moi i hidel-u'u \\
        \textsc{conn} \textsc{loc.up} \textsc{rnm} top{\til}top-\textsc{locv}-\textsc{foc} \textsc{conn} \textsc{nm} chest one 3\textsc{nh}.\textsc{a} hang-\textsc{dir.down} \\
        \trans `and from up at the top hung down a chest' [G.\ref{ex:text7-69}]
    \ex 
        \gll dau manga baiti ma doda'a 'a yo ma bangiheli \\
        \textsc{loc.down} 3\textsc{hpl}.\textsc{poss} grave \textsc{rnm} inside \textsc{foc} 3\textsc{hpl}.\textsc{a} \textsc{mid} flute \\
        \trans `down inside their grave they played the flute' [I.\ref{ex:text9-26}]
    \ex 
        \gll dai o a'el-o'a mi ma guguule ho \\
        \textsc{loc.sea} \textsc{nm} water-\textsc{locv} 1\textsc{pl.ex}.\textsc{a} \textsc{mid} play thus \\
        \trans `seawards by the water we (ex.) will play' [F.\ref{ex:text6-81}]
    \ex 
        \gll do'a manga ngi-o'a yo arababu \\
        \textsc{locv} 3\textsc{hpl}.\textsc{poss} place-\textsc{locv} 3\textsc{hpl}.\textsc{a} native.violin \\
        \trans `there at their place they were playing the violin' [C.\ref{ex:text3-38}]
    \end{xlist}
\end{exe}

There are two examples of locationals followed by the third person non-human pronoun \textit{ena} in adnominal function (\ref{ex:locational+ena}). This is reminiscent of the demonstratives \textit{ge-ena} and \textit{ne-ena} (see \sectref{subsec:neena_geena}).
Indeed, \textit{dau ena} functions as a adnominal demonstrative in (\ref{ex:dau_ena}). The function of \textit{do'a ena-u} in (\ref{ex:doka_enau}) is unclear.

\begin{exe}
    \ex 
    \label{ex:locational+ena}
    \begin{xlist}
    \ex 
        \gll o gogunane dau ena ai pihanga ya odo{\til}odomo \\
        \textsc{nm} ant \textsc{loc.down} \textsc{pro}.3\textsc{nh} 1\textsc{sg}.\textsc{poss} banana 3\textsc{hpl}>3\textsc{nh} \textsc{rdpl}{\til}eat \\
        \trans `the ants down here are eating my bananas' [D.\ref{ex:text4-8}]
        \label{ex:dau_ena}
    \ex 
        \gll ma wali{-ie} ena ona do'a ena-u \\
        \textsc{rnm} door-\textsc{dir.up} \textsc{pro.}3\textsc{nh} \textsc{pro}.3\textsc{hpl} \textsc{locv} \textsc{pro}.3\textsc{nh}-\textsc{foc} \\
        \trans `?they [came] up to the door' [C.\ref{ex:text3-41}] 
        \label{ex:doka_enau}
    \end{xlist}
\end{exe}

\subsection{Combinations of locationals and directionals}\label{subsec:locationals+directionals}

Directionals can be suffixed to locationals to form complex demonstratives (\ref{ex:locational+directionals_demonstratives}) and adverbials (\ref{ex:locationals+demonstratives_adverbials}). 

\begin{exe}
    \ex 
    \label{ex:locational+directionals_demonstratives}
    \begin{xlist}
    \ex 
        \gll awi ilingi ma {dau-ie} o hala'a  \\
        3\textsc{sg.m}.\textsc{poss} tooth \textsc{rnm} \textsc{loc.up}-\textsc{dir.up} \textsc{nm} silver \\
        \trans `his upper teeth (were) silver' [B.\ref{ex:text2-83}]
    \ex 
        \gll {dau-u'u} o gurahi \\
        \textsc{loc.down}-\textsc{dir.down} \textsc{nm} gold \\
        \trans `the lower [teeth] were gold' [B.\ref{ex:text2-83}]
    \end{xlist}
\end{exe}

%spatial adverb
\begin{exe}
    \ex 
    \label{ex:locationals+demonstratives_adverbials}
    \begin{xlist}
    \ex
        \gll {dau-ie} yo i hi-duba \\
        \textsc{loc.up}-\textsc{dir.up} 3\textsc{hpl}.\textsc{a} 3\textsc{hpl}.\textsc{u} \textsc{caus}-fall \\
        \trans `from above they fell' [D.\ref{ex:text4-55}]
    \ex 
        \gll ena {dau-ie} yo tutu'u \\
        \textsc{pro}.3\textsc{nh} \textsc{loc.up}-\textsc{dir.up} 3\textsc{hpl}.\textsc{a} arrive \\
        \trans `when they arrived upwards' [E.\ref{ex:text5-73}]
    \ex 
        \gll {dina-o'o} \\
        \textsc{loc.land}-\textsc{dir.sea} \\
        \trans `[we've come] from the land seawards' [F.\ref{ex:text6-145}]
    \end{xlist}
\end{exe}

In a number of examples (\ref{ex:locationals+directionals_ng}), locationals and directionals are separated by <ng> (glossed \textsc{n}). These may be mutated forms of the directionals (see \sectref{subsec:onset_mutation}). Their function is unclear.\footnote{According to \citet[41]{yoshida1980}, these forms are proximate in Tabaru (but see \cite[104]{taylor1984a}).}

\begin{exe}
    \ex
    \label{ex:locationals+directionals_ng}
    \begin{xlist}
    \ex 
        \gll 'o-ne ani 'o'ata {dae-ngo'o} wo uti ja'o ma wehara \\
        \textsc{emph}-\textsc{prox} 2\textsc{sg}.\textsc{poss} man \textsc{loc.sea}-\textsc{n}:\textsc{dir.down} 3\textsc{sg.m}.\textsc{a} descend nature \textsc{rnm} ugly \\
        \trans `like this your (sg.) husband ?[shall] descend seawards [with] his ugly appearance' [F.\ref{ex:text6-10}]
    \ex 
        \gll abei'a {dauwe-ngo'o} mi ma hi-dudutuumo \\
        well \textsc{loc.down}-\textsc{n:dir.sea} 1\textsc{pl.ex}.\textsc{a} \textsc{mid} \textsc{caus}-take.a.raft \\
        \trans `?come, [let]'s (ex.) follow the river with a raft down to the sea' [B.\ref{ex:text2-68}] 
    \ex 
        \gll ma dodoao ena {dai-ngiha} manga hidete 'a i hori{\til}hori \\
        \textsc{rnm} after \textsc{pro}.3\textsc{nh} \textsc{loc.sea}-\textsc{n}:\textsc{dir.land} 3\textsc{hpl}.\textsc{poss} sail \textsc{foc} 3\textsc{nh}.\textsc{a} near \\
        \trans `and following that their sail came closer and closer landwards from the sea' [F.\ref{ex:text6-101}]
    \ex 
        \gll ma ma heli de o upa~upaha ho o {doa-ngu'u} aha i-tubu'u \\
        but \textsc{rnm} spout \textsc{conn} \textsc{nm} bamboo.joint thus \textsc{emph} \textsc{locv}-\textsc{n}:\textsc{dir.down} as.a.consequence 3\textsc{nh}.\textsc{a}-top:\textsc{dir.down} \\
        \trans `but the spout [he let] through a bamboo joint and so it dripped down far away' [F.\ref{ex:text6-117}--\ref{ex:text6-118}]
    \end{xlist}
\end{exe}

\subsection{Temporal usage of directionals}

In combination with certain nouns and verbs, \textsc{ventive} and \textsc{itive} directionals \textit{ino} and \textit{i'a} form temporal adverbials.  \textit{I'a} is attested with this function in isolation. These forms serve to shift the narrative time to a point after the previous events.

\textit{Ino} is amply attested with \textit{duanga} `finish'. \textit{Duangino} preceded by the relational noun marker \textit{ma} or the third person non-human actor index \textit{i} means `afterwards' or `after that' (see examples in (\ref{ex:duangino})).

\begin{exe}
    \ex 
    \label{ex:duangino}
    \begin{xlist}
    \ex 
        \gll de yo odomo ma duang-ino i wutu-oau \\
        \textsc{conn} 3\textsc{hpl}.\textsc{a} eat \textsc{rnm} finish-\textsc{dir.ven} 3\textsc{nh}.\textsc{a} night-\textsc{perf} \\
        \trans `and after they had eaten, it had become night' [E.\ref{ex:text5-52}]
    \ex 
        \gll de yo odomo; ho i duang-ino de wo tagi o 'aho wo aho'o \\
        \textsc{conn} 3\textsc{hpl}.\textsc{a} eat thus 3\textsc{nh}.\textsc{a} finish-\textsc{dir.ven} \textsc{conn} 3\textsc{sg.m}.\textsc{a} go \textsc{nm} dog 3\textsc{sg.m}.\textsc{a} call \\
        \trans `and they ate; and after that he went [and] called the dogs' [F.\ref{ex:text6-73}--\ref{ex:text6-74}]
    \end{xlist}
\end{exe}

In isolation, \textit{i'a} means `then, later' (\ref{ex:i'a}). 

\begin{exe}
    \ex 
    \label{ex:i'a}
        \gll de mi hi-huhu-oa-hi ma bilanga; {i'a} 'o-geena awi ngomaha i dudungu \\
        \textsc{conn} 3\textsc{sg.f}>3\textsc{sg.m} \textsc{caus}-breast-\textsc{lim}-\textsc{still} \textsc{rnm} sister \textsc{dir.itv} \textsc{emph}-\textsc{dist:pro.3nh} 3\textsc{sg.m}.\textsc{poss} throat 3\textsc{nh}.\textsc{a} dry \\
        \trans `and she, the sister, nursed him first; later his throat was dry [again]' [G.\ref{ex:text7-46}--\ref{ex:text7-47}]
\end{exe}

Frequently, \textit{i'a} is combined with \textit{togumu} `finish' to mean `after that' (\ref{ex:togumi'a}).

\begin{exe}
    \ex 
    \label{ex:togumi'a}
        \gll ma a mo olu'u; i togum-ia-de wo uti-ou \\
        but \textsc{foc} 3\textsc{sg.f}.\textsc{a} refuse 3\textsc{nh.a} finish-\textsc{dir.itv}-\textsc{conn} 3\textsc{sg.m}.\textsc{a} descend-\textsc{already} \\
        \trans `but she refused, after that he descended' [B.\ref{ex:text2-31}--\ref{ex:text2-32}]
\end{exe}

\section{Other suffixes}
\label{sec:other_suffixes}

This section deals with a number of suffixes that occur with verbs, nouns and other parts of speech but that do not belong to the spatial domain:\footnote{Note that I use `suffix' for elements that are written conjoined with the preceding word.}
`Already'/Focus \textit{-ou}, Limitative \textit{-o'a}, Perfect \textit{-o'au}, \textit{-ohi} `still, first' and \textit{-oli} `again, too'. All these suffixes share a similar phonological form and often lose their initial <o> when attached to a host.

In combination with the the negator \textit{-ua}, \textit{-ua-ou} means `not anymore' (lit. `already not') and \textit{-ua-ohi} means `not yet' (lit. `still not'). In this way, Modole shows features of a phasal polarity system (see \cite{kramer2017}).

\subsection{`Still, first': \textit{-ohi}}

The suffix \textit{-ohi} (also spelled <hi> and <ahi>) usually means `still' (\ref{ex:ohi_still}) but can also mean `first', especially in imperative contexts (\ref{ex:ohi_first}).\footnote{The co-lexification of `still' and `first' is very common cross-linguistically, see \citet[112]{persohn2024}.}
I always gloss \textit{-ohi} as \textsc{still}.

\begin{exe}
    \ex
    \begin{xlist}
    \ex
    \label{ex:ohi_still}
        \gll gena-'a de yo tagi ho o de'u tumudingi de o ngaili tumudingi ya paha,  ma a yo ihen{-ohi} wo ali \\
        \textsc{dist:pro.3nh-locv} \textsc{conn} 3\textsc{hpl}.\textsc{a} go thus \textsc{nm} mountain seven \textsc{conn} \textsc{nm} river seven 3\textsc{nh}>3\textsc{nh} leave but \textsc{foc} 3\textsc{hpl}.\textsc{a} hear-\textsc{still} 3\textsc{sg.m}.\textsc{a} cry \\
        \trans `then they went, over seven hills and seven rivers they left, but they still heard him cry' [A.\ref{ex:text1-10}--\ref{ex:text1-11}]
    \ex 
    \label{ex:ohi_first}
        \gll na'o no i modo'a-de, halingou ani halalomu moi to wi topo'o{-ahi} \\
        \textsc{cond} 2\textsc{sg}.\textsc{a} 1\textsc{sg}.\textsc{u} marry-\textsc{conn} necessary 2\textsc{sg}.\textsc{poss} slave one 1\textsc{sg}.\textsc{a} 3\textsc{sg.m}.\textsc{u} stab-\textsc{still} \\
        \trans `if you (sg.) want to marry me, I first must stab one of your (sg.) slaves [dead]' [B.\ref{ex:text2-75}]
    \end{xlist}
\end{exe}

%Extended function `more'
In one example, \textit{-ohi} means `more' (\ref{ex:ohi_more}; compare Dutch `Kom, \textit{nog} (`still') wat landwaarts').

\begin{exe}
    \ex 
    \label{ex:ohi_more}
        \gll abei'a iha{-hi} \\
        well \textsc{dir.land}-\textsc{still} \\
        \trans  `come, a little bit more landwards' [B.\ref{ex:text2-57}] \\
\end{exe}

%with nouns
In combination with a personal pronoun, \textit{-ohi} is attested in the meaning `first' (\ref{ex:ohi_pronoun}).

\begin{exe}
    \ex 
    \label{ex:ohi_pronoun}
        \gll de muna{-hi} mu ma gowo \\
        \textsc{conn} \textsc{pro}.3\textsc{sg.f}-\textsc{still} 3\textsc{sg.f}.\textsc{a} \textsc{mid} wash.hair.with.coconut.milk \\
        \trans `and she first washed and oiled her hair' [F.\ref{ex:text6-20}]
\end{exe}

\subsection{`Not yet' \textit{-ua-ohi}}

The negator \textit{-ua} and \textit{-ohi} combined express `not yet'. There is no example of this combination marked on a verb.
Together with the emphatic marker \textit{'o} , \textit{-ua-ohi} forms an independent exclamation `not yet!' (\ref{ex:ouahi}).

\begin{exe}
    \ex 
    \label{ex:ouahi}
        \gll {o-ua-hi} o ta bot-ua-hi \\
        \textsc{emph}-\textsc{neg}-\textsc{still} \textsc{emph} 1\textsc{sg}>3\textsc{nh} finish-\textsc{neg}-\textsc{still} \\
        \trans `not yet, I'm not yet done with it' [D.\ref{ex:text4-22}]
\end{exe}

The suffix \textit{-hi} is once attested with the prohibitive \textit{uwa}, also meaning `not yet'.

\begin{exe}
    \ex 
        \gll bere'i uwa-hi mia ngagootili mio diai-ohi ho \\
        old.person \textsc{proh}-\textsc{still} 1\textsc{pl.ex}.\textsc{poss} proa 1\textsc{pl.ex}.\textsc{a} finish-\textsc{still} thus \\
        \trans `old woman, not yet, we (ex.) are still making our (ex.) proa' [F.\ref{ex:text6-88}]
\end{exe}

\subsection{`Already' and focus: \textit{-ou}}

The analysis of \textit{-ou} (also spelled <u>, <o> and <au>) is difficult.\footnote{My attempts to shed light on the function of \textit{-ou} in the texts with the help of a Modole native speaker proved unsuccessful since its usage has changed a lot in the past 100 years. For the modern speaker, \textit{-ou} is only a prospective marker that translates Indonesian \textit{hampir} `almost'. It apparently no longer occurs in any of the other verbal functions discussed here.}

When marked on verbs, \textit{-ou} expresses anteriority (but not completion) of events, especially in a series of events; and it conveys a prospective reading, especially in direct speech. These two, seemingly contrary, functions are commonly found expressed by the same markers in languages of Southeast Asia and Indonesia \citep{olsson2013}. I gloss \textit{-ou} as \textsc{already}. I provide a more extensive discussion of its functions below.
With nouns, \textit{-ou} functions as a focus marker (glossed \textsc{foc}. The functional difference of \textit{-ou} marked on verbs and nouns may go back to the same function in an earlier stage of the language but I keep them apart synchronically.

Often, \textit{-ou} occurs on verbs marking a series of events building up on each other. It then expresses anteriority of previous events. In the example in (\ref{ex:ou_anteriority}), the first event (the king firing a cannon) is marked with \textit{-ou} (\ref{ex:ou_anteriority_a}) while the next event (the older sisters returning) is not (\ref{ex:ou_anteriority_b}). The event of the older sisters' return is expressed again in a tail-head-linkage--like construction but this time the verb \textit{lio} `return' is marked with \textit{-ou} (\ref{ex:ou_anteriority_c}). The next event (the old woman asking) is again not marked with \textit{-ou} (\ref{ex:ou_anteriority_d}). As in this example, the last event in the event series commonly introduces direct speech, i.e. \textit{temo} `say, speak'.

\begin{exe}
    \ex 
    \label{ex:ou_anteriority}
    \begin{xlist}
    \ex
        \gll ena ma 'oana o hupera wo tu'u{-au} \\
        \textsc{pro.}3\textsc{nh} \textsc{rnm} king \textsc{nm} cannon 3\textsc{sg.m}.\textsc{a} fire-\textsc{already} \\
        \trans `then the king fired a cannon' [E.\ref{ex:text5-91}]
        \label{ex:ou_anteriority_a}
    \ex
        \gll de ona ma lia'a yo lio \\
        \textsc{conn} \textsc{pro.}3\textsc{hpl} \textsc{rnm} older.sibling 3\textsc{hpl}.\textsc{a} return \\
        \trans `and they, the older sisters, returned' [E.\ref{ex:text5-92}]
        \label{ex:ou_anteriority_b}
    \ex 
        \gll ona ma a lio{-au} \\
        \textsc{pro.}3\textsc{hpl} \textsc{rnm} \textsc{foc} return-\textsc{already} \\
        \trans `they had returned' [E.\ref{ex:text5-93}]
        \label{ex:ou_anteriority_c}
    \ex 
        \gll De ma bere'i mo temo \\
        \textsc{conn} \textsc{rnm} old.person 3\textsc{sg.f}.\textsc{a} say \\
        \trans `And the old woman said' [E.\ref{ex:text5-94}]
        \label{ex:ou_anteriority_d}
    \end{xlist}
\end{exe}

However, \textit{-ou} does not always mark anteriority. In one example, displayed in (\ref{ex:MOD_ou_simul}), \textit{wo tagiou} `he went' is preceded by \textit{de waktu} `when', expressing simultaneity of events. Analyzing \textit{-ou} as an anteriority marker is hence not appropriate.

\begin{exe}
    \ex 
        \gll de waktu wo tagi{-ou} ma lia'a wa hoana ma dodoto awi {ali{\textasciitilde}ali} \\
        \textsc{conn} time 3\textsc{sg.m}.\textsc{a} go-\textsc{already} \textsc{rnm} older.sibling 3\textsc{sg.m}>3\textsc{nh} ?take.away \textsc{rnm} younger.sibling 3\textsc{sg.m}.\textsc{poss} ring \\
        \trans `and when the older brother went, he took away the rings of the younger one' [H.\ref{ex:text8-38}]
        \label{ex:MOD_ou_simul}
\end{exe}

The suffix \textit{-ou} can express completion (\ref{ex:ou_compl}) but it does not entail it. In the example in (\ref{ex:ou_not_compl}), the event expressed by the verb marked with \textit{-ou}, \textit{wi tomangou} `he woke him up', fails and is hence uncompleted. Therefore, \textit{-ou} cannot be analyzed as a perfective marker.

%-ou as completion
\begin{exe}
    \ex 
        \gll bere'i, 'aano neena mia iha de yo rari a {ma moi} de a i ma hi-dapat-ou \\
        old.person a.moment.ago \textsc{prox}:\textsc{pro}.3\textsc{nh} 1\textsc{pl.ex}>3\textsc{nh} \textsc{dir.land} \textsc{conn} 3\textsc{hpl.a} weed \textsc{foc} once \textsc{conn} \textsc{foc} 3\textsc{nh.a} \textsc{mid} \textsc{caus}-?open-\textsc{already} \\
        \trans `old man, a moment ago we (ex.) went landwards and they weeded and at once it was done' [D.\ref{ex:text4-32}--\ref{ex:text4-33}]
        \label{ex:ou_compl}
\end{exe}

%ou without completion
\begin{exe}
    \ex 
        \gll 'o-ne ma lia'a wi tomang{-ou} ma dodoto, ma wo maoa \\
         \textsc{emph-prox} \textsc{rnm} older.sibling \textsc{3sgm>3sgm} wake.up-\textsc{already} \textsc{rnm} younger.sibling but 3\textsc{sg.m}.\textsc{a} wake.up:\textsc{neg} \\
        \trans `so the older brother woke up the younger one, but he did not wake up' [H.\ref{ex:text8-36}]
        \label{ex:ou_not_compl}
\end{exe}

In other examples still, \textit{-ou} has a prospective reading (\ref{ex:ou_prospective}). In the example, the older brother proposes to his younger brother that they should return home but the younger brother does not end up doing so.
The prospective function of \textit{-ou} is only found in direct speech while the anteriority function is present both in narration and direct speech. 

\begin{exe}
    \ex 
        \gll de ma lia'a wo temo: ``na {inou} po {liou}.'' \\
        \textsc{conn} \textsc{rnm} older.sibling 3\textsc{sg.m}.\textsc{a} say 2\textsc{sg}>3\textsc{nh} \textsc{dir.ven}:\textsc{already} 1\textsc{pl.in} return:\textsc{already} \\
        \trans `and the older one said: ``Come here, [let]'s (in.) return!'' ' [E.\ref{ex:text5-33}] %(also [E.\ref{ex:text5-33}])
        \label{ex:ou_prospective}
\end{exe}

When occurring on nouns and pronouns, \textit{-ou} seems to mark focus (\ref{ex:ou_focus_noun}). 
Sometimes, verbs marked with \textit{-ou} are in focus as well (\ref{ex:ou_focus_verb}). However, there are lots of examples for verbs in focus without \textit{-ou} (see [B.\ref{ex:text2-40}--\ref{ex:text2-41}]). 

\begin{exe}
    \ex 
    \begin{xlist}
    \ex 
        \gll na'o  o geena de ngoi-ou \\
        \textsc{cond} \textsc{emph} \textsc{dist}:\textsc{pro}.3\textsc{nh} \textsc{conn} \textsc{pro}.1\textsc{sg}-\textsc{foc} \\
        \trans `if that's the case, it's me' [E.\ref{ex:text5-156}]
        \label{ex:ou_focus_noun}
    \ex
        \gll na'o no wi dioma-hi de ngoi to hila-u \\
        \textsc{cond} 2\textsc{sg}.\textsc{a} 3\textsc{sg.m}.\textsc{u} return-\textsc{still} \textsc{conn} \textsc{pro}.1\textsc{sg} 1\textsc{sg}.\textsc{a} first-\textsc{already} \\
        \trans `if you (sg.) return to him, then I will [go] forward' [A.\ref{ex:text1-14}] 
        \label{ex:ou_focus_verb}
    \end{xlist} 
\end{exe}


\subsection{`Anymore': \textit{-ua-ou}}

Parallel to `not yet', which is formed as the negation of \textit{-ohi} `still', `anymore' is expressed by the negation of \textit{-ou} (\ref{ex:wau}). The portmanteau suffix is usually spelled <wau>.

\begin{exe}
    \ex 
    \label{ex:wau}
    \begin{xlist}
    \ex 
        \gll gena-'a-de mi 'ula, ho o wo ali-wau \\
        \textsc{dist:apro.3nh-locv-conn} 3\textsc{sg.f}>3\textsc{sg.m} give thus \textsc{emph} 3\textsc{sg.m}.\textsc{a} cry-\textsc{neg}:\textsc{already} \\
        \trans `then she gave [it] to him, so that he wouldn't cry anymore' [G.\ref{ex:text7-10}--\ref{ex:text7-11}]
    \ex 
        \gll ma ma ino, ena {ma mi} pihanga o utu moi 'oiwa-u \\
        but 3\textsc{sg.f}>3\textsc{nh} \textsc{dir.ven} \textsc{pro}.3\textsc{nh} \textsc{rnm}:3\textsc{sg.f}.\textsc{poss} banana \textsc{nm} bunch one \textsc{neg.exist}-\textsc{already} \\
        \trans `but she went there, and of her bananas one bunch was not there anymore' [H.\ref{ex:text8-59}]
    \end{xlist}
\end{exe}
 
\subsection{Limitative: \textit{-o'a}}

The function of \textit{-o'a} (also spelled <oa> and <'a>) is as difficult to discern as that of \textit{-ou}. I gloss \textit{-o'a} as Limitative (\textsc{lim}). It is somewhat similar in function to \textit{-ou} and both markers are sometimes translated as `already'.\footnote{In the North Halmahera languages Galela, Tobelo and Loloda, \textit{-oka}, cognate to \textit{-o'a} means `already' while no cognates of \textit{-ou} are found. Pagu and Tabaru, on the other hand, have both \textit{-oka} and \textit{-ou}.} 

In contrast to \textit{-ou}, \textit{-o'a} does not occur in event series. It frequently occurs with states (`sleep'; see (\ref{ex:oka_states})) or achievements leading to states (`close', `hide'; see (\ref{ex:oka_achievements})). Maybe \textit{-o'a} marks the onset of such events and their subsequent duration, hence the term Limitative.
The example in (\ref{ex:oka_pros}) also shows that \textit{-o'a} can have a prospective reading.

%states
\begin{exe}
    \ex
    \label{ex:oka_states}
        \gll 'ano to nihu'u{-o'a} de o tiba mo tobi'i de o goda mi haiti-au \\
        a.moment.ago 1\textsc{sg}.\textsc{a} sleep-\textsc{lim} \textsc{conn} \textsc{nm} bamboo 3\textsc{sg.f}.\textsc{a} hack \textsc{conn} \textsc{nm} k.o.spirit 3\textsc{sg.f}.\textsc{u} seize-\textsc{already} \\
        \trans `just now I was sleeping and and she chopped bamboo and the \textit{goda} spirit seized her' [C.\ref{ex:text3-28}--\ref{ex:text3-29}]
\end{exe}

%achievements
\begin{exe}
    \ex
    \label{ex:oka_achievements}
    \begin{xlist}
    \ex 
        \gll de una ma doguulu u ma iun{-o'a} o manuru ma goa-'a \\
        \textsc{conn} \textsc{pro}.3\textsc{sg.m} \textsc{rnm} young.man 3\textsc{sg.m}.\textsc{a} \textsc{mid} hide-\textsc{lim} \textsc{nm} jasmine \textsc{rnm} ?lower.part-\textsc{locv} \\
        \trans `and he, the young man, hid under a jasmine [bush]' [E.\ref{ex:text5-86}]
    \ex 
        \gll wo temo: ``no ma ruwut{-o'a},'' de mu ma ruwut{-o'a} \\
        3\textsc{sg.m}.\textsc{a} say 2sg.\textsc{a} \textsc{mid} close.eyes-\textsc{lim} \textsc{conn} 3\textsc{sg.f}.\textsc{a} \textsc{mid} close-\textsc{lim} \\
        \trans `[and] he said: ``close your (sg.) eyes,'' and she closed her eyes' [F.\ref{ex:text6-22}--\ref{ex:text6-24}]
        \label{ex:oka_pros}
    \end{xlist}
\end{exe}

\subsection{Perfect: \textit{-o'au}}

The suffixes \textit{-o'a} and \textit{-ou} can be combined. The portmanteau suffix is spelled <o'au> or <oau> and glossed \textsc{perf}. It conveys a perfect meaning, i.e. ``the continuing present relevance of a past situation'' \citep[52]{comrie1976}; see examples in (\ref{ex:o'au}). 

\begin{exe}
    \ex
    \label{ex:o'au}
    \begin{xlist}
    \ex 
        \gll de na'o wo lamo'{-o'au}, {ma moi} wo hupu-o o a'ele Itio \\
        \textsc{conn} \textsc{cond} 3\textsc{sg.m}.\textsc{a} big-\textsc{perf} once 3\textsc{sg.m}.\textsc{a} exit-\textsc{already} \textsc{nm} water Itio \\
        \trans ‘and when he had grown up, he once left the Itio river’ [A.\ref{ex:text1-16}] 
    \ex 
        \gll de yo odomo ma duang-ino i wutu{-oau} ho yo ma idu \\
        \textsc{conn} 3\textsc{hpl}.\textsc{a} eat \textsc{rnm} finish-\textsc{dir.ven} 3\textsc{nh}.\textsc{a} night-\textsc{perf} thus 3\textsc{hpl}.\textsc{a} \textsc{mid} sleep \\
        \trans ‘and after they had eaten, it had become night, so they slept’ [E.\ref{ex:text5-52}--\ref{ex:text5-53}]
    \end{xlist}
\end{exe}

There is no example of a verb marked with \textit{-o'a-ou} being negated. It is therefore possible that a negative perfect does not exist in Modole. In cases where a negative perfect would occur in English, a simple negation is used in Modole (\ref{neg_perf}). 

\begin{exe}
    \ex 
    \label{neg_perf}
        \gll ti ni poha-u, ngomi o ni mi dodoa-wa \\
        1\textsc{sg}.\textsc{a} \textsc{2pl.u} beat-\textsc{already} \textsc{pro}.1\textsc{pl.ex} \textsc{emph} \textsc{2pl.u} 1\textsc{pl.ex.a} do-\textsc{neg} \\
        \trans `I will strike you (pl.) [dead], we (ex.) haven't done anything to you (pl.)' [D.\ref{ex:text4-46}]
\end{exe}

\subsection{Sequential repetition?: \textit{-o'a-ohi}}

The combination of \textit{-o'a} and \textit{-ohi}, spelled <o'ahi> and <oahi>, occurs only in text 7 with the verb \textit{huhu} `nurse' in several parallel sentences (\ref{ex:o'a-ohi}). In this context, \textit{-ohi} expresses `first'.
It is possible that \textit{-o'a-ohi} expresses a sequential repetition of an event. In the texts, I gloss \textit{-o'a} and \textit{-ohi} separately.

\begin{exe}
    \ex
    \label{ex:o'a-ohi}
    \begin{xlist}
    \ex 
        \gll yai, o raja Molo'u awi ngomaha i dudungu, ho no wi huhu-oa-hi \\
        mother, \textsc{nm} king Moluccas 3\textsc{sg.m}.\textsc{poss} throat 3\textsc{nh}.\textsc{a} dry, thus 2\textsc{sg}.\textsc{a} 3\textsc{sg.m} breast-\textsc{lim}-\textsc{still} \\
        \trans `mother, the throat of the King of the Moluccas is dry, so nurse him first' [G.\ref{ex:text7-33}, G.\ref{ex:text7-44}, G.\ref{ex:text7-48}]
    \ex 
        \gll ho ya iha de mi hi-huhu-oa-hi ma bilanga \\
        thus 3\textsc{hpl}>3\textsc{nh} \textsc{dir.land} \textsc{conn} 3\textsc{sg.f}>3\textsc{sg.m} \textsc{caus}-breast-\textsc{lim}-\textsc{still} \textsc{rnm} older.sibling \\
        \trans `so they [went] landwards and she, the sister, first nursed him' [G.\ref{ex:text7-37}, G.\ref{ex:text7-41}]
    \end{xlist}
\end{exe}


\subsection{`Again, too': \textit{-oli}}

The suffix \textit{-oli} (also spelled <li> and <ali>) means `again' or `too' with verbs (\ref{ex:oli_verbs}) and nouns (\ref{ex:oli_nouns}). It is glossed \textsc{again}.

%again
\begin{exe}
    \ex 
    \label{ex:oli_verbs}
    \begin{xlist}
    \ex 
        \gll de {ma mi} la'o yo dauhu ho i pelanga{-li} \\
        \textsc{conn} \textsc{rnm}=3\textsc{sg.f}.\textsc{poss} eye 3\textsc{hpl}.\textsc{a} oil thus 3\textsc{nh}.\textsc{a} open-\textsc{again} \\
        \trans `and her eyes they rubbed with oil, so they opened again' [F.\ref{ex:text6-49}]
    \ex 
        \gll de wa hulo'o{-li} awi {ali{\textasciitilde}ali} \\
        \textsc{conn} 3\textsc{sg.m}>3\textsc{nh} send-\textsc{again} 3\textsc{sg.m}.\textsc{poss} ring \\
        \trans `and he also sent his rings' [E.\ref{ex:text5-43}]
    \end{xlist}
\end{exe}

\begin{exe}
    \ex 
    \label{ex:oli_nouns}
    \begin{xlist}
    \ex 
        \gll de o wange ma hohoru-oa{-li} ma 'oana awi ngoa wa ino \\
        \textsc{conn} \textsc{nm} sun \textsc{rnm} west-\textsc{locv}-\textsc{again} \textsc{rnm} king 3\textsc{sg.m}.\textsc{poss} child 3\textsc{sg.m}>3\textsc{nh} \textsc{dir.ven} \\
        \trans `and the son of the Western King came too' [D.\ref{ex:text4-82}]
    \ex 
        \gll de o mo'a ma ho'a{-ali} mi de-tege-'u \\
        \textsc{conn} \textsc{nm} palm \textsc{rnm} leaf-\textsc{again} 3\textsc{sg.f}>3\textsc{sg.m} \textsc{appl}-drip-\textsc{dir.down} \\
        \trans `and she again pressed out [her breast] into a palm leaf for him' [G.\ref{ex:text7-45}]
    \end{xlist}
\end{exe}

\section{Numerals}

\noindent The numerals `one', `two', `three', `four', `six' and `seven' are attested in the stories. They are displayed in \tabref{tab:numerals}. I have added the remaining numerals from other sources.

\begin{table}
    \caption{Modole numerals}
    \label{tab:numerals}
    \begin{tabularx}{\textwidth}{llX}
        \lsptoprule
        \textsc{digit}&\textsc{modole}&\textsc{comment}\\
        \midrule
        1& \textit{moi}&\\
        2& \textit{mididi} / \textit{modidi}&Both \textit{mididi} and \textit{modidi} are attested four times each.\\
        3& \textit{ha(a)nge}&`Three' is attested once as \textit{hange} and twice as \textit{haange}.\\
        4& \textit{hoata}&\\
        5& [\textit{motoa}]&(\cite{wada1980}, \cite{stokhof1980a})\\
        6& \textit{butanga}&\\
        7& \textit{tumudingi} / \textit{tumidingi}&\textit{Tumidingi} is attested only once, \textit{tumudingi} 28 times.\\
        8& [\textit{tuangele}]&(\cite{ellen1916b})\\
        9& [\textit{siwo}]&(\cite{wada1980}, \cite{stokhof1980a})\\
        \lspbottomrule 
    \end{tabularx}
\end{table}

I am unable to tell if there is a strict functional difference between the two forms of `two' (\textit{mididi} and \textit{modidi}). In three of the four attested examples of \textit{mididi}, the numeral functions as a verb. \textit{Modidi} never functions as a verb. 

Numerals can occur with temporal suffixes and directionals (\ref{ex:numeral_infl}).

\begin{exe}
	\ex 
	\label{ex:numeral_infl}
	\begin{xlist}
	\ex 
		\gll awi baju moi-ohi \\
		3\textsc{sg.m}.\textsc{poss} shirt one-\textsc{still} \\
		\trans `he still had one shirt' [H.\ref{ex:text8-47}]
	\ex 
		\gll o wange moi-u \\
		\textsc{nm} sun one-\textsc{foc} \\
		\trans `one day' [H.\ref{ex:text8-2}]
	\ex 
		\gll ma tadi-'a moi-o'a \\
		\textsc{rnm} pole-\textsc{locv} one-\textsc{locv} \\
		\trans `one pole' [I.\ref{ex:text9-3}, J.\ref{ex:text10-8}]
	\ex
		\gll o wange tumuding-ino \\
		\textsc{nm} day seven-\textsc{dir.ven} \\
		\trans `?seven days from now on' [E.\ref{ex:text5-157}]
	\end{xlist}
\end{exe}

\subsection{Attributive numerals}\label{subsec:attributive_numerals}

\noindent Attributive numerals can be constructed in different ways. In most examples, they simply follow their dependent (\ref{ex:numerals_jux}).

\begin{exe}
	\ex 
	\label{ex:numerals_jux}
	\begin{xlist}
	\ex 
		\gll o ngaili moi \\
		\textsc{nm} river one \\
		\trans `a river' [H.\ref{ex:text8-25}]   
	\ex 
		\gll ani ngoa'a modidi \\
		2\textsc{sg}.\textsc{poss} child two \\
		\trans `your (sg.) two children' [F.\ref{ex:text6-37}]
		\label{ex:numerals_jux_human}
	\ex 
		\gll o wutu haange \\
		\textsc{nm} night three \\
		\trans `three nights' [I.\ref{ex:text9-24}]
	\ex 
		\gll o ngaili tumudingi \\
		\textsc{nm} river seven \\
		\trans `seven rivers' [A.\ref{ex:text1-10}, B.\ref{ex:text2-42}, E.\ref{ex:text5-140}]
	\end{xlist}
\end{exe}

Numerals can also occur with indices in a relative clause (see \sectref{subsec:relative_clause}). In all but one example, the measured entity is a human being. The only other case features a fish as the measured entity. 
In the case of the third person plural, the actor-undergoer combination \textit{ya} 3\textsc{hpl}>3\textsc{nh} is used in all examples but one (\ref{ex:numerals_ya}). 
For all other attested persons (first person singular, third person non-human, first person plural exclusive), only the actor index occurs (\ref{ex:PP_actor}).

\begin{exe}
	\ex 
	\label{ex:numerals_ya}
	\begin{xlist}
	\ex 
		\gll ma 'oana awi ngoa'a ya butanga \\
		\textsc{rnm} king 3\textsc{sg.m}.\textsc{poss} child 3\textsc{hpl}>3\textsc{nh} six \\
		\trans `the six daughters of the king' [E.\ref{ex:text5-107}]
	\ex 
		\gll ma 'oana awi ngoa'a ya tumuding-o'a \\
		\textsc{rnm} king 3\textsc{sg.m}.\textsc{poss} child 3\textsc{hpl}>3\textsc{nh} seven-\textsc{loc} \\
		\trans `the seven children of the king' [F.\ref{ex:text6-5}]
	\end{xlist}
\end{exe}

\begin{exe}
	\ex 
	\label{ex:numerals_actor}
	\begin{xlist}
	\ex 
		\gll o bere'i mo moi \\
		\textsc{nm} old.person 3\textsc{sg.f}.\textsc{a} one \\
		\trans `an old woman' [A.\ref{ex:text1-17}]
	\ex 
		\gll o bibiti i moi \\
		\textsc{nm} bibiti 3\textsc{nh}.\textsc{a} one \\
		\trans `a \textit{bibiti} fish' [B.\ref{ex:text2-4}]
	\ex 
		\gll ngoi neena mi-mididi-o'a \\
		\textsc{pro}.1\textsc{sg} \textsc{prox}:\textsc{pro}.3\textsc{nh} 1\textsc{pl.ex}.\textsc{a}-two-\textsc{locv} \\
		\trans `we (ex.) were two' [E.\ref{ex:text5-134}]
	\end{xlist}
\end{exe}

Human measured entities can also occur with juxtaposed numerals (see example (\ref{ex:numerals_jux_human}) above). 
It is possible that in these constructions the referent is indefinite, unspecific and/or inactivated in contrast to the human entities with verbal numerals. \textit{Moi} may then serve as an indefinite article (\ref{ex:moi_indef}).

\begin{exe}
	\ex 
	\label{ex:moi_indef}
	\begin{xlist}
	\ex 
		\gll o nyawa moi  \\
		\textsc{nm} person one \\
		\trans `someone' [B.\ref{ex:text2-23}, C.\ref{ex:text3-2}, F.\ref{ex:text6-2}]
	\ex 
		\gll o moholehe moi \\
		\textsc{nm} maiden one \\
		\trans `a maiden' [I.\ref{ex:text9-2}]
	\end{xlist}
\end{exe}

\subsection{Numeral classifiers}

Attributive numerals sometimes occur with classifiers. They are not obligatory in Modole and in fact not particularly frequent.

Only two classifiers are attested: \textit{oti} for coconuts (\ref{ex:oti}) and \textit{utu} for a bunch of bananas and houses (\ref{ex:utu}).

\begin{exe}
	\ex 
	\begin{xlist}
	\ex 
		\gll {o} oti tumudingi, ho muna {o} oti hoata, una {o} oti haange \\
		\textsc{nm} \textsc{class} seven thus \textsc{pro}.3\textsc{sg.f} \textsc{nm} \textsc{class} four \textsc{pro}.3\textsc{sg.m} \textsc{nm} \textsc{class} three \\
		\trans `seven [coconuts], four [for] her, three [for] him' [A.\ref{ex:text1-33}]
		\label{ex:oti}
	\ex 
		\gll o wo'a o utu moi \\
		\textsc{nm} house \textsc{nm} \textsc{class} one \\
		\trans `one house' [G.\ref{ex:text7-49}]
		\label{ex:utu}
	\end{xlist}
\end{exe}

%ma pihanga o utu moi `one bunch of bananas' [H.55, 8.59]

%o igono o ngodumu moi `one coconut' [H.23] ngodumu also means `whole'

\subsection{Pronominal numerals}

There are three examples of numerals serving as pronouns. In the first example, \textit{moi} `one' has no additional marking (\ref{ex:pron_num_moi}). In the other two examples, the numeral occurs with an actor index and the numeral phrase seems to be a clarifying afterthought (see (\ref{ex:pro_num_actor1}) and (\ref{ex:pro_num_actor2})).

\begin{exe}
	\ex
	\begin{xlist}
	\ex 
		\gll {moi} de ma mede ma giau-o'a, de {moi} ma wange ma nyonyie-'a \\
		one \textsc{conn} \textsc{rnm} moon \textsc{rnm} new-\textsc{locv} \textsc{conn} one \textsc{rnm} sun \textsc{rnm} rise-\textsc{locv} \\
		\trans `and one [will be born] with the new moon and one with the sunrise’ [F.\ref{ex:text6-38}--\ref{ex:text6-39}, F.\ref{ex:text6-156}]
		\label{ex:pron_num_moi}
	\ex 
		\gll ena ya uoli ya butanga \\
		\textsc{pro}.3\textsc{nh} 3\textsc{hpl}>3\textsc{nh} descend:\textsc{again} 3\textsc{hpl}>3\textsc{nh} six \\
		\trans `then they descended again, the six [sisters]' [E.\ref{ex:text5-104}]
		\label{ex:pro_num_actor1}
	\ex 
		\gll yo boa-u-ali, ya tumudingi \\
		3\textsc{hpl}.\textsc{a} come-\textsc{already}-\textsc{again}, 3\textsc{hpl}>3\textsc{nh} seven \\
		\trans `they came again, the seven [daughters of the king]' [E.\ref{ex:text5-84}--\ref{ex:text5-85}]
		\label{ex:pro_num_actor2}
	\end{xlist}
\end{exe}

\subsection{Numerals with verbs}

The only attested adverbial numeral is \textit{ma moi} `once' (\ref{ex:mamoi}).

\begin{exe}
	\ex 
	\label{ex:mamoi}
	\begin{xlist}
	\ex 
		\gll naga {ma moi} yo labea \\
		\textsc{exist} once 3\textsc{hpl}.\textsc{a} fish.with.net \\
		\trans `once they went fishing with a net' [B.\ref{ex:text2-2}]
	\ex 
		\gll de {ma moi} ma bere'i mo tagi ami bail-i'a \\
		\textsc{conn} once \textsc{rnm} old.person 3\textsc{sg.f}.\textsc{a} go 3\textsc{sg.f}.\textsc{poss} garden-\textsc{dir.itv} \\
		\trans `and once the old woman went to her garden' [D.\ref{ex:text4-18}]
	\end{xlist}
\end{exe}

There is one example where the numeral \textit{tumudingi} `seven' seems to modify the noun \textit{nani} `layer' serving as a predicate in a kind of relative clause (\ref{ex:num_rel}).

\begin{exe}
	\ex 
		\gll awi baju u ma nani tumudingi-'a \\
		3\textsc{sg.m}.\textsc{poss} shirt 3\textsc{sg.m}.\textsc{a} \textsc{mid} layer seven-?\textsc{locv} \\
		\trans `seven layers of shirts' (lit. `his shirts he layered seven') [H.\ref{ex:text8-42}] \\
		\label{ex:num_rel}
\end{exe}

\subsection{Ordinal numerals}

There is only one potential example of an ordinal number: \textit{o hara ma hange} `third time' (\ref{ex:num_ord}). This is constructed with the numeral \textit{hange} `three' connected to the measured entity \textit{o hara} `time' by means of the relational noun marker \textit{ma}.

\begin{exe}
	\ex 
		\gll de wo ahoo-u wo temo: ``Apu'' hi-adono o hara ma hange aha mo ihene \\
		\textsc{conn} 3\textsc{sg.m}.\textsc{a} shout-\textsc{already} 3\textsc{sg.m}.\textsc{a} say grandmother \textsc{caus}-arrive \textsc{nm} time \textsc{rnm} three as.a.consequence 3\textsc{sg.f}.\textsc{a} hear \\
		\trans `and he shouted, saying: ``Grandmother!'' [only] at the third time she heard [him]' [H.\ref{ex:text8-63}--\ref{ex:text8-64}]
        \label{ex:num_ord}
\end{exe}

\section{Demonstratives}    \label{sec:demonstratives}

Modole demonstratives can be quite complex but are made up of only a handful of elements. This section deals with demonstratives based on the morphemes \textit{ne} and \textit{ge}. Demonstratives based on locationals and directionals are discussed in \sectref{subsec:locationals+directionals}.

\begin{table}
    \caption{Modole demonstratives}
    \label{tab:demonstratives}    
    \begin{tabularx}{\textwidth}{XXXXXXXX} 
    \lsptoprule
        \multicolumn{2}{c}{\textsc{nominal}}&  \multicolumn{3}{c}{\textsc{adverbial}}&  \multicolumn{2}{c}{\textsc{predicative}}&\textsc{exist.}\\ \midrule
  \textsc{adn}.& \textsc{pron}.& \textsc{loc}.& \textsc{temp}.& \textsc{manner}& \textsc{pres}.& \textsc{ident}.& \\ \midrule
  \textit{ne},    
  \textit{ge}, \textit{neena}& \textit{ge}& \textit{ne}, \textit{ge}, \textit{geena} (+\textit{o'a})& \textit{ne}, \textit{geena} + \textit{o'a} + \textit{de}& \textit{'one}, \textit{'oneena}, \textit{'ogeena}& \textit{neena}, \textit{geena}, \textit{beneena}, \textit{begeena}, \textit{na} \textit{ne}& \textit{neena}, \textit{geena}& \textit{naga}, \textit{nage}\\
  \lspbottomrule
    \end{tabularx}
    \parbox{\textwidth}{\footnotesize\raggedright
        {Abbreviations: adn. = adnominal, pron. = pronominal, loc. = locational, pres. = presentative\\
        ident. = identifier, exist. = existential, temp. = temporal}\\
      }
\end{table}

\subsection{\textit{Ne} and \textit{ge}}

The basic demonstratives are \textit{ne} `proximal' (\textsc{prox}) and \textit{ge} `distal' (\textsc{dist}).
They are used adnominally with nouns (\ref{ex:ne_ge_adnom}) and, in one case, a personal pronoun (\ref{ex:ge_adnom_PP}).

%adnominal with noun
\begin{exe}
    \ex
    \label{ex:ne_ge_adnom}
    \begin{xlist}
    \ex 
        \gll ai huhu ne ma hongona no wi gaha \\
        1\textsc{sg}.\textsc{poss} breast \textsc{prox} \textsc{rnm} hald 2\textsc{sg}.\textsc{a} 3\textsc{sg.m}.\textsc{u} take \\
        \trans `take this half of my breast for him' [G.\ref{ex:text7-55}]
    \ex 
        \gll de ma ngoa ge wa odomo duga{\textasciitilde}duga a o heleene \\
        \textsc{conn} \textsc{rnm} child \textsc{dist} 3\textsc{sg.m}>3\textsc{nh} eat only \textsc{foc} \textsc{nm} eggplant \\
        \trans `and that child ate only eggplants' [A.\ref{ex:text1-15}]  
    \end{xlist}
\end{exe}

%adnominal with PP!
\begin{exe}
    \ex 
        \gll ona ge ya tumudingi yo hila \\
        \textsc{pro}.3\textsc{hpl} \textsc{dist} 3\textsc{nh}>3\textsc{nh} seven 3\textsc{hpl}.\textsc{a} first \\
        \trans `and they, those seven, [went] first' [B.\ref{ex:text2-70}]
        \label{ex:ge_adnom_PP}
\end{exe}

\textit{Ge} is also attested in pronominal function (\ref{ex:ge_pro}).

\begin{exe}
    \ex 
        \gll a ge mo tu{\til}tuulu \\
        \textsc{foc} \textsc{dist} 3\textsc{sg.f}.\textsc{a} \textsc{rdpl}{\til}follow \\
        \trans `the one that follows' [E.\ref{ex:text5-71}]
        \label{ex:ge_pro}
\end{exe}

\textit{Ne} and \textit{ge} also function as locationals (\ref{ex:ne_ge_spat}) and temporal adverbs (\ref{ex:ne_temp}), i.e. \textit{ne} means `here' and `now' and \textit{ge} `there' and `then'.\footnote{There is no clear example of \textit{ge} as a temporal adverb. In most cases, both a temporal and a locational interpretation is possible.} 

%locational adverb
\begin{exe}
\ex
\begin{xlist}
    \ex 
        \gll apu, po wi tila {ne} o hum-u'u \\
        grandmother 1\textsc{pl.in}.\textsc{a} 3\textsc{sg.m}.\textsc{u} push \textsc{prox} \textsc{nm} well-\textsc{dir.down} \\
        \trans `grandmother! [let]'s (in.) push him here down the well' [A.\ref{ex:text1-46}]
    \ex 
        \gll de ge ma 'oana awi we'ata yo ngamo \\
        \textsc{conn} \textsc{dist} \textsc{rnm} king 3\textsc{sg.m}.\textsc{poss} wife 3\textsc{hpl}.\textsc{a} angry \\
        \trans `and there the wives of the king were angry' [B.\ref{ex:text2-61}]  
    \end{xlist}
    \label{ex:ne_ge_spat}
\end{exe}

%temporal adverb
\begin{exe}
    \ex 
        \gll de ne ma we'ata mo temo \\
        conn \textsc{prox} \textsc{rnm} wife 3\textsc{sg.f}.\textsc{a} say \\
        \trans `and now the wife said' [F.\ref{ex:text6-63}]
        \label{ex:ne_temp}
\end{exe}

\textit{Ge} and \textit{ne} can then be combined with a number of other elements which will be introduced in the following sections. Like all parts of speech, they can also be marked with \textit{-ou}.
Sometimes, \textit{ge}-forms are suffixed with the connector \textit{de}.
Forms based on \textit{ge} are more frequent and varied in the corpus. 

\subsection{\textit{Neena} and \textit{geena}}\label{subsec:neena_geena}

\textit{Ne} and \textit{ge} can be combined with the third person non-human pronoun \textit{ena}, yielding \textit{neena} (also spelled <nena>) and \textit{geena} (also spelled <gena>). 
There appears to be no clear-cut functional difference between the forms with and without \textit{ena}. 

\textit{Neena} frequently occurs in adnominal function (\ref{ex:neena_adnom}) and \textit{geena} is attested once as a locational adverb (\ref{ex:geena_spat}).

%adnominal demonstrative
\begin{exe}
    \ex 
    \label{ex:neena_adnom}
    \begin{xlist}
    \ex 
        \gll ai pihanga neena o'ia ya odomo? \\
        1\textsc{sg}.\textsc{poss} banana \textsc{prox}:\textsc{pro}.3\textsc{nh} what 3\textsc{nh}>3\textsc{nh} eat \\
        \trans `these bananas of mine, who has eaten them?' [D.\ref{ex:text4-3}]
    \ex 
        \gll ngona {neena} o'ia-'a ngona de na ino? \\
        \textsc{pro}.2\textsc{sg} \textsc{prox:pro.3nh} what-\textsc{locv} \textsc{pro}.2\textsc{sg} \textsc{conn} 2\textsc{sg}>3\textsc{nh} \textsc{dir.ven} \\
        \trans `you (sg.) here, where are you (sg.) coming from?' [A.\ref{ex:text1-23}]
    \end{xlist}
\end{exe}

%locational adverb
\begin{exe}
    \ex 
        \gll de gena mu ma idu \\
        \textsc{conn} \textsc{dist:pro.3nh} 3\textsc{sg.f}.\textsc{a} \textsc{mid} lie \\
        \trans `and there she slept' [B.\ref{ex:text2-52}]
    \label{ex:geena_spat}
\end{exe}

In contrast to \textit{ne} and \textit{ge}, \textit{neena} and \textit{geena} also function as predicative demonstratives. In the examples in (\ref{ex:geena_neena_temp}), they serve as presentative demonstratives that ``help organize the temporal flow of discourse'' \citep[19]{killian2022}.

%presentative
\begin{exe}
    \ex 
    \label{ex:geena_neena_temp}
    \begin{xlist}
    \ex 
        \gll ho na'o una wo hawu, {geena} muna mo uti \\
        thus \textsc{cond} \textsc{pro}.3\textsc{sg.m} \textsc{3sgm.A} climb \textsc{dist:pro.3nh} \textsc{pro}.3\textsc{sg.f} 3\textsc{sg.f}.\textsc{a} descend \\
        \trans `so when he climbed (a mountain), then she descended' [B.\ref{ex:text2-43}]
    \ex 
        \gll {neena} nia u'u nio hi-atubele o hepa{\til}hepa \\
        \textsc{prox:pro.3nh} 2>3\textsc{nh} \textsc{dir.down} 2\textsc{pl}.\textsc{a} \textsc{caus}-provoke \textsc{nm} \textsc{rdpl}{\til}kick \\
        \trans `now you (pl.) go down to play ball' [A.\ref{ex:text1-57}]    
    \end{xlist}
\end{exe}

In (\ref{ex:geena_neena_ident}), \textit{neena} and \textit{geena} serve as demonstrative identifiers (see \cite[20]{killian2022}).\footnote{According to \citet{killian2022}, demonstrative identifiers but not presentative demonstratives can answer or ask the question `What is that?'. I consider the question in (\ref{ex:geena_neena_ident_b}) as an answer to this question since it is formally identical to the declarative clause `These are your \textit{legundi} fruits'.}

%identificational
\begin{exe}
    \ex 
    \label{ex:geena_neena_ident}
    \begin{xlist}
    \ex 
        \gll o ngo'u i tangi, o wogono i hoho, {geena} o'ia? \\
        \textsc{nm} forest.pigeon 3\textsc{nh}.\textsc{a} perch \textsc{nm} crow 3\textsc{nh}.\textsc{a} fly \textsc{dist:pro.3nh} what \\
        \trans `the [white] forest pigeon sits down, the [black] crow flies [away], what's that?' [J.\ref{ex:text10-10}]
    \ex
        \gll {neena} to ngona ani lagudi ma howo'o? \\
        \textsc{prox:pro.3nh} \textsc{poss.hum} \textsc{pro}.2\textsc{sg} 2\textsc{sg}.\textsc{poss} legundi \textsc{rnm} fruit \\
        \trans `are these your (sg.) \textit{legundi} fruits?' [A.\ref{ex:text1-94}]
        \label{ex:geena_neena_ident_b}
    \end{xlist}
\end{exe}

\subsection{\textit{'one} and \textit{'ogeena}}\label{subsec:demonstratives-'o_forms}

\textit{Ne} (\ref{ex:'one}) and \textit{geena} (\ref{ex:'ogeena}) can be prefixed with or be preceded by the element \textit{'o} to form manner adverbs \citep{dixon2003} meaning `like this, in this way' and `like that, in that way'. 

\begin{exe}
    \ex 
    \label{ex:'one}
    \begin{xlist}
    \ex 
        \gll uwa-u! 'o-ne ani dea de ani eha ni ngamo-ho \\
        \textsc{proh-foc} \textsc{emph}-\textsc{prox} 2\textsc{sg}.\textsc{poss} father \textsc{conn} 2\textsc{sg}.\textsc{poss} mother \textsc{2sg.u} angry-thus \\
        \trans `don't do that! Because then (= like this) your (sg.) father and your (sg.) mother will be angry with you (sg.)' [B.\ref{ex:text2-38}]
    \ex 
        \gll tole, ho 'o-ne awi 'iau wa gowo\\
        sister thus \textsc{emph}-\textsc{prox} 3\textsc{sg.m}.\textsc{poss} part 3\textsc{sg.m}.\textsc{a} wash.with.coconut \\
        \trans `sister, so like this what part of him [shall] he wash and oil' [F.\ref{ex:text6-16}]
    \end{xlist}
\end{exe}

\begin{exe}
    \ex 
    \label{ex:'ogeena}
    \begin{xlist}
    \ex 
        \gll na'o o geena de ngoi-ou nia dea \\
        \textsc{cond} \textsc{emph} \textsc{dist:pro.3nh} \textsc{conn} \textsc{pro}.1\textsc{sg}-\textsc{foc} 2\textsc{pl}.\textsc{poss} father \\
        \trans `if that's the case, I am your (pl.) father' [F.\ref{ex:text6-165}]
    \ex
        \gll ma lia'a awi baju u ma nani tumuding-i'a, awi haana ma 'o-geena \\
        \textsc{rnm} older.sibling 3\textsc{sg.m}.\textsc{poss} shirt 3\textsc{sg.m}.\textsc{a} \textsc{mid} layer seven-\textsc{dir.itv} 3\textsc{sg.m}.\textsc{poss} pants \textsc{rnm} \textsc{emph}-\textsc{dist:pro.3nh} \\
        \trans `the older brother [wore] seven layers of shirts [and] his pants [were] like that' [H.\ref{ex:text8-42}--\ref{ex:text8-43}]
    \end{xlist}
\end{exe}

\subsection{\textit{Beneena} and \textit{begeena}}

\textit{Neena} and \textit{geena} can also be prefixed with \textit{be-}. These forms function as presentative predicative demonstratives (see \cite[14]{killian2022}).

\begin{exe}
    \ex 
    \begin{xlist}
    \ex
        \gll be-neena 'a o holaibi ma aunu \\
        ?here-\textsc{prox}:\textsc{pro}.3\textsc{nh} \textsc{foc} \textsc{nm} hummingbird \textsc{rnm} blood \\
        \trans `here, [it] is just the blood of the hummingbird' [F.\ref{ex:text6-126}]    
    \ex 
        \gll be-geena 'a o u'u bobitino, de o puniti \\
        ?here-\textsc{dist:pro.3nh} \textsc{foc} \textsc{nm} fire burned.object, \textsc{conn} \textsc{nm} coconut.husk \\
        \trans `here are only charcoal and coconut bark' [F.\ref{ex:text6-53}]
    \end{xlist}
\end{exe}

\subsection{\textit{Na ne}, \textit{naga} and \textit{nage}}

In one instance (\ref{ex:na ne}), \textit{ne} occurs after \textit{na}, meaning `now'. This is presumably also a presentational predicative demonstrative.

\begin{exe}
    \ex 
        \gll na ne ma tanu ani we'ata ma ya tutumiding-o'a \\
        ?here \textsc{prox} \textsc{rnm} \textsc{mod} 2\textsc{sg}.\textsc{poss} wife \textsc{rnm} 3\textsc{nh}>3\textsc{nh} seven-\textsc{locv} \\
        \trans `now your (sg.) wives surely number seven already' [B.\ref{ex:text2-63}]
    \label{ex:na ne}
\end{exe}

\textit{Na} potentially also forms part of the two existentials \textit{naga} and \textit{nage}. The latter sometimes seem to function as an adnominal demontrative (\ref{ex:nage_dem}).
%naga as indefinite pronoun/adverb? and nage as definite?

\begin{exe}
\ex 
\gll o igono nage no na du-puniti\\
     \textsc{nm}	coconut	\textsc{exist}.\textsc{prox}	2\textsc{sg}.\textsc{a}	1\textsc{pl}.\textsc{in}.\textsc{u}	\textsc{appl}-coconut.husk\\
\trans `strip this coconut of its husk for us (in.)' [F.\ref{ex:text6-14}]
\label{ex:nage_dem}
\end{exe}


\subsection{\textit{Geena'a}}

\textit{Geena} is often suffixed with \textit{-o'a} meaning `there', or `then' in combination with \textit{de}.\footnote{The semantic distinction between \textit{geena'a} and \textit{geena'a de} was contributed by a native speaker.}

\begin{exe}
    \ex 
    \begin{xlist}
    \ex 
        \gll gena-'a de yo tagi ho o de'u tumudingi de o ngaili tumudingi ya paha \\
        \textsc{dist:pro.3nh}-\textsc{locv} \textsc{conn} 3\textsc{hpl}.\textsc{a} go thus \textsc{nm} mountain seven \textsc{conn} \textsc{nm} river seven 3\textsc{nh}>3\textsc{nh} leave \\
        \trans `then they went, over seven hills and seven rivers they left' [A.\ref{ex:text1-10}]
    \ex 
        \gll ma a wi dioma-ali gena-'a \\
        but \textsc{foc} 3\textsc{sg.m}.\textsc{u} return-again \textsc{dist:pro.3nh}-\textsc{locv} \\
        \trans `but they returned to him again over there' [A.\ref{ex:text1-8}]
    \end{xlist}
\end{exe}

\subsection{\textit{Geena} plus locationals}

\textit{Geena} can be combined with locationals and then expresses `there in the direction designated by the locational' (\ref{ex:locationals+demonstratives}). In (\ref{ex:locationals+demonstratives_oka}), \textit{geena} is marked with the locative  \textit{-o'a}. 
The form \textit{gena'adau(de)} (\textit{ge-ena-o'a-dau-de}) serves as a temporal adverbial meaning `then'. 

\begin{exe}
    \ex 
    \label{ex:locationals+demonstratives}
    \begin{xlist}
    \ex
        \gll {ge-da'u} ma inomo to wi 'ula \\
        \textsc{dist}-\textsc{loc.up} \textsc{rnm} food \textsc{1sg.a} \textsc{3sgm.u} give\\
        \trans `the food up there I gave to him' [G.\ref{ex:text7-23}]
    \ex 
        \gll de {genaa-dau} de yo ma togumu \\
        \textsc{conn} \textsc{dist:pro.3nh}:\textsc{locv}-\textsc{loc.down} \textsc{conn} 3\textsc{hpl}.\textsc{a} \textsc{mid} stop \\
        \trans `and there they [both] stopped' [E.\ref{ex:text5-27}]
        \label{ex:locationals+demonstratives_oka}
    \end{xlist}
\end{exe}

\subsection{\textit{Ge} plus directionals}\label{subsec:ge+direc}

\textit{Ge} is once combined with a directional prefixed with \textit{ng-} (\ref{ex:ge+directional}). The complex form functions as a pronominal.

\begin{exe}
    \ex 
        \gll 'aano ge-ngu'u tumudingi \\
        a.moment.ago \textsc{dist}-\textsc{n:dir.down} seven \\
        \trans `a moment ago, these seven down there' [E.\ref{ex:text5-70}]
        \label{ex:ge+directional}
\end{exe}

\section{Possession}

In Modole, the possessive phrase follows the strict order [possessor--possessed].
The possessor is expressed by a possessive marked for person and gender and optionally a noun phrase. 

\subsection{Possessives}\label{subsec:possession-possessives}

Possessives distinguish the same person-number combinations as actor and undergoer indices. They are clearly derived from the undergoer indices (with the exception of the third person plural human and the third person non-human possessives).
To form the singular possessives, an initial /a/ is added to the undergoer indices.
In the first person plural exclusive and the second person plural possessives, /a/ follows the index.
The first person plural inclusive possessive consists of the undergoer index plus an element \textit{nga}. The same element occurs in the third person plural possessive, but the form \textit{manga} is unrelated to the undergoer index \textit{'i}.
The form of the third person non-human possessive is identical to the relational noun marker \textit{ma}.
The possessives are displayed in \tabref{tab:possessives}.


\begin{table}
    \caption{Modole Possessives}
    \label{tab:possessives}
    \begin{tabularx}{.25\textwidth}{ll} \lsptoprule
        1\textsc{sg} & \textit{ai}\\
        2\textsc{sg} & \textit{ani}\\ 
        3\textsc{sg.f} & \textit{ami}\\ 
        3\textsc{sg.m} & \textit{awi}\\ 
        1\textsc{pl.in} & \textit{nanga}\\ 
        1\textsc{pl.ex} & \textit{mia}\\ 
        2\textsc{pl} & \textit{nia}\\ 
        3\textsc{hpl} & \textit{manga}\\ 
        3\textsc{nh} & \textit{ma}\\ \lspbottomrule
    \end{tabularx}
\end{table}

Possessives always precede the head, i.e. the possessed. Examples are given in (\ref{ex:possessives}). There is no example of a genuine possessive phrase with a non-human possessor (e.g. of an animal possessing something) but see \sectref{subsec:possession-relational_cons} for quasi-possessive relational constructions.

\begin{exe}
    \ex 
    \begin{xlist}
        \ex \textit{ai baili} `my garden' [B.\ref{ex:text2-15}]
        \ex \textit{ani dea de ani eha} `your (sg.) father and your (sg.) mother' [B.\ref{ex:text2-38}]
        \ex \textit{ami utu} `her hair' [B.\ref{ex:text2-53}]
        \ex \textit{awi roehe} `his body' [A.\ref{ex:text1-49}]
        \ex \textit{nanga danongo} `our (in.) grandchildren' [F.\ref{ex:text6-72}]
        \ex \textit{mia ilanga} `our (ex.) brother' [A.\ref{ex:text1-43}]
        \ex \textit{nia eha} `your (pl.) mother' [F.\ref{ex:text6-169}]
        \ex \textit{manga damunu} `their (human) drum' [D.\ref{ex:text4-88}]
    \end{xlist}
    \label{ex:possessives}
\end{exe}

\subsection{The particle \textit{to}}

The particle \textit{to} obligatorily precedes personal pronouns and noun phrases referring to human possessors. 
The \textit{to}-phrase is usually followed by a possessive that also expresses the possessor, and the possessed (see examples in (\ref{ex:to_with_poss})). 

%with pronoun
\begin{exe}
    \ex 
    \label{ex:to_with_poss}
    \begin{xlist}
    \ex
        \gll to ngoi ai hohan-iha \\
        \textsc{poss.hum} \textsc{pro}.1\textsc{sg} 1\textsc{sg}.\textsc{poss} landing.place-\textsc{dir.land} \\
        \trans `my landing place landwards' [F.\ref{ex:text6-147}]
    \ex 
        \gll neena to ngona ani lagudi ma howo'o? \\
        \textsc{prox:pro.3nh} \textsc{poss.hum} \textsc{pro}.2\textsc{sg} 2\textsc{sg}.\textsc{poss} legundi \textsc{rnm} fruit \\
        \trans `are these your (sg.) \textit{legundi} fruits?' [A.\ref{ex:text1-94}]
    \ex 
        \gll to muna ami io'o \\
        \textsc{poss.hum} \textsc{pro}.3\textsc{sg.f} 3\textsc{sg.f}.\textsc{poss} excrement \\
        \trans `her excrement' [B.\ref{ex:text2-65}]
    \end{xlist}
\end{exe}

Possessive constructions with \textit{to} can be headless.
In the example in (\ref{ex:to_no_possessed}), \textit{to} precedes the possessor noun phrase \textit{ai mana'i} `my friends' and the personal pronoun \textit{ona} `they'. The possessed noun phrase (`garden') is implicit. 

\begin{exe}
    \ex 
        \gll hababu to ai mana'i to ona yo do-todanga ya boto-au ho \\
        because \textsc{poss.hum} 1\textsc{sg}.\textsc{poss} friend \textsc{poss.hum} \textsc{pro}.3\textsc{hpl} \textsc{3hpl.a} \textsc{appl}-cut.down 3\textsc{nh}>\textsc{3nh} finish-\textsc{already} thus \\
        \trans `because those of my friends are already weeded' (lit. `because my friend's their [garden] ...') [B.\ref{ex:text2-16}] \\
        \label{ex:to_no_possessed}
\end{exe}

\subsection{Relational construction with \textit{ma}} \label{subsec:possession-relational_cons}

Often, two nouns connected with \textit{ma} are not in a possessive relationship, strictly speaking. The relationship is rather a part-whole relationship, general association or attribution (see examples in (\ref{ex:relation_cons})). I will call these \textit{relational constructions}.

\begin{exe}
    \ex 
    \label{ex:relation_cons}
    \begin{xlist}
    \ex 
        \gll o lagudi ma howo'o \\
        \textsc{nm} legundi \textsc{rnm} fruit \\
        \trans `\textit{legundi} fruits' [A.\ref{ex:text1-1}]
    \ex 
        \gll o wange ma dumun-o'a \\
        \textsc{nm} sun \textsc{rnm} dive-\textsc{locv} \\
        \trans `West' (lit. `at the diving of the sun') [A.\ref{ex:text1-116}] \\
    \ex 
        \gll yo ma palitana ma gota ma homoa-'a,\\
     	3\textsc{hpl}.\textsc{a}	\textsc{mid}	jump.\textsc{up}	\textsc{rnm}	wood	\textsc{rnm}	other-\textsc{dir}.\textsc{itv}\\
        \trans `but when it fell down, they jumped to another tree' [D.\ref{ex:text4-49}]
    \end{xlist}
\end{exe}

\subsection{Complex attributive possession}

Possessive phrases can be recursive. In the the example in (\ref{ex:compl_poss1}) the possessor \textit{ma 'oana} `the king' is attributed by a relational construction \textit{o wanga ma dumuno'a} followed by the possessive \textit{awi} `his' plus the possessed noun phrase \textit{humu ma jojaga} `well guardian', which itself is a relational construction.
The whole possessor noun phrase \textit{[ma 'oana] [o wange ma dumuno'a]} `Western King' is not a relational construction since \textit{wange} occurs with the noun marker \textit{o}, not \textit{ma}. This is probably to signal that the second noun phrase is not dependent but attributive to the first one (`Western King' or `king in the west' not `king of the west'). The same is the case in the example in (\ref{ex:compl_poss2}) where \textit{o 'uho} `cuscus' in the possessor noun phrase is juxtaposed to the possessor \textit{ma 'oana} `king', yielding `King Cuscus' not `the king's cuscus'. Compare this to the use of \textit{o} with materials discussed in \sectref{sec:noun_phrase-noun_markers}.

\begin{exe}
    \ex
    \begin{xlist}
    \ex 
        \gll ma 'oana o wange ma dumun-o'a awi humu ma jojaga \\
        \textsc{rnm} king \textsc{nm} sun \textsc{rnm} dive-\textsc{locv} 3\textsc{sg.m}.\textsc{poss} well \textsc{rnm} guardian \\
        \trans `the well guardian of the Western King' (lit. `the king at the sun's diving his well's guardian') [A.\ref{ex:text1-17}] \\
    \label{ex:compl_poss1}
    \ex
        \gll ngomi neena ma 'oana o 'uho awi ngoa'a \\
        \textsc{pro}.1\textsc{pl.ex} \textsc{prox}:\textsc{pro}.3\textsc{nh} \textsc{rnm} king \textsc{nm} cuscus 3\textsc{sg.m}.\textsc{poss} child \\
        \trans `we (ex.) are the children of King Cuscus' [F.\ref{ex:text6-152}]
    \label{ex:compl_poss2}
    \end{xlist}
\end{exe}

\subsection{Predicative possession}

There is just one example of affirmative predicative possession in the texts (\ref{ex:pred_poss}). The possessor \textit{ani} `your (sg.)' is preceded by the connector \textit{de}. The possessed noun \textit{igono} `coconut' is marked with the locative \textit{-o'a}, which may have an existential function here.\footnote{This construction is also reported by \citet[87-89]{peranginangin2018} for Pagu.} For the construction of negated predicative possession see \sectref{subsec:neg_poss_exist}.

\begin{exe}
    \ex 
        \gll bote de ani igon-o'a \\
        maybe \textsc{conn} 2\textsc{sg}.\textsc{poss} coconut-\textsc{locv} \\
        \trans `do you (sg.) maybe have coconuts?' (lit. `with you exist coconuts?') [A.\ref{ex:text1-27}] \\
    \label{ex:pred_poss}
\end{exe}

\section{Verbless clauses}

Clauses in Modole can be divided into verbal clauses and verbless clauses. Verbal clauses contain a predicate marked with actor and/or undergoer indices.\footnote{There are no infinite verbs in Modole.} As previously mentioned, predicates in Modole represent a wide range of semantic classes and include roots that are nominal in many languages.

Verbless clauses are defined as clauses that do not contain a verb. Modole has no copula.
There are two main types of verbless clauses: existential clauses and presentational/identificational clauses. In addition, equative clauses are sometimes verbless as well.

\subsection{Existential clauses}
\label{ex:subsec:existential_clauses}

In existential clauses, `exist' is expressed by the particles \textit{naga} or \textit{nage} (\ref{ex:existentials}). These are usually found in clause-initial position. While \textit{naga} seems to be the default existential, \textit{nage} only occurs in proximate scenarios `there is here'.
All stories but two (\textref{chapter:text01} and \textref{chapter:text10}) start with an existential clause.

\begin{exe}
    \ex 
    \label{ex:existentials}
    \begin{xlist}
    \ex 
        \gll {naga} o bere'i moi \\
        \textsc{exist} \textsc{nm} old.person one \\
        \trans `there was an old woman' [D.\ref{ex:text4-2}]    
    \ex 
        \gll o igono ena {naga} \\
        \textsc{nm} coconut \textsc{pro}.3\textsc{nh} \textsc{exist} \\
        \trans `there are coconuts' [A.\ref{ex:text1-29}]
    \ex 
        \gll {nage} ani bilanga to mi modoa-'a \\
        \textsc{exist}.\textsc{prox} 2\textsc{sg}.\textsc{poss} sister 1\textsc{sg}.\textsc{a} 3\textsc{sg.f}.\textsc{u} marry-\textsc{lim} \\
        \trans `your (sg.) sister here, I [want] to marry her' [F.\ref{ex:text6-111}]
    \end{xlist}
\end{exe}

\subsection{Presentational and identificational clauses}

Presentational (`this is X', (\ref{ex:presentational})) and identificational clauses (`it is X', (\ref{ex:identificational})) are often not marked in any special way, but sometimes demonstratives occur (see \sectref{sec:demonstratives}).

\begin{exe}
    \ex 
    \label{ex:}
    \begin{xlist}
    \ex 
        \gll to ngoi ai lagudi ma howo'o \\
        \textsc{poss.hum} \textsc{pro}.1\textsc{sg} 1\textsc{sg}.\textsc{poss} legundi \textsc{rnm} fruit \\
        \trans `these are my \textit{legundi} fruits' (lit. `my \textit{legundi} fruits') [A.\ref{ex:text1-96}]\\ 
        \label{ex:presentational}
    \ex 
        \gll ena a ma {omo{\textasciitilde}omo} \\
        \textsc{pro}.3\textsc{nh} \textsc{foc} \textsc{rnm} annelid \\
        \trans `it was a worm' [E.\ref{ex:text5-61}]
        \label{ex:identificational}
    \end{xlist}
\end{exe}

\subsection{Verbless equative clauses}\label{subsec:verbless_clauses-equative}

Equative clauses (`X is Y', `X is also Y', `X has the attribute Y', etc.) are sometimes verbless in Modole (see \sectref{subsubsec:verbal_clauses-equative} for verbal equative clauses).

\begin{exe}
    \ex 
    \begin{xlist}
    \ex 
        \gll awi ilingi ma da'u-ie o hala'a \\
        3\textsc{sg.m}.\textsc{poss} tooth \textsc{rnm} \textsc{loc.up}-\textsc{dir.up} \textsc{nm} silver \\
        \trans `his upper teeth were silver' [B.\ref{ex:text2-83}]
    \ex
        \gll ngomi neena ma 'oana o 'uho awi ngoa'a \\
        \textsc{pro}.1\textsc{pl.ex} \textsc{prox}:\textsc{pro}.3\textsc{nh} \textsc{rnm} king \textsc{nm} cuscus 3\textsc{sg.m}.\textsc{poss} child \\
        \trans `we (ex.) are the children of King Cuscus' [F.\ref{ex:text6-152}]
    \end{xlist}
\end{exe}

\section{Verbal clauses}

Verbal clauses contain a predicate marked with an actor and/or undergoer index. Based on the index/indices present, I distinguish intransitive actor clauses (only actor index), intransitive undergoer clauses (only undergoer index) and transitive clauses (actor and undergoer indices). Note that this classification is solely based on the indices (and the semantic role of their referents) and not on the occurrence of external arguments.\footnote{I acknowledge that this is an unconventional way of labeling, but since a study of the lexical valency of verbs and the further triggers for the presence or absence of undergoer indexing is impossible given the small size of the corpus, this is the best option at hand.}

\subsection{Intransitive actor clauses}

Intransitive actor clauses contain a predicate marked with only an actor index (\ref{ex:intr_actor}).
Predicates can be ``traditional'' intransitive verbs such as `go' and `stand' but also lexemes that are treated as nominal in many languages in equative or locational clauses.

\begin{exe}
    \ex 
    \label{ex:intr_actor}
    \begin{xlist}
    \ex 
        \gll 'a to olu'u to tagi-ou \\
        \textsc{foc} 1\textsc{sg}.\textsc{a} refuse 1\textsc{sg}.\textsc{a} go-\textsc{already} \\
        \trans `I refuse, I will go away' [G.\ref{ex:text7-28}]
    \ex 
        \gll nio go{\til}gogel-ou nia dea-'a \\
        2\textsc{pl}.\textsc{a} \textsc{rdpl}{\til}sit-\textsc{already} 2\textsc{pl}.\textsc{poss} father-\textsc{locv} \\
        \trans `you (pl.) will stay with your (pl.) father' [G.\ref{ex:text7-30}]
    \end{xlist}
\end{exe}

The actor can be additionally expressed by a noun phrase. It can precede (\ref{ex:intr_actor_NP_pre}, \ref{ex:intr_actor_pro_pre}) or follow the predicate (\ref{ex:intr_actor_NP_follow}) and is morphologically unmarked.

\begin{exe}
    \ex 
    \begin{xlist}
    \ex 
        \gll ma gogunane i hano, i temo \\
        \textsc{rnm} ant 3\textsc{nh}.\textsc{a} ask 3\textsc{nh}.\textsc{a} say \\
        \trans `the ants asked, saying' [D.\ref{ex:text4-20}]
        \label{ex:intr_actor_NP_pre}
    \ex 
        \gll ngomi mio pa{\til}pahiara \\
        \textsc{pro}.1\textsc{pl.ex} 1\textsc{pl.ex}.\textsc{a} \textsc{rdpl}{\til}walk \\
        \trans `we'll (ex.) just go for a walk' [A.\ref{ex:text1-62}]
        \label{ex:intr_actor_pro_pre}
    \ex 
        \gll mo hano ma eha, mo temo \\
        3\textsc{sg.f}.\textsc{a} ask \textsc{rnm} mother 3\textsc{sg.f}.\textsc{a} say \\
        \trans `and she, the mother, asked, saying' [F.\ref{ex:text6-50}]
        \label{ex:intr_actor_NP_follow}
    \end{xlist}
\end{exe}

The absence of an undergoer index does not preclude the occurrence of an object. In (\ref{ex:intr_actor_DO}), \textit{diai} `make' governs the direct object \textit{o ngotili moi} `a proa'.

\begin{exe}
    \ex 
        \gll yo diai o ngotili moi \\
        3\textsc{hpl}.\textsc{a} make \textsc{nm} proa one \\
        \trans `they made a proa' [F.\ref{ex:text6-84}]
        \label{ex:intr_actor_DO}
\end{exe}

With verbs of speaking, the indirect object is often marked with the \textsc{itive} directional \textit{i'a} (\ref{ex:intr_actor_IO}). 

\begin{exe}
    \ex 
    \label{ex:intr_actor_IO}
    \begin{xlist}
    \ex 
        \gll mo hi-ngahu ma berei'a \\
        3\textsc{sg.f}.\textsc{a} \textsc{caus}-report \textsc{rnm} old.person:\textsc{dir.itv} \\
        \trans `she told the old man' [D.\ref{ex:text4-31}]
    \ex 
        \gll wo jaji mia eha-'a \\
        3\textsc{sg.m}.\textsc{a} promise 1\textsc{pl.ex}.\textsc{poss} mother-\textsc{dir.itv} \\
        \trans `he promised our (ex.) mother' [F.\ref{ex:text6-154}]
    \end{xlist}
\end{exe}

\subsubsection{Verbal equative clauses}\label{subsubsec:verbal_clauses-equative}

Equative clauses are verbless in many languages. In Modole, the predicate is usually marked with an actor index ((\ref{ex:intr_equa}); see Section (\ref{subsec:verbless_clauses-equative}) for examples of verbless equative clauses in Modole).

\begin{exe}
    \ex 
    \label{ex:intr_equa}
    \begin{xlist}
    \ex 
        \gll {wo} hungi{\textasciitilde}hungi \\
        3\textsc{sg.m}.\textsc{a} search.for.game.with.dogs \\
        \trans `he was a hunter' [C.\ref{ex:text3-3}]
    \ex 
        \gll ma to muna ami io'o ma bounu {i} hemo \\
        but \textsc{poss.hum} \textsc{pro}.3\textsc{sg.f} excrement \textsc{rnm} smell 3\textsc{nh}.\textsc{a} sweet \\
        \trans `but the smell of her excrement was sweet' [B.\ref{ex:text2-65}] 
    \end{xlist}
\end{exe}

\subsubsection{Locational clauses}

In locational clauses the Figure is expressed by a noun phrase and/or actor index and the Ground by the predicate (\ref{ex:locational_clause}).

\begin{exe}
    \ex 
        \gll mo 'o-gorona-'a \\
        3\textsc{sg.f}.\textsc{a} \textsc{emph}-middle-\textsc{locv} \\
        \trans `she is in the middle' [E.\ref{ex:text5-108}]
        \label{ex:locational_clause}
\end{exe}



\subsection{Intransitive undergoer clauses}\label{subsec:verbal_clauses-intransitive_undergoer}

In intransitive undergoer clauses, the predicate is only marked with an undergoer index. 
Such predicates occur quite frequently, since the third person non-human actor index almost never surfaces (see \sectref{subsubsec:pronouns_indices-indices-actor_undergoer}), resulting in a predicate plus only an undergoer index. I treat this as a case of gapping\footnote{The absence of the third person non-human actor index \textit{i} is not due to regular sound change; compare \textit{igono} `coconut' where the initial /i/ is retained.} and consider them as transitive clauses (see \sectref{subsec:transitive}).

This leaves only four predicates that are intransitive and always occur with only an undergoer index: \textit{to'ata} `(be a) witch', \textit{hawini} `hungry', \textit{ngamoho} `angry' and \textit{ma'e'e} `ashamed' (see examples in (\ref{ex:intr_undergoer})).

\begin{exe}
    \ex 
    \label{ex:intr_undergoer}
    \begin{xlist}
    \ex 
        \gll na'o o ngo apu mi to'ata \\
        \textsc{cond} \textsc{nm} \textsc{hon.fem} grandmother 3\textsc{sg.f}.\textsc{u} witch \\
        \trans `if the grandmother is a witch' [E.\ref{ex:text5-40}]
    \ex 
        \gll o 'eto'o ma ube ta odomo i hawini ho \\
        \textsc{nm} sago \textsc{rnm} bit 1\textsc{sg}>3\textsc{nh} eat 1\textsc{sg}.\textsc{u} hungry thus \\
        \trans `[can] I eat some sago bread? I'm hungry' [E.\ref{ex:text5-14}]
    \ex 
        \gll 'o-ne ani dea de ani eha ni ngamo-ho \\
        \textsc{emph}-\textsc{prox} 2\textsc{sg.f}.\textsc{poss} father \textsc{conn} 2\textsc{sg.f}.\textsc{poss} mother \textsc{2sg.u} angry-thus \\
        \trans `because then your (sg.) father and your (sg.) mother will be angry with you (sg.)' [B.\ref{ex:text2-38}]
    \ex 
        \gll de una ma geri wi ma'e'e \\
        \textsc{conn} \textsc{pro}.3\textsc{sg.m} \textsc{rnm} in-law 3\textsc{sg.m}.\textsc{u} ashamed \\
        \trans `and he, the brother-in-law, was ashamed' [E.\ref{ex:text5-17}]
    \end{xlist}
\end{exe}

\subsection{Transitive clauses}
\label{subsec:transitive}

Predicates in transitive clauses occur with an actor and an undergoer index, unless the actor index is gapped. 
Usually, the undergoer index expresses the direct object but there are cases where the undergoer is the indirect object, e.g. a Beneficiary (see \sectref{subsec:ditranstive_clauses} on ditransitive clauses). 
If the direct object is additionally expressed by a noun phrase, it is unflagged (see examples in (\ref{ex:trans})).

\begin{exe}
    \ex 
    \label{ex:trans}
    \begin{xlist}
    \ex 
        \gll halingou ani halalomu moi to wi topo'o-ahi \\
        necessary 2\textsc{sg}.\textsc{poss} slave one 1\textsc{sg}.\textsc{a} 3\textsc{sg.m}.\textsc{u} stab-\textsc{still} \\
        \trans `I first must stab one of your (sg.) slaves' [B.\ref{ex:text2-75}]
    \ex 
        \gll ani eha, de ani dea o no i ma'e-wa-u \\
        2\textsc{sg}.\textsc{poss} mother \textsc{conn} 2\textsc{sg}.\textsc{poss} father \textsc{emph} 2\textsc{sg}.\textsc{a} 3\textsc{hpl}.\textsc{u} see-\textsc{neg}-\textsc{already} \\
        \trans `you (sg.) will not find your (sg.) mother and your (sg.) father anymore' [B.\ref{ex:text2-47}]
    \end{xlist}
\end{exe}

The examples in (\ref{ex:trans_gapped}) show cases where the actor index is gapped.

\begin{exe}
    \ex 
    \label{ex:trans_gapped}
    \begin{xlist}
    \ex 
        \gll bote ngona ani lolabi {mi} tadi? \\
        maybe \textsc{pro}.2\textsc{sg} 2\textsc{sg}.\textsc{poss} kris 3\textsc{sg.f}.\textsc{u} stab \\
        \trans `maybe your (sg.) kris has stabbed her?' [I.\ref{ex:text9-16}]
    \ex 
        \gll de ona ma moholehe ya tumudingi yo temo: ``Apu, po wi tila ne o humu'u'', de {wi} tila \\
        \textsc{conn} \textsc{pro}.3\textsc{hpl} \textsc{rnm} maiden 3\textsc{hpl}>3\textsc{nh} seven 3\textsc{hpl}.\textsc{a} say grandmother \textsc{1pl.ex.a} \textsc{3sgm.u} push \textsc{prox} \textsc{nm} well:\textsc{dir.down} \textsc{conn} 3\textsc{sg.m}.\textsc{u} push \\
        \trans `and they, the seven maidens, said: ``Grandmother! [let]'s (in.) push him here down the well,'' and they pushed him [in]'' [A.\ref{ex:text1-45}--\ref{ex:text1-47}]
    \end{xlist}
\end{exe}

\subsubsection{Ditransitive clauses}\label{subsec:ditranstive_clauses}

Ditransitive clauses in Modole are a subtype of transitive clauses. 
The predicate in ditransitive clauses governs a direct object which, however, is not expressed by the undergoer index. Instead, the undergoer index refers to an indirect object, e.g. a beneficiary or recipient.
 
Based on the data from the texts, Modole is a secundative type language. The recipient or beneficiary (the indirect object in languages like English) is marked as undergoer on the verb while the theme or patient (the direct object in English) is unmarked and also often unexpressed. For Pagu, \citet[203]{peranginangin2018} states that undergoer marking operates along the following semantic hierarchy: 
\begin{quote}
human recipient/beneficiary > human patient > non-human object
\end{quote}
Accordingly, the referent highest in the hierarchy will be marked as undergoer on the verb. Based on the Modole texts, it is impossible to say whether the undergoer marking follows a strict secundative alignment or is governed by a semantic hierarchy, since there are no examples with a non-human recipient or beneficiary involved.\footnote{My Modole informant accepted both undergoer marking of a non-human recipient or of a human patiens for the verb \textit{'ula} `give'.}

Some verbs in the corpus are inherently ditransitive and can occur in ditransitive clauses without additional morphology. These are \textit{(')ula} `give'\footnote{\textit{(')ula} is marked with a directional several times (see examples in (\ref{ex:ditrans_ino}) and (\ref{ex:ditrans_i'a})) but there are also examples without any additional marking.}, \textit{diai} `make', \textit{gaha} `take' and \textit{ha'ai} `cook' (see examples in (\ref{ex:ditrans})).

\begin{exe}
    \ex 
    \label{ex:ditrans}
    \begin{xlist}
    \ex   
        \gll ge-da'u ma inomo to wi 'ula \\
        \textsc{dist}-\textsc{loc.up} \textsc{rnm} food 1\textsc{sg}.\textsc{a} 3\textsc{sg.m}.\textsc{u} give \\
        \trans `the food up there I gave to him' [G.\ref{ex:text7-23}]
    \ex 
        \gll mi ha'ai o bira \\
        3\textsc{sg.f}>3\textsc{sg.m} cook \textsc{nm} rice \\
        \trans `she cooked rice for him' [H.\ref{ex:text8-69}]
    \end{xlist}
\end{exe}

%Some verbs have an inherent direct object and the undergoer index always refers to the indirect object, e.g. \textit{peleti} `make up hair' (\ref{ex:peleti}).

%\begin{exe}
%\ex 
%\gll iti no i peleti \\
%only 2\textsc{sg}.\textsc{a} 1\textsc{sg}.\textsc{u} make.up.hair \\
%\trans `just make up my hair' [B.26]
%\label{ex:peleti}
%\end{exe}

Transitive verbs can be used in ditransitive clauses when they are marked with the applicative prefix \textit{dV-} ((\ref{ex:ditrans_dV}); see \sectref{subsec:applicative}).

\begin{exe}
    \ex 
        \gll o igono nage no na du-puniti \\
        \textsc{nm} coconut \textsc{prox}.\textsc{exist} 2\textsc{sg}.\textsc{a} 1\textsc{pl.in}.\textsc{u} \textsc{appl}-coconut.husk \\
        \trans `strip this coconut of its husk for us (in.)' [F.\ref{ex:text6-14}]
    \label{ex:ditrans_dV}
\end{exe}

%check whether \textit{i'a} is not o'a and if the directional don't have spatial meaning
In some ditransitive clauses, the predicate is marked with the \textsc{ventive} and \textsc{itive} directionals \textit{ino} and \textit{i'a}. They may have an applicative function. 
In most examples with \textit{ino}, the recipient referent is a first or second person, but third person referents occur as well (\ref{ex:ditrans_ino}), while in examples with \textit{i'a}, the referent is always a third person (\ref{ex:ditrans_i'a}).

\begin{exe}
    \ex 
    \label{ex:ditrans_ino}
    \begin{xlist}
    \ex 
        \gll bere'i, ni mi 'ula-no o bebeoto \\
        old.person \textsc{2sg.a} \textsc{1pl.ex.u} give-\textsc{dir.ven} \textsc{nm} knife \\
        \trans `old woman, give us (ex.) a knife' [F.\ref{ex:text6-80}]
    \ex 
        \gll bere'i, nage o inomo i ti{\til}tipo'o'a, no 'i gai'i-no \\
        old.person \textsc{prox}.\textsc{exist} \textsc{nm} food 3\textsc{nh}.\textsc{a} \textsc{rdpl}{\til}short:\textsc{locv} \textsc{2sg.a} \textsc{3hpl.u} take-\textsc{dir.ven} \\
        \trans ‘old woman, there is a little bit of food left, take [it] for them’ [F.\ref{ex:text6-71}--\ref{ex:text6-72}]
    \end{xlist}
\end{exe}

\begin{exe}
    \ex  
        \gll o inomo no wi 'ula-'a \\
        \textsc{nm} food \textsc{2sg.a} \textsc{3sgm.u} give-\textsc{dir.itv} \\
        \trans `give him food to eat' [G.\ref{ex:text7-4}]
        \label{ex:ditrans_i'a}
\end{exe}

\subsection{Imperative clauses}

Imperative clauses are neither syntactically nor morphologically marked in any special way and hence formally indistinguishable from regular intransitive (\ref{ex:imp_intra}) and transitive clauses (\ref{ex:imp_trans}).

\begin{exe}
    \ex 
    \begin{xlist}
    \ex 
        \gll 'a nio lio-u nia dea-'a \\
        \textsc{foc} 2\textsc{pl}.\textsc{a} return-\textsc{already} 2\textsc{pl}.\textsc{poss} father-\textsc{dir.itv} \\
        \trans `return to your (pl.) father' [G.\ref{ex:text7-35}]
        \label{ex:imp_intra}
    \ex 
        \gll ni mi aha ani woa-'a \\
        \textsc{2sg.a} \textsc{1pl.ex.u} carry 2\textsc{sg}.\textsc{poss} house-\textsc{dir.itv} \\
        \trans `take us (ex.) to your (sg.) house' [D.\ref{ex:text4-10}]
        \label{ex:imp_trans}
    \end{xlist}
\end{exe}

Prohibitive clauses differ from regular negative clauses in that they are introduced by the negator \textit{uwa} (see \sectref{sec:negation}).

\section{Complex clauses}

In this section, I will briefly discuss complex clause structures, namely relative clauses and serial verb constructions. 

\subsection{Relative clauses}\label{subsec:relative_clause}

Modole does not have a dedicated means to mark relative clauses but there are some structures that can be interpreted as such. In \sectref{subsubsec:verbal_clauses-equative}, I describe verbal equative clauses. Such clauses can be used as arguments of verbs. 
Consider the example in (\ref{ex:relative_clause_simple}). The verb form \textit{i moi} `it is one' modifies \textit{o bibiti}, the undergoer argument of the verb form \textit{ya amono} `they catch it'. Attributive numerals are often constructed as relative clauses in Modole.

\begin{exe}
    \ex 
        \gll \textup{[}o bibiti i moi\textup{]} ya amono \\
        \textsc{nm} k.o.fish \textsc{3nh.a} one \textsc{3nh>3nh} catch.fish \\
        \trans `they caught a \textit{bibiti} fish' [B.\ref{ex:text2-4}]
    \label{ex:relative_clause_simple}
\end{exe}

Structures resembling headless relative clauses are also found.
In the example in (\ref{ex:rel_clause}), the clause \textit{ma lagudi ma howo'o wo jaga{\textasciitilde}jaga} `he guards the \textit{legundi} fruits' is an intransitive actor clause serving as the actor argument of the verb \textit{wa ino} `he comes here.'
As in this example, the verb in relative clauses is often reduplicated. 

\begin{exe}
    \ex 
        \gll de \textup{[}ma lagudi ma howo'o wo jaga{\textasciitilde}jaga\textup{]} wa ino mamane-'a \\
        \textsc{conn} \textsc{rnm} legundi \textsc{rnm} fruit 3\textsc{sg.m}.\textsc{a} guard 3\textsc{sg.m}>3\textsc{nh} \textsc{dir.ven} lover-\textsc{locv} \\
        \trans `and the \textit{legundi} guard approached his lover' (lit. `and he (who) guards the \textit{legundi} fruits approached his lover') [A.\ref{ex:text1-76}] \\
    \label{ex:rel_clause}
\end{exe}

\subsection{Serial verb constructions}

For the pupose of this grammar sketch, I define a serial verb construction in Modole as two verb forms without anything in between. Both verb forms occur with indices. This is similar to the definition in \citet{vanstaden2008}.
There are two very common types of serial verb constructions in which both verbs share the same actor referent. In the first, a verb of speaking is followed by \textit{temo} `say' (\ref{ex:SVC_temo}).
The second common type consists of a directed motion verb combined with another verb (\ref{ex:SVC_dir}). 

\begin{exe}
    \ex
    \begin{xlist}
    \ex
    \gll ma bere'i mo hano mo temo\\
    \textsc{rnm}	old.person	3\textsc{sg.f}.\textsc{a}	ask	3\textsc{sg.f}.\textsc{a}	say\\
    \trans `the old woman asked, saying:' [E.\ref{ex:text5-69}]
    \label{ex:SVC_temo}
    \ex 
    \gll o dogulu moi wa ino wo tahe\\
     \textsc{nm}	young.man	one	3\textsc{sg.m}>3\textsc{nh}	\textsc{dir}.\textsc{ven}	3\textsc{sg.m}.\textsc{a}	court\\
    \trans `a young man came to court [her]' [I.\ref{ex:text9-4}]
    \label{ex:SVC_dir}
     \end{xlist}
\end{exe}

Adverbials are also constructed as intransitive actor predicates in serial verb constructions (\ref{ex:adverb}).

\begin{exe}
    \ex
        \gll de mi ma'e i tiai \\
        \textsc{conn} 3\textsc{sg.f}>3\textsc{sg.m} see 3\textsc{nh}.\textsc{a} straight \\
        \trans `and she saw him well' [B.\ref{ex:text2-35}]
        \label{ex:adverb}
\end{exe}

\section{Negation}\label{sec:negation}

Modole has one general negation marker \textit{-ua} (\ref{ex:neg}). It is suffixed to the final lexeme of the clause (usually a verb) or phrase it has scope over. It can only be followed by the temporal suffixes. After vowels, \textit{-ua} is spelled <wa>. The spelling <uwa> occurs as well.

\begin{exe}
    \ex 
    \label{ex:neg}
    \begin{xlist}
    \ex 
        \gll mo ma 'o'ana a wo nga'u{-wa} \\
        but \textsc{rnm} king \textsc{foc} 3\textsc{sg.m}.\textsc{a} believe-\textsc{neg} \\
        \trans `but the king did not believe [it]' [F.\ref{ex:text6-113}]
    \ex 
        \gll o to ngoi{-wa}-u ai lagudi ma howo'o \\
        \textsc{emph} 1\textsc{sg}.\textsc{a} \textsc{pro}.1\textsc{sg}-\textsc{neg}-\textsc{already} 1\textsc{sg}.\textsc{poss} legundi \textsc{rnm} fruit \\
        \trans `my \textit{legundi} fruits are not mine' [A.\ref{ex:text1-105}]
    \end{xlist}
\end{exe}

\subsection{Prohibitive clauses}

Prohibitive clauses are introduced with \textit{uwa} (\ref{ex:proh}), a variant of \textit{-ua}. 

\begin{exe}
    \ex 
    \label{ex:proh}
    \begin{xlist}
    \ex 
        \gll {uwa} na poha \\
        \textsc{proh} 2\textsc{sg}>3\textsc{nh} beat \\
        \trans `don't beat it dead' [B.\ref{ex:text2-6}]
    \ex 
        \gll {uwa} no ali \\
        \textsc{proh} 2\textsc{sg}.\textsc{a} cry \\
        \trans `don't cry' [E.\ref{ex:text5-121}]
    \end{xlist}
\end{exe}

\textit{Uwa} is also used in isolation for `no'.

\begin{exe}
    \ex 
        \gll o {uwa}, to mi hi-gu-mada-wa \\
        \textsc{emph} \textsc{neg} 1\textsc{sg}.\textsc{a} 3\textsc{sg.m}.\textsc{u} \textsc{caus}-\textit{gu-}abandon-\textsc{neg} \\
        \trans `no, I won't let her alone' [C.\ref{ex:text3-8}]
\end{exe}

\subsection{Negated possession and existence}\label{subsec:neg_poss_exist}

Possession is negated in the form of a negative existential clause. It is formed either with \textit{'a-ua} `\textsc{foc}-\textsc{neg}' or \textit{'o'iwa}. The latter certainly features emphatic \textit{'o} and \textit{ua} but I do not know what exactly the middle element \textit{(')i} is. I therefore gloss it as \textsc{exist.neg} `negative existential'.

\begin{exe}
    \ex 
    \begin{xlist}
    \ex 
        \gll ai danongo ma {'a-ua} \\
        1\textsc{sg}.\textsc{poss} grand.child but \textsc{foc}-\textsc{neg} \\
        \trans `I don't have a grandchild' (lit. `my grandchild doesn't exist') [E.\ref{ex:text5-46}]\\
    \ex 
        \gll to muna ami pahita'a {'oiwa-u} \\
        \textsc{poss.hum} \textsc{pro}.3\textsc{sg.f} 3\textsc{sg.f}.\textsc{poss} mask \textsc{exist.neg}-\textsc{already} \\
        \trans `her mask was gone' (lit. `her mask is not anymore') [E.\ref{ex:text5-117}]
    \end{xlist}
\end{exe}

\section{Questions}

Modole distinguishes polar and content questions. Polar questions are formed liked declarative clauses. Content questions contain a question word.

\subsection{Polar questions}

Polar questions are not morphologically or syntactically marked in any way (\ref{ex:polar_ques}).

\begin{exe}
	\ex 
    \label{ex:polar_ques}
    \begin{xlist}
	\ex
        \gll neena to ngona ani lagudi ma howo'o? \\
        \textsc{prox:pro.3nh} \textsc{poss.hum} \textsc{pro}.2\textsc{sg} 2\textsc{sg}.\textsc{poss} legundi \textsc{rnm} fruit \\
        \trans `are these your (sg.) \textit{legundi} fruits?' [A.\ref{ex:text1-94}]
	\ex 
        \gll ngoi to wi na'o? \\
        \textsc{pro}.1\textsc{sg} 1\textsc{sg}.\textsc{a} 3\textsc{sg.m}.\textsc{u} know \\
        \trans `do I know about him?' [E.\ref{ex:text5-21}]
	\ex 
        \gll bote ngona ani lolabi mi tadi? \\
        maybe \textsc{pro}.2\textsc{sg} 2\textsc{sg}.\textsc{poss} kris 3\textsc{sg.f}.\textsc{u} stab \\
        \trans `maybe your (sg.) kris has stabbed her?' [I.\ref{ex:text9-16}]
    \end{xlist}
\end{exe}


\subsection{Content questions}

Content questions are formed with question words. 

\textit{O'ia} means `what?' or `who?' if the actor is assumed to be non-human or their gender is unknown (\ref{ex:o'ia}). It occurs in subject position.

\begin{exe}
	\ex 
	\label{ex:o'ia}
	\begin{xlist}
	\ex 
		\gll apu {o'ia} ma rame? \\
		grandmother what \textsc{rnm} feast \\
		\trans `grandmother, what feast is this?' [E.\ref{ex:text5-75}]
	\ex 
		\gll ai pihanga neena {o'ia} ya odomo? \\
		1\textsc{sg}.\textsc{poss} banana \textsc{prox}:\textsc{pro}.3\textsc{nh} what 3\textsc{nh}>3\textsc{nh} eat \\
		\trans `these bananas of mine, who has eaten them?' [D.\ref{ex:text4-3}]
	\end{xlist}
\end{exe}

If \textit{o'ia} is the predicate it occurs at the end of the clause (\ref{ex:o'ia_pred}).

\begin{exe}
	\ex 
	\label{ex:o'ia_pred}
	\begin{xlist}
	\ex 
		\gll o a'ele ma muhuti'a {o'ia}? \\
		\textsc{nm} water \textsc{rnm} pearl what \\
		\trans `what are the pearls of water?' [J.\ref{ex:text10-2}]
	\ex 
		\gll wange ca'o mo dodengo, wange cumu mo halai; geena {o'ia}? \\
		sun hour 3\textsc{sg.f}.\textsc{a} k.o.dance sun ?evening 3\textsc{sg.f}.\textsc{a} k.o.dance \textsc{dist:pro.3nh} what \\
		\trans `?in the morning she dances the \textit{dodengo} dance, in the evening she dances the \textit{cakalele} dance, what is that?' [J.\ref{ex:text10-6}]
	\end{xlist}
\end{exe}

In one example, \textit{o'ia} in combination with the optative particle \textit{tanu} translates as `how?' (\ref{ex:o'ia_how}).

\begin{exe}
	\ex 
		\gll yai, o hininga {o'ia} tanu po lio-u? \\
		mother \textsc{nm} heart what \textsc{mod} 1\textsc{pl.in}.\textsc{a} return-\textsc{already} \\
		\trans `mother, how may we (in.) return?' [G.\ref{ex:text7-52}]
		\label{ex:o'ia_how}
\end{exe}

\textit{O'ia'a} (< \textit{o'ia-o'a}) is used to ask `from where?' (\ref{ex:o'ia'a}). It is probably used when the focus is on the source location.

\begin{exe}
	\ex 
		\gll ngona {neena} {o'ia-'a} ngona de na ino? \\
		\textsc{pro}.2\textsc{sg} \textsc{prox}:\textsc{pro}.3\textsc{nh} where-\textsc{locv} \textsc{pro}.2\textsc{sg} \textsc{conn} 2\textsc{sg}>3\textsc{nh} \textsc{dir.ven} \\
		\trans `you (sg.) here, where are you (sg.) coming from?' [A.\ref{ex:text1-23}]    
		\label{ex:o'ia'a}
\end{exe}

When the focus is on the goal location (the deictic center), \textit{o'iano} (<\textit{o'ia-ino} `where here?' is used instead (\ref{ex:o'iano}).

\begin{exe}
	\ex 
	\label{ex:o'iano}
	\begin{xlist}
	\ex     
		\gll {o'ia-no} ngini ma nyawa? \\
		where-\textsc{dir.ven} \textsc{pro}.2\textsc{pl} \textsc{rnm} person \\
		\trans `where have you (pl.) [come from] here?' [F.\ref{ex:text6-150}]
	\ex 
		\gll o 'eto'o {o'ia-no}? \\
		\textsc{nm} sago what-\textsc{dir.ven} \\
		\trans `sago bread, where here [do I have that]?' [E.\ref{ex:text5-15}]
	\end{xlist}
\end{exe}

\textit{I dodoa} (< \textit{i dodoa} `3\textsc{nh.a} do') means `why?' (\ref{ex:i_dodoa}). It occurs at the beginning of the question.

\begin{exe}
	\ex 
    \label{ex:i_dodoa}
    \begin{xlist}
	\ex 
        \gll {i dodoa} ho o-na dawong-ua? \\
        why thus \textsc{emph-}\textsc{pro}.3\textsc{hpl} take-\textsc{neg} \\
        \trans `why don't you (sg.)take them?' [A.\ref{ex:text1-80}]
	\ex 
        \gll {i dodooa} ho no tagi? \\
        why thus 2\textsc{sg}.\textsc{a} go \\
        \trans `why did you (sg.) go [away]?' [E.\ref{ex:text5-30}]
    \end{xlist}
\end{exe}

When asking `who?' with a human referent in mind, \textit{nagoona} (< \textit{naga-ona}) is used for both actor (\ref{ex:nagoona}) and undergoer (\ref{ex:nagoona_U}).

\begin{exe}
	\ex 
    \label{ex:nagoona}
    \begin{xlist}
	\ex 
        \gll ai 'uho, {nagoona} ya tana? \\
        1\textsc{sg}.\textsc{poss} cucus \textsc{exist}:3\textsc{hpl} 3\textsc{nh}>3\textsc{nh} raise \\
        \trans `my cuscus, who shall raise it?' [F.\ref{ex:text6-4}]
	\ex 
        \gll {nagoona} yo tutu'u ai hohan-o'a? \\
        \textsc{exist}:3\textsc{hpl} 3\textsc{hpl}.\textsc{a} arrive 1\textsc{sg}.\textsc{poss} landing.place-\textsc{locv} \\
        \trans `who is arriving at my landing place?' [F.\ref{ex:text6-142}]
    \end{xlist}
\end{exe}

\begin{exe}
	\ex 
        \gll danongo 'a'ano ma 'oana awi ngoa'a ya butanga ani hininga ma hu'a {nagoona}? \\
        grandchild a.moment.ago \textsc{rnm} king 3\textsc{sg.m}.\textsc{poss} child 3\textsc{nh}>3\textsc{nh} six 2\textsc{sg}.\textsc{poss} heart \textsc{rnm} like \textsc{exist}:\textsc{3hpl} \\
        \trans `grandson, of the six daughters of the king a moment ago, who do you (sg.) desire?' [E.\ref{ex:text5-107}]
    \label{ex:nagoona_U}
\end{exe}

\textit{Ho} `thus, then so' also serves as a question word roughly meaning `how about?' (\ref{ex:ho_ques}). In this function it always occurs in the same kind of question, asking about the condition or state of a human or object.

\begin{exe}
	\ex 
    \label{ex:ho_ques}
    \begin{xlist}
	\ex 
        \gll tole, {ho} ma ngo'a? \\
        sister, thus \textsc{rnm} child \\
        \trans `sisters, how are the kids?' [F.\ref{ex:text6-51}]    
	\ex 
        \gll {ho} m'ai biara? \\
        thus \textsc{rnm}:1\textsc{sg}.\textsc{poss} biara \\
        \trans `how about my \textit{biara}?' [G.\ref{ex:text7-14}]
    \end{xlist}
\end{exe}

\section{Other}

In this section, I will discuss miscellaneous particles and other elements.

\subsection{Coordination and subordination}

Phrases and clauses are coordinated using \textit{de} `and, with' (see below), \textit{e'ola} `or' and \textit{ma} `but' (see [A.58]). 
Subordinators are \textit{na'o} `if, when', \textit{la} `so that', \textit{aha} `as a consequence, therefore', \textit{ho} `thus' and \textit{ngaro} `just, even if'. Subordinated clauses do not differ from matrix clauses besides the presence of the subordinator. I define them as subordinated since they are in some sense semantically dependent on the matrix clause. 
Larger chunks of text are separated by various elements meaning `then' (\textit{gena'a de}, \textit{i'a}, \textit{ena} and \textit{done}) or `after' (\textit{ma duangino}, \textit{i togumi'a}).

\textit{De} coordinates both phrases (\ref{ex:coor_phrase}) and clauses (\ref{ex:coor_clause}). 
\textit{De} has a wider function than `and' in English. In (\ref{ex:coor_phrase}), it separates the temporal adverbial \textit{o roeae'a} `during a riot' from the rest of the sentence.
I therefore gloss it as \textsc{conn} `connector' in the texts. An example for \textit{de} meaning `with' is given in (\ref{ex:de_with}).

\begin{exe}
    \ex 
    \begin{xlist}
    \ex 
        \gll o roeae-'a de yo loa {ma} eha de {ma} dea \\
        \textsc{nm} riot:\textsc{locv} \textsc{conn} 3\textsc{hpl}.\textsc{a} flee \textsc{rnm} mother \textsc{conn} \textsc{rnm} father \\
        \trans `during a riot they fled, the mother and the father' [A.\ref{ex:text1-2}]
        \label{ex:coor_phrase}
    \ex 
        \gll de yo loa, de ya i'a de o gota de yo doa-de \\
        \textsc{conn} 3\textsc{hpl}.\textsc{a} flee \textsc{conn} 3\textsc{nh}>3\textsc{nh} \textsc{dir.itv} \textsc{conn} \textsc{nm} tree \textsc{conn} 3\textsc{hpl}.\textsc{a} climb-\textsc{conn} \\
        \trans `and they ran away, and they went away and climbed a tree' [D.\ref{ex:text4-42}]
        \label{ex:coor_clause}
    \ex 
        \gll de ami bebeoto mo hi-legono \\
        \textsc{conn} 3\textsc{sg.f}.\textsc{poss} knife 3\textsc{sg.f}.\textsc{a} \textsc{caus}-catch \\
        \trans `and she caught [it] with her knife' (lit. `she made her knife catch [it]') [F.\ref{ex:text6-124}] \\
        \label{ex:de_with}
    \end{xlist}
\end{exe}

\subsection{Modality marker?: \textit{tanu}}

In the texts, I labeled \textit{tanu} \textsc{mod} `modality marker', for want of a better analysis. In two of the four attestations, \textit{tanu} introduces a request and may be translated as `please' (\ref{ex:tanu_please1}, \ref{ex:tanu_please2}). In another example, it seems to express epistemic modality (\ref{ex:tanu_epis}). In the fourth example, it translates as `may' in a deontic modality sense (\ref{ex:tanu_deo}). 

\begin{exe}
    \ex 
    \begin{xlist}
    \ex 
        \gll aba! tanu no da-pa'o o ngotili moi la to hi-noa ai bibiti \\
        father \textsc{mod} 2\textsc{sg}.\textsc{a} \textsc{appl}-nail \textsc{nm} boat one so.that 1\textsc{sg}.\textsc{a} \textsc{caus}-put \textsc{1sg.poss} bibiti \\
        \trans `father! please make a proa (little boat), so that I can put my \textit{bibiti} fish [in there]' [B.\ref{ex:text2-11}]    
        \label{ex:tanu_please1}
    \ex 
        \gll aba, tanu {ma moi} de ai baili no do-todang-ohi \\
        father \textsc{mod} once \textsc{conn} 1\textsc{sg}.\textsc{poss} garden 2\textsc{sg}.\textsc{a} \textsc{appl}-cut-\textsc{still} \\
        \trans `father! you (sg.) still have to weed my garden' [B.\ref{ex:text2-15}]
        \label{ex:tanu_please2}
    \ex 
        \gll na ne ma tanu ani we'ata ma ya tu{\til}tumiding-o'a \\
        ?here \textsc{prox} \textsc{rnm} \textsc{mod} 2\textsc{sg}.\textsc{poss} woman but 3\textsc{nh}>3\textsc{nh} \textsc{rdpl}{\til}seven-\textsc{lim} \\
        \trans `now your (sg.) women surely number seven already' [B.\ref{ex:text2-63}] %funtion of ma - coordinating dependent phrases, preceding head
        \label{ex:tanu_epis}
    \ex 
        \gll yai, o hininga o'ia tanu po lio-u? \\
        mother \textsc{nm} heart what \textsc{mod} 1\textsc{pl.in}.\textsc{a} return-\textsc{already} \\
        \trans `mother, how may we (in.) return?' [G.\ref{ex:text7-52}]
        \label{ex:tanu_deo}
    \end{xlist}
\end{exe}

\subsection{Focus: \textit{'a}}

\textit{'a}, also spelled <a> is a focus marker, i.e. it highlights one scenario among several alternatives (see examples in (\ref{ex:focus})).

\begin{exe}
    \ex 
    \label{ex:focus}
    \begin{xlist}
    \ex 
        \gll Neena ngomi mia u'u, \\
        \textsc{prox:pro.3nh} \textsc{pro}.1\textsc{pl.ex} 1\textsc{pl.ex}>3\textsc{nh} \textsc{dir.down} \\
        \trans `And now we'll (ex.) go down,'
    \ex 
        \gll mio hi-a-tubele, o hepa{\til}hepa ma 'o-ua, \\
        1\textsc{pl.ex}.\textsc{a} \textsc{caus-vpl}-provoke \textsc{nm} \textsc{rdpl}{\til}kick \textsc{rnm} \textsc{emph}-\textsc{neg} \\
        \trans `we'll (ex.) play football, no,'
    \ex 
        \gll o namo ma 'o-ua, \\
        \textsc{nm} chicken \textsc{rnm} \textsc{emph}-\textsc{neg} \\
        \trans `[we'll let] the cocks [fight], no,'
        \label{ex:'o-ua}
    \ex
        \gll ho {a} ngomi mio papahiara. \\
        thus \textsc{foc} \textsc{pro}.1\textsc{pl.ex} 1\textsc{pl.ex}.\textsc{a} go.for.a.walk \\
        \trans `so we'll (ex.) just go for a walk.' [A.\ref{ex:text1-59}--      \ref{ex:text1-62}]
    \end{xlist}
\end{exe}

\subsection{Emphasis?: \textit{'o}}\label{subsec:'o}

\textit{'o} (also spelled <o>) may likewise be a focus marker. In the texts, I tentatively gloss it \textsc{emph} `emphatic' (\ref{ex:emph}).
In the example in (\ref{ex:'o1}), it is translated as `of course not' (`welneen').

\begin{exe}
    \ex 
    \label{ex:emph}
    \begin{xlist}
    \ex 
        \gll ngoi 'o i to'at-ua \\
        \textsc{pro}.1\textsc{sg} \textsc{emph} 1\textsc{sg}.\textsc{u} witch-\textsc{neg} \\
        \trans `I am not a witch' [A.\ref{ex:text1-21}]
        \label{ex:'o1}
    \ex 
        \gll de awi hogo wo hi-do-dowanga de wo 'i li'o ma o ya adon-uwa \\
        \textsc{conn} 3\textsc{sg.m}.\textsc{poss} pubic.hair 3\textsc{sg.m}.\textsc{a} \textsc{caus}-\textsc{appl-}extend \textsc{conn} 3\textsc{sg.m}.\textsc{a} 3\textsc{hpl}.\textsc{u} bind but \textsc{emph} 3\textsc{nh}>3\textsc{nh} reach-\textsc{neg} \\
        \trans `and he extended his pubic hair to bind them but it did not reach it' [F.\ref{ex:text6-96}--\ref{ex:text6-97}]
    \end{xlist}
\end{exe}
