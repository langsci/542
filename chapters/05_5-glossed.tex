\ea\label{ex:text5-1}
O lia'a de o dodoto\\
\gll o	lia'a	de	o	dodoto\\
     \textsc{nm}	older.sibling	\textsc{conn}	\textsc{nm}	younger.sibling\\
\glt `The older and the younger brother'
\z

\ea\label{ex:text5-2}
Naga o njawa ja mididi, ma lia'a de ma dodoto.\\
\gll naga	o	nyawa	ya	mididi	ma	lia'a	de	ma	dodoto\\
     \textsc{exist}	\textsc{nm}	person	3\textsc{nh}>3\textsc{nh}	two	\textsc{rnm}	older.sibling	\textsc{conn}	\textsc{rnm}	younger.sibling\\
\glt `There were two people, an older and a younger brother.'\footnote{MZ: Ellen refers to the two brothers as `the oldest' and `the youngest'. In fact, \textit{lia'a} and \textit{dodoto} simply mean `older sibling' and `younger sibling' respectively.}
\z

\ea\label{ex:text5-3}
Ma lia'a o aho wo aho'o,\\
\gll ma	lia'a	o	aho	wo	aho'o\\
     \textsc{rnm}	older.sibling	\textsc{nm}	dog	3\textsc{sg.m}.\textsc{a}	call\\
\glt `The older one called for the dogs'\footnote{GJE: went hunting}
\z

\ea\label{ex:text5-4}
de ma dodoto wo tagi wo alo.\\
\gll de	ma	dodoto	wo	tagi	wo	alo\\
     \textsc{conn}	\textsc{rnm}	younger.sibling	3\textsc{sg.m}.\textsc{a}	go	3\textsc{sg.m}.\textsc{a}	beat.sago\\
\glt `and the younger one went beating sago.'
\z


\ea\label{ex:text5-5}
Ma lia'a wo tagi o aho wo aho'o,\\
\gll ma	lia'a	wo	tagi	o	aho	wo	aho'o\\
     \textsc{rnm}	older.sibling	3\textsc{sg.m}.\textsc{a}	go	\textsc{nm}	dog	3\textsc{sg.m}.\textsc{a}	call\\
\glt `The older one went to call the dogs'\footnote{GJE: hunting with the dogs}
\z

\ea\label{ex:text5-6}
de ma we'ati'a wo temo:\\
\gll de	ma	we'ata-i'a	wo	temo\\
     \textsc{conn}	\textsc{rnm}	wife-\textsc{dir}.\textsc{itv}	3\textsc{sg.m}.\textsc{a}	say\\
\glt `and to his wife he said:'
\z

\newpage
\ea\label{ex:text5-7}
 ``Na'o o ma ube ani geri wo boano,\\
\gll na'o	'o	ma	ube	ani	geri	wo	boa-ino\\
     \textsc{cond}	\textsc{emph}	\textsc{rnm}	little.bit	2\textsc{sg}.\textsc{poss}	sibling-in-law	3\textsc{sg.m}.\textsc{a}	come-\textsc{dir}.\textsc{ven}\\
\glt `{``}If in a little while your (sg.) brother-in-law comes here'
\z

\ea\label{ex:text5-8}
la o 'eto'o wo gaaho'o, de no wi 'ula'a.''\\
\gll la	o	'eto'o	wo	gaho'o	de	no wi	'ula-i'a\\
     so.that	\textsc{nm}	sago.bread	3\textsc{sg.m}.\textsc{a}	request	\textsc{conn}	2\textsc{sg}.\textsc{a} 3\textsc{sg.m}.\textsc{u}	give-\textsc{dir.itv}\\
\glt `and he asks for sago bread, then you (sg.) [have to] give [it] to him.'''\footnote{MZ: Sago -- the starch extracted from the sago palm (genus \textit{Metroxylon}) -- used to be the staple food in the North Moluccas. After being baked into blocks it is called sago bread.}
\z

\ea\label{ex:text5-9}
De ma we'ata mo temo:\\
\gll de	ma	we'ata	mo	temo\\
     \textsc{conn}	\textsc{rnm}	wife	3\textsc{sg.f}.\textsc{a}	say\\
\glt `And the wife said:'
\z

\ea\label{ex:text5-10}
 ``ngaro 'a no tagi,\\
\gll ngaro	'a	no	tagi\\
     just	\textsc{foc}	2\textsc{sg}.\textsc{a}	go\\
\glt `{``}Just go,'
\z

\ea\label{ex:text5-11}
a to wi 'ula'a ho'';\\
\gll 'a	to	wi	'ula-i'a	ho\\
     \textsc{foc}	1\textsc{sg}.\textsc{a}	3\textsc{sg.m}.\textsc{u}	give-?\textsc{dir.itv}	thus\\
\glt `I'll give [it] to him,'''
\z


\ea\label{ex:text5-12}
de wo tagi ho wo paha'a, ma dodoto wo boa.\\
\gll de	wo	tagi	ho	wo	paha-o'a	ma	dodoto	wo	boa\\
     \textsc{conn}	3\textsc{sg.m}.\textsc{a}	go	thus	3\textsc{sg.m}.\textsc{a}	leave-\textsc{lim}	\textsc{rnm}	younger.sibling	3\textsc{sg.m}.\textsc{a}	come\\
\glt `and he went and when he was gone, his younger brother came.'
\z

\newpage
\ea\label{ex:text5-13}
De wa ino de wo gaaho'o ma 'eto'o wo temo:\\
\gll de	wa	ino	de	wo	gaho'o	ma	'eto'o	wo	temo\\
     \textsc{conn}	3\textsc{sg.m}>3\textsc{nh}	\textsc{dir}.\textsc{ven}	\textsc{conn}	3\textsc{sg.m}.\textsc{a}	request	\textsc{rnm}	sago.bread	3\textsc{sg.m}.\textsc{a}	say\\
\glt `And he came and asked for sago bread, saying:'
\z

\ea\label{ex:text5-14}
 ``dawu, o 'eto'o ma ube ta odomo i hawini ho.''\\
\gll dawu	o	'eto'o	ma	ube	ta	odomo	i	hawini	ho\\
     sister-in-law	\textsc{nm}	sago.bread	\textsc{rnm}	little.bit	1\textsc{sg}>3\textsc{nh}	eat	1\textsc{sg}.\textsc{u}	hungry	thus\\
\glt `{``}Sister-in-law, [can] I eat some sago bread? I'm hungry.'''
\z

\ea\label{ex:text5-15}
De muna mo temo: ``o 'eto'o o'iano?'',\\
\gll de	muna	mo	temo	o	'eto'o	o'ia-ino\\
     \textsc{conn}	\textsc{pro}.3\textsc{sg.f}	3\textsc{sg.f}.\textsc{a}	say	\textsc{nm}	sago.bread	what-\textsc{dir}.\textsc{ven}\\
\glt `And she said: ``Sago bread, where here [do I have that]?'''
\z

\ea\label{ex:text5-16}
naga la awi po'o i lodidi.\\
\gll naga	la	awi	po'o	i	lodidi\\
     \textsc{exist}	so.that	3\textsc{sg.m}.\textsc{poss}	belly	3\textsc{nh}.\textsc{a}	painful\\
\glt `and his bowels rumbled\footnote{GJE: from hunger}.'\footnote{MZ: It is possible that \textit{naga} is part of the question in the previous clause.}
\z

\ea\label{ex:text5-17}
De una ma geri wi ma'e'e de wo tagi,\\
\gll de	una	ma	geri	wi	ma'e'e	de	wo	tagi\\
     \textsc{conn}	\textsc{pro}.3\textsc{sg.m}	\textsc{rnm}	sibling-in-law	3\textsc{sg.m}.\textsc{u}	ashamed	\textsc{conn}	3\textsc{sg.m}.\textsc{a}	go\\
\glt `And he, the brother-in-law, was ashamed and he went [away],'
\z

\ea\label{ex:text5-18}
ho wo hawu o de'u tumudingou.\\
\gll ho	wo	hawu	o	de'u	tumudingi-ou\\
     thus	3\textsc{sg.m}.\textsc{a}	climb	\textsc{nm}	mountain	seven-\textsc{foc}\\
\glt `so he climbed seven mountains.'
\z

\newpage
\ea\label{ex:text5-19}
Ma lia'a wo boa, de wo hano ma we'ai'a wo temo:\\
\gll ma	lia'a	wo	boa	de	wo	hano	ma	we'a-i'a	wo	temo\\
     \textsc{rnm}	older.sibling	3\textsc{sg.m}.\textsc{a}	come	\textsc{conn}	3\textsc{sg.m}.\textsc{a}	ask	\textsc{rnm}	wife-\textsc{dir}.\textsc{itv}	3\textsc{sg.m}.\textsc{a}	say\\
\glt `And the older brother came [back] and he asked his wife, saying:'
\z

\ea\label{ex:text5-20}
 ``Ho ani geri?''\\
\gll ho	ani	geri\\
     thus	2\textsc{sg}.\textsc{poss}	sibling-in-law\\
\glt `{``}How is your (sg.) brother-in-law?'''
\z

\ea\label{ex:text5-21}
De muna mo temo: ``Ngoi to wi na'o?''\\
\gll de	muna	mo	temo	ngoi	to	wi	na'o\\
     \textsc{conn}	\textsc{pro}.3\textsc{sg.f}	3\textsc{sg.f}.\textsc{a}	say	\textsc{pro}.1\textsc{sg}	1\textsc{sg}.\textsc{a}	3\textsc{sg.m}.\textsc{u}	know\\
\glt `And she said: ``Do I know [about] him?'''
\z

\ea\label{ex:text5-22}
De awi hininga'a wo temo:\\
\gll de	awi	hininga-o'a	wo	temo\\
     \textsc{conn}	3\textsc{sg.m}.\textsc{poss}	heart-\textsc{locv}	3\textsc{sg.m}.\textsc{a}	say\\
\glt `And he said to himself:'
\z

\ea\label{ex:text5-23}
 ``neena, o 'eto'o wo gaaho'o de o mi 'ulawa,\\
\gll neena	o	'eto'o	wo	gaho'o	de	'o	mi	'ula-ua\\
     \textsc{prox}:\textsc{pro}.3\textsc{nh}	\textsc{nm}	sago.bread	3\textsc{sg.m}.\textsc{a}	request	\textsc{conn}	\textsc{emph}	3\textsc{sg.f}>3\textsc{sg.m}	give-\textsc{neg}\\
\glt `{``}[It is like] this, he has asked for sago bread and she has not given [it] to him,'
\z

\ea\label{ex:text5-24}
ho wo tagioau.''\\
\gll ho	wo	tagi-o'au\\
     thus	3\textsc{sg.m}.\textsc{a}	go-\textsc{perf}\\
\glt `so he is gone.'''
\z

\newpage
\ea\label{ex:text5-25}
De wi tuulu ho o de'u tumidingi wa hawu\\
\gll de	wi	tuulu	ho	o	de'u	tumidingi	wa	hawu\\
     \textsc{conn}	3\textsc{sg.m}>3\textsc{sg.m}	follow	thus	\textsc{nm}	mountain	seven	3\textsc{sg.m}>3\textsc{nh}	climb\\
\glt `And he followed him\footnote{GJE: his younger brother}, climbing seven mountains'
\z

\ea\label{ex:text5-26}
de o aluhu wo poha de u ma togumu.\\
\gll de	o	aluhu	wo	poha	de	u	ma	togumu\\
     \textsc{conn}	\textsc{nm}	root	3\textsc{sg.m}.\textsc{a}	hit	\textsc{conn}	3\textsc{sg.m}.\textsc{a}	\textsc{mid}	finish\\
\glt `and he hit a tree root and stopped.'
\z

\ea\label{ex:text5-27}
De genaadau de jo ma togumu\\
\gll de	geena-o'a-dau	de	yo	ma	togumu\\
     \textsc{conn}	\textsc{dist}:\textsc{pro}.3\textsc{nh}-\textsc{locv}-\textsc{loc}.\textsc{down}	\textsc{conn}	3\textsc{hpl}.\textsc{a}	\textsc{mid}	finish\\
\glt `And then they [both] stopped,'
\z

\ea\label{ex:text5-28}
jo ma ena-ena'a;\\
\gll yo	ma	(C)V(C)V{\textasciitilde}ena'a\\
     3\textsc{hpl}.\textsc{a}	\textsc{mid}	\textsc{rdpl}{\textasciitilde}areca\\
\glt `they chewed betel'
\z

\ea\label{ex:text5-29}
de ma lia'a wo hano wo temo:\\
\gll de	ma	lia'a	wo	hano	wo	temo\\
     \textsc{conn}	\textsc{rnm}	older.sibling	3\textsc{sg.m}.\textsc{a}	ask	3\textsc{sg.m}.\textsc{a}	say\\
\glt `and the older one asked, saying:'
\z

\ea\label{ex:text5-30}
 ``I dodooa ho no tagi?''\\
\gll i	dodoa	ho	no	tagi\\
     3\textsc{nh}.\textsc{a}	why	thus	2\textsc{sg}.\textsc{a}	go\\
\glt `{``}Why did you (sg.) go [away]?'''
\z

\ea\label{ex:text5-31}
De wo temo: ``o'iawa ho ani we'ata o 'eto'o to gaaho'o\\
\gll de	wo	temo	o'ia-ua	ho	ani	we'ata	o	'eto'o	to	gaho'o\\
     \textsc{conn}	3\textsc{sg.m}.\textsc{a}	say	what-\textsc{neg}	thus	2\textsc{sg}.\textsc{poss}	wife	\textsc{nm}	sago.bread	1\textsc{sg}.\textsc{a}	request\\
\glt `And he said: ``[It's] nothing, I asked your (sg.) wife for sago bread'
\z

\ea\label{ex:text5-32}
de a mo i 'ulawa.''\\
\gll de	'a	mo	i	'ula-ua\\
     \textsc{conn}	\textsc{foc}	3\textsc{sg.f}.\textsc{a}	1\textsc{sg}.\textsc{u}	give-\textsc{neg}\\
\glt `and she did not give [it] to me.'''
\z

\ea\label{ex:text5-33}
De ma lia'a wo temo: ``na inou po liou.''\\
\gll de	ma	lia'a	wo	temo	na	ino-ou	po	lio-ou\\
     \textsc{conn}	\textsc{rnm}	older.sibling	3\textsc{sg.m}.\textsc{a}	say	2\textsc{sg}>3\textsc{nh}	\textsc{dir}.\textsc{ven}-\textsc{already}	1\textsc{pl}.\textsc{in}.\textsc{a}	return-\textsc{already}\\
\glt `And the older one said: ``Come here, [let]'s (in.) return!'''
\z

\ea\label{ex:text5-34}
De una wo temo: ``a to liowau.''\\
\gll de	una	wo	temo	'a	to	lio-ua-ou\\
     \textsc{conn}	\textsc{pro}.3\textsc{sg.m}	3\textsc{sg.m}.\textsc{a}	say	\textsc{foc}	1\textsc{sg}.\textsc{a}	return-\textsc{neg}-\textsc{already}\\
\glt `And he\footnote{GJE: the youngest} said: ``I won't return anymore.'''
\z

\ea\label{ex:text5-35}
De a jo ma'a tili'utou;\\
\gll de	'a	yo	ma-'a	tili'utu-ou\\
     \textsc{conn}	\textsc{foc}	3\textsc{hpl}.\textsc{a}	\textsc{mid}-\textsc{vpl}	?turn.back-\textsc{already}\\
\glt `And they turned their backs to each other\footnote{GJE: separated},'\footnote{MZ: \textit{Tili'utu} is not attested in the wordlist. In \posscitet{ellen1916c} Pagu wordlist, \textit{tilikut} is glossed `sit back to back' (``rug aan rug zitten'') and \citet{kotynski2022a} glosses \textit{tilikutu} in Tabaru as `put behind, turn around (?)'.}
\z

\ea\label{ex:text5-36}
una ma lia'a wo lio manga wo'a'a,\\
\gll una	ma	lia'a	wo	lio	manga	wo'a-i'a\\
     \textsc{pro}.3\textsc{sg.m}	\textsc{rnm}	older.sibling	3\textsc{sg.m}.\textsc{a}	return	3\textsc{hpl}.\textsc{poss}	house-\textsc{dir}.\textsc{itv}\\
\glt `he, the oldest, returned to their house,'
\z

\ea\label{ex:text5-37}
de una ma dodoto wo tagi ma homoaau,\\
\gll de	una	ma	dodoto	wo	tagi	ma	homoa-ou\\
     \textsc{conn}	\textsc{pro}.3\textsc{sg.m}	\textsc{rnm}	younger.sibling	3\textsc{sg.m}.\textsc{a}	go	\textsc{rnm}	other-\textsc{foc}\\
\glt `and he, the younger one, went to foreign [lands],'
\z

\ea\label{ex:text5-38}
de wo tagi ho mi doto'a o bere'i moi.\\
\gll de	wo	tagi	ho	mi	dota-o'a	o	bere'i	moi\\
     \textsc{conn}	3\textsc{sg.m}.\textsc{a}	go	thus	3\textsc{sg.m}>3\textsc{sg.f}	come.to-\textsc{lim}	\textsc{nm}	old.person	one\\
\glt `and he went and came to an old woman.'
\z

\ea\label{ex:text5-39}
De wo hulo'o o 'aho wo temo:\\
\gll de	wo	hulo'o	o	'aho	wo	temo\\
     \textsc{conn}	3\textsc{sg.m}.\textsc{a}	send	\textsc{nm}	dog	3\textsc{sg.m}.\textsc{a}	say\\
\glt `And he sent the dog, saying:'\footnote{MZ: Here and in the following lines, Ellen assumes several dogs. However, in [E.\ref{ex:text5-40}], the second person \textit{singular} indices \textit{na} and \textit{no} occur, suggesting that there is only one dog. In [E.\ref{ex:text5-41}] the index <jo> occurs but I consider it a typo for \textit{ya} `\textsc{3nh>3nh}'.}
\z

\ea\label{ex:text5-40}
 ``'aho na iha la na'o o ngo apu mi to'ata de no hingahu.\\
\gll 'aho	na	iha	la	na'o	o	ngo	apu	mi	to'ata	de	no	hi-ngahu\\
     dog	2\textsc{sg}>3\textsc{nh}	\textsc{dir}.\textsc{land}	so.that	\textsc{cond}	\textsc{nm}	\textsc{hon}.\textsc{fem}	grandmother	3\textsc{sg.f}.\textsc{u}	witch	\textsc{conn}	2\textsc{sg}.\textsc{a}	\textsc{caus}-report\\
\glt `{``}Dog, go landwards and if the grandmother is a witch, then tell [me]!'''
\z

\ea\label{ex:text5-41}
De jo iha de i lioo de i temo:\\
\gll de	ya	iha	de	i	lio-ou	de	i	temo\\
     \textsc{conn}	3\textsc{nh}>3\textsc{nh}	\textsc{dir}.\textsc{land}	\textsc{conn}	3\textsc{nh}.\textsc{a}	return-\textsc{already}	\textsc{conn}	3\textsc{nh}.\textsc{a}	say\\
\glt `And it went landwards and returned and said:'
\z

\ea\label{ex:text5-42}
 ``o mi to'atua'';\\
\gll 'o	mi	to'ata-ua\\
     \textsc{emph}	3\textsc{sg.f}.\textsc{u}	witch-\textsc{neg}\\
\glt `{``}She is not a witch,'''
\z

\ea\label{ex:text5-43}
de wa hulo'oli awi ali-ali de ja o'o\\
\gll de	wa	hulo'o-oli	awi	ali{\textasciitilde}ali	de	ya	o'o\\
     \textsc{conn}	3\textsc{sg.m}>3\textsc{nh}	send-\textsc{again}	3\textsc{sg.m}.\textsc{poss}	ring	\textsc{conn}	3\textsc{nh}>3\textsc{nh}	\textsc{dir}.\textsc{sea}\\
\glt `and he also sent his ring and it came seawards'
\z

\ea\label{ex:text5-44}
de i temo: ``o mi to'atua.''\\
\gll de	i	temo	'o	mi	to'ata-ua\\
     \textsc{conn}	3\textsc{nh}.\textsc{a}	say	\textsc{emph}	3\textsc{sg.f}.\textsc{u}	witch-\textsc{neg}\\
\glt `and it said: ``She is not a witch.'''
\z

\largerpage
\ea\label{ex:text5-45}
De wo aho'o u ma himaiti wo temo: ``Apu! Apu!''\\
\gll de	wo	aho'o	u	ma	hi-maiti	wo	temo	apu  apu\\
     \textsc{conn}	3\textsc{sg.m}.\textsc{a}	call	3\textsc{sg.m}.\textsc{a}	\textsc{mid}	\textsc{caus}-show	3\textsc{sg.m}.\textsc{a}	say	grandmother  grandmother\\
\glt `And he shouted and showed himself [and] he said: ``Grandmother! Grandmother!'''
\z

\ea\label{ex:text5-46}
de mo temo: ``hai, ai danongo ma 'aua.''\\
\gll de	mo	temo	hai	ai	danongo	ma	'a-ua\\
     \textsc{conn}	3\textsc{sg.f}.\textsc{a}	say	hey	1\textsc{sg}.\textsc{poss}	grandchild	\textsc{rnm}	\textsc{foc}-\textsc{neg}\\
\glt `and she said: ``Hey, I don't have a grandchild.'''
\z

\ea\label{ex:text5-47}
De una wo temo:\\
\gll de	una	wo	temo\\
     \textsc{conn}	\textsc{pro}.3\textsc{sg.m}	3\textsc{sg.m}.\textsc{a}	say\\
\glt `And he said:'
\z

\ea\label{ex:text5-48}
 ``Apu, bote ni to'ata, ho uwa no i ao!''\\
\gll apu	bote	ni	to'ata	ho	uwa	no	i	ao\\
     grandmother	maybe	2\textsc{sg}.\textsc{u}	witch	thus	\textsc{proh}	2\textsc{sg}.\textsc{a}	1\textsc{sg}.\textsc{u}	take\\
\glt `{``}Grandmother, maybe you (sg.) are a witch, but don't take me!'''
\z

\ea\label{ex:text5-49}
De muna mo temo: ``O i to'atua, o njawa ma aua.''\\
\gll de	muna	mo	temo	'o	i	to'ata-ua	o	nyawa	ma	aua\\
     \textsc{conn}	\textsc{pro}.3\textsc{sg.f}	3\textsc{sg.f}.\textsc{a}	say	\textsc{emph}	1\textsc{sg}.\textsc{u}	witch-\textsc{neg}	\textsc{nm}	person	\textsc{rnm}	genuine\\
\glt `And she said: ``I'm not a witch [but] a real human.'''
\z

\ea\label{ex:text5-50}
De ja i'a ami wo'a'a,\\
\gll de	ya	i'a	ami	wo'a-i'a\\
     \textsc{conn}	3\textsc{nh}>3\textsc{nh}	\textsc{dir}.\textsc{itv}	3\textsc{sg.f}.\textsc{poss}	house-\textsc{dir}.\textsc{itv}\\
\glt `And they went to her house,'
\z

\newpage
\ea\label{ex:text5-51}
de mo ha'ai ma bira ho jo odomo,\\
\gll de	mo	ha'ai	ma	bira	ho	yo	odomo\\
     \textsc{conn}	3\textsc{sg.f}.\textsc{a}	cook	\textsc{rnm}	rice	thus	3\textsc{hpl}.\textsc{a}	eat\\
\glt `and she cooked rice and they ate,'
\z

\ea\label{ex:text5-52}
de jo odomo ma duangino i wutu-oau\\
\gll de	yo	odomo	ma	duanga-ino	i	wutu-o'au\\
     \textsc{conn}	3\textsc{hpl}.\textsc{a}	eat	\textsc{rnm}	finish-\textsc{dir}.\textsc{ven}	3\textsc{nh}.\textsc{a}	night-\textsc{perf}\\
\glt `and after they had eaten, it had become night, '
\z

\ea\label{ex:text5-53}
ho jo ma idu.\\
\gll ho	yo	ma	idu\\
     thus	3\textsc{hpl}.\textsc{a}	\textsc{mid}	sleep\\
\glt `so they slept.'
\z

\ea\label{ex:text5-54}
De o wutu gogoranaau o omo-omo i ota'a manga wo'au,\\
\gll de	o	wutu	CV{\textasciitilde}gorona-ou	o	omo{\textasciitilde}omo	i	ota'a	manga	wo'a-ou\\
     \textsc{conn}	\textsc{nm}	night	\textsc{rdpl}{\textasciitilde}middle-\textsc{foc}	\textsc{nm}	annelid	3\textsc{nh}.\textsc{a}	fall	3\textsc{hpl}.\textsc{poss}	house-\textsc{foc}\\
\glt `And at midnight, a worm fell into their house,'\footnote{MZ: The form \textit{omo-omo} is glossed as `ringworm' (`annelid') here and in the wordlist. I do not know what species it refers to.}
\z

\ea\label{ex:text5-55}
de wo hidodo'ana wo temo:\\
\gll de	wo	hi-dodo'ana	wo	temo\\
     \textsc{conn}	3\textsc{sg.m}.\textsc{a}	\textsc{caus}-startle	3\textsc{sg.m}.\textsc{a}	say\\
\glt `and he was startled [by it] [and] said:'
\z

\ea\label{ex:text5-56}
 ``'iani a muna''\\
\gll 'iani	'a	muna\\
     ?must	\textsc{foc}	\textsc{pro}.3\textsc{sg.f}\\
\glt `{``}Certainly that's her'''\footnote{MZ: Cognates of \textit{'iani} mean `need, must' in several Mainland North Halmahera languages.}
\z

\newpage
\ea\label{ex:text5-57}
de wo aho'o wo temo:\\
\gll de	wo	aho'o	wo	temo\\
     \textsc{conn}	3\textsc{sg.m}.\textsc{a}	call	3\textsc{sg.m}.\textsc{a}	say\\
\glt `and he called, saying:'
\z

\ea\label{ex:text5-58}
 ``Apu o'iau naga?''\\
\gll apu	o'ia-ou	naga\\
     grandmother	what-\textsc{foc}	\textsc{exist}\\
\glt `{``}Grandmother, what is it?'''
\z

\ea\label{ex:text5-59}
de mo temo: ``geena naga omo-omo i ota'a.''\\
\gll de	mo	temo	geena	naga	omo{\textasciitilde}omo	i	ota'a\\
     \textsc{conn}	3\textsc{sg.f}.\textsc{a}	say	\textsc{dist}:\textsc{pro}.3\textsc{nh}	\textsc{exist}	annelid	3\textsc{nh}.\textsc{a}	fall\\
\glt `and she said: ``There is a worm that has fallen down.'''
\z

\ea\label{ex:text5-60}
De ma tu'u o hilo,\\
\gll de	ma	tu'u	o	hilo\\
     \textsc{conn}	3\textsc{sg.f}>3\textsc{nh}	light	\textsc{nm}	torch\\
\glt `And she lighted a torch,'
\z

\ea\label{ex:text5-61}
ena a ma omo-omo, gena'a i otauau.\\
\gll ena	'a	ma	omo{\textasciitilde}omo	geena-o'a	i	ota'a-o'au\\
     \textsc{pro}.3\textsc{nh}	\textsc{foc}	\textsc{rnm}	annelid	\textsc{dist}:\textsc{pro}.3\textsc{nh}-\textsc{locv}	3\textsc{nh}.\textsc{a}	fall-\textsc{perf}\\
\glt `and it was a worm that had fallen there.'
\z

\ea\label{ex:text5-62}
De ma dewelano ma 'oana o wange ma hiwa'a awi ngo'a o moholehe ja tumudingi\\
\gll de	ma	dewela-ino	ma	'oana	o	wange	ma	hiwa-o'a	awi	ngo'a	o	moholehe	ya	tumudingi\\
     \textsc{conn}	\textsc{rnm}	tomorrow-\textsc{dir}.\textsc{ven}	\textsc{rnm}	king	\textsc{nm}	sun	\textsc{rnm}	shine-\textsc{locv}	3\textsc{sg.m}.\textsc{poss}	child	\textsc{nm}	maiden	3\textsc{nh}>3\textsc{nh}	seven\\
\glt `And the next day seven daughters of the Eastern King'
\z

\ea\label{ex:text5-63}
ja u'u jo ma pahiara;\\
\gll ya	u'u	yo	ma	pahiara\\
     3\textsc{nh}>3\textsc{nh}	\textsc{dir}.\textsc{down}	3\textsc{hpl}.\textsc{a}	\textsc{mid}	walk\\
\glt `came walking southwards;'
\z

\ea\label{ex:text5-64}
de jo u'u, de jo temo:\\
\gll de	yo	u'u	de	yo	temo\\
     \textsc{conn}	3\textsc{hpl}.\textsc{a}	\textsc{dir}.\textsc{down}	\textsc{conn}	3\textsc{hpl}.\textsc{a}	say\\
\glt `they went southwards and they said:'
\z

\ea\label{ex:text5-65}
 ``Apu ho ani bounu ma hemo i'a-i'a a o toginita'a manga bounu.\\
\gll apu	ho	ani	bounu	ma	hemo	i'a{\textasciitilde}i'a	'a	o	toginita-'a	manga	bounu\\
     grandmother	thus	2\textsc{sg}.\textsc{poss}	smell	\textsc{rnm}	sweet	\textsc{rdpl}{\textasciitilde}\textsc{dir}.\textsc{itv}	\textsc{foc}	\textsc{nm}	supernatural.being-?\textsc{locv}	3\textsc{hpl}.\textsc{poss}	smell\\
\glt `{``}Grandmother, your (sg.) smell is sweet, like the smell of supernatural beings.'''
\z

\ea\label{ex:text5-66}
De muna mo temo:\\
\gll de	muna	mo	temo\\
     \textsc{conn}	\textsc{pro}.3\textsc{sg.f}	3\textsc{sg.f}.\textsc{a}	say\\
\glt `And she said:'
\z

\ea\label{ex:text5-67}
 ``danongo o'iawa, ai woa tamudja i pagoto ho'';\\
\gll danongo	o'ia-ua	ai	wo'a	tamuja	i	pagoto	ho\\
     grandchild	what-\textsc{neg}	1\textsc{sg}.\textsc{poss}	house	jasmine	3\textsc{nh}.\textsc{a}	smell	thus\\
\glt `{``}Grandchildren, [there is] nothing -- [in] my house jasmine flowers smell [like this],'''
\z

\ea\label{ex:text5-68}
gena'ade a o a'elu'u jo gila-gila jo uti.\\
\gll geena-o'a-de	'a	o	a'ele-u'u	yo	gila{\textasciitilde}gila	yo	uti\\
     \textsc{dist}:\textsc{pro}.3\textsc{nh}-\textsc{locv}-\textsc{conn}	\textsc{foc}	\textsc{nm}	water-\textsc{dir}.\textsc{down}	3\textsc{hpl}.\textsc{a}	straight	3\textsc{hpl}.\textsc{a}	descend\\
\glt `then they descended into the water.'
\z

\ea\label{ex:text5-69}
De ma bere'i mo hano mo temo:\\
\gll de	ma	bere'i	mo	hano	mo	temo\\
     \textsc{conn}	\textsc{rnm}	old.person	3\textsc{sg.f}.\textsc{a}	ask	3\textsc{sg.f}.\textsc{a}	say\\
\glt `And the old woman asked\footnote{GJE: [asked] the younger brother, who had come to her}, saying:'
\z

\ea\label{ex:text5-70}
 ``'Aano gengu'u tumudingi, a o'ia'a no 'i igo?''\\
\gll 'a'ano	ge-ngu'u	tumudingi	'a	o'ia-o'a	no	'i	igo\\
     just.now	\textsc{dist}-\textsc{n:dir.down}	seven	\textsc{foc}	what-\textsc{locv}	2\textsc{sg}.\textsc{a}	3\textsc{hpl}.\textsc{u}	love\\
\glt `{``}A moment ago, those seven down there, ?which ones do you (sg.) like?'''\footnote{MZ: The second part may mean `where are the ones that you like?'}
\z

\ea\label{ex:text5-71}
De wo temo: ``A ge mo tutuulu'';\\
\gll de	wo	temo	'a	ge	mo	CV{\textasciitilde}tuulu\\
     \textsc{conn}	3\textsc{sg.m}.\textsc{a}	say	\textsc{foc}	\textsc{dist}	3\textsc{sg.f}.\textsc{a}	\textsc{rdpl}{\textasciitilde}follow\\
\glt `And he said: ``The one that follows,'''\footnote{GJE: so the last one}
\z

\ea\label{ex:text5-72}
de ma 'oana o hupera wo tu'u ho a jo lioau.\\
\gll de	ma	'oana	o	hupera	wo	tu'u	ho	'a	yo	lio-ou\\
     \textsc{conn}	\textsc{rnm}	king	\textsc{nm}	k.o.small.cannon	3\textsc{sg.m}.\textsc{a}	shoot	thus	\textsc{foc}	3\textsc{hpl}.\textsc{a}	return-\textsc{already}\\
\glt `and the king fired a cannon, so they returned.'
\z

\ea\label{ex:text5-73}
Ena dauie jo tutu'u,\\
\gll ena	da'u-ie	yo	tutu'u\\
     \textsc{pro}.3\textsc{nh}	\textsc{loc}.\textsc{up}-\textsc{dir}.\textsc{up}	3\textsc{hpl}.\textsc{a}	arrive\\
\glt `When they arrived upwards,'\footnote{GJE: at the king's palace}
\z

\ea\label{ex:text5-74}
de manga damunu de manga liwanga a i ma hidodiota'a;\\
\gll de	manga	damunu	de	manga	liwanga	'a	i	ma	hididiota-o'a\\
     \textsc{conn}	3\textsc{hpl}.\textsc{poss}	drum	\textsc{conn}	3\textsc{hpl}.\textsc{poss}	gong	\textsc{foc}	3\textsc{nh}.\textsc{a}	\textsc{mid}	drum.beating-\textsc{lim}\\
\glt `and their drums and gongs were beaten;'
\z

\ea\label{ex:text5-75}
de wo hano wo temo: ``Apu o'ia ma rame?''\\
\gll de	wo	hano	wo	temo	apu	o'ia	ma	rame\\
     \textsc{conn}	3\textsc{sg.m}.\textsc{a}	ask	3\textsc{sg.m}.\textsc{a}	say	grandmother	what	\textsc{rnm}	feast\\
\glt `and he asked, saying: ``Grandmother, what feast is this?'''\footnote{GJE: there was a feast}
\z

\ea\label{ex:text5-76}
De mo temo: ``jo ato-ato o banahana ma doa'a ho jo ma'o atopo'o.''\\
\gll de	mo	temo	yo	ato{\textasciitilde}ato	o	banahana	ma	doa'a	ho	yo	ma-'o	'a-topo'o\\
     \textsc{conn}	3\textsc{sg.f}.\textsc{a}	say	3\textsc{hpl}.\textsc{a}	say	\textsc{nm}	k.o.palm	\textsc{rnm}	?top	thus	3\textsc{hpl}.\textsc{a}	\textsc{mid}-\textsc{emph}	\textsc{vpl}-stab\\
\glt `And she said: ``They say, they stab with tops of banahana plants\footnote{MZ: I do not know what kind of species \textit{banahana} refers to.} at each other.'''\footnote{GJE: a kind of game; MZ: Ellen translates \textit{jo ma'o atopo'o} as `they throw [...] at each other' (``men werpt elkaar [...]''). The root \textit{topo'o} and its cognates in other Core North Halmaheran languages is always glossed `stab', `pierce', etc. elsewhere.}
\z

\ea\label{ex:text5-77}
De wo temo: ``Apu, na'o ja aunu de po 'i gegelelo?''\\
\gll de	wo	temo	apu	na'o	ya	aunu	de	po	'i	CV{\textasciitilde}gelelo\\
     \textsc{conn}	3\textsc{sg.m}.\textsc{a}	say	grandmother	\textsc{cond}	3\textsc{nh}>3\textsc{nh}	blood	\textsc{conn}	1\textsc{pl}.\textsc{in}.\textsc{a}	3\textsc{hpl}.\textsc{u}	\textsc{rdpl}{\textasciitilde}watch\\
\glt `And he said: ``Grandmother, if they bleed, [will] we (in.) go watch them?'''
\z

\ea\label{ex:text5-78}
De jo tagi ho nge'omo ma oa de jo ma gogelu'u.\\
\gll de	yo	tagi	ho	nge'omo	ma	oa	de	yo	ma	gogele-u'u\\
     \textsc{conn}	3\textsc{hpl}.\textsc{a}	go	thus	way	\textsc{rnm}	?end	\textsc{conn}	3\textsc{hpl}.\textsc{a}	\textsc{mid}	sit-\textsc{dir}.\textsc{down}\\
\glt `And they went and at the end of the path they sat down.'
\z


\ea\label{ex:text5-79}
De muna mo hepa\\
\gll de	muna	mo	hepa\\
     \textsc{conn}	\textsc{pro}.3\textsc{sg.f}	3\textsc{sg.f}.\textsc{a}	kick\\
\glt `And she kicked'\footnote{GJE: played too, but what she played with is unclear}
\z

\ea\label{ex:text5-80}
ho i ota'a jo u'u awi mumuo'a aha awi taratibioali.\\
\gll ho	i	ota'a	ya	u'u	awi	mumu-o'a	aha	awi	taratibi-o'a-oli\\
     thus	3\textsc{nh}.\textsc{a}	fall	3\textsc{nh}>3\textsc{nh}	\textsc{dir}.\textsc{down}	3\textsc{sg.m}.\textsc{poss}	head-\textsc{locv}	as.a.consequence	3\textsc{sg.m}.\textsc{poss}	?lap-\textsc{locv}-\textsc{again}\\
\glt `so it fell down on his head and also on his lap.'\footnote{MZ: \textit{Taratibi} means `sit cross-legged' in several Core North Halmahera languages. \citet[335]{hueting1908b} also glosses it as `de kuil of ruimte die, aldus zittende, tusschen de beenen komt' (`the hole or space that is between the legs when sitting [cross-legged]').}
\z

\ea\label{ex:text5-81}
De unali wo hepa,\\
\gll de	una-oli	wo	hepa\\
     \textsc{conn}	\textsc{pro}.3\textsc{sg.m}-\textsc{again}	3\textsc{sg.m}.\textsc{a}	kick\\
\glt `And he kicked\footnote{GJE: played} as well, '
\z

\ea\label{ex:text5-82}
ho i ota'a ja u'u ami mumuo'a aha ami taratibio'ali.\\
\gll ho	i	ota'a	ya	u'u	ami	mumu-o'a	aha	ami	taratibi-o'a-oli\\
     thus	3\textsc{nh}.\textsc{a}	fall	3\textsc{nh}>3\textsc{nh}	\textsc{dir}.\textsc{down}	3\textsc{sg.f}.\textsc{poss}	head-\textsc{locv}	as.a.consequence	3\textsc{sg.f}.\textsc{poss}	?lap-\textsc{locv}-\textsc{again}\\
\glt `so it fell down on her head [and] on her lap too.'
\z

\ea\label{ex:text5-83}
Gena'ade a jo liou,\\
\gll geena-o'a-de	'a	yo	lio-ou\\
     \textsc{dist}:\textsc{pro}.3\textsc{nh}-\textsc{locv}-\textsc{conn}	\textsc{foc}	3\textsc{hpl}.\textsc{a}	return-\textsc{already}\\
\glt `Then they returned,'
\z

\ea\label{ex:text5-84}
ha ato ma dadaao ena jo boauali,\\
\gll ho	ato	ma	dadaao	ena	yo	boa-ou-oli\\
     thus	\textsc{excl}	\textsc{rnm}	after	\textsc{pro}.3\textsc{nh}	3\textsc{hpl}.\textsc{a}	come-\textsc{already}-\textsc{again}\\
\glt `but look, after that they came again,'
\z

\ea\label{ex:text5-85}
ja tumudingi jo ma hipahita'a o halaibia.\\
\gll ya	tumudingi	yo	ma	hi-pahita'a	o	halaibia\\
     3\textsc{nh}>3\textsc{nh}	seven	3\textsc{hpl}.\textsc{a}	\textsc{mid}	\textsc{caus}-wear.costume	\textsc{nm}	mask\\
\glt `the seven [and] they were disguised.'
\z

\ea\label{ex:text5-86}
De una ma doguulu u ma iuno'a o manuru ma goa'a,\\
\gll de	una	ma	doguulu	u	ma	iunu-o'a	o	manuru	ma	goa-o'a\\
     \textsc{conn}	\textsc{pro}.3\textsc{sg.m}	\textsc{rnm}	young.man	3\textsc{sg.m}.\textsc{a}	\textsc{mid}	hide-\textsc{lim}	\textsc{nm}	jasmine	\textsc{rnm}	?lower.part-\textsc{locv}\\
\glt `And he, the young man, hid under a jasmine [bush],'
\z

\newpage
\ea\label{ex:text5-87}
de to muna ma dodoto ami pahita'a wi dahaba'a una'a,\\
\gll de	to	muna	ma	dodoto	ami	pahita'a	wi	dV-haba'a	una-i'a\\
     \textsc{conn}	\textsc{poss}.\textsc{hum}	\textsc{pro}.3\textsc{sg.f}	\textsc{rnm}	younger.sibling	3\textsc{sg.f}.\textsc{poss}	mask	3\textsc{sg.m}.\textsc{u}	\textsc{appl}-touch.by.accident	\textsc{pro}.3\textsc{sg.m}-\textsc{dir}.\textsc{itv}\\
\glt `and the mask of the youngest hit him by accident'
\z

\ea\label{ex:text5-88}
de o a'elu'u jo utiou de wa iuno'a.\\
\gll de	o	a'ele-u'u	yo	uti-ou	de	wa	iunu-o'a\\
     \textsc{conn}	\textsc{nm}	water-\textsc{dir}.\textsc{down}	3\textsc{hpl}.\textsc{a}	descend-\textsc{already}	\textsc{conn}	3\textsc{sg.m}>3\textsc{nh}	hide-\textsc{lim}\\
\glt `and they descended into the water and he hid it.'\footnote{GJE: her mask}
\z

\ea\label{ex:text5-89}
Ho ato ma lia'a jo tagi,\\
\gll ho	ato	ma	lia'a	yo	tagi\\
     thus	\textsc{excl}	\textsc{rnm}	older.sibling	3\textsc{hpl}.\textsc{a}	go\\
\glt `And look, the older sisters went'
\z

\ea\label{ex:text5-90}
ena to muna ma mi pahita'a 'o'iwau;\\
\gll ena	to	muna	ma-ami	pahita'a	'o'iwa-ou\\
     \textsc{pro}.3\textsc{nh}	\textsc{poss}.\textsc{hum}	\textsc{pro}.3\textsc{sg.f}	\textsc{rnm}-3\textsc{sg.f}.\textsc{poss}	mask	\textsc{neg}.\textsc{exist}-\textsc{foc}\\
\glt `and her mask was gone'\footnote{GJE: of the youngest sister}
\z

\ea\label{ex:text5-91}
ena ma 'oana o hupera wo tu'uau,\\
\gll ena	ma	'oana	o	hupera	wo	tu'u-ou\\
     \textsc{pro}.3\textsc{nh}	\textsc{rnm}	king	\textsc{nm}	k.o.small.cannon	3\textsc{sg.m}.\textsc{a}	shoot-\textsc{already}\\
\glt `then the king fired a cannon'
\z

\ea\label{ex:text5-92}
de ona ma lia'a jo lio\\
\gll de	ona	ma	lia'a	yo	lio\\
     \textsc{conn}	\textsc{pro}.3\textsc{plh}	\textsc{rnm}	older.sibling	3\textsc{hpl}.\textsc{a}	return\\
\glt `and they, the older sisters, returned,'
\z

\ea\label{ex:text5-93}
ona ma a lioau.\\
\gll ona	ma	'a	lio-ou\\
     \textsc{pro}.3\textsc{plh}	but	\textsc{foc}	return-\textsc{already}\\
\glt `they had already returned.'
\z

\ea\label{ex:text5-94}
De ma bere'i mo temo:\\
\gll de	ma	bere'i	mo	temo\\
     \textsc{conn}	\textsc{rnm}	old.person	3\textsc{sg.f}.\textsc{a}	say\\
\glt `And the old woman said:'\footnote{GJE: to the youngest sister who had lost her mask}
\z

\ea\label{ex:text5-95}
 ``uwa no ali la no doa de'u'',\\
\gll uwa	no	ali	la	no	doa	de'u\\
     \textsc{proh}	2\textsc{sg}.\textsc{a}	weep	so.that	2\textsc{sg}.\textsc{a}	climb	mountain\\
\glt `{``}Don't cry, go up,'''
\z

\ea\label{ex:text5-96}
de mo doa, ho mo temo:\\
\gll de	mo	doa	ho	mo	temo\\
     \textsc{conn}	3\textsc{sg.f}.\textsc{a}	climb	thus	3\textsc{sg.f}.\textsc{a}	say\\
\glt `and she went up and said:'
\z

\ea\label{ex:text5-97}
 ``ai leletongo ena u ma himaitino'u'',\\
\gll ai	CV{\textasciitilde}letongo	ena	u	ma	hi-maiti-ino-u'u\\
     1\textsc{sg}.\textsc{poss}	\textsc{rdpl}{\textasciitilde}shine	\textsc{pro}.3\textsc{nh}	3\textsc{sg.m}.\textsc{a}	\textsc{mid}	\textsc{caus}-show-\textsc{dir}.\textsc{ven}-\textsc{dir}.\textsc{down}\\
\glt `{``}It is my mask that he is showing,'''
\z

\ea\label{ex:text5-98}
de a mi modo'oau.\\
\gll de	'a	mi	modo'a-o'au\\
     \textsc{conn}	\textsc{foc}	3\textsc{sg.f}>3\textsc{sg.m}	marry-\textsc{perf}\\
\glt `and she married him.'
\z

\newpage
\ea\label{ex:text5-99}
De ma 'oana wo hulo'o jo 'i hideehe muna de ma o'ata;\\
\gll de	ma	'oana	wo	hulo'o	yo	'i	hi-dV-ehe	muna	de	ma	o'ata\\
     \textsc{conn}	\textsc{rnm}	king	3\textsc{sg.m}.\textsc{a}	send	3\textsc{hpl}.\textsc{a}	3\textsc{hpl}.\textsc{u}	\textsc{caus}-\textsc{appl}-fetch	\textsc{pro}.3\textsc{sg.f}	\textsc{conn}	\textsc{rnm}	husband\\
\glt `And the king sent to fetch them, her and her husband,'
\z

\ea\label{ex:text5-100}
de ja u'u jo 'i ehe, de jo hiie,\\
\gll de	ya	u'u	yo	'i	ehe	de	yo	hi-ie\\
     \textsc{conn}	3\textsc{nh}>3\textsc{nh}	\textsc{dir}.\textsc{down}	3\textsc{hpl}.\textsc{a}	3\textsc{hpl}.\textsc{u}	fetch	\textsc{conn}	3\textsc{hpl}.\textsc{a}	\textsc{caus}-\textsc{dir}.\textsc{up}\\
\glt `and they descended, fetched them and brought [them] up'
\z


\ea\label{ex:text5-101}
de jo 'i hirame-rame o wutu tumudingi de o wange tumudingi.\\
\gll de	yo	'i	hi-rame{\textasciitilde}rame	o	wutu	tumudingi	de	o	wange	tumudingi\\
     \textsc{conn}	3\textsc{hpl}.\textsc{a}	3\textsc{hpl}.\textsc{u}	\textsc{caus}-\textsc{rdpl}{\textasciitilde}feast	\textsc{nm}	night	seven	\textsc{conn}	\textsc{nm}	sun	seven\\
\glt `and they feasted them for seven nights and seven days.'
\z

\ea\label{ex:text5-102}
De ma lia'a wo tuulu,\\
\gll de	ma	lia'a	wo	tuulu\\
     \textsc{conn}	\textsc{rnm}	older.sibling	3\textsc{sg.m}.\textsc{a}	follow\\
\glt `And the older brother followed,'
\z

\ea\label{ex:text5-103}
ho wo totara ma bere'inoli,\\
\gll ho	wo	totara	ma	bere'i-ino-oli\\
     thus	3\textsc{sg.m}.\textsc{a}	come.to	\textsc{rnm}	old.person-\textsc{dir}.\textsc{ven}-\textsc{again}\\
\glt `and he also reached the old woman,'\footnote{MZ: The form \textit{totara} is likely a borrowing since Modole has no /r/ phoneme. In Tabaru, \textit{totara} means `arrive, convey, communicate' \citep{kotynski2022a} and in Tobelo `run out on, come out on' \citep[370]{hueting1908b}. These are likely cognate to Modole \textit{tota} that occurs elsewhere (\textsc{pcnh} *r becomes {\o} in Modole in the context a\_a).}
\z

\newpage
\ea\label{ex:text5-104}
ena ja uoli ja butanga,\\
\gll ena	ya	u'u-oli	ya	butanga\\
     \textsc{pro}.3\textsc{nh}	3\textsc{nh}>3\textsc{nh}	\textsc{dir}.\textsc{down}-\textsc{again}	3\textsc{nh}>3\textsc{nh}	six\\
\glt `then they descended again, the six [sisters],'\footnote{GJE: also into the river; MZ: It is more likely that they descended from the dwelling of the king down to where the old woman lives.}
\z

\ea\label{ex:text5-105}
jo ma hipahita'a holoibia.\\
\gll yo	ma	hi-pahita'a	holoibia\\
     3\textsc{hpl}.\textsc{a}	\textsc{mid}	\textsc{caus}-wear.costume	mask\\
\glt `[and] they were disguised.'
\z

\ea\label{ex:text5-106}
De ma bere'i mo hano mo temo:\\
\gll de	ma	bere'i	mo	hano	mo	temo\\
     \textsc{conn}	\textsc{rnm}	old.person	3\textsc{sg.f}.\textsc{a}	ask	3\textsc{sg.f}.\textsc{a}	say\\
\glt `And the old woman asked, saying:'
\z


\ea\label{ex:text5-107}
 ``Danongo 'a'ano ma 'oana awi ngoa'a ja butanga ani hininga ma hu'a nagoona?''\\
\gll danongo	'a'ano	ma	'oana	awi	ngoa'a	ya	butanga	ani	hininga	ma	hu'a	nago-ona\\
     grandchild	just.now	\textsc{rnm}	king	3\textsc{sg.m}.\textsc{poss}	child	3\textsc{nh}>3\textsc{nh}	six	2\textsc{sg}.\textsc{poss}	heart	\textsc{rnm}	like	\textsc{exist}-\textsc{pro}.3\textsc{plh}\\
\glt `{``}Grandson, of the six daughters of the king a moment ago, who do you (sg.) desire?'''
\z

\ea\label{ex:text5-108}
De wo temo: ``'Age mo 'ogorona'a'',\\
\gll de	wo	temo	'a-ge	mo	'o-gorona-o'a\\
     \textsc{conn}	3\textsc{sg.m}.\textsc{a}	say	\textsc{foc}-\textsc{dist}	3\textsc{sg.f}.\textsc{a}	\textsc{emph}-middle-\textsc{locv}\\
\glt `And he said: ``That one in the middle,'''
\z

\ea\label{ex:text5-109}
ena ma 'oana uhupera wo tu'u,\\
\gll ena	ma	'oana	o-hupera	wo	tu'u\\
     \textsc{pro}.3\textsc{nh}	\textsc{rnm}	king	\textsc{nm}-k.o.small.cannon	3\textsc{sg.m}.\textsc{a}	shoot\\
\glt `then the king fired a cannon,'
\z

\newpage
\ea\label{ex:text5-110}
ho a jo lioali.\\
\gll ho	'a	yo	lio-oli\\
     thus	\textsc{foc}	3\textsc{hpl}.\textsc{a}	return-\textsc{again}\\
\glt `so they returned again.'
\z

\ea\label{ex:text5-111}
Ma dadaao ena ja uoli\\
\gll ma	dadaao	ena	ya	u'u-oli\\
     \textsc{rnm}	after	\textsc{pro}.3\textsc{nh}	3\textsc{nh}>3\textsc{nh}	\textsc{dir}.\textsc{down}-\textsc{again}\\
\glt `After that they descended again'
\z

\ea\label{ex:text5-112}
jo ma hipahita'a o holaibia;\\
\gll yo	ma	hi-pahita'a	o	holaibia\\
     3\textsc{hpl}.\textsc{a}	\textsc{mid}	\textsc{caus}-wear.costume	\textsc{nm}	mask\\
\glt `[and] they were disguised;'
\z

\ea\label{ex:text5-113}
de una u ma iuno'a o manuru ma goa'a;\\
\gll de	una	u	ma	iunu-o'a	o	manuru	ma	goa-o'a\\
     \textsc{conn}	\textsc{pro}.3\textsc{sg.m}	3\textsc{sg.m}.\textsc{a}	\textsc{mid}	hide-\textsc{lim}	\textsc{nm}	jasmine	\textsc{rnm}	?lower.part-\textsc{locv}\\
\glt `and he hid under a jasmine [bush]'
\z

\ea\label{ex:text5-114}
de to muna o gorona'a ami pahita'a i haba'a una'a,\\
\gll de	to	muna	o	gorona-o'a	ami	pahita'a	i	haba'a	una-i'a\\
     \textsc{conn}	\textsc{poss}.\textsc{hum}	\textsc{pro}.3\textsc{sg.f}	\textsc{nm}	middle-\textsc{locv}	3\textsc{sg.f}.\textsc{poss}	mask	3\textsc{nh}.\textsc{a}	touch.by.accident	\textsc{pro}.3\textsc{sg.m}-\textsc{dir}.\textsc{itv}\\
\glt `and the mask of the middle one accidentally hit him'
\z

\ea\label{ex:text5-115}
de o a'elu'u jo utio'u de wa iuno'a.\\
\gll de	o	a'ele-u'u	yo	uti-u'u	de	wa	iunu-o'a\\
     \textsc{conn}	\textsc{nm}	water-\textsc{dir}.\textsc{down}	3\textsc{hpl}.\textsc{a}	descend-\textsc{dir}.\textsc{down}	\textsc{conn}	3\textsc{sg.m}>3\textsc{nh}	hide-\textsc{lim}\\
\glt `and they descended into the water and he hid it.'\footnote{GJE: her mask}
\z

\ea\label{ex:text5-116}
Ho ata ma 'oana o hupera wo tu'u ena,\\
\gll ho	ato	ma	'oana	o	hupera	wo	tu'u	ena\\
     thus	\textsc{excl}	\textsc{rnm}	king	\textsc{nm}	k.o.small.cannon	3\textsc{sg.m}.\textsc{a}	shoot	\textsc{pro}.3\textsc{nh}\\
\glt `And look, the king fired a cannon'
\z


\ea\label{ex:text5-117}
to muna ami pahita'a 'oiwau;\\
\gll to	muna	ami	pahita'a	'oiwa-ou\\
     \textsc{poss}.\textsc{hum}	\textsc{pro}.3\textsc{sg.f}	3\textsc{sg.f}.\textsc{poss}	mask	\textsc{neg}.\textsc{exist}-\textsc{foc}\\
\glt `and her mask was gone'
\z

\ea\label{ex:text5-118}
ena ma lia'a jo liou,\\
\gll ena	ma	lia'a	yo	lio-ou\\
     \textsc{pro}.3\textsc{nh}	\textsc{rnm}	older.sibling	3\textsc{hpl}.\textsc{a}	return-\textsc{already}\\
\glt `and the older sisters returned'
\z

\ea\label{ex:text5-119}
de muna a mo ali,\\
\gll de	muna	'a	mo	ali\\
     \textsc{conn}	\textsc{pro}.3\textsc{sg.f}	\textsc{foc}	3\textsc{sg.f}.\textsc{a}	weep\\
\glt `and she cried'
\z

\ea\label{ex:text5-120}
de ma bere'i mo temo:\\
\gll de	ma	bere'i	mo	temo\\
     \textsc{conn}	\textsc{rnm}	old.person	3\textsc{sg.f}.\textsc{a}	say\\
\glt `and the old woman said:'
\z

\ea\label{ex:text5-121}
 ``Uwa no ali, la doade.''\\
\gll uwa	no	ali	la	doa-de\\
     \textsc{proh}	2\textsc{sg}.\textsc{a}	weep	so.that	climb-\textsc{conn}\\
\glt `{``}Don't cry, but go up.'''
\z

\ea\label{ex:text5-122}
De mo doa ho mo temo:\\
\gll de	mo	doa	ho	mo	temo\\
     \textsc{conn}	3\textsc{sg.f}.\textsc{a}	climb	thus	3\textsc{sg.f}.\textsc{a}	say\\
\glt `And she climbed up and said:'
\z

\newpage
\ea\label{ex:text5-123}
 ``ai leletongo ena a una.''\\
\gll ai	CV{\textasciitilde}letongo	ena	'a	una\\
     1\textsc{sg}.\textsc{poss}	\textsc{rdpl}{\textasciitilde}shine	\textsc{pro}.3\textsc{nh}	\textsc{foc}	\textsc{pro}.3\textsc{sg.m}\\
\glt `{``}My mask [is with] him.'''
\z

\ea\label{ex:text5-124}
De awi utu ma hongona hala'a, ma hongona o gurahi,\\
\gll de	awi	utu	ma	hongona	hala'a	ma	hongona	o	gurahi\\
     \textsc{conn}	3\textsc{sg.m}.\textsc{poss}	hair	\textsc{rnm}	half	silver	\textsc{rnm}	half	\textsc{nm}	gold\\
\glt `And half of his hair was silver [and] the [other] half was gold,'
\z

\ea\label{ex:text5-125}
de awi ilingi ma dauie o gurahi, dauu'u o hala'a.\\
\gll de	awi	ilingi	ma	da'u-ie	o	gurahi	dau-u'u	o	hala'a\\
     \textsc{conn}	3\textsc{sg.m}.\textsc{poss}	tooth	\textsc{rnm}	\textsc{loc}.\textsc{up}-\textsc{dir}.\textsc{up}	\textsc{nm}	gold	\textsc{loc}.\textsc{down}-\textsc{dir}.\textsc{down}	\textsc{nm}	silver\\
\glt `and his upper teeth were gold, [his] lower [teeth] silver.'
\z

\ea\label{ex:text5-126}
De ma 'oana wo hulo'o ho jo 'i ehe,\\
\gll de	ma	'oana	wo	hulo'o	ho	yo	'i	ehe\\
     \textsc{conn}	\textsc{rnm}	king	3\textsc{sg.m}.\textsc{a}	send	thus	3\textsc{hpl}.\textsc{a}	3\textsc{hpl}.\textsc{u}	fetch\\
\glt `And the king sent to fetch them,'
\z

\ea\label{ex:text5-127}
de jo u'u jo 'i ehe\\
\gll de	yo	u'u	yo	'i	ehe\\
     \textsc{conn}	3\textsc{hpl}.\textsc{a}	\textsc{dir}.\textsc{down}	3\textsc{hpl}.\textsc{a}	3\textsc{hpl}.\textsc{u}	fetch\\
\glt `and they descended to fetch them'
\z

\ea\label{ex:text5-128}
de ma bere'i a mo ali.\\
\gll de	ma	bere'i	'a	mo	ali\\
     \textsc{conn}	\textsc{rnm}	old.person	\textsc{foc}	3\textsc{sg.f}.\textsc{a}	weep\\
\glt `and the old woman cried.'
\z

\ea\label{ex:text5-129}
De wo temo: ``Uwa no ali la na ino a dede ngone,''\\
\gll de	wo	temo	uwa	no	ali	la	na	ino	'a	de{\textasciitilde}de	ngone\\
     \textsc{conn}	3\textsc{sg.m}.\textsc{a}	say	\textsc{proh}	2\textsc{sg}.\textsc{a}	weep	so.that	2\textsc{sg}>3\textsc{nh}	\textsc{dir}.\textsc{ven}	\textsc{foc}	\textsc{rdpl}{\textasciitilde}\textsc{conn}	\textsc{pro}.1\textsc{pl}.\textsc{in}\\
\glt `And he\footnote{GJE: the older brother} said: ``Don't cry but come here with us (in.),'''
\z

\ea\label{ex:text5-130}
de jo hiie,\\
\gll de	yo	hi-ie\\
     \textsc{conn}	3\textsc{hpl}.\textsc{a}	\textsc{caus}-\textsc{dir}.\textsc{up}\\
\glt `and they brought [them] up,'
\z

\ea\label{ex:text5-131}
ja ie de wo hano wo temo:\\
\gll ya	ie	de	wo	hano	wo	temo\\
     3\textsc{nh}>3\textsc{nh}	\textsc{dir}.\textsc{up}	\textsc{conn}	3\textsc{sg.m}.\textsc{a}	ask	3\textsc{sg.m}.\textsc{a}	say\\
\glt `they went up and he\footnote{GJE: the king} asked, saying:'
\z

\ea\label{ex:text5-132}
 ``Ngona o'ia no ma njawa,\\
\gll ngona	o'ia	no	ma	nyawa\\
     \textsc{pro}.2\textsc{sg}	what	2\textsc{sg}.\textsc{a}	\textsc{mid}	person\\
\glt `{``}You, what kind of human are you (sg.),'
\z

\ea\label{ex:text5-133}
ho ani rupa de ani gonaga i ho a o djini de widadario'a.\\
\gll ho	ani	rupa	de	ani	gonaga	i	ho	'a	o	jini	de	widadari-o'a\\
     thus	2\textsc{sg}.\textsc{poss}	shape	\textsc{conn}	2\textsc{sg}.\textsc{poss}	appearance	3\textsc{nh}.\textsc{a}	thus	\textsc{foc}	\textsc{nm}	jinn	\textsc{conn}	nymph-\textsc{locv}\\
\glt `because your (sg.) shape and appearance resemble a jinn and nymph\footnote{MZ: \textit{Widadari} was borrowed into several Core North Halmahera languages from Javanese \textit{widyādharī} `nymph, divine maiden' \citep[133]{zurbuchen1976}, itself derived from Sanskrit \textit{vidyādharī} `bearer of wisdom' \citep[22]{platenkamp2015}.}.'''\footnote{MZ: The particle \textit{ho} seems to function as a verb here (also in [F.\ref{ex:text6-8}], [F.\ref{ex:text6-12}] and [F.\ref{ex:text6-95}]).}
\z


\ea\label{ex:text5-134}
De wo temo: ``Djou lamo-lamo ngoi neena mimididio'a,\\
\gll de	wo	temo	jou	lamo{\textasciitilde}lamo	ngoi	neena	mi-mididi-o'a\\
     \textsc{conn}	3\textsc{sg.m}.\textsc{a}	say	lord	great	\textsc{pro}.1\textsc{sg}	\textsc{prox}:\textsc{pro}.3\textsc{nh}	1\textsc{pl}.\textsc{ex}.\textsc{a}-two-\textsc{locv}\\
\glt `And he said: ``Great lord\footnote{The phrase \textit{jou lamo-lamo} is borrowed from Ternate. The Modole cognate of \textit{lamo} is \textit{lamo'o} or \textit{amo'o}.}, I here, we (ex.) were two,'
\z

\ea\label{ex:text5-135}
ngoi o 'aho to aho'o'a,\\
\gll ngoi	o	'aho	to	aho'o-o'a\\
     \textsc{pro}.1\textsc{sg}	\textsc{nm}	dog	1\textsc{sg}.\textsc{a}	call-\textsc{lim}\\
\glt `I called the dogs'\footnote{GJE: was a hunter}
\z

\ea\label{ex:text5-136}
de una, ai dodoto wo alo\\
\gll de	una	ai	dodoto	wo	alo\\
     \textsc{conn}	\textsc{pro}.3\textsc{sg.m}	1\textsc{sg}.\textsc{poss}	younger.sibling	3\textsc{sg.m}.\textsc{a}	beat.sago\\
\glt `and he, my younger brother, beat sago'
\z


\ea\label{ex:text5-137}
de to temo: `Na'o ani geri wo boano de o eto'o no wi ula'ä.'\\
\gll de	to	temo	na'o	ani	geri	wo	boa-ino	de	o	eto'o	no	wi	'ula-'a\\
     \textsc{conn}	1\textsc{sg}.\textsc{a}	say	\textsc{cond}	2\textsc{sg}.\textsc{poss}	sibling-in-law	3\textsc{sg.m}.\textsc{a}	come-\textsc{dir}.\textsc{ven}	\textsc{conn}	\textsc{nm}	sago.bread	2\textsc{sg}.\textsc{a}	3\textsc{sg.m}.\textsc{u}	give-?\textsc{lim}\\
\glt `and I said: `If your (sg.) brother-in-law comes here, give him sago bread.'{'}\footnote{GJE: to his wife he said that, according to the preceding narrative}
\z

\ea\label{ex:text5-138}
De wa ino wo gaaho'o de o mi 'ulawa,\\
\gll de	wa	ino	wo	gaho'o	de	'o	mi	'ula-ua\\
     \textsc{conn}	3\textsc{sg.m}>3\textsc{nh}	\textsc{dir}.\textsc{ven}	3\textsc{sg.m}.\textsc{a}	request	\textsc{conn}	\textsc{emph}	3\textsc{sg.f}>3\textsc{sg.m}	give-\textsc{neg}\\
\glt `And he came here to ask [for it], and she did not give [it] to him,'
\z

\ea\label{ex:text5-139}
de wi ma'e'e de wo tagi,\\
\gll de	wi	ma'e'e	de	wo	tagi\\
     \textsc{conn}	3\textsc{sg.m}.\textsc{u}	ashamed	\textsc{conn}	3\textsc{sg.m}.\textsc{a}	go\\
\glt `and he was ashamed and he went [away],'
\z

\ea\label{ex:text5-140}
ho de'u tumudingi de o ngaili tumudingi a u ma hiadono,\\
\gll ho	de'u	tumudingi	de	o	ngaili	tumudingi	'a	u	ma	hi-adono\\
     thus	mountain	seven	\textsc{conn}	\textsc{nm}	river	seven	\textsc{foc}	3\textsc{sg.m}.\textsc{a}	\textsc{mid}	\textsc{caus}-reach\\
\glt `and over seven mountains and seven rivers he made it,'
\z

\ea\label{ex:text5-141}
de to wi tuulu.\\
\gll de	to	wi	tuulu\\
     \textsc{conn}	1\textsc{sg}.\textsc{a}	3\textsc{sg.m}.\textsc{u}	follow\\
\glt `and I followed him.'
\z

\ea\label{ex:text5-142}
De to wi daeni'a\\
\gll de	to	wi	daene-i'a\\
     \textsc{conn}	1\textsc{sg}.\textsc{a}	3\textsc{sg.m}.\textsc{u}	find-\textsc{dir}.\textsc{itv}\\
\glt `And I found him'
\z

\ea\label{ex:text5-143}
de mi ma ena-ena'a i togumu de to temo:\\
\gll de	mi	ma	(C)V(C)V{\textasciitilde}ena'a	i	togumu	de	to	temo\\
     \textsc{conn}	1\textsc{pl}.\textsc{ex}.\textsc{a}	\textsc{mid}	\textsc{rdpl}{\textasciitilde}areca	3\textsc{nh}.\textsc{a}	finish	\textsc{conn}	1\textsc{sg}.\textsc{a}	say\\
\glt `and after we (ex.) had finished chewing betel, I said:
\z

\ea\label{ex:text5-144}
 `Na inou po liou;'\\
\gll na	ino-ou	po	lio-ou\\
     2\textsc{sg}>3\textsc{nh}	\textsc{dir}.\textsc{ven}-\textsc{already}	1\textsc{pl}.\textsc{in}.\textsc{a}	return-\textsc{already}\\
\glt `{`}Come here, [let]'s (in.) return;'{'}
\z

\ea\label{ex:text5-145}
de wo temo: `Wolo, ngoi ai ma'eou to lio ai duduui'a ho;\\
\gll de	wo	temo	wolo	ngoi	'a-i	ma'e'e-ou	to	lio	ai	duduu-i'a	ho\\
     \textsc{conn}	3\textsc{sg.m}.\textsc{a}	say	well	\textsc{pro}.1\textsc{sg}	\textsc{foc}-1\textsc{sg}.\textsc{u}	ashamed-\textsc{already}	1\textsc{sg}.\textsc{a}	return	1\textsc{sg}.\textsc{poss}	?back-\textsc{dir}.\textsc{itv}	thus\\
\glt `and he said: `Well, I'm ashamed to return ?my back there;'\footnote{MZ: Based on Ellen's translation, the form <duduui'a> may contain \textit{dudun} `back'. The absence of /n/ is unexplained.}
\z

\ea\label{ex:text5-146}
ho ngona no liou, la ngoi ma to tagiou.'{''}\\
\gll ho	ngona	no	lio-ou	la	ngoi	ma	to	tagi-ou\\
     thus	\textsc{pro}.2\textsc{sg}	2\textsc{sg}.\textsc{a}	return-\textsc{already}	but	\textsc{pro}.1\textsc{sg}	but	1\textsc{sg}.\textsc{a}	go-\textsc{already}\\
\glt `so you (sg.) return, but I will go on.'{''}{'}
\z

\ea\label{ex:text5-147}
De ma 'oana wo temo:\\
\gll de	ma	'oana	wo	temo\\
     \textsc{conn}	\textsc{rnm}	king	3\textsc{sg.m}.\textsc{a}	say\\
\glt `And the king said:'
\z

\ea\label{ex:text5-148}
 ``Abei'a ani dopo-dopo ma harumu na ai'ino,\\
\gll abei'a	ani	dopo{\textasciitilde}dopo	ma	harumu	na	'ai'i-ino\\
     well	2\textsc{sg}.\textsc{poss}	kris	\textsc{rnm}	sheath	2\textsc{sg}>3\textsc{nh}	take.out-\textsc{dir}.\textsc{ven}\\
\glt `{``}Well then, take your (sg.) kris\footnote{MZ: Ellen equates both \textit{dopo-dopo} and \textit{lolabi} (see \textref{chapter:text09}) with the Javanese \textit{k(e)ris} dagger, whence English \textit{kris}. Based on their shape, both forms are borrowings.} sheath out [of your belt],'
\z

\ea\label{ex:text5-149}
la ai bebeoto to hiharumu'',\\
\gll la	ai	bebeoto	to	hi-harumu\\
     so.that	1\textsc{sg}.\textsc{poss}	knife	1\textsc{sg}.\textsc{a}	\textsc{caus}-sheath\\
\glt `so that I [may] put my knife in it,'''
\z

\ea\label{ex:text5-150}
de wo hiharumu 'a i ma tero.\\
\gll de	wo	hi-harumu	'a	i	ma	tero\\
     \textsc{conn}	3\textsc{sg.m}.\textsc{a}	\textsc{caus}-sheath	\textsc{foc}	3\textsc{nh}.\textsc{a}	\textsc{mid}	beautiful\\
\glt `and he put [it] into the sheath and it fit.'
\z

\ea\label{ex:text5-151}
De wo temo: ``abei'a ani ali-ali to ma hinoa;''\\
\gll de	wo	temo	abei'a	ani	ali{\textasciitilde}ali	to	ma	hi-noa\\
     \textsc{conn}	3\textsc{sg.m}.\textsc{a}	say	well	2\textsc{sg}.\textsc{poss}	ring	1\textsc{sg}.\textsc{a}	\textsc{mid}	\textsc{caus}-put\\
\glt `And he\footnote{GJE: the king} said: ``Well then, [let me] put your (sg.) ring on,'''
\z

\ea\label{ex:text5-152}
de u ma hinoa, 'a i tero.\\
\gll de	u	ma	hi-noa	'a	i	tero\\
     \textsc{conn}	3\textsc{sg.m}.\textsc{a}	\textsc{mid}	\textsc{caus}-put	\textsc{foc}	3\textsc{nh}.\textsc{a}	beautiful\\
\glt `and he put [it] on, and it fit.'
\z

\ea\label{ex:text5-153}
De wo temo: ``Abei'a ani hahawi po hiatogo'o'',\\
\gll de	wo	temo	abei'a	ani	hahawi	po	hi-'a-togo-o'o\\
     \textsc{conn}	3\textsc{sg.m}.\textsc{a}	say	well	2\textsc{sg}.\textsc{poss}	coconut.shell	1\textsc{pl}.\textsc{in}.\textsc{a}	\textsc{caus}-\textsc{vpl}-add-\textsc{dir}.\textsc{sea}\\
\glt `And he said [again]: ``Well then, [let]'s (in.) connect your (sg.) coconut shells,'''
\z

\ea\label{ex:text5-154}
de jo hiatogo'o 'a i ma tero,\\
\gll de	yo	hi-'a-togo-o'o	'a	i	ma	tero\\
     \textsc{conn}	3\textsc{hpl}.\textsc{a}	\textsc{caus}-\textsc{vpl}-add-\textsc{dir}.\textsc{sea}	\textsc{foc}	3\textsc{nh}.\textsc{a}	\textsc{rnm}	beautiful\\
\glt `and they connected them [and] they fit,'\footnote{GJE: these three methods seem to be tested to find out if one is still family}
\z

\ea\label{ex:text5-155}
de ma 'oana awi gogono'o o wa po'a de wo temo:\\
\gll de	ma	'oana	awi	gogono'o	'o	wa	po'a	de	wo	temo\\
     \textsc{conn}	\textsc{rnm}	king	3\textsc{sg.m}.\textsc{poss}	sternum	\textsc{emph}	3\textsc{sg.m}>3\textsc{nh}	slap	\textsc{conn}	3\textsc{sg.m}.\textsc{a}	say\\
\glt `and the king beat his chest and he said:'
\z


\ea\label{ex:text5-156}
 ``na'o o geena de ngoiou;''\\
\gll na'o	'o geena	de	ngoi-ou\\
     \textsc{cond}	\textsc{emph} \textsc{dist}:\textsc{pro}.3\textsc{nh}	\textsc{conn}	\textsc{pro}.1\textsc{sg}-\textsc{foc}\\
\glt `{``}If that's the case, it's me,'''\footnote{MZ: The younger brother has become king.}
\z

\ea\label{ex:text5-157}
de jo ma hirame-rame o wutu tumudingi, de o wange tumudingino.\\
\gll de	yo	ma	hi-rame{\textasciitilde}rame	o	wutu	tumudingi	de	o	wange	tumudingi-ino\\
     \textsc{conn}	3\textsc{hpl}.\textsc{a}	\textsc{mid}	\textsc{caus}-\textsc{rdpl}{\textasciitilde}feast	\textsc{nm}	night	seven	\textsc{conn}	\textsc{nm}	sun	seven-\textsc{dir}.\textsc{ven}\\
\glt `and they celebrated for seven nights and seven days ?from now on.'
\z
