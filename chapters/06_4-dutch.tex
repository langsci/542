\subsection*{De buidelrat}

Er was een mensch (vrouw), zij baarde een buidelrat.\footnote{MZ: The Dutch word \textit{buidelrat} nowadays usually refers to an opossum but was commonly used for cuscuses at the beginning of the 20th century.} En de moeder vroeg betreffende hem (die buidelrat), zij zeide: ``Mijn buidelrat, wie zal hem grootbrengen?'' En (van) de koning zijn zeven kinderen (dochters) de jongste, zij wilde (het wel). En zij (de zes andere dochters) zeiden: ``Zusje, (zijt) gij bedroefd over de mannen (hebt gij een afkeer van de mannen), dat gij nu een buidelrat (gaat) trouwen?'' En zij zweeg maar, (en) toen een volgenden dag ook (was) het hun feest. En zij (de zusters) zeiden: ``Morgen vieren wij feest overdag, dan (zal) je man, (die) daarginds zeewaarts (is ook) afdalen (om met ons feest te vieren) (met) zijn leelijk uiterlijk.''

En zij zweeg (weer), toen vierden zij hier feest en de buidelrat zeide: ``Ontdoe gij die cocosnooten aldaar voor ons van hun bast, opdat wij ons haar wasschen en beolieën''; en de oudere (zusters) zeiden: ``Zusje, alzoo wat van hem (zal) hij wasschen en beolieën (want een buidelrat heeft slechts heel kort haar).'' En zij ontbastte (de cocosnaot) om hun haar te wasschen en te beolieën, zij gingen (daarop) zeewaarts, naar het strand, vlak bij het zeewater en waschten en beolieden (aldaar) hun haar. En hij zeide: ``gij ook eerst,'' en zij waschte en beoliede haar haar. En toen zij klaar was met het wasschen en beolieën zeide zij: ``gij ook,'' en hij (deed het) ook, (toen) zeide hij: ``sluit je oogen,'' en zij sloot hare oogen, toen zeide zij: ``het schittert'' dat was, dat hij zijne vermomming uitgetrokken had.

De helft van zijn haar (was) zilver, zijn (andere) helft (was) goud (en) de helft van zijne tanden (was) zilver (en) de (andere) helft (was) goud. En zij gingen terug en zij (de andere zusters) troffen hem aan en zij die oudere (zusters) zeiden: ``Zusje, (dat zal) onze man (zijn)'' (en) hij zeide: ``Neen, (zij) slechts (is) mijne vrouw,'' (en) zij zeiden: ``Wat (is hij) onze zwager?'' en hij zeide: ``Neen'' (hij wilde dat liever niet zijn omdat zij hem vroeger, toen hij zijne vermomming nog niet afgelegd had, bespot hadden.)

Alzoo ging hij ook (weg) met een bezwaard hart en zijne vrouw deed hij beloven, zeggende: ``Als ge bevalt en twee kinderen krijgt, een (wordt dan geboren) met nieuwe maan, (en) een met zonsopgang.''

Toen hij nu vertrokken was, beviel zij. En de oudere (zusters) zeiden: ``je oogen (zullen) wij dichtkleven,'' en zij kleefden ze dicht, en zij beviel. En toen ze geboren waren, brachten zij ze met eene kist naar buiten en zij (die zusters namelijk) lieten ze (de kinderen) er mee wegdrijven; daarna staken ze houtskool in haar vagina en cocosnotenbast. En hare oogen bestreken ze (met olie), alzoo gingen ze weer open. En zij, de moeder namelijk, vroeg, zij zeide: ``Zusjes, hoe staat het met de kinderen?'' En zij zeiden: ``de kinderen, vanwaar? Kijk hier slechts houtskool en cocosnotenbast.'' En die kinderen (in de kist) dreven aan wal aan de aanlegplaats van den Generaal. En de vrouw van den Generaal ging poepen, en zij hurkte neer op die kist, en daarin huilden die kinderen.

En zij liep hard landwaarts en gaf (het) te kennen, zij zeide: ``Oudje, aan zee op de aanlegplaats, in een (stuk) hout huilen er kinderen;'' en hij zeide: ``Oudje, (laten) wij dat (hout stuk) slaan.''

En zij sloegen het (stuk), zij sloegen het en die kinderen waren twee.

En toen zeide de vrouw: ``Oudje, (laten) wij ze voor ons slachten,'' en hij zeide: ``neen, opdat wij ze boodschappen kunnen laten doen (ze als bedienden kunnen gebruiken).'' Toen voedden zij ze op tot ze groot waren geworden.

(Eens) rookten zij een rookhekje vol menschen(vleesch) en een rookhekje vol varkens(vleesch). En zij aten en zij zeiden, de kinderen daar: ``grootvader, wij hebben veel eten, dat ons verboden is te eten,'' en hij zeide: ``Oudje, er is daar nog een zeer klein stukje eten, neem het voor hen, (laten) onze kleinkinderen ook eten (van) het overvloedige voedsel.''

En zij aten, (en) daarna ging hij de honden roepen (op de jacht) en hij zeide: ``Oudje, verwijder je niet van onze kinderen, (totdat) zij groot (volwassen) zijn.'' En zij zeide: ``Oudje, ga maar, ik zal ze niet verlaten.''

\newpage
En hij ging, toen hij (de honden) geroepen had, (was gaan jagen) zeiden zij (namelijk die twee kinderen): ``Oudje, geef ons een mes, (want) aan zee aan het water gaan wij spelen.'' En zij zeide: ``kleinkinderen, vooruit maar, speelt maar;'' en zij (gingen) spelen, zij gingen en zij maakten een steenen bootje volgens het model van een djuanganaboot (een Sultansvaartuig). Zij maakten zijne zeilen, zijne masten, zijne roeispanen, zijne planken; (en) toen riep het oudje, zij zeide: ``Kleinkinderen, komt landwaarts;'' en zij zij zeiden: ``Oudje, nog niet, ons bootje (willen) wij nog eerst maken.'' En het zeil zetten zij overeind en zij gingen (voeren weg). Daarna kwam hij (de Generaal) (terug) en hij vroeg: ``Hoe is het met onze kleinkinderen?'' En zij zeide: ``Zooeven speelden ze aan zee bij de aanlegplaats.''

En hij klom in den kanariboom, alzoo zag hij hun zeil nog (zoo klein) als een sigaretje (zoo ver waren ze al weg gevaren).

En hij bond zijn schaamharen aan elkaar (maakte er als het ware een langen draad van) maar zij reikten niet tot aan het zeil (der kinderen), en hij zeide: ``Oudje, trek je schaamharen uit, opdat wij ze er mee (met de mijne) verbinden (en) ze vastbinden.'' En hij verlengde ze er mee en hij wierp hen er mee. En zij bonden hen met deze schaamharendraad, en daarop kwam hun zeil steeds dichter van zee naar land.

En zij (die kinderen in de boot) hakten (met de bijl) die draad, maar die wilde niet, (brak niet), zij hakten met het hakmes, maar hij wilde niet (breken), zij streken er afval van cocosnootenpitten op, opdat kijk de muizen hem zouden doorknagen, maar hij wilde niet (stuk) en zij kwamen steeds dichter bij hen.

En zij gingen weer en aan de aanlegplaats des konings kwamen zij aan wal en de broeder zeide: ``Doe eene broek aan'', en zij deed een broek aan (van deze twee kinderen schijnt dus de een een jongen en de ander een meisje geweest te zijn).

En zij kwamen van boord en de koning zeide: ``Uwe zuster daar, ik (wil) met haar trouwen''; en hij (de broer namelijk) zeide: ``Groote heer, maar wij zijn slechts mannen.'' Maar de koning geloofde (dat) niet, en hij de broeder zeide (toen): ``Kom, gij gelooft (dat) niet, (laat) hij (eigenlijk zij) dan urineeren.'' En hij urineerde, maar de straal liet (liet zij) door een bamboegeleiding gaan, en een heel eind weg kwam het neer. En hij, (de koning) zeide: ``'t is waar!'' maar daarna bloedde haar schaamdeel en haar broek deed haar pijn. En de koning zag dat en hij zeide: ``'t is niet waar ('t is) eene vrouw;'' doch een kolibritje kwam boven haar hoofd (vliegen) en met haar mes ving zij het op, alzoo (werd) hare hand bebloed en zij zeide: ``hier dan ('t is) slechts bloed van het kolibritje,'' en de koning zeide: ``'t is waar.''

En hij zeide: ``gaat maar,'' en zij gingen, zeewaarts roeiden zij, (en) zij was naakt. En hij zeide: ``gelijk de vrouwen (roeien) moet gij ons roeien.'' En zij roeide het, zij duwde het (het bootje al roeiende) en het zonk, vooruit, nog eens, zij duwde het en het zonk (weer wat), vooruit zij duwde het en het zonk (weer wat) en (toen) was het bootje heelemaal weg (verzonken) en (toen) zeide zij: ``gij kunt het (bootje) ook niet achterlaten.''

En zij gingen weer, en zij kwamen aan wal aan de aanlegplaats huns vaders, daar aan zee aan zijn aanlegplaats kwamen zij aan wal. En de koning (hun vader de vroeger vermomde buidelrat) zeide: ``Wie kwam daar aan wal aan mijn aanlegplaats?'' en hij zeide, de koning namelijk: ``Daar van zee, haal hen.'' en zij ('s konings dienaren zeker) gingen zeewaarts om hen te halen (en) zij zeiden: ``Van het land naar de zee (komen wij);'' de koning zegt: ``waarom landen zij aan mijn aanlegplaats?'' En zij zeiden: ``als het zoo is, laat hij dan zeewaarts komen;'' en hij kwam zeewaarts en toen hij aan zee gekomen was vroeg hij, hij zeide: ``Vanwaar (zijt) gij menschen (gekomen)?'' En zij zeiden: ``Wij hier zijn kinderen van koning buidelrat. Met een bezwaard hart is hij gegaan en hij deed onze moeder belooven, hij zeide: `Gij hier (zult) bevallen, en je kinderen (zullen) twee (zijn), een (aal) met de nieuwe maan en een (ander) in den morgenstond (geboren worden).' En moeder beviel, en haar oogen kleefden zij dicht en in haar schaamdeel staken zij houtskool en cocosbast en daarna brachten onze oudere moeders (de oudere zusters der moeder, die hier meestal oudere moeder genoemd worden) ons in een kist naar buiten en lieten ons daarmee wegdrijven. En wij dreven aan land aan de aanlegplaats van den Generaal en hij voedde ons op tot wij groot geworden waren. En hij (de Generaal) riep de honden (ging op jacht) en wij gingen, wij maakten een steenen bootje.''

En de koning zeide: ``Als dat zoo is dan ben ik ook uw vader, alzoo komt uit de boot;'' en zij vroegen, zij zeiden: ``Hoe is het met moeder?'' En hij zeide: ``uwe moeder is aan zee, zij is alleen, (en) zij volgt naar boven.'' (het verblijf van een vorst ligt meestal op een heuvel aan zee).

En zij zeiden: ``Als wij uit de boot komen, laten uwe zes vrouwen dan dienst doen als rolhouten om daarover de boot op te slepen.''

En de koning zeide: ``vooruit maar'' en hij gebruikte ze (die vrouwen) als dwarshouten, dus hunne beenderen braken en de uiteinden gingen naar boven staan.

Daarna haalde men de moeder en men wreef haar met cocosnootenbast, daarna waschte men haar mooi (schoon) in water, daarop waschte men haar met zilverwater en daarna met goudwater. Toen waren de helft van haar haren zilveren en de (andere) helft gouden, haar tanden de helft (waren) zilveren, (en) de (andere) helft (waren) gouden.
