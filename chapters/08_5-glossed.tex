\ea\label{ex:text8-1}
Naga o njawa ja mididi jo puaha\\
\gll naga	o	nyawa	ya	mididi	yo	puaha\\
     \textsc{exist}	\textsc{nm}	person	3\textsc{nh}>3\textsc{nh}	two	3\textsc{hpl}.\textsc{a}	fast\\
\glt `Two people who fasted'
\z

\ea\label{ex:text8-2}
Naga o wange moiu jo 'ohama o Higi'a jo puaha;\\
\gll naga	o	wange	moi-ou	yo	'ohama	o	Higi-'a	yo	puaha\\
     \textsc{exist}	\textsc{nm}	sun	one-\textsc{foc}	3\textsc{hpl}.\textsc{a}	enter	\textsc{nm}	mosque-?\textsc{dir}.\textsc{itv}	3\textsc{hpl}.\textsc{a}	fast\\
\glt `On one day they entered the mosque, they fasted,'\footnote{MZ: The mention of a mosque shows that this story comes from a Muslim context. Muslims lived at the coast of Kao bay at the beginning of the 20th century.}
\z

\newpage
\ea\label{ex:text8-3}
de ma lia'a wo djadji ma eha'a, wo temo:\\
\gll de	ma	lia'a	wo	jaji	ma	eha-i'a	wo	temo\\
     \textsc{conn}	\textsc{rnm}	older.sibling	3\textsc{sg.m}.\textsc{a}	promise	\textsc{rnm}	mother-\textsc{dir}.\textsc{itv}	3\textsc{sg.m}.\textsc{a}	say\\
\glt `and the older brother urged the mother, saying:'
\z

\ea\label{ex:text8-4}
``Jai, mia namo ua nia poha,\\
\gll yai	mia	namo	uwa	nia	poha\\
     mother	1\textsc{pl}.\textsc{ex}.\textsc{poss}	chicken	\textsc{proh}	2\textsc{pl}>3\textsc{nh}	hit\\
\glt `{``}Mother, don't slaughter our (ex.) chicken,'
\z

\ea\label{ex:text8-5}
ngaro o ma ube o'iawa mia odomo, mio lionoma.\\
\gll ngaro	'o	ma	ube	o'ia-ua	mia	odomo	mio	lio-ino-ma\\
     just	\textsc{emph}	\textsc{rnm}	little.bit	what-\textsc{neg}	1\textsc{pl}.\textsc{ex}.\textsc{poss}	eat	1\textsc{pl}.\textsc{ex}.\textsc{a}	return-\textsc{dir}.\textsc{ven}-\textsc{rnm}\\
\glt `even if there is nothing to eat for us (ex.), we (ex.) [will] return here.'''
\z

\ea\label{ex:text8-6}
De ma eha mo temo:\\
\gll de	ma	eha	mo	temo\\
     \textsc{conn}	\textsc{rnm}	mother	3\textsc{sg.f}.\textsc{a}	say\\
\glt `And the mother said:'
\z

\ea\label{ex:text8-7}
``Ouwa, 'o mia pahawa,\\
\gll ouwa	'o	mia	poha-ua\\
     \textsc{exclproh}	\textsc{emph}	1\textsc{pl}.\textsc{ex}>3\textsc{nh}	hit-\textsc{neg}\\
\glt `{``}No, we (ex.) won't strike it [dead],'
\z

\ea\label{ex:text8-8}
ngaro 'a nio tagi.''\\
\gll ngaro	'a	nio	tagi\\
     just	\textsc{foc}	2\textsc{pl}.\textsc{a}	go\\
\glt `just go.'''
\z

\ea\label{ex:text8-9}
De na'o jo puaha i duangou jo lio.\\
\gll de	na'o	yo	puaha	i	duanga-ou	yo	lio\\
     \textsc{conn}	\textsc{cond}	3\textsc{hpl}.\textsc{a}	fast	3\textsc{nh}.\textsc{a}	finish-\textsc{already}	3\textsc{hpl}.\textsc{a}	return\\
\glt `And when their fasting was finished, they returned.'
\z

\ea\label{ex:text8-10}
De ma lia'a wo hanou, wo temo:\\
\gll de	ma	lia'a	wo	hano-ou	wo	temo\\
     \textsc{conn}	\textsc{rnm}	older.sibling	3\textsc{sg.m}.\textsc{a}	ask-\textsc{already}	3\textsc{sg.m}.\textsc{a}	say\\
\glt `And the older brother asked, saying:'
\z

\ea\label{ex:text8-11}
``Jai, mia namo ena naga?''\\
\gll yai	mia	namo	ena	naga\\
     mother	1\textsc{pl}.\textsc{ex}.\textsc{poss}	chicken	\textsc{pro}.3\textsc{nh}	\textsc{exist}\\
\glt `{``}Mother, is our (ex.) chicken [still] around?'''
\z

\ea\label{ex:text8-12}
Done ma eha mo temo:\\
\gll done	ma	eha	mo	temo\\
     then	\textsc{rnm}	mother	3\textsc{sg.f}.\textsc{a}	say\\
\glt `Then the mother said:'
\z

\ea\label{ex:text8-13}
``hai, ngoa'a, nia namo ta pohaau;''\\
\gll hai	ngoa'a	nia	namo	ta	poha-ou\\
     hey	child	2\textsc{pl}.\textsc{poss}	chicken	1\textsc{sg}>3\textsc{nh}	hit-\textsc{already}\\
\glt `{``}Hey, children, I slaughtered your (pl.) chicken,'''
\z

\ea\label{ex:text8-14}
de jo ngamo, jo temo:\\
\gll de	yo	ngamo	yo	temo\\
     \textsc{conn}	3\textsc{hpl}.\textsc{a}	angry	3\textsc{hpl}.\textsc{a}	say\\
\glt `and they were angry, they said:'
\z

\ea\label{ex:text8-15}
``i dodooha mia namo na poha?''\\
\gll i	dodooha	mia	namo	na	poha\\
     3\textsc{nh}.\textsc{a}	?possible	1\textsc{pl}.\textsc{ex}.\textsc{poss}	chicken	2\textsc{sg}>3\textsc{nh}	hit\\
\glt `{``}How is it possible [that] you (sg.) slaughtered our (ex.) chicken?'''\footnote{MZ: The meaning and internal structure of \textit{dodooha} is unclear. For Galela, \citet[121]{vanbaarda1895} translates \textit{dodooha} as `wat te doen?, hoe het te doen?, wat er aan te doen of te veranderen?' (`what to do?, how to do it?, what to do or change about it?').}
\z

\ea\label{ex:text8-16}
De ma eha mo temo:\\
\gll de	ma	eha	mo	temo\\
     \textsc{conn}	\textsc{rnm}	mother	3\textsc{sg.f}.\textsc{a}	say\\
\glt `And the mother said:'
\z

\ea\label{ex:text8-17}
``hebabu 'one o'iawa mia odomo, dadi ta poha.''\\
\gll hebabu	'o-ne	o'ia-ua	mia	odomo	dadi	ta	poha\\
     because	\textsc{emph}-\textsc{prox}	what-\textsc{neg}	1\textsc{pl}.\textsc{ex}>3\textsc{nh}	eat	therefore	1\textsc{sg}>3\textsc{nh}	hit\\
\glt `{``}Because we (ex.) had nothing to eat, therefore I slaughtered it.'''
\z

\ea\label{ex:text8-18}
Gena'ade jo temo: ``Hebabu mia namo na pohaau,\\
\gll geena-o'a-de	yo	temo	hebabu	mia	namo	na	poha-ou\\
     \textsc{dist}:\textsc{pro}.3\textsc{nh}-\textsc{locv}-\textsc{conn}	3\textsc{hpl}.\textsc{a}	say	because	1\textsc{pl}.\textsc{ex}.\textsc{poss}	chicken	2\textsc{sg}>3\textsc{nh}	hit-\textsc{already}\\
\glt `Then they said: ``Because you (sg.) slaughtered our (ex.) chicken,'
\z

\ea\label{ex:text8-19}
dadi mio tagiou.''\\
\gll dadi	mio	tagi-ou\\
     therefore	1\textsc{pl}.\textsc{ex}.\textsc{a}	go-\textsc{already}\\
\glt `we (ex.) will go [away].'''
\z

\ea\label{ex:text8-20}
I togumi'a de jo tagi jo ma hibilaono\\
\gll i	togumu-i'a	de	yo	tagi	yo	ma	hi-bilaono\\
     3\textsc{nh}.\textsc{a}	finish-\textsc{dir}.\textsc{itv}	\textsc{conn}	3\textsc{hpl}.\textsc{a}	go	3\textsc{hpl}.\textsc{a}	\textsc{mid}	\textsc{caus}-provisions\\
\glt `After that they went [and] they took provisions [for the way]:'
\z


\ea\label{ex:text8-21}
o omo-omo tumudingi,\\
\gll o	omo{\textasciitilde}omo	tumudingi\\
     \textsc{nm}	watermelon	seven\\
\glt `seven watermelons,'\footnote{MZ: The form <omo-omo> for `watermelon' is not attested in any other Core North Halmahera language and I do not know what species it refers to. Note that in [E.\ref{ex:text5-54}], <omo-omo> means `annelid.'}
\z

\ea\label{ex:text8-22}
o namo ma gohi tumudingi,\\
\gll o	namo	ma	gohi	tumudingi\\
     \textsc{nm}	chicken	\textsc{rnm}	egg	seven\\
\glt `seven chicken eggs'
\z

\ea\label{ex:text8-23}
de o igono o ngodumu moi.\\
\gll de	o	igono	o	ngodumu	moi\\
     \textsc{conn}	\textsc{nm}	coconut	\textsc{nm}	whole	one\\
\glt `and one whole coconut.'
\z

\ea\label{ex:text8-24}
Gena'ade jo tagi-jo tagi,\\
\gll geena-o'a-de	yo	tagi yo	tagi\\
     \textsc{dist}:\textsc{pro}.3\textsc{nh}-\textsc{locv}-\textsc{conn}	3\textsc{hpl}.\textsc{a}	go  3\textsc{hpl}.\textsc{a}	go\\
\glt `Then they walked on,'
\z

\ea\label{ex:text8-25}
ho jo dota'a o ngaili moi\\
\gll ho	yo	dota-'a	o	ngaili	moi\\
     thus	3\textsc{hpl}.\textsc{a}	come.to-?\textsc{lim}	\textsc{nm}	river	one\\
\glt `until they came to a river'\footnote{MZ: Cognates of \textit{dota'a} mean `bring' in several Mainland North Halmahera languages and it is also glossed as `bring' in the wordlist. However, these semantics seem less felicitous here. Possibly, \textit{dota} is the onset-mutated equivalent of \textit{tota}, which is glossed `uitkomen aan' (`come out at') in the wordlist and `come to' by me.}
\z

\ea\label{ex:text8-26}
de gena'a jo ma togumu,\\
\gll de	geena-o'a	yo	ma	togumu\\
     \textsc{conn}	\textsc{dist}:\textsc{pro}.3\textsc{nh}-\textsc{locv}	3\textsc{hpl}.\textsc{a}	\textsc{mid}	finish\\
\glt `and there they rested,'
\z

\ea\label{ex:text8-27}
hebabu o wutuoau.\\
\gll hebabu	o	wutu-o'au\\
     because	\textsc{nm}	night-\textsc{perf}\\
\glt `because it had become night.'
\z

\ea\label{ex:text8-28}
De jo ma idu-idude'u,\\
\gll de	yo	ma	idu{\textasciitilde}idu-de'u\\
     \textsc{conn}	3\textsc{hpl}.\textsc{a}	\textsc{mid}	\textsc{rdpl}{\textasciitilde}sleep-?\\
\glt `And they lay down [to sleep]'\footnote{MZ: The function of \textit{de'u} is unclear. A connection to \textit{de'u} `mountain, high place' seems less felicitous in this context.}
\z

\newpage
\ea\label{ex:text8-29}
ma lia'a wo temo: ``na'o 'o ma ube to ni tomanga,\\
\gll ma	lia'a	wo	temo	na'o	'o	ma	ube	to	ni	tomanga\\
     \textsc{rnm}	older.sibling	3\textsc{sg.m}.\textsc{a}	say	\textsc{cond}	\textsc{emph}	\textsc{rnm}	little.bit	1\textsc{sg}.\textsc{a}	2\textsc{sg}.\textsc{u}	wake.up\\
\glt `and the older one said: ``In a little while I'll wake you (sg.) up,'
\z

\ea\label{ex:text8-30}
de no ma tutumu mao,\\
\gll de	no	ma	tutumu	mao\\
     \textsc{conn}	2\textsc{sg}.\textsc{a}	\textsc{mid}	go.see	wake.up\\
\glt `?then watch carefully,'\footnote{MZ: The meaning of \textit{mao} is unclear. The cognate in Tobelo means `feel' or `conscious' \citep[239]{hueting1908b}. The Galela cognate \textit{malo} also means `wake up' \citep[252]{vanbaarda1895} and this seems to be the semantics in [H.\ref{ex:text8-36}]. However, it does not really fit with Ellen's translation.}
\z

\ea\label{ex:text8-31}
hebabu o ma ube o wutu gogoronanohi,\\
\gll hebabu	'o	ma	ube	o	wutu	CV{\textasciitilde}goronan-ohi\\
     because	\textsc{emph}	\textsc{rnm}	little.bit	\textsc{nm}	night	\textsc{rdpl}{\textasciitilde}middle-\textsc{still}\\
\glt `because in a little while, when it is still midnight,'
\z

\ea\label{ex:text8-32}
o djara i na tulu ngone'';\\
\gll o	jara	i	na	tulu	ngone\\
     \textsc{nm}	horse	3\textsc{nh}.\textsc{a}	1\textsc{pl}.\textsc{in}.\textsc{u}	transport	\textsc{pro}.1\textsc{pl}.\textsc{incl}\\
\glt `a horse\footnote{GJE: a mystic horse} will pick us (in.) up,'''
\z

\ea\label{ex:text8-33}
de ma dodoto wo temo:\\
\gll de	ma	dodoto	wo	temo\\
     \textsc{conn}	\textsc{rnm}	younger.sibling	3\textsc{sg.m}.\textsc{a}	say\\
\glt `and the younger one said:'
\z


\ea\label{ex:text8-34}
``na'o geena, de halingou no i tomanga ua no i mada.''\\
\gll na'o	geena	de	halingou	no	i	tomanga	uwa	no	i	mada\\
     \textsc{cond}	\textsc{dist}:\textsc{pro}.3\textsc{nh}	\textsc{conn}	necessary	2\textsc{sg}.\textsc{a}	1\textsc{sg}.\textsc{u}	wake.up	\textsc{proh}	2\textsc{sg}.\textsc{a}	1\textsc{sg}.\textsc{u}	abandon\\
\glt `{``}If that's the case, you (sg.) must wake me up, don't abandon me.'''
\z

\ea\label{ex:text8-35}
De gena'a o wutu gogoronanohi o djara i boau,\\
\gll de	geena-o'a	o	wutu	CV{\textasciitilde}goronan-ohi	o	jara	i	boa-ou\\
     \textsc{conn}	\textsc{dist}:\textsc{pro}.3\textsc{nh}-\textsc{locv}	\textsc{nm}	night	\textsc{rdpl}{\textasciitilde}middle-\textsc{still}	\textsc{nm}	horse	3\textsc{nh}.\textsc{a}	come-\textsc{already}\\
\glt `And when it was still midnight the horse came,'
\z

\ea\label{ex:text8-36}
'one ma lia'a wi tomangou ma dodoto, ma wo maowa.\\
\gll 'o-ne	ma	lia'a	wi	tomanga-ou	ma	dodoto	ma	wo	mao-ua\\
     \textsc{emph}-\textsc{prox}	\textsc{rnm}	older.sibling	3\textsc{sg.m}>3\textsc{sg.m}	wake.up-\textsc{already}	\textsc{rnm}	younger.sibling	but	3\textsc{sg.m}.\textsc{a}	wake.up-\textsc{neg}\\
\glt `so the older brother woke up the younger one, but he did not wake up.'
\z

\ea\label{ex:text8-37}
Done a wo tagiou ma lia'a;\\
\gll done	'a	wo	tagi-ou	ma	lia'a\\
     then	\textsc{foc}	3\textsc{sg.m}.\textsc{a}	go-\textsc{already}	\textsc{rnm}	older.sibling\\
\glt `Then the older brother went,'
\z

\ea\label{ex:text8-38}
de waktu wo tagiou ma lia'a wa hoana ma dodoto awi ali-ali\\
\gll de	waktu	wo	tagi-ou	ma	lia'a	wa	hoana	ma	dodoto	awi	ali{\textasciitilde}ali\\
     \textsc{conn}	time	3\textsc{sg.m}.\textsc{a}	go-\textsc{already}	\textsc{rnm}	older.sibling	3\textsc{sg.m}>3\textsc{nh}	?take.away	\textsc{rnm}	younger.sibling	3\textsc{sg.m}.\textsc{poss}	ring\\
\glt `and when the older brother went, he took away the rings of the younger one,'\footnote{MZ: Based on Ellen's translation, <hoana> means `take away'. This form is elsewhere unattested for Modole and other Core North Halmahera languages. It is possible that <hoana> is a typo for <hoaha> = \textit{ho-aha} with \textit{aha} meaning `to carry.'}
\z


\ea\label{ex:text8-39}
la jo ma hialiti,\\
\gll la	yo	ma	hi-aliti\\
     so.that	3\textsc{hpl}.\textsc{a}	\textsc{mid}	\textsc{caus}-exchange\\
\glt `so that they exchanged [them],'
\z

\newpage
\ea\label{ex:text8-40}
la to ma lia'a wi hinoa ma dodotao awi ali-ali,\\
\gll la	to	ma	lia'a	wi	hi-noa	ma	dodoto	awi	ali{\textasciitilde}ali\\
     so.that	\textsc{poss}.\textsc{hum}	\textsc{rnm}	older.sibling	3\textsc{sg.m}>3\textsc{sg.m}	\textsc{caus}-put	\textsc{rnm}	younger.sibling	3\textsc{sg.m}.\textsc{poss}	ring\\
\glt `so that the older one put his rings on the younger'
\z

\ea\label{ex:text8-41}
la to ma dodoto u ma hinoa unao.\\
\gll la	to	ma	dodoto	u	ma	hi-noa	una-ou\\
     so.that	\textsc{poss}.\textsc{hum}	\textsc{rnm}	younger.sibling	3\textsc{sg.m}.\textsc{a}	\textsc{mid}	\textsc{caus}-put	\textsc{pro}.3\textsc{sg.m}-\textsc{foc}\\
\glt `and those of the younger one he put on himself.'
\z

\ea\label{ex:text8-42}
Ma lia'a awi badju u ma nani tumudingi'a,\\
\gll ma	lia'a	awi	baju	u	ma	nani	tumudingi-'a\\
     \textsc{rnm}	older.sibling	3\textsc{sg.m}.\textsc{poss}	k.o.shirt	3\textsc{sg.m}.\textsc{a}	\textsc{mid}	layers	seven-?\textsc{locv}\\
\glt `The older brother [wore] seven layers of shirts'
\z

\ea\label{ex:text8-43}
awi haana ma 'ogeena,\\
\gll awi	haana	ma	'o-geena\\
     3\textsc{sg.m}.\textsc{poss}	pants	\textsc{rnm}	\textsc{emph}-\textsc{dist}:\textsc{pro}.3\textsc{nh}\\
\glt `[and] his pants [were] like that,'\footnote{GJE: thus also seven layers on top of each other}
\z

\ea\label{ex:text8-44}
de na'o o nge'omo wa ma'e ma djaga awi badju wo dugumo,\\
\gll de	na'o	o	nge'omo	wa	ma'e	ma	jaga	awi	baju	wo	dV-gumo\\
     \textsc{conn}	\textsc{cond}	\textsc{nm}	way	3\textsc{sg.m}>3\textsc{nh}	see	\textsc{rnm}	branch	3\textsc{sg.m}.\textsc{poss}	k.o.shirt	3\textsc{sg.m}.\textsc{a}	\textsc{appl}-throw\\
\glt `and when he came to a crossroads on the way, he dropped his shirt,'
\z


\ea\label{ex:text8-45}
i'a wa ma'e wo dugumo,\\
\gll i'a	wa	ma'e	wo	dV-gumo\\
     \textsc{dir}.\textsc{itv}	3\textsc{sg.m}>3\textsc{nh}	see	3\textsc{sg.m}.\textsc{a}	\textsc{appl}-throw\\
\glt `then he came [to a crossroads again] [and] he dropped [one shirt],'
\z

\ea\label{ex:text8-46}
wa ma'e, wo dugumo,\\
\gll wa	ma'e	wo	dV-gumo\\
     3\textsc{sg.m}>3\textsc{nh}	see	3\textsc{sg.m}.\textsc{a}	\textsc{appl}-throw\\
\glt `he came [to a crossroads again] [and] he dropped [one down],'
\z

\ea\label{ex:text8-47}
hiadono awi badju moiohi,\\
\gll hiadono	awi	baju	moi-ohi\\
     until	3\textsc{sg.m}.\textsc{poss}	k.o.shirt	one-\textsc{still}\\
\glt `until he still had one shirt [left]'
\z

\ea\label{ex:text8-48}
de awi haana moiohi\\
\gll de	awi	haana	moi-ohi\\
     \textsc{conn}	3\textsc{sg.m}.\textsc{poss}	pants	one-\textsc{still}\\
\glt `and still one pair of pants'
\z

\ea\label{ex:text8-49}
de ma duulu awi dopo-dopo ma harumu wa umo\\
\gll de	ma duulu awi dopo{\textasciitilde}dopo ma harumu wa	umo\\
     \textsc{conn} \textsc{rnm} later 3\textsc{sg.m}.\textsc{poss}	kris	\textsc{rnm}	sheath	3\textsc{sg.m}>3\textsc{nh}	throw\\
\glt `and after that he dropped his kris sheath'
\z

\ea\label{ex:text8-50}
de i paapa o nge'omo ma djaga o gubali'a.\\
\gll de	i	paapa	o	nge'omo	ma	jaga	o	gubali-i'a\\
     \textsc{conn}	3\textsc{nh}.\textsc{a}	bounce.back	\textsc{nm}	way	\textsc{rnm}	branch	\textsc{nm}	left-\textsc{dir}.\textsc{itv}\\
\glt `and it bounced back on the left\footnote{MZ: Ellen translates \textit{gubali} as `right' (``rechtschen''). However, in the wordlist it is glossed as `left' and this is also the semantics found in all other Core North Halmahera languages where it is attested.} path [at the crossroads].'\footnote{GJE: It is a common habit of the natives to mark the right way at a crossroads in unknown territory by laying a twig on the path so that followers see which way they must take.}
\z

\ea\label{ex:text8-51}
De ma lia'a wa nonu o nge'omo o guida'a,\\
\gll de	ma	lia'a	wa	nonu	o	nge'omo	o	guida'a\\
     \textsc{conn}	\textsc{rnm}	older.sibling	3\textsc{sg.m}>3\textsc{nh}	follow.path	\textsc{nm}	way	\textsc{nm}	right\\
\glt `And the older brother followed the rightward\footnote{MZ: Here, Ellen translates \textit{guida'a} as `left' (``links'') while in the wordlist and elsewhere it means `right'.} path'
\z

\ea\label{ex:text8-52}
de ma dodoto wa nonu o nge'omo o gubali'a.\\
\gll de	ma	dodoto	wa	nonu	o	nge'omo	o	gubali-i'a\\
     \textsc{conn}	\textsc{rnm}	younger.sibling	3\textsc{sg.m}>3\textsc{nh}	follow.path	\textsc{nm}	way	\textsc{nm}	left-\textsc{dir}.\textsc{itv}\\
\glt `and the younger brother\footnote{GJE: who hence woke up later} followed the leftward path.'
\z

\ea\label{ex:text8-53}
De ma dodoto wo tota o bere'i'a,\\
\gll de	ma	dodoto	wo	tota	o	bere'i-i'a\\
     \textsc{conn}	\textsc{rnm}	younger.sibling	3\textsc{sg.m}.\textsc{a}	come.to	\textsc{nm}	old.person-\textsc{dir}.\textsc{itv}\\
\glt `And the younger brother came to an old woman,'
\z

\ea\label{ex:text8-54}
ma ma didihila'a u ma iunuahi o pihanga ma goa'a.\\
\gll ma	ma	CV{\textasciitilde}dV-hila-'a	u	ma	iunu-ohi	o	pihanga	ma	goa-o'a\\
     but	\textsc{rnm}	\textsc{rdpl}{\textasciitilde}\textsc{appl}-first-?\textsc{locv}	3\textsc{sg.m}.\textsc{a}	\textsc{mid}	hide-\textsc{still}	\textsc{nm}	banana	\textsc{rnm}	?lower.part-\textsc{locv}\\
\glt `but first he hid under a banana plant.'
\z

\ea\label{ex:text8-55}
De ma pihanga o utu moi i umu'u,\\
\gll de	ma	pihanga	o	utu	moi	i	umu'u\\
     \textsc{conn}	\textsc{rnm}	banana	\textsc{nm}	tross	one	3\textsc{nh}.\textsc{a}	ripe\\
\glt `And one bunch of bananas [was] ripe,'
\z

\ea\label{ex:text8-56}
de gena'u u ma gou-gou.\\
\gll de	geena-u'u	u	ma	gou{\textasciitilde}gou\\
     \textsc{conn}	\textsc{dist}:\textsc{pro}.3\textsc{nh}-\textsc{dir}.\textsc{down}	3\textsc{sg.m}.\textsc{a}	\textsc{mid}	eat.raw\\
\glt `and down there he ate [them] raw.'
\z

\ea\label{ex:text8-57}
Ma moi ami baili'a mo tagi ami bira mo utu'u,\\
\gll ma	moi	ami	baili-i'a	mo	tagi	ami	bira	mo	utu'u\\
     \textsc{rnm}	one	3\textsc{sg.f}.\textsc{poss}	garden-\textsc{dir}.\textsc{itv}	3\textsc{sg.f}.\textsc{a}	go	3\textsc{sg.f}.\textsc{poss}	rice	3\textsc{sg.f}.\textsc{a}	harvest\\
\glt `Once she went to her garden to harvest her rice,'
\z


\ea\label{ex:text8-58}
de mo digila-gila mami pihanga.\\
\gll de	mo	dV-gila{\textasciitilde}gila	ma ami	pihanga\\
     \textsc{conn}	3\textsc{sg.f}.\textsc{a}	\textsc{appl}-straight	\textsc{rnm} 3\textsc{sg.f}.\textsc{poss}	banana\\
\glt `?and she went straight to her bananas.'\footnote{MZ: As mentioned for [A.\ref{ex:text1-88}] above, \textit{gila-gila} likely means `straight'. Ellen's translation has`and she passed her bananas' (``en zij kwam voorbij hare bananen'').}
\z

\ea\label{ex:text8-59}
Ma ma ino, ena ma mi pihanga o utu moi 'oiwau;\\
\gll ma	ma	ino	ena	ma-ami	pihanga	o	utu	moi	'oiwa-ou\\
     but	3\textsc{sg.f}>3\textsc{nh}	\textsc{dir}.\textsc{ven}	\textsc{pro}.3\textsc{nh}	\textsc{rnm}-3\textsc{sg.f}.\textsc{poss}	banana	\textsc{nm}	tross	one	\textsc{neg}.\textsc{exist}-\textsc{foc}\\
\glt `But she came there, and of her bananas one bunch was not there anymore;'
\z

\ea\label{ex:text8-60}
``hei, neena o'ia ja odomo?\\
\gll hei	neena	o'ia	ya	odomo\\
     hey	\textsc{prox}:\textsc{pro}.3\textsc{nh}	what	3\textsc{nh}>3\textsc{nh}	eat\\
\glt `[she said]: ``Hey, who ate this?'
\z

\ea\label{ex:text8-61}
poelua o heheana.''\\
\gll poelua	o	heheana\\
     maybe	\textsc{nm}	?earwig\\
\glt `Maybe the ?earwigs.'''\footnote{MZ: I do not know what species \textit{heheana} refers to.}
\z

\ea\label{ex:text8-62}
De mo temo: ``A de iti ata ma'ehi la ta poha'a ho.''\\
\gll de	mo	temo	'a	de	iti	'a-ta	ma'e-ohi	la	ta	poha-'a	ho\\
     \textsc{conn}	3\textsc{sg.f}.\textsc{a}	say	\textsc{foc}	\textsc{conn}	only	\textsc{foc}-1\textsc{sg}>3\textsc{nh}	see-\textsc{still}	so.that	1\textsc{sg}>3\textsc{nh}	hit-?\textsc{lim}	thus\\
\glt `And she said: ``If I only find them, then I [will] strike them [dead].'''
\z

\ea\label{ex:text8-63}
De wo ahoou wo temo:\\
\gll de	wo	aho'o-ou	wo	temo\\
     \textsc{conn}	3\textsc{sg.m}.\textsc{a}	call-\textsc{already}	3\textsc{sg.m}.\textsc{a}	say\\
\glt `And he\footnote{GJE: the younger brother who was still sitting below the banana plant} shouted, saying:'
\z

\ea\label{ex:text8-64}
``Apu'' hiadono o hara ma hange aha mo ihene,\\
\gll apu	hiadono	o	hara	ma	hange	aha	mo	ihene\\
     grandmother	until	\textsc{nm}	time	\textsc{rnm}	three	as.a.consequence	3\textsc{sg.f}.\textsc{a}	hear\\
\glt `{``}Grandmother!'' [Only] at the third time she heard [him],'
\z

\ea\label{ex:text8-65}
de mo temo: ``Wah nagoona nena naga, jo apu-apu?''\\
\gll de	mo	temo	wah	naga-ona	neena	naga	yo	apu{\textasciitilde}apu\\
     \textsc{conn}	3\textsc{sg.f}.\textsc{a}	say	hey	\textsc{exist}-\textsc{pro}.3\textsc{plh}	\textsc{prox}:\textsc{pro}.3\textsc{nh}	\textsc{exist}	3\textsc{hpl}.\textsc{a}	\textsc{rdpl}{\textasciitilde}grandmother\\
\glt `and she said: ``Hey, who here [is calling] `grandmother, grandmother'?'''
\z

\ea\label{ex:text8-66}
De u ma himaiti,\\
\gll de	u	ma	hi-maiti\\
     \textsc{conn}	3\textsc{sg.m}.\textsc{a}	\textsc{mid}	\textsc{caus}-show\\
\glt `And he revealed himself,'
\z

\ea\label{ex:text8-67}
de mi o'o de mo temo:\\
\gll de	mi	o'o	de	mo	temo\\
     \textsc{conn}	3\textsc{sg.f}>3\textsc{sg.m}	\textsc{dir}.\textsc{sea}	\textsc{conn}	3\textsc{sg.f}.\textsc{a}	say\\
\glt `and she came seawards to him and said:'
\z

\ea\label{ex:text8-68}
``hai, ai ngoa'a ma aua.''\\
\gll hai	ai	ngoa'a	ma	aua\\
     hey	1\textsc{sg}.\textsc{poss}	child	\textsc{rnm}	genuine\\
\glt `{``}Hey, my real son.'''\footnote{GJE: real human in contrast to a supernatural being}\\
\z

\ea\label{ex:text8-69}
Gena'a de mi ha'ai o bira de wo odomo.\\
\gll geena-o'a	de	mi	ha'ai	o	bira	de	wo	odomo\\
     \textsc{dist}:\textsc{pro}.3\textsc{nh}-\textsc{locv}	\textsc{conn}	3\textsc{sg.f}>3\textsc{sg.m}	cook	\textsc{nm}	rice	\textsc{conn}	3\textsc{sg.m}.\textsc{a}	eat\\
\glt `Then she cooked rice for him and he ate.'
\z
