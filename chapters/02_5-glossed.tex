\ea\label{ex:text2-1}
O Bibiti\\
\gll o	bibiti\\
     \textsc{nm}	k.o.fish\\
\glt `The \textit{bibiti} fish'\footnote{MZ: According to \citet[62]{vanbaarda1895}, a \textit{bibiti} is ``a sea-fish with a plump, round body and a spiny skin'' (``een zeevisch met dik rond lichaam en stekelige huid''). It is said to be poisonous.}
\z

\ea\label{ex:text2-2}
Naga ma moi jo labea,\\
\gll naga	ma	moi	yo	labea\\
     \textsc{exist}	\textsc{rnm}	one	3\textsc{hpl}.\textsc{a}	fish.with.net\\
\glt `Once they went fishing with a net;'
\z

\ea\label{ex:text2-3}
ma eha de ami ngoa'a,\\
\gll ma	eha	de	ami	ngoa'a\\
     \textsc{rnm}	mother	\textsc{conn}	3\textsc{sg.f}.\textsc{poss}	child\\
\glt `a mother and her child,'
\z

\ea\label{ex:text2-4}
de o bibiti i moi ja amono.\\
\gll de	o	bibiti	i	moi	ya	amono\\
     \textsc{conn}	\textsc{nm}	k.o.fish	3\textsc{nh}.\textsc{a}	one	3\textsc{nh}>3\textsc{nh}	catch.fish\\
\glt `and they caught a \textit{bibiti} fish.'
\z

\ea\label{ex:text2-5}
Ma eha mo temo: ``pa poha'a,''\\
\gll ma	eha	mo	temo	pa	poha-'a\\
     \textsc{rnm}	mother	3\textsc{sg.f}.\textsc{a}	say	1\textsc{pl}.\textsc{in}>3\textsc{nh}	hit-?\textsc{lim}\\
\glt `The mother said: ''[Let]'s (in.) strike it [dead].'''
\z

\ea\label{ex:text2-6}
de muna mo temo: ``uwa na poha,\\
\gll de	muna	mo	temo	uwa	na	poha\\
     \textsc{conn}	\textsc{pro}.3\textsc{sg.f}	3\textsc{sg.f}.\textsc{a}	say	\textsc{proh}	2\textsc{sg}>3\textsc{nh}	hit\\
\glt `And she said: ``Don't strike it [dead],'
\z

\ea\label{ex:text2-7}
ai guguule ho,''\\
\gll ai	gugule	ho\\
     1\textsc{sg}.\textsc{poss}	toy	thus\\
\glt `[let it be] my plaything,'''
\z

\ea\label{ex:text2-8}
gena'ade mo hileotie o gohapanato'a.\\
\gll geena-o'a-de	mo	hi-leoto-ie	o	gohapanata-o'a\\
     \textsc{dist}:\textsc{pro}.3\textsc{nh}-\textsc{locv}-\textsc{conn}	3\textsc{sg.f}.\textsc{a}	\textsc{caus}-supine-\textsc{dir}.\textsc{up}	\textsc{nm}	k.o.tree-\textsc{locv}\\
\glt `then she placed [it] up\footnote{GJE: [in] a forked branch} in a \textit{gohapanata} tree\footnote{MZ: I have been unable to identify the \textit{gohapanata} (also spelled <gahapanata>).}.'
\z

\ea\label{ex:text2-9}
Ho ato! ma tutumu ena ma gahapanata ma ja oma-omangieu,\\
\gll ho	ato ma	tutumu	ena	ma	gohapanata	ma	ya	(C)V(C)V{\textasciitilde}omanga-ie-ou\\
     thus	\textsc{excl} 3\textsc{sg.f}>3\textsc{nh}	go.see	\textsc{pro}.3\textsc{nh}	\textsc{rnm}	k.o.tree	but	3\textsc{nh}>3\textsc{nh}	\textsc{rdpl}{\textasciitilde}full-\textsc{dir}.\textsc{up}-\textsc{foc}\\
\glt `But look! she went to see it [and] [the fish]\footnote{MZ: The \textit{bibiti} is/are henceforth treated as plural in the Dutch translation. It is impossible to gather from the Modole text whether there is one very large fish or many small ones.} filled the \textit{gohapanata} tree completely,'\footnote{GJE: the space between the branch and stem}
\z

\ea\label{ex:text2-10}
de ma ngoa'a mo temo:\\
\gll de	ma	ngoa'a	mo	temo\\
     \textsc{conn}	\textsc{rnm}	child	3\textsc{sg.f}.\textsc{a}	say\\
\glt `and the child said:'
\z

\ea\label{ex:text2-11}
 ``Aba! tanu no dapa'o o ngotili moi la to hinoa ai bibiti.''\\
\gll aba    tanu	no	dV-pa'u	o	ngotili	moi	la	to	hi-noa	ai	bibiti\\
     father \textsc{mod}	2\textsc{sg}.\textsc{a}	\textsc{appl}-nail	\textsc{nm}	proa	one	so.that	1\textsc{sg}.\textsc{a}	\textsc{caus}-put	1\textsc{sg}.\textsc{poss}	k.o.fish\\
\glt `{``}Father! please make a proa, so that I can put my \textit{bibiti} fish [in there].'''\footnote{MZ: \textit{Ngotili} is the Modole word for the type of boat known in Malay as \textit{perahu} and in English as \textit{proa} (photograph of a Halmaheran proa: \url{http://hdl.handle.net/1887.1/item:831618}, accessed 4 October 2025).}
\z

\ea\label{ex:text2-12}
Gena'ade mo hinoa,\\
\gll geena-o'a-de	mo	hi-noa\\
     \textsc{dist}:\textsc{pro}.3\textsc{nh}-\textsc{locv}-\textsc{conn}	3\textsc{sg.f}.\textsc{a}	\textsc{caus}-put\\
\glt `Then she put [it in there],'
\z

\ea\label{ex:text2-13}
ho ato! ma tutumu ena ja oma-omangieu.\\
\gll ho	ato ma	tutumu	ena	ya	(C)V(C)V{\textasciitilde}omanga-ie-ou\\
     thus	\textsc{excl} 3\textsc{sg.f}>3\textsc{nh}	go.see	\textsc{pro}.3\textsc{nh}	3\textsc{nh}>3\textsc{nh}	\textsc{rdpl}{\textasciitilde}full-\textsc{dir}.\textsc{up}-\textsc{already}\\
\glt `but look! she looked there [and saw] that they [the fish] had filled it.'
\z

\ea\label{ex:text2-14}
De muna mo temo:\\
\gll de	muna	mo	temo\\
     \textsc{conn}	\textsc{pro}.3\textsc{sg.f}	3\textsc{sg.f}.\textsc{a}	say\\
\glt `And she said:'
\z

\ea\label{ex:text2-15}
 ``Aba, tanu ma moi de ai baili no dotodangohi,\\
\gll aba	tanu	ma	moi	de	ai	baili	no	dV-todanga-ohi\\
     father	\textsc{mod}	\textsc{rnm}	one	\textsc{conn}	1\textsc{sg}.\textsc{poss}	garden	2\textsc{sg}.\textsc{a}	\textsc{appl}-cut.down-\textsc{still}\\
\glt `{``}Father, you (sg.) still have to weed my garden,'
\z

\ea\label{ex:text2-16}
hababu to ai mana'i to ona jo dotodanga ja botoau ho.''\\
\gll hababu	to	ai	mana'i	to	ona	yo	dV-todanga	ya	boto-ou	ho\\
     because	\textsc{poss}.\textsc{hum}	1\textsc{sg}.\textsc{poss}	friend	\textsc{poss}.\textsc{hum}	\textsc{pro}.3\textsc{plh}	3\textsc{hpl}.\textsc{a}	\textsc{appl}-cut.down	3\textsc{nh}>3\textsc{nh}	finish-\textsc{already}	thus\\
\glt `because those of my friends are already weeded.'''
\z

\ea\label{ex:text2-17}
De wo temo: ``Aba to dotodanga, ho no hilahau'';\\
\gll de	wo	temo	aba	to	dV-todanga	ho	no	hila-iha-ou\\
     \textsc{conn}	3\textsc{sg.m}.\textsc{a}	say	father	1\textsc{sg}.\textsc{a}	\textsc{appl}-cut.down	thus	2\textsc{sg}.\textsc{a}	first-\textsc{dir}.\textsc{land}-\textsc{already}\\
\glt `And he said: ``I'll weed [it], but you (sg.) go landwards first,'''\footnote{MZ: As the gardens are located far inland, the \textsc{land} directional \textit{iha} is associated with going to the garden.}
\z

\ea\label{ex:text2-18}
de mo hihau,\\
\gll de	mo	hi-iha-ou\\
     \textsc{conn}	3\textsc{sg.f}.\textsc{a}	\textsc{caus}-\textsc{dir}.\textsc{land}-\textsc{already}\\
\glt `?and she went landwards,'\footnote{MZ: or maybe `she brought them [the fish] landwards'}
\z

\ea\label{ex:text2-19}
ho ato ma iha ena jo dotodangoau\\
\gll ho	ato	ma	iha	ena	yo	dV-todanga-o'au\\
     thus	\textsc{excl}	3\textsc{sg.f}>3\textsc{nh}	\textsc{dir}.\textsc{land}	\textsc{pro}.3\textsc{nh}	3\textsc{hpl}.\textsc{a}	\textsc{appl}-cut.down-\textsc{perf}\\
\glt `and look, when she came landwards they had already weeded [it]'\footnote{MZ: The fish are seemingly referred to as \textit{yo} `third person plural human actor'. This may be a typo for \textit{ya} `\textsc{3nh>3nh}' or \textit{yo} is used for an unspecified actor. Agentivity may also play a role.}
\z

\ea\label{ex:text2-20}
ma ngutu'u ma ja hiato'a.\\
\gll ma	ngutu'u	ma	ya	hiata-o'a\\
     \textsc{rnm}	root	but	3\textsc{nh}>3\textsc{nh}	pull.out-\textsc{lim}\\
\glt `[and] the roots had been pulled out.'
\z

\ea\label{ex:text2-21}
De ma o'o de mo temo:\\
\gll de	ma	o'o	de	mo	temo\\
     \textsc{conn}	3\textsc{sg.f}>3\textsc{nh}	\textsc{dir}.\textsc{sea}	\textsc{conn}	3\textsc{sg.f}.\textsc{a}	say\\
\glt `And she went seawards and she said:'
\z

\ea\label{ex:text2-22}
 ``Aba, ngarouwau na iha, ta botoau ho, to dotodanga.''\\
\gll aba	ngaro-uwa-ou	na	iha	ta	boto-ou	ho	to	dV-todanga\\
     father	just-\textsc{proh}-\textsc{foc}	2\textsc{sg}>3\textsc{nh}	\textsc{dir}.\textsc{land}	1\textsc{sg}>3\textsc{nh}	finish-\textsc{already}	thus	1\textsc{sg}.\textsc{a}	\textsc{appl}-cut.down\\
\glt `{``}Father, just don't go inland, I'm already done with the weeding.'''
\z

\ea\label{ex:text2-23}
De ma wutu'u o njawa moi wa ino mi dahe,\\
\gll de	ma	wutu-u'u	o	nyawa	moi	wa	ino	mi	dahe\\
     \textsc{conn}	\textsc{rnm}	night-\textsc{dir}.\textsc{down}	\textsc{nm}	person	one	3\textsc{sg.m}>3\textsc{nh}	\textsc{dir}.\textsc{ven}	3\textsc{sg.m}>3\textsc{sg.f}	court\\
\glt `And at night someone came to court her,'\footnote{The roots \textit{dahe} and \textit{tahe} are both translated as ``vrijen'' (`to court'). They are derived from \textsc{pcnh} *taSe `crawl' and in this extended semantics refer to a man sneaking into the bedroom of a woman at night.}
\z

\ea\label{ex:text2-24}
de muna 'a mo olu'u, mo temo:\\
\gll de	muna	'a	mo	olu'u	mo	temo\\
     \textsc{conn}	\textsc{pro}.3\textsc{sg.f}	\textsc{foc}	3\textsc{sg.f}.\textsc{a}	refuse	3\textsc{sg.f}.\textsc{a}	say\\
\glt `and she refused, saying:'
\z

\ea\label{ex:text2-25}
 ``Ngoi neena ta ino o to mamanewa.''\\
\gll ngoi	neena	ta	ino	'o	to	mamane-ua\\
     \textsc{pro}.1\textsc{sg}	\textsc{prox}:\textsc{pro}.3\textsc{nh}	1\textsc{sg}>3\textsc{nh}	\textsc{dir}.\textsc{ven}	\textsc{emph}	1\textsc{sg}.\textsc{a}	lover-\textsc{neg}\\
\glt `{``}I came here not to court.'''
\z

\ea\label{ex:text2-26}
De una wo temo: ``Ma na'o o geenade iti no i peleti'',\\
\gll de	una	wo	temo	ma	na'o	'o	geena-de	iti	no	i	peleti\\
     \textsc{conn}	\textsc{pro}.3\textsc{sg.m}	3\textsc{sg.m}.\textsc{a}	say	but	\textsc{cond}	\textsc{emph}	\textsc{dist}:\textsc{pro}.3\textsc{nh}-\textsc{conn}	only	2\textsc{sg}.\textsc{a}	1\textsc{sg}.\textsc{u}	do.hair\\
\glt `And he said: ``If that's the case, will you (sg.) just do my hair,'''\footnote{GJE: a sign of affection much appreciated by boys}
\z

\newpage
\ea\label{ex:text2-27}
ma 'a mo olu'u,\\
\gll ma	'a	mo	olu'u\\
     but	\textsc{foc}	3\textsc{sg.f}.\textsc{a}	refuse\\
\glt `but she refused,'
\z

\ea\label{ex:text2-28}
 ``ma iti no i dauhu,''\\
\gll ma	iti	no	i	dauhu\\
     \textsc{rnm}	only	2\textsc{sg}.\textsc{a}	1\textsc{sg}.\textsc{u}	oil\\
\glt `{``}Will you (sg.) just oil me,'''
\z

\ea\label{ex:text2-29}
ma a mo olu'u,\\
\gll ma	'a	mo	olu'u\\
     but	\textsc{foc}	3\textsc{sg.f}.\textsc{a}	refuse\\
\glt `but she refused,'
\z

\ea\label{ex:text2-30}
 ``ma iti no na dopoga o ena'a la po ma ena'a,''\\
\gll ma	iti	no	na	dV-poga	o	ena'a	la	po	ma	ena'a\\
     but	only	2\textsc{sg}.\textsc{a}	1\textsc{pl}.\textsc{in}.\textsc{u}	\textsc{appl}-split	\textsc{nm}	areca	so.that	1\textsc{pl}.\textsc{in}.\textsc{a}	\textsc{mid}	areca\\
\glt `{``}But just peel betel nuts for us (in.), so that we (in.) [can] chew betel,'''
\z

\ea\label{ex:text2-31}
ma a mo olu'u;\\
\gll ma	'a	mo	olu'u\\
     but	\textsc{foc}	3\textsc{sg.f}.\textsc{a}	refuse\\
\glt `but she refused,'
\z

\ea\label{ex:text2-32}
i togumiade wo utiou,\\
\gll i	togumu-i'a-de	wo	uti-ou\\
     3\textsc{nh}.\textsc{a}	finish-\textsc{dir}.\textsc{itv}-\textsc{conn}	3\textsc{sg.m}.\textsc{a}	descend-\textsc{already}\\
\glt `afterwards, he descended,'\footnote{GJE: from a house on poles;  MZ: Houses in the area were traditionally built on low poles (see the photograph in \sectref{subsec:land}).}
\z

\ea\label{ex:text2-33}
ami paro ma ngi o beta-beta moi.\\
\gll ami	paro	ma	ngi	o	beta{\textasciitilde}beta	moi\\
     3\textsc{sg.f}.\textsc{poss}	?betel	\textsc{rnm}	box	\textsc{nm}	bunch[fruit]	one\\
\glt `?her betel box was a betel bag.'\footnote{MZ: Ellen translates this as `his betel box was a betel bag (which is carried on the back)'. \textit{Ami} is the third person singular \textit{feminine} (not masculine!) possessive pronoun. The word form \textit{paro} is unclear to me. In several \cnhl, a \textit{beta-beta} is a kind of shawl used to carry things on the back. If \textit{ami} is erroneous for \textit{awi}, the clause could mean `with his betel box in a sling on his back'.}
\z

\ea\label{ex:text2-34}
De gena'ade wi dihiwa o wange\\
\gll de	geena-o'a-de	wi	dV-hiwa	o	wange\\
     \textsc{conn}	\textsc{dist}:\textsc{pro}.3\textsc{nh}-\textsc{locv}-\textsc{conn}	3\textsc{sg.m}.\textsc{u}	\textsc{appl}-shine	\textsc{nm}	sun\\
\glt `And then the sun shone on him'
\z

\ea\label{ex:text2-35}
de mi ma'e i tiai, de mo temo:\\
\gll de	mi	ma'e	i	tiai	de	mo	temo\\
     \textsc{conn}	3\textsc{sg.f}>3\textsc{sg.m}	see	3\textsc{nh}.\textsc{a}	straight	\textsc{conn}	3\textsc{sg.f}.\textsc{a}	say\\
\glt `and she saw him well and she said:'\footnote{GJE: it had become lighter}
\z

\ea\label{ex:text2-36}
 ``Na inoou la to ni peletiou,''\\
\gll na	ino-ou	la	to	ni	peleti-ou\\
     2\textsc{sg}>3\textsc{nh}	\textsc{dir}.\textsc{ven}-\textsc{already}	so.that	1\textsc{sg}.\textsc{a}	2\textsc{sg}.\textsc{u}	do.hair-\textsc{already}\\
\glt `{``}Come (sg.) here so that I can do your (sg.) hair,'''
\z

\ea\label{ex:text2-37}
ma a wo olu'u wo temo:\\
\gll ma	'a	wo	olu'u	wo	temo\\
     but	\textsc{foc}	3\textsc{sg.m}.\textsc{a}	refuse	3\textsc{sg.m}.\textsc{a}	say\\
\glt `but he refused, saying:'
\z


\ea\label{ex:text2-38}
``Uwau! 'one ani dea de ani eha ni ngamoho;''\\
\gll uwa-ou 'o-ne	ani	dea	de	ani	eha	ni	ngamo-ho\\
     \textsc{proh}-\textsc{foc} \textsc{emph}-\textsc{prox}	2\textsc{sg}.\textsc{poss}	father	\textsc{conn}	2\textsc{sg}.\textsc{poss}	mother	2\textsc{sg}.\textsc{u}	angry-thus\\
\glt `{``}Don't do that! Because then your father and your mother will be angry with you (sg.),'''
\z

\ea\label{ex:text2-39}
de a mo olu'u ho a mi ni'i.\\
\gll de	'a	mo	olu'u	ho	'a	mi	ni'i\\
     \textsc{conn}	\textsc{foc}	3\textsc{sg.f}.\textsc{a}	refuse	thus	\textsc{foc}	3\textsc{sg.f}>3\textsc{sg.m}	follow\\
\glt `and she refused and followed him.'
\z

\ea\label{ex:text2-40}
Ho na'o una wo uti ma dodanga ma de'u geena muna mo palene ma goade,\\
\gll ho	na'o	una	wo	uti	ma	dodanga	ma	de'u	geena	muna	mo	palene	ma	goa-de\\
     thus	\textsc{cond}	\textsc{pro}.3\textsc{sg.m}	3\textsc{sg.m}.\textsc{a}	descend	\textsc{rnm}	?felled.tree	\textsc{rnm}	mountain	\textsc{dist}:\textsc{pro}.3\textsc{nh}	\textsc{pro}.3\textsc{sg.f}	3\textsc{sg.f}.\textsc{a}	climb	\textsc{rnm}	?lower.part-\textsc{conn}\\
\glt `And when he descended from the top of a ?felled tree, then she climbed up on its lower end,'\footnote{MZ: \textit{Dodanga} is given in the wordlist as `omgehakte hoornen' (`felled horns'). The meaning of \textit{hoornen} is unclear, it may be a certain type of tree. \textit{Dodanga} is potentially related to \textit{todanga} `cut', hence the word form is tentatively glossed as `felled tree'. \textit{De'u} means `mountain' but can also refer to a high place in general.}
\z

\ea\label{ex:text2-41}
ma na'o una wo palene, geena muna mo uti.\\
\gll ma	na'o	una	wo	palene	geena	muna	mo	uti\\
     but	\textsc{cond}	\textsc{pro}.3\textsc{sg.m}	3\textsc{sg.m}.\textsc{a}	climb	\textsc{dist}:\textsc{pro}.3\textsc{nh}	\textsc{pro}.3\textsc{sg.f}	3\textsc{sg.f}.\textsc{a}	descend\\
\glt `but when he climbed up, then she descended.'
\z

\ea\label{ex:text2-42}
Gena'ade ho jo honga o ngaili tumudingi, de o de'u tumudingi,\\
\gll geena-o'a-de	ho	yo	honga	o	ngaili	tumudingi	de	o	de'u	tumudingi\\
     \textsc{dist}:\textsc{pro}.3\textsc{nh}-\textsc{locv}-\textsc{conn}	thus	3\textsc{hpl}.\textsc{a}	follow	\textsc{nm}	river	seven	\textsc{conn}	\textsc{nm}	mountain	seven\\
\glt `Then they followed seven rivers and [over] seven mountains,'
\z

\ea\label{ex:text2-43}
ho na'o una wo hawu, geena muna mo uti,\\
\gll ho	na'o	una	wo	hawu	geena	muna	mo	uti\\
     thus	\textsc{cond}	\textsc{pro}.3\textsc{sg.m}	3\textsc{sg.m}.\textsc{a}	climb	\textsc{dist}:\textsc{pro}.3\textsc{nh}	\textsc{pro}.3\textsc{sg.f}	3\textsc{sg.f}.\textsc{a}	descend\\
\glt `but when he climbed, then she descended,'\footnote{GJE: [climb] a mountain (MZ: also in next line)}
\z

\newpage
\ea\label{ex:text2-44}
ma na'o una wo uti, geena muna mo hawu.\\
\gll ma	na'o	una	wo	uti	geena	muna	mo	hawu\\
     but	\textsc{cond}	\textsc{pro}.3\textsc{sg.m}	3\textsc{sg.m}.\textsc{a}	descend	\textsc{dist}:\textsc{pro}.3\textsc{nh}	\textsc{pro}.3\textsc{sg.f}	3\textsc{sg.f}.\textsc{a}	climb\\
\glt `and when he descended, then she climbed.'
\z

\ea\label{ex:text2-45}
De ja ma'e'a o a'ele moi, ma de'uno o gumooanga moi i taurino,\\
\gll de	ya	ma'e-o'a	o	a'ele	moi	ma	de'u-ino	o	gumoanga	moi	i	tauru-ino\\
     \textsc{conn}	3\textsc{nh}>3\textsc{nh}	see-\textsc{lim}	\textsc{nm}	water	one	\textsc{rnm}	mountain-\textsc{dir}.\textsc{ven}	\textsc{nm}	liana	one	3\textsc{nh}.\textsc{a}	?pull-\textsc{dir}.\textsc{ven}\\
\glt `And they found a river ?over which a liana hung,'\footnote{MZ: \textit{Tauru} means `to pull' in several \cnhl.}
\z

\ea\label{ex:text2-46}
gena'ade mi damaou, jo ma enaou, de wo temo:\\
\gll geena-o'a-de	mi	dama-ou	yo	ma	ena-ou	de	wo	temo\\
     \textsc{dist}:\textsc{pro}.3\textsc{nh}-\textsc{locv}-\textsc{conn}	3\textsc{sg.f}>3\textsc{sg.m}	wait-\textsc{already}	3\textsc{hpl}.\textsc{a}	\textsc{mid}	areca-\textsc{already}	\textsc{conn}	3\textsc{sg.m}.\textsc{a}	say\\
\glt `then she awaited him, [and] they chewed betel, and he said:'
\z

\ea\label{ex:text2-47}
``No liou! 'one ani eha, de ani dea o no i ma'ewau,''\\
\gll no	lio-ou  'o-ne	ani	eha	de	ani	dea	'o	no	'i	ma'e-ua-ou\\
     2\textsc{sg}.\textsc{a}	return-\textsc{already}    \textsc{emph}-\textsc{prox}	2\textsc{sg}.\textsc{poss}	mother	\textsc{conn}	2\textsc{sg}.\textsc{poss}	father	\textsc{emph}	2\textsc{sg}.\textsc{a}	3\textsc{hpl}.\textsc{u}	see-\textsc{neg}-\textsc{already}\\
\glt `{``}Return! Otherwise, you (sg.) will not find your mother and your father anymore,'''
\z

\ea\label{ex:text2-48}
de muna mo temo: ``a to olu'u, a to ni ni'i;''\\
\gll de	muna	mo	temo	'a	to	olu'u	'a	to	ni	ni'i\\
     \textsc{conn}	\textsc{pro}.3\textsc{sg.f}	3\textsc{sg.f}.\textsc{a}	say	\textsc{foc}	1\textsc{sg}.\textsc{a}	refuse	\textsc{foc}	1\textsc{sg}.\textsc{a}	2\textsc{sg}.\textsc{u}	follow\\
\glt `and she said: ``I don't want to, I [will] follow you (sg.),'''
\z

\newpage
\ea\label{ex:text2-49}
gena'ade wo temo: ``geenade no ma ruwuto'a,''\\
\gll geena-o'a-de	wo	temo	geena-de	no	ma	ruwutu-o'a\\
     \textsc{dist}:\textsc{pro}.3\textsc{nh}-\textsc{locv}-\textsc{conn}	3\textsc{sg.m}.\textsc{a}	say	\textsc{dist}:\textsc{pro}.3\textsc{nh}-\textsc{conn}	2\textsc{sg}.\textsc{a}	\textsc{mid}	close.eyes-\textsc{lim}\\
\glt `then he said: ``Then close your (sg.) eyes,'''
\z

\ea\label{ex:text2-50}
de mu ma ruwuto'a,\\
\gll de	mu	ma	ruwutu-o'a\\
     \textsc{conn}	3\textsc{sg.f}.\textsc{a}	\textsc{mid}	close.eyes-\textsc{lim}\\
\glt `and she closed them,'
\z

\ea\label{ex:text2-51}
ho ato ma pelangie ena o de'u ma goronanou.\\
\gll ho	ato	ma	pelanga-ie	ena	o	de'u	ma	gorona-ino-ou\\
     thus	\textsc{excl}	3\textsc{sg.f}>3\textsc{nh}	open-\textsc{dir}.\textsc{up}	\textsc{pro}.3\textsc{nh}	\textsc{nm}	mountain	\textsc{rnm}	middle-\textsc{dir}.\textsc{ven}-\textsc{foc}\\
\glt `but look, she opened them here in the middle of the mountain.'\footnote{MZ: The form \textit{gorona} is the onset mutated equivalent of Proto-Mainland North Halmahera *korona `middle, center'. It is attested in several languages. In this line as well as in [H.\ref{ex:text8-31}] and [H.\ref{ex:text8-35}], the root has a final <n> not attested in any other language. Here I analyze it as the \textsc{ventive} directional \textit{-ino}, as suggested by Ellen's translation `here on the middle of the mountain' (``hier midden op den berg''). The other two occurrences do not suggest this analysis.
The phrase \textit{o de'u ma goronanou} literally translates as `the middle of the mountain'. According to my Modole informant, this refers to the mountain ridge, not the peak of the mountain.}
\z

\ea\label{ex:text2-52}
Genade o gumoanga moi i taurino, de gena mu ma idu,\\
\gll geena-de	o	gumoanga	moi	i	tauru-ino	de	geena	mu	ma	idu\\
     \textsc{dist}:\textsc{pro}.3\textsc{nh}-\textsc{conn}	\textsc{nm}	liana	one	3\textsc{nh}.\textsc{a}	?pull-\textsc{dir}.\textsc{ven}	\textsc{conn}	\textsc{dist}:\textsc{pro}.3\textsc{nh}	3\textsc{sg.f}.\textsc{a}	\textsc{mid}	sleep\\
\glt `There hung a liana and there she slept,'
\z


\ea\label{ex:text2-53}
de ami utu mu ma bilinganaau ho a'ele o di'o modidino.\\
\gll de	ami	utu	mu	ma	bilingana-ou	ho	a'ele	o	di'o	modidi-ino\\
     \textsc{conn}	3\textsc{sg.f}.\textsc{poss}	hair	3\textsc{sg.f}.\textsc{a}	\textsc{mid}	spread-\textsc{already}	thus	water	\textsc{nm}	bay	two-\textsc{dir}.\textsc{ven}\\
\glt `and she spread out her hair over two river bays.'
\z

\ea\label{ex:text2-54}
De o nao'o i ma noa ho ma tingi ma dja i ewou.\\
\gll de	o	nao'o	i	ma	noa	ho	ma	tingi	ma	ja	i	ewo-ou\\
     \textsc{conn}	\textsc{nm}	fish	3\textsc{nh}.\textsc{a}	\textsc{mid}	put	thus	\textsc{rnm}	?tight	\textsc{rnm}	net	3\textsc{nh}.\textsc{a}	full-\textsc{already}\\
\glt `And fish filled [her hair] up to the top, tight like a case net that is completely full.'
\z

\ea\label{ex:text2-55}
Ma moi wo duo, de wo hidoduo ami utu,\\
\gll ma	moi	wo	duo	de	wo	hi-dV-duo	ami	utu\\
     \textsc{rnm}	one	3\textsc{sg.m}.\textsc{a}	?fish	\textsc{conn}	3\textsc{sg.m}.\textsc{a}	\textsc{caus}-\textsc{appl}-?fish	3\textsc{sg.f}.\textsc{poss}	hair\\
\glt `Once he ?fished, and caught her hair,'
\z

\ea\label{ex:text2-56}
de wo temo: ``Neena o'iau bali a manga utu.''\\
\gll de	wo	temo	neena	o'ia-ou	bali	'a	manga	utu\\
     \textsc{conn}	3\textsc{sg.m}.\textsc{a}	say	\textsc{prox}:\textsc{pro}.3\textsc{nh}	what-\textsc{foc}	?\textsc{filler}	\textsc{foc}	3\textsc{hpl}.\textsc{poss}	hair\\
\glt `and he said: ``Here now, whose hair is this?'''
\z

\ea\label{ex:text2-57}
Wo temo: ``abei'a ihahi''\\
\gll wo temo	abei'a	iha-ohi\\
     3\textsc{sg.m}.\textsc{a} say	well	\textsc{dir}.\textsc{land}-\textsc{still}\\
\glt `He said: ``Come, a little bit more towards the land,'''
\z

\ea\label{ex:text2-58}
de ja iha, ona o gumoango'a mo to'uno mo nihuo'a,\\
\gll de	ya	iha	ona	o	gumoanga-o'a	mo	to'u-ino	mo	nihu-o'a\\
     \textsc{conn}	3\textsc{nh}>3\textsc{nh}	\textsc{dir}.\textsc{land}	\textsc{pro}.3\textsc{plh}	\textsc{nm}	liana-\textsc{locv}	3\textsc{sg.f}.\textsc{a}	?across-\textsc{dir}.\textsc{ven}	3\textsc{sg.f}.\textsc{a}	sleep-\textsc{lim}\\
\glt `and they\footnote{GJE: there seem to be two people now} went landwards [and] ?she slept on a liana'\footnote{MZ: \textit{To'u} is translated as `step over something' (`over iets stappen') in the wordlist.}
\z

\ea\label{ex:text2-59}
de mi hibalene o ngootilu'u.\\
\gll de	mi	hi-balene	o	ngotili-u'u\\
     \textsc{conn}	3\textsc{sg.f}.\textsc{u}	\textsc{caus}-embark	\textsc{nm}	proa-\textsc{dir}.\textsc{down}\\
\glt `and they loaded her into the proa.'
\z

\ea\label{ex:text2-60}
De ja i'a\\
\gll de	ya	i'a\\
     \textsc{conn}	3\textsc{nh}>3\textsc{nh}	\textsc{dir}.\textsc{itv}\\
\glt `And they went'
\z

\ea\label{ex:text2-61}
de ge ma 'oana awi we'ata jo ngamo\\
\gll de	ge	ma	'oana	awi	we'ata	yo	ngamo\\
     \textsc{conn}	\textsc{dist}	\textsc{rnm}	king	3\textsc{sg.m}.\textsc{poss}	wife	3\textsc{hpl}.\textsc{a}	angry\\
\glt `and there the wives of the king were angry,'
\z

\ea\label{ex:text2-62}
jo temo: ``ho ngewe'a no ma dotoomu de;\\
\gll yo	temo	ho	ngewe'a	no	ma	dV-toomu	de\\
     3\textsc{hpl}.\textsc{a}	say	thus	woman	2\textsc{sg}.\textsc{a}	\textsc{mid}	\textsc{appl}-gather	\textsc{conn}\\
\glt `they said: ``You (sg.) collect women;'
\z

\ea\label{ex:text2-63}
na ne ma tanu ani we'ata ma ja tutumidingo'a.''\\
\gll na	ne	ma	tanu	ani	we'ata	ma	ya	CV{\textasciitilde}tumidingi-o'a\\
     ?here	\textsc{prox}	but	\textsc{mod}	2\textsc{sg}.\textsc{poss}	wife	but	3\textsc{nh}>3\textsc{nh}	\textsc{rdpl}{\textasciitilde}seven-\textsc{locv}\\
\glt `now your (sg.) wives surely number seven already.'''\footnote{GJE: The fisherman seems to have been a king.}
\z

\ea\label{ex:text2-64}
De jo temo: ``Mo tero ho; bei'a mi ma o'o'';\\
\gll de	yo	temo	mo	tero	ho	abei'a	mi	ma o'o\\
     \textsc{conn}	3\textsc{hpl}.\textsc{a}	say	3\textsc{sg.f}.\textsc{a}	beautiful	thus	well	1\textsc{pl}.\textsc{ex}.\textsc{a}	\textsc{mid} defecate\\
\glt `And they\footnote{GJE: the women} said: ``She is beautiful; well, [let]'s (ex.) poop;'''
\z

\ea\label{ex:text2-65}
ma jo ma o'o, ma to muna ami io'o ma bounu i hemo;\\
\gll ma	yo	ma	o'o	ma	to	muna	ami	io'o	ma	bounu	i	hemo\\
     but	3\textsc{hpl}.\textsc{a}	\textsc{mid}	defecate	but	\textsc{poss}.\textsc{hum}	\textsc{pro}.3\textsc{sg.f}	3\textsc{sg.f}.\textsc{poss}	excrement	\textsc{rnm}	smell	3\textsc{nh}.\textsc{a}	sweet\\
\glt `then they pooped, but the smell of her excrement was sweet,'\footnote{MZ: A sweet smell seems to be a sign of a supernatural being (compare [E.\ref{ex:text5-65}]).}
\z

\newpage
\ea\label{ex:text2-66}
de jo temoli: ``mo tero ho, bei'a mi o hie,''\\
\gll de	yo	temo-oli	mo	tero	ho	abei'a	mio	hie\\
     \textsc{conn}	3\textsc{hpl}.\textsc{a}	say-\textsc{again}	3\textsc{sg.f}.\textsc{a}	beautiful	thus	well	1\textsc{pl}.\textsc{ex}.\textsc{a}	fart\\
\glt `and they said again: ``She is beautiful; come, [let]'s (ex.) fart,'''
\z

\ea\label{ex:text2-67}
de jo hie, ma ato! muna ami hie ma bounu i hemo.\\
\gll de	yo	hie	ma	ato   muna	ami	hie	ma	bounu	i	hemo\\
     \textsc{conn}	3\textsc{hpl}.\textsc{a}	fart	but	\textsc{excl}  \textsc{pro}.3\textsc{sg.f}	3\textsc{sg.f}.\textsc{poss}	fart	\textsc{rnm}	smell	3\textsc{nh}.\textsc{a}	sweet\\
\glt `and they farted, but look! the smell of her fart was sweet.'
\z

\ea\label{ex:text2-68}
De jo temoli: ``abei'a dauwengo'o mi ma hidudutuumo;''\\
\gll de	yo	temo-oli	abei'a	dauwe-ngo'o	mi	ma	hi-CV{\textasciitilde}dV-tuumo\\
     \textsc{conn}	3\textsc{hpl}.\textsc{a}	say-\textsc{again}	well	\textsc{loc}.\textsc{down}-\textsc{n:dir.sea}	1\textsc{pl}.\textsc{ex}.\textsc{a}	\textsc{mid}	\textsc{caus-rdpl}{\textasciitilde}\textsc{appl}-?follow.with.raft\\
\glt `And they said again: ``Come, [let]'s (ex.) ?follow the river with a raft down to the sea,'''\footnote{MZ: The form \textit{hidudutuumo} is unclear to me. It is translated as `follow with a raft' but does not occur in the world list or any dictionary of a {\cnhl}. It may be derived from \textsc{pcnh} *tuuru `follow'.}
\z

\ea\label{ex:text2-69}
de jo ma hidudutuumo.\\
\gll de	yo	ma	hi-CV{\textasciitilde}dV-tuumo\\
     \textsc{conn}	3\textsc{hpl}.\textsc{a}	\textsc{mid}	\textsc{caus-rdpl}{\textasciitilde}\textsc{appl}-?follow.with.raft\\
\glt `and they followed the river with a raft.'
\z

\ea\label{ex:text2-70}
Ona ge ja tumudingi jo hila,\\
\gll ona	ge	ya	tumudingi	yo	hila\\
     \textsc{pro}.3\textsc{plh}	\textsc{dist}	3\textsc{nh}>3\textsc{nh}	seven	3\textsc{hpl}.\textsc{a}	first\\
\glt `They, the seven, [went] first,'
\z

\ea\label{ex:text2-71}
de jo i hitadi, ho a ma moiu'u.\\
\gll de	yo	'i	hi-tadi	ho	'a	ma	moi-u'u\\
     \textsc{conn}	3\textsc{hpl}.\textsc{a}	3\textsc{hpl}.\textsc{u}	\textsc{caus}-thrust	thus	\textsc{foc}	\textsc{rnm}	one-\textsc{dir}.\textsc{down}\\
\glt `and they crashed\footnote{GJE: with that raft against the bank} and went down at once\footnote{GJE: all perished}.'
\z

\ea\label{ex:text2-72}
De munali.\\
\gll de	muna-oli\\
     \textsc{conn}	\textsc{pro}.3\textsc{sg.f}-\textsc{again}\\
\glt `And she too.'\footnote{GJE: took a raft}
\z

\ea\label{ex:text2-73}
De muna hidahini ho, a mo hiiha mo lio.\\
\gll de	muna	hi-dahini	ho	'a	mo	hi-iha	mo	lio\\
     \textsc{conn}	\textsc{pro}.3\textsc{sg.f}	\textsc{caus}-float	thus	\textsc{foc}	3\textsc{sg.f}.\textsc{a}	\textsc{caus}-\textsc{dir}.\textsc{land}	3\textsc{sg.f}.\textsc{a}	return\\
\glt `And she [also] drifted away with it, [and] she [drifted] landwards returning.'\footnote{GJE: thus against the current}
\z

\ea\label{ex:text2-74}
De ma 'oana wo temo: ``To ni modoa'a.''\\
\gll de	ma	'oana	wo	temo	to	ni	modo'a-'a\\
     \textsc{conn}	\textsc{rnm}	king	3\textsc{sg.m}.\textsc{a}	say	1\textsc{sg}.\textsc{a}	2\textsc{sg}.\textsc{u}	marry-?\textsc{lim}\\
\glt `And the king said: ``I will marry you (sg.).'''
\z

\ea\label{ex:text2-75}
De mo temo: ``na'o no i modo'ade, halingou ani halalomu moi to wi topo'oahi.''\\
\gll de	mo	temo	na'o	no	i	modo'a-de	halingou	ani	halalomu	moi	to	wi	topo'o-ohi\\
     \textsc{conn}	3\textsc{sg.f}.\textsc{a}	say	\textsc{cond}	2\textsc{sg}.\textsc{a}	1\textsc{sg}.\textsc{u}	marry-\textsc{conn}	necessary	2\textsc{sg}.\textsc{poss}	slave	one	1\textsc{sg}.\textsc{a}	3\textsc{sg.m}.\textsc{u}	stab-\textsc{still}\\
\glt `And she said: ``If you (sg.) want to marry me, I first must stab one of your (sg.) slaves [dead].'''
\z

\ea\label{ex:text2-76}
De mi topo'o ho a wo hidulubade,\\
\gll de	mi	topo'o	ho	'a	wo	hi-dV-luba-de\\
     \textsc{conn}	3\textsc{sg.f}>3\textsc{sg.m}	stab	thus	\textsc{foc}	3\textsc{sg.m}.\textsc{a}	\textsc{caus}-\textsc{appl}-fall.down-\textsc{conn}\\
\glt `And she stabbed him, so that he fell down'\footnote{GJE: fell down dead}
\z

\ea\label{ex:text2-77}
de wi hihao o iwaja;\\
\gll de	wi	hi-hao	o	iwaya\\
     \textsc{conn}	3\textsc{sg.m}.\textsc{u}	\textsc{caus}-wrap	\textsc{nm}	kind.of.taro\\
\glt `and he was bound with taro [leaves]'
\z


\ea\label{ex:text2-78}
de o wo'a ma homoa'a wi tingai'a.\\
\gll de	o	wo'a	ma	homoa-o'a	wi	tinga-i'a\\
     \textsc{conn}	\textsc{nm}	house	\textsc{rnm}	other-\textsc{locv}	3\textsc{sg.m}.\textsc{u}	separate-\textsc{dir}.\textsc{itv}\\
\glt `and they separated him in another house.'
\z

\ea\label{ex:text2-79}
Ha ato! o wutu tumudingi ena a u ma babangiheli,\\
\gll ho	ato o	wutu	tumudingi	ena	'a	u	ma	CV{\textasciitilde}bangiheli\\
     thus	\textsc{excl} \textsc{nm}	night	seven	\textsc{pro}.3\textsc{nh}	\textsc{foc}	3\textsc{sg.m}.\textsc{a}	\textsc{mid}	\textsc{rdpl}{\textasciitilde}flute\\
\glt `But look! after seven nights he played the flute,'
\z

\ea\label{ex:text2-80}
de a wo wangoali.\\
\gll de	'a	wo	wango-oli\\
     \textsc{conn}	\textsc{foc}	3\textsc{sg.m}.\textsc{a}	live-\textsc{again}\\
\glt `and he was alive again.'
\z

\ea\label{ex:text2-81}
De awi tjapa awi hongona o hala'a, ma hongona o gurahi,\\
\gll de	awi	capa	awi	hongona	o	hala'a	ma	hongona	o	gurahi\\
     \textsc{conn}	3\textsc{sg.m}.\textsc{poss}	face	3\textsc{sg.m}.\textsc{poss}	half	\textsc{nm}	silver	\textsc{rnm}	half	\textsc{nm}	gold\\
\glt `And half of his face was silver [and] the [other] half was gold,'
\z

\ea\label{ex:text2-82}
awi utu ma ma hongona o hala'a ma hougona gurahi,\\
\gll awi	utu	ma	ma	hongona	o	hala'a	ma	hongona	gurahi\\
     3\textsc{sg.m}.\textsc{poss}	hair	\textsc{rnm}	\textsc{rnm}	half	\textsc{nm}	silver	\textsc{rnm}	half	gold\\
\glt `half of his hair was silver, the [other] half gold,'
\z

\ea\label{ex:text2-83}
awi ilingi ma dauie o hala'a, dauu'u o gurahi.\\
\gll awi	ilingi	ma	da'u-ie	o	hala'a	dau-u'u	o	gurahi\\
     3\textsc{sg.m}.\textsc{poss}	tooth	\textsc{rnm}	\textsc{loc}.\textsc{up}-\textsc{dir}.\textsc{up}	\textsc{nm}	silver	\textsc{loc}.\textsc{down}-\textsc{dir}.\textsc{down}	\textsc{nm}	gold\\
\glt `his upper teeth were silver [and] the lower ones were gold.'
\z

\ea\label{ex:text2-84}
De ma 'oana wo temo: ``Ngoioli no i topo'o;''\\
\gll de	ma	'oana	wo	temo	ngoi-oli	no	i	topo'o\\
     \textsc{conn}	\textsc{rnm}	king	3\textsc{sg.m}.\textsc{a}	say	\textsc{pro}.1\textsc{sg}-\textsc{again}	2\textsc{sg}.\textsc{a}	1\textsc{sg}.\textsc{u}	stab\\
\glt `And the king said: ``Stab me too,'''
\z

\ea\label{ex:text2-85}
de mo temo: ``uwau, 'one a na hongonaau.''\\
\gll de	mo	temo	uwa-ou	'o-ne	'a	na	hongona-ou\\
     \textsc{conn}	3\textsc{sg.f}.\textsc{a}	say	\textsc{proh}-\textsc{foc}	\textsc{emph}-\textsc{prox}	\textsc{foc}	2\textsc{sg}>3\textsc{nh}	half-\textsc{already}\\
\glt `and she said: ``No, because then you (sg.) will [also] become half.'''
\z

\ea\label{ex:text2-86}
Ma wo ributu ho de mi topo'o;\\
\gll ma	wo	ributu	ho	de	mi	topo'o\\
     but	3\textsc{sg.m}.\textsc{a}	withstand	thus	\textsc{conn}	3\textsc{sg.f}>3\textsc{sg.m}	stab\\
\glt `But he withstood her, so she stabbed him;'
\z

\ea\label{ex:text2-87}
de a wo hongonaau o wo liowau.\\
\gll de	'a	wo	hongona-ou	'o	wo	lio-ua-ou\\
     \textsc{conn}	\textsc{foc}	3\textsc{sg.m}.\textsc{a}	half-\textsc{already}	\textsc{emph}	3\textsc{sg.m}.\textsc{a}	return-\textsc{neg}-\textsc{already}\\
\glt `and he became half [and] he did not return anymore.'\footnote{GJE: i.e. became half silver and half gold}
\z
