\subsection*{Er waren twee menschen, die gingen vasten}

Op zekeren dag gingen zij in de moskee, zij (gingen) vasten, en de oudste (broer) deed de moeder beloven, hij zeide: ``Moeder, slacht onze kip niet, al is er ook geen klein beetje meer voor ons te eten, (want) wij keeren toch terug.''

En de moeder zeide: ``Neen, wij (zullen) ze niet slaan (dooden), ga maar;'' en na hun vasten keerden zij terug.

En de oudste vroeg, hij zeide: ``Moeder, is onze kip er nog?'' Alzoo zeide de moeder: ``He, kind, je kip (heb) ik geslacht;'' en zij (waren) boos, zij zeiden: ``Hoe is het mogelijk, (dat) gij onze kip geslacht hebt?'' En de moeder zeide: ``omdat wij geen eten meer hadden, heb ik ze geslacht.'' Toen zeiden zij: ``Omdat gij onze kip geslacht hebt, gaan wij nu weg.''

Daarop gingen zij (en) zij namen zich teerkost (voor onder weg) zeven watermeloenen, zeven hoendereieren en een cocosnoot.

Toen liepen zij al maar door tot zij kwamen aan eene rivier en aldaar rustten zij, want het was avond (geworden).

En zij legden zich neer (om te slapen) en de oudste zeide: ``over een poosje maak ik je wakker, en let gij dan goed op, want over een poosje als het nog middernacht is, (zal) een (fabelachtig) paard ons meenemen;'' en de jongste zeide: ``als dat zoo is, dan (moet) gij mij wekken (en) moogt ge mij niet laten (slapen).''

En toen het nog middernacht was kwam het paard, alzoo wekte de oudste (broer) den jongste, maar hij werd niet wakker.

Alzoo ging de oudste, en toen de oudste (zou) gaan, ontdeed hij den jongste van zijne ringen; toen werden zij met elkaar verwisseld en de oudste deed den jongste de zijne aan en die van den jongste zichzelf.

De oudste (broeder) had zeven baadjes over elkaar aan, zijne broeken ook alzoo (dus ook zeven over elkaar), en wanneer hij aan een tweesprong op den weg kwam, wierp hij er zijn baadje op neer. Vervolgens kwam hij (weer) aan een (tweesprong) (en) hij wierp er (een baadje) op neer, hij kwam (weer aan een) en wierp (er weer een) op neer, totdat hij nog een baadje had en nog een broek en daarna wierp hij zijn krisscheede neer en deze wipte weer op en viel neer op den rechtschen weg van den tweesprong. (Het is eene goede gewoonte van de inlanders, om in onbekend land bij een tweesprong op den weg een teeken, meestal een takje neer te leggen en wel zoo dat achteraankomers kunnen zien welken weg zij moeten inslaan om niet te verdwalen.)

En de oudste (broeder) sloeg den weg in, die naar links afweek en de jongste (die dus later toch wakker geworden zal zijn en hem toen volgde) sloeg den weg in naar rechts.

En de jongste kwam totaan een oud vrouwtje, maar eerst verborg hij zich onder de bananenplanten. En een tros bananen (was) rijp, en aldaar at hij (ze op).

Eens ging zij naar haar tuin om hare rijst te oogsten, en zij kwam voorbij hare bananen. Maar zij ging er naar toe, (en) van deze hare bananen was er een tros niet meer; ``he,'' (zei ze) ``wie (heeft) deze nu opgegeten? misschien de oorwormen.''

En zij zeide: ``als ik ze maar vind, dan (zal) ik ze (dood)-slaan;'' en hij (die jongste broer, die nog onder de bananenplanten zat) riep, hij zeide: ``Moedertje!'' tot driekeer toe, toen eerst hoorde zij (het), en zij zeide: ``he, wie is hier, die moedertje! moedertje! roept?'' En hij vertoonde zich, en zij zeewaarts (naar hem toe) en zij zeide: ``he, mijn zoontje'' (werkelijk mensch, in tegenstelling van bovenaardsch wezen). Toen kookte zij rijst voor hem en hij at.
